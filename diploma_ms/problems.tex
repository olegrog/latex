Для апробации изложенного численного метода решения кинетического уравнения Больцмана
рассмотрена наиболее фундаментальная одномерная задача динамики разреженного газа "--- 
течение газа между двумя параллельными пластинами.

\begin{wrapfigure}{r}{4cm}
	\vspace{-10pt}
	\centering
	\begin{tikzpicture}[dashdot/.style={dash pattern=on .4pt off 3pt on 4pt off 3pt},
						>=latex',thick, scale=1]
		\fill[gray!20] (0,0) -- (3.6,0) -- (3.6,1.5) -- (0,1.5) -- cycle;
		\draw[dashdot] (-.2,0) -- (3.8,0);
		\draw[very thick] (0,1.5) -- (3.6,1.5);
		\draw[<->] (.7,0) -- (.7,.75) node[left] {\(\dfrac{L}{2}\)} -- (.7,1.5);
		\draw[<->] (3.2,0) node[above] {\(z\)} -- (2.4,0) -- (2.4,.8) node[right] {\(x\)};
	\end{tikzpicture}
	\vspace{-20pt}
	\caption{Схема задачи}\label{fig:parallel_plates}
	\vspace{-10pt}
\end{wrapfigure}

Расстояние между пластинами фиксировано и равно \(L\).
Коэффициент аккомодации \(\alpha=1\), что соответсвует полному диффузному отражению молекул от пластин.
Координатные оси направлены в соответствии со схемой на рис.~\ref{fig:parallel_plates}.
Изучается стационарное состояние одноатомного идеального газа.
В качестве молекулярной потенциала используется модель твердых сфер.

В зависимости от природы, действующих на разреженный газ сил, можно выделить 4 различных физических явления:

\begin{enumerate}
	\item \textit{Течение Куэтта}. Газ приводится в движение за счёт относительной скорости пластин.
		В классической газовой динамики задача ограничивается изучением тензора напряжений,
		при кинетическом рассмотрении кроме потока массы возникают также тепловые потоки,
		как вдоль пластин, так и в перпендикулярном направлении.
	\item \textit{Перенос тепла}. Вследствие разности температур пластин возникает поперечный поток тепла.
		В модели сплошной среды он пропорционален градиенту температур,
		но в пределе бесстолновительного газа он зависит только от разности температур.
	\item \textit{Течение Пуазёйля}\footnote
		{
			Поскольку ранее по типографским правилам буквы \textit{е} и \textit{ё} не различались,
			распространённым стало чтение фамилии Пуазёйль через букву \textit{е}, однако здесь мы будем придерживаться
			классической транслитерации французских слов.
		}. В этом случае возникает продольный поток газа вследствие заданного градиента давления.
		Из уравнений Навье"--~Стокса легко получается параболический профиль скоростей между пластинами.
		В разреженном газе возникает эффект \textit{скольжения} газа вдоль пластин, появляются тепловые потоки.
		Кроме того, возникает \textit{парадокс Кнудсена} "--- существование минимума потока массы в зависимости от числа Кнудсена.
	\item \textit{Тепловая траспирация} (англ. \textit{transpiration} "--- просачивание).
		Рассматривая протекание газа через неравномерно нагретые пористые вещества, Осборн Рейнольдс в 1880 году заметил [],
		что вдоль твёрдых поверхностей с градиентом температур возникает так называемое \textit{тепловое скольжение} (\textit{thermal creep}) газа.
		В это же время Джеймс Максвелл [] привел теоретическое обоснование этого явления на основе предположения,
		что неравномерное температурное распределение в газе приводит к внутренним напряжениям.
		Тепловое скольжение происходит в пристеночном слое газа толщиной порядка длины свободного пробега,
		поэтому этот эффект исчезает в гидродинамическом пределе.
\end{enumerate}

Будем рассматривать линейные приближения задач, которые соответствуют малым градиентам макропараметров.
В такой постановке они хорошо изучены для всего диапазона чисел Кнудсена, и при этом подробно табулированы с высокой точностью.
Целью моделирования является верификация проекционного метода решения кинетического уравнения Больцмана
на основе точного решения линеаризованного приближения.

