Динамика разреженных газов изучает явления, имеющие место при произвольном
отношении длины (времени) свободного пробега молекул к характерному размеру (времени) явления.
Изучаемые явления могут быть сколь угодно далекими от равновесных.
Их исследование требует в общем случае учета молекулярной структуры газа,
кинетического описания, применения уравнения Больцмана.

Кинетическое уравнение Больцмана по количеству информации, необходимой для описания газа,
находится между уравнением Лиувилля, справедливого для произвольной системы, состоящей из конечного числа частиц,
и гидродинамическими уравнениями Навье"--~Стокса, которые служат моделью сплошной среды.
Модель идеального газа позволяет нам вывести уравнение Больцмана из уравнения Лиувилля,
т.\,е. осуществить переход от многочастичной функции распределения к одночастичной,
а в предельном случае малых длин пробега можно ограничиться непосредственно уравнениями эволюции макропараметров,
которые являются основой классической газовой динамики.
Поэтому только с помощью кинетической теории, из анализа уравнения Больцмана, можно обоснованно
вывести уравнения Эйлера и Навье"--~Стокса, установить область их применимости,
снабдить их правильными начальными и граничными условиями, коэффициентами переноса.

В круг классических задач динамики разреженных газов входят, например,
задачи об обтекании летательных аппаратов, движущихся на больших высотах, о движении газов в вакуумных аппаратах,
ультразвуковых колебаниях в газах, структуре ударных волн, неравновесных течениях и т.\,д.
В XXI веке в связи с бурным развитием нанотехнологий особо выделилось направление Gas in MEMS/NEMS [].
Действительно, длина свободного пробега газа при нормальных условиях порядка сотен нанометров,
поэтому в системах соответствующего размера уже приходится использовать модель разреженного газа.
Несмотря на то что сейчас многие исследователи активно пытаются расширить область применения 
уравнений Навье"--~Стокса до чисел Кнудсена порядка \(\Kn\approx0.1\), вводя в т.\,ч. условия скольжения высоких порядков [],
очевидно, что этот путь носит временный характер.

Обычно связующим звеном между уравнениями Навье"--~Стокса и кинетической теорией служит разложение Чепмена"--~Энскога [],
однако это разложение имеет множество недостатков. Неясным является степень аппроксимации получаемых гидродинамических уравнений.
В первом приближении это уравнения Эйлера, во втором "--- уравнения Навье"--~Стокса, в третьем "--- уравнения Барнетта.
Непонятным остаётся статус рядов макропараметров. Проблемы также возникают при попытке решить краевую задачу.
Поэтому в настоящее время основным инструментом асимптотического исследования явлется разложение Гильберта [].
В частности, с его помощью показано, что существует важный класс задач в континуальном пределе (\(\Kn\to0\)),
для которых ни уравнения Эйлера, ни уравнения Навье"--~Стокса \textit{не} дают верное решение.
Причиной тому служит так называемый \textit{эффект призрака} [] (\textit{ghost effect}).
Говоря общим языком, было установлено, что нечто, не существующее в пределе \(\Kn\to0\),
но имееющее место для конечных чисел Кнудсена \(\Kn\), \textit{конечным} образом влияет на поведение газа в этом пределе.
Сегодня существует множество примеров таких явлений, обусловленных неоднородностью температурного [] и скоростного [] поля.
Эффект призрака возникает в том числе в классических задачах, таких как цилидрическое течения Куэтта [], задача Бенарда [], в газовых смесях [].
Этот факт в большой степени должен концентрировать внимание современной науки и инженерии на кинетической теории.

Аналитическое исследование уравнения Больцмана сложно и трудоёмко, точное решение возможно получить лишь для очень узкого спектра задач.
Некоторые классические задачи динамики разреженного газа удаётся выразить в квадратурах в рамках линеаризованного уравнения Больцмана.
Зачастую некоторые качественные оценки поведения разреженного газа можно получить на основе модельных уравнений [],
однако в этом случае практически невозможно оценить получаемую ошибку, обусловленную приближённостью соответствующей модели.
Асимптотический анализ [] позволяет получить сложные системы уравнения гидродинамического типа, описывающие поведение газа для малых \(\Kn\).
Тем не менее в диапазоне чисел Кнудсена вплоть до свободномолекулярных течений, необходимо прибегать к непосредственному решению уравнения Больцмана.
Для достаточно разреженного газа статистическое моделирование методом DSMC [] даёт удовлетворительные результаты.
Этот метод может применяться (и применяется для одномерных и двумерных задач) в том числе для не очень больших чисел Кнудсена,
однако в таком случае требуются большие вычислительные ресурсы, дабы преодолеть значительные статистические флуктуации.

Ввиду сложности нелинейного интеграла столкновения долгое время не существовало консервативного конечно-разностного метода решения 
уравнения Больцмана. Эта проблема была впервые решена в рамках \textit{проекционного метода дискретных ординат} [].
Что касается инженерных задач, то на сегодняшний день хорошо развиты \textit{проблемно-моделирующие среды} (\textit{problem solving environment})
как на основе уравнений Навье"--~Стокса, так и с использованием метода DSMC.
Кроме того, применяются методики «сшивания» решений полученных обоими методами [].
Всё же диапазон чисел Кнудсена порядка единицы, называемый \textit{переходным} режимом, остаётся незаполненным соответствующим
надёжным и проверенным численным методом. В качестве такового предлагается использовать проекционный метод дискретных ординат.
Данная работа посвящена попытке частичной верификации и оценке точности этого метода,
который показал хорошую сходимость к точным решениям линейной теории в переходном режиме.
Его использование для малых чисел Кнудсена (менее 0.001) затруднено ввиду высоких вычислительных затрат (но в гораздо меньшей степени, чем DSMC),
а в области сильно разреженного газа он обладает недостатками, свойственными любому методу дискретных ординат:
наличию выделенных направлений в дискретной скоростной сетке.

Автор искренне надеется, что проекционный метод дискретных ординат в будущем сможет занять свою нишу,
а универсальный инженерный инструмент моделирования идеального газа будет сочетать все три описанные выше подхода.

