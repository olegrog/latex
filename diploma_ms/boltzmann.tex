\subsection{Молекулярная структура газа}
Кинетическая теория основывается на гипотезе о том, что все вещества, в том числе и газы, состоят из молекул.
Газом называется совокупность молекул, находящихся на столь больших расстояниях друг от друга,
что молекулы большую часть времени слабо взаимодействуют друг с другом.
Короткие промежутки времени, в течение которых молекулы сильно взаимодействуют, рассматриваются как \textit{столкновения}.
Если осредненной по времени потенциальной энергией взаимодействия молекул можно пренебречь
по сравнению с их кинетической энергией, то газ называется \textit{идеальным} (\textit{perfect}).
Практически газы из нейтральных молекул при давлениях до сотен атмосфер могут рассматриваться как идеальные.
До этих же давлений вероятность тройных столкновений мала по сравнению с вероятностью двойных (или парных).
В идеальном газе объём, занятый молекулами, мал по сравнению с объёмом, занятым газом.
Если \(d_m\) "--- эффективный диаметр молекулы, то асимптотически в идеальном газе \(\rho d_m^3/m \to 0 \),
где \(\rho/m\) "--- число молекул в единице объема.
Назовем идеальный газ \textit{газом Больцмана}, если отношение длины пробега молекул в этом
газе к характерному размеру течения \(L\) конечно, т.\,е. если \(L\rho d_m^2/m=\mathrm{const}\),
поскольку длина пробега обратно пропорциональна \(\rho d_m^2/m\).
Если \( L\rho d_m^2/m \to 0 \), то такой газ будем называть \textit{газом Кнудсена}.

Предполагается, что движение молекул может быть описано с помощью классической ньютоновской механики.
Квантовые эффекты существенны лишь при очень низких температурах и для легких молекул (водород, гелий, электроны).
Для водорода и гелия квантовые поправки существенны уже при нормальных условиях.
Большинство же газов сжижается при температуре, при которой еще нет необходимости применять квантовую теорию столкновения молекул.
Квантовые эффекты необходимо учитывать при неупругих столкновениях атомов и молекул 
(возбуждение внутренних степеней свободы молекул, возбуждение электронных уровней и т.\,п.).
Потенциалы упругих взаимодействий молекул также могут быть вычислены лишь с помощью квантовой механики.
Однако при известном потенциале взаимодействия упругие столкновения могут быть рассмотрены классически.

Релятивистские эффекты существенны лишь при очень больших температурах (больших скоростях молекул).
Практически эти эффекты можно не учитывать при температурах порядка десятков и сотен тысяч градусов.
Для водорода, например, средняя скорость молекул при температуре в 105~K равняется 0.0001 скорости света.
Даже скорость электрона при такой температуре составляет тысячные доли скорости света.

Таким образом, рассматриваемая теория идеального газа с учетом парных столкновений в рамках классической механики
удовлетворительно описывает движение газа в широком диапазоне температур и давлений
(для температур от десятков градусов Кельвина до сотен тысяч и для давлений до сотен атмосфер).

\subsection{Законы взаимодействия молекул}
Хотя в принципе квантовомеханический расчет потенциала взаимодействия возможен для любых молекул,
на практике обычно пользуются эмпирическими и полуэмпирическими законами взаимодействия.
Приведем некоторые наиболее распространенные из них:
\begin{enumerate}
    \item \textit{Упругие шары}.
Пусть \(d\) "--- диаметр шара, тогда потенциал взаимодействия двух шаров представляется в виде
\[U_{HS}(r) = \begin{cases}
	\infty	& \text{при } r < d, \\
	0		& \text{при } r > d.
\end{cases}\]
Хотя эта модель грубо моделирует лишь короткодействующие силы отталкивания,
она весьма часто употребляется в расчетах благодаря своей простоте.
Эта модель также очень широко применяется для качественных рассмотрений,
связанных с процессом столкновения молекул,
так как для твердых шаров процесс столкновения представляется наиболее наглядно.
Более того, при применении других, более сложных законов взаимодействия обычно вводят так называемые
\textit{эффективные сечения} столкновений, с помощью которых столкновение реальных молекул заменяется
столкновением в некотором смысле эквивалентных им упругих сфер.
	\item \textit{Центры отталкивания с потенциалом} \[U_M(r) = \frac{K}{r^{s-1}}.\]
Величину \(s-1\) (а иногда и \(s\)) называют показателем отталкивания,
благодаря чисто математическим удобствам широкое распространение получили гипотетические
молекулы с показателем отталкивания \(s=5\), называемые \textit{максвелловскими} [].
Газ, состоящий из таких молекул, называют \textit{максвелловским газом}.
Ближе к реальным значения \(s\), большие 5, например 9 или 12.
Максвелловские молекулы слишком «мягкие», в то время как упругие сферы слишком «жесткие».
Реальные потенциалы взаимодействия лежат между этими двумя наиболее распространенными модельными потенциалами.
Следует заметить, что применение весьма удобных в математическом отношении максвелловских молекул иногда
приводит к неверным эффектам. Так, например, в максвелловском газе отсутствует явление термодиффузии.
	\item \textit{Потенциал Леннард-Джонса} \[ U_{LJ}(r) = \frac{d}{r^n} - \frac{e}{r^m} .\]
Первый член описывает отталкивание с показателем \(n\), второй "--- притяжение; \(d\), \(e\) "--- константы.
Чаще всего применяется так называемый \textit{(6-12)-потенциал Леннард-Джонса} []:
\[ U_{LJ}(r) = 4\varepsilon\left[\left(\frac{d}{r}\right)^{12} - \left(\frac{d}{r}\right)^6\right] .\]
Шестая степень убывания потенциала моделирует электростатическое диполь-дипольное и дисперсионное притяжение.
Двенадцатая степень убывания отталкивающего потенциала выбрана из соображений математического удобства.
В то же время она моделирует достаточно жесткое отталкивание.
При \(r=d\) потенциал равен нулю. Величина \(\varepsilon\) характеризует глубину потенциальной ямы.
\end{enumerate}

\subsection{Уравнение Лиувилля}

Рассмотрим систему, состоящую из \(N\) одинаковых частиц, каждая из которых имеет массу \(m\).
Возьмём \(6N\)-мерное \(\Gamma\) пространство, состоящее из положения \(\boldsymbol{X}^{(i)}\)
и скорости \(\boldsymbol\xi^{(i)}\) частицы \(i\) (\(i=1,2,\dots, N\)).
Пусть \(f^N\), функция \(6N + 1\) переменных \(\boldsymbol{X}^{(1)},\dots,\boldsymbol{X}^{(N)}\),
\(\boldsymbol{\xi}^{(1)},\dots,\boldsymbol{\xi}^{(N)}\) и \(t\), "---
\(N\)-частичная функция плотности вероятности нахождения частицы \(i\) (\(i=1,2,\dots, N\))
в точке \(\boldsymbol{X}^{(i)}\) и со скоростью \(\boldsymbol\xi^{(i)}\) в \(\Gamma\) пространстве 
с \(6N\) размерностью в момент времени \(t\).

Поведение функции распределения \(f^N\) определяется \textit{уравнением Лиувилля}
\begin{equation}\label{eq:Liouville}
	\frac{\partial f^N}{\partial t} + \sum_{i=1}^{N}\left(
		\boldsymbol\xi^{(i)}\frac{\partial f^N}{\partial \boldsymbol{X}^{(i)}} + 
		\boldsymbol{F}^{(i)}\frac{\partial f^N}{\partial \boldsymbol{\xi}^{(i)}}\right) = 0,
\end{equation}
которое является точным уравнением для \(N\)-частичной системы, подчиняющейся закону Ньютона.
\(mF^{(i)}\) "--- сила, действующая на частицу \(i\).

С помощью \textit{цепочки уравнений Грэда} [] (\textit{Grad hierarchy})\footnote{
	\textit{Цепочка уравнений Боголюбова} [] (\textit{BBGKY hierarchy}) позволяет осуществить аналогичный переход
	для молекулярных потенциалов с конечным радиусом действия, например, для модели твёрдых сфер.
} можно формально перейти от уравнения Лиувилля к уравнению Больцмана,
от динамики \(N\)-частичной системы к больцмановскому газу, описываемому одночастичной функцией распределения.
Для этого необходимо ввести следующие предположения.
\begin{enumerate}
	\item Так называемый \textit{предельный переход Грэда"--~Больцмана}:
устремляем число частиц к бесконечности \(N\to\infty\),
зануляем эффективный диапазон действия межмолекулярного взаимодействия \(d_m\to0\),
сохраняя фиксированным значение \(Nd_m^2=\mathrm{const}\).
В таком случае можно рассматривать только парные столкновения.
	\item Вероятности нахождения любых двух частиц в фазовом пространстве независимы:
\begin{equation}\label{eq:chaos}
	f_2(\boldsymbol{X}^{(1)}, \boldsymbol{\xi}^{(1)}, \boldsymbol{X}^{(2)}, \boldsymbol{\xi}^{(2)}, t) =
	f_1(\boldsymbol{X}^{(1)}, \boldsymbol{\xi}^{(1)}, t) f_1(\boldsymbol{X}^{(2)}, \boldsymbol{\xi}^{(2)}, t).
\end{equation}
Такое предположение называют \textit{молекулярным хаосом}.

Единственным механизмом, который может приводить к установлению хаоса или нарушать его, является механизм столкновений молекул.
С одной стороны, очевидно, что столкновения нарушают условия хаоса,
поскольку положение и скорости только что столкнувшихся молекул коррелированы.
Однако, с другой стороны, вероятность вторичного столкновения этих же молекул стремится к нулю при \(N\to\infty\).
Прежде чем молекулы столкнутся вторично, каждая из них испытает огромное число столкновений с другими молекулами.
Поэтому можно ожидать, что условие хаоса сохраняется, если оно выполнялось в начальный момент.
В подавляющем большинстве случаев реальные молекулярные системы удовлетворяют этому требованию.
В тех исключительных случаях, когда условие молекулярного хаоса в начальный момент или на границах
не выполняется, можно ожидать, что молекулярный хаос установится за время порядка времени между столкновениями.

Для получения уравнения Больцмана достаточно выполнение \textit{одностороннего} (\textit{one-side}) условия хаоса:
равенство \eqref{eq:chaos} выполняется для непосредственно сталкивающихся частиц, т.\,е. до столкновения.
Можно формально показать [], что при \(N\to\infty\), во-первых, почти любое начальное состояние хаотично,
во-вторых, одностороннее условие хаоса сохраняется.
Даже если начальное состояние было бы упорядоченным, то в силу принципа неопределённости Гейзенберга для прицельного угла столкновения
начальная корреляция была бы разрушена за время порядка времени между столкновениями.

	\item Наконец, необходимо предположение о \textit{медленном изменении} функции распределения
в пространстве \(\boldsymbol{X}\) на расстояниях порядка диаметра взаимодействия \(d_m\)
и в течение времени порядка \(d_m/\xi_0\) (\(\xi_0\) "--- характерная скорость частиц).
Часто это свойство функции распределения называют \textit{однородностью в \(d_m\)-масштабе}.
В силу этого предположения уравнение Больцмана не способно описывать поведение частиц на молекулярном уровне.

\end{enumerate}

Основанное на \textit{обратимых законах механики} уравнение Больцмана описывает \textit{необратимые процессы}.
Именно вводя предположение о молекулярном хаосе,
мы отступили от чисто механического (детерминированного) обратимого описания движении системы.
Вероятностный характер описания газа обусловлен также вероятностными начальными и граничными условиями.

Далее ограничимся резюмированием основных определений, формул и фундаментальных свойств, относящихся к уравнению Больцмана.
Детальное объяснение и вывод можно найти в классических трудах [].

\subsection{Уравнение Больцмана}

Пусть \(X_i\) (или \(\boldsymbol X\)) "--- декартовы координаты нашего физического пространства,
\(\xi_i\) (или \(\boldsymbol\xi\)) "--- молекулярная скорость.
Пусть число \(\D N\) молекул в шестимерном элементе объёма \(\D X_1\D X_2\D X_3\D\xi_1\D\xi_2\D\xi_3\)
(\(\D\boldsymbol X\D\boldsymbol\xi\) для краткости) выражено как
\[ \D N = \frac1{m} f(\boldsymbol X,\boldsymbol\xi,t)\D\boldsymbol X\D\boldsymbol\xi, \]
где \(m\) "--- масса молекулы, \(t\) "--- время.
Тогда \(f\) или \(f/m\), которая является функция семи переменных \(\boldsymbol X\), \(\boldsymbol\xi\) и t,
называется \textit{функцией распределения} скоростей молекул газа.

Макроскопические переменные: плотность газа \(\rho\), скорость потока \(v_i\),
температура \(T\), давление \(p\), удельная внутренняя энергии \(e\), тензор напряжений \(p_{ij}\)
и вектор теплового потока \(q_i\), "--- в точке \(\boldsymbol X\) и в момент времени \(t\) определены следующими моментами \(f\):

\begin{align}\label{eq:macro}
	\rho &= \int f(\boldsymbol X,\boldsymbol\xi,t)\boldsymbol{\dd\xi}, \\
	v_i &= \frac1{\rho}\int\xi_i f(\boldsymbol X,\boldsymbol\xi,t)\boldsymbol{\dd\xi}, \\
	3RT &= \frac1{\rho}\int (\xi_i-v_i)^2 f(\boldsymbol X,\boldsymbol\xi,t)\boldsymbol{\dd\xi}, \\
	p &= \frac1{3}\int (\xi_i-v_i)^2 f(\boldsymbol X,\boldsymbol\xi,t)\boldsymbol{\dd\xi} = R\rho T, \\
	e &= \frac1{\rho}\int\frac1{2} (\xi_i-v_i)^2 f(\boldsymbol X,\boldsymbol\xi,t)\boldsymbol{\dd\xi} = \frac3{2}RT, \\
	p_{ij} &= \int (\xi_i-v_i)(\xi_j-v_j) f(\boldsymbol X,\boldsymbol\xi,t)\boldsymbol{\dd\xi}, \\
	q_i &= \int\frac1{2} (\xi_i-v_i)(\xi_j-v_j)^2 f(\boldsymbol X,\boldsymbol\xi,t)\boldsymbol{\dd\xi}, \\
\end{align}
где \(R\) "--- удельная газовая постоянная, т.\,е. постоянная Больцмана \(k_B\) (\(=1.380658\times 10^{-23} \text{Дж·К}^{-1}\)), делённая на \(m\).
Трехмерное интегрирование по \(\boldsymbol\xi\) здесь и далее проводится во всём пространстве \(\boldsymbol\xi\).
Эти определения согласуются с аналогами из классической газодинамики.

Масса \(M\), импульс \(P_i\), и энергия \(EF\), переносимая от газа
до границы, в точке \(\boldsymbol X\) на ее единицу площади и в единицу времени задаются как
\begin{align}\label{eq:macro_transfer}
	M &= -\int (\xi_j-v_{wj}) n_j f(\boldsymbol X,\boldsymbol\xi,t)\boldsymbol{\dd\xi} = -n_j\rho(v_j-v_{wj}), \\
	P_i &= -\int \xi_i(\xi_j-v_{wj}) n_j f(\boldsymbol X,\boldsymbol\xi,t)\boldsymbol{\dd\xi} = -n_j\left[p_{ij}+\rho v_i(v_j-v_{wj})\right], \\
	EF &= -\int\frac1{2} \xi_i^2(\xi_j-v_{wj}) n_j f(\boldsymbol X,\boldsymbol\xi,t)\boldsymbol{\dd\xi} 
		= -n_j\left[q_j+p_{ij}v_i+\rho\left(e+\frac1{2}v_i^2\right)(v_j-v_{wj})\right], \\
\end{align}
где \(v_{wi}\) "--- скорость границы, \(n_i\) "--- единичный вектор нормали к границы, направленный в сторону газа.
В отсутствие потока массы (\(M=0\)) через границу,
\[ P_i = -n_j p_{ij}, \quad EF = -n_j (q_j+p_{ij}v_i). \]

Поведение функции распределения \(f\) определяется \textit{уравнением Больцмана}:
\begin{equation}\label{eq:Boltzmann}
	\frac{\partial f}{\partial t} + \xi_i\frac{\partial f}{\partial X_i}
	+ F_i\frac{\partial f}{\partial\xi_i} = J(f,f),
\end{equation}
\begin{equation}\label{eq:integral}
	J(f,g) = \frac1{2m}\int_{\boldsymbol\alpha,\boldsymbol\xi_*}
	(f'g'_*+f'_*g'-fg_*-f_*g) B\left(\frac{|\alpha_j V_j|}{V},V\right)\dd\Omega(\boldsymbol\alpha)\boldsymbol{\dd\xi_*},
\end{equation}
где
\[ \left.\begin{array}{l l}
	f=f(X_i,\xi_i,t), & f_*=f(X_i,\xi_{i*},t), \\
	f'=f(X_i,\xi_i',t), & f'_*=f(X_i,\xi'_{i*},t), \\
	\xi_i'=\xi_i+\alpha_i\alpha_j V_j, & \xi_{i*}'=\xi_{i*}-\alpha_i\alpha_j V_j, \\
	V_i=\xi_{i*}-\xi_i, & V=(V_i^2)^{1/2} = |V_i| \\
\end{array}\right\}\]
и \(mF_i\) "--- внешняя сила, действующая на молекулу, \(\alpha_i\) (или \(\boldsymbol\alpha\)) "--- единичный вектор,
выражающий изменение направления молекулярной скорости из-за столкновения молекул,
\(\Omega(\alpha)\) "--- элемент телесного угла в направлении \(\alpha_i\),
\(B(|\alpha_i V_i|/V,V)\) "--- неотрицательная функция своих аргументов, определяемая межмолекулярным потенциалом,
например, для газа, состоящего из твердых сфер молекул с диаметром \(d_m\), \(B=d_m^2 |\alpha_i V_i|/2\).
Интегрирование по \(\xi_{i*}\) и \(\alpha_i\) проводятся по всему пространству \(\xi_{i*}\) и
по всем направлениям \(\alpha_i\) (весь сферической поверхности) соответственно.
Интеграл \(J(f,f)\) называется \textit{интегралом столкновения} или столкновительным членом уравнения Больцмана.

Момент \(\int\varphi(\boldsymbol\xi)J(f,g)\boldsymbol{\dd\xi}\), где \(\varphi\) "--- произвольная функция от \(\boldsymbol\xi\),
удовлетворяет соотношению симметрии
\[ \int\varphi(\boldsymbol\xi)J(f,g)\boldsymbol{\dd\xi} = \frac1{8m}\int(\varphi+\varphi_*\varphi'-\varphi'_*)
	(f'g'_*+f'_*g'-fg_*-f_*g) B\dd\Omega(\boldsymbol\alpha)\boldsymbol{\dd\xi_*}\boldsymbol{\dd\xi}. \]
Выбирая в качестве \(\varphi\) \(1\), \(\xi_i\), \(\xi_i^2\), легко видеть, что
\begin{equation}\label{eq:conser_moments}
	\int \begin{pmatrix} 1 \\ \xi_i \\ \xi_i^2 \end{pmatrix} J(f,g)\boldsymbol{\dd\xi}=0,
\end{equation}
поскольку \(\xi_i+\xi_{i*}=\xi'_i+\xi'_{i*}\) и \(\xi_i^2+\xi_{i*}^2={\xi'_i}^2+{\xi'_{i*}}^{\!\!\!2}\).

Умножая уравнения Больцмана \eqref{eq:Boltzmann} на \(1\), \(\xi_i\) или \(\xi_i^2\)
и интегрируя результат по всему пространству \(\boldsymbol\xi\), получим следующие уравнения сохранения:
\begin{gather}\label{eq:conservation}
	\frac{\partial\rho}{\partial t} + \frac{\partial}{\partial X_i}(\rho v_i) = 0, \\
	\frac{\partial}{\partial t}(\rho v_i) + \frac{\partial}{\partial X_j}(\rho v_i v_j + p_{ij}) = \rho F_i, \\
	\frac{\partial}{\partial t}\left[\rho\left(e+\frac1{2}v_i^2\right)\right] + 
		\frac{\partial}{\partial X_j}\left[\rho v_j\left(e+\frac1{2}v_i^2\right) + v_i p_{ji} + q_j\right] = \rho v_j F_j, \label{eq:conservation1}
\end{gather}
где сила \(F_i\) предполагается не зависящей от молекулярной скорости \(\boldsymbol\xi\).
Столкновительный член обращается в нуль при интегрировании (ввиду \eqref{eq:conser_moments}).
Это \textit{уравнения сохранения} массы, импульса и энергии в классической газовой динамике,
которые дополняются соответствующими формами \(p_{ij}\) и \(q_i\), чтобы замкнуть систему \eqref{eq:conservation}--\eqref{eq:conservation1}.
Например,
\begin{equation}\label{eq:Euler}
	p_{ij}=p\delta_{ij}, \quad q_i=0,
\end{equation}
или
\begin{equation}\label{eq:Navier-Stokes}
	p_{ij}=p\delta_{ij}-\mu\left(\frac{\partial v_i}{\partial X_j}+\frac{\partial v_j}{\partial X_i}-\frac2{3}\frac{\partial v_k}{\partial X_k}\delta_{ij}\right)
	- \mu_B\frac{\partial v_k}{\partial X_k}\delta_{ij}, \quad q_i=-\lambda\frac{\partial T}{\partial X_i},
\end{equation}
где \(\mu\), \(\mu_B\), \(\lambda\), называемые \textit{вязкостью}, \textit{объёмной (второй) вязкотью} и 
\textit{теплопроводностью} газа соответственно, являются функциями температуры.
Множество уравнений с тензором напряжений и потоком тепла, выраженными как \eqref{eq:Euler}, называются \textit{уравнениями Эйлера},

\subsubsection{Граничные условия}
\subsubsection{Безразмерные переменные}

В работе будем придерживаться современных обозначений, принятых 
в классической монографии Соуна ``Молекулярная динамика''~\cite{Sone2007}.
Безразмерные величины будем отмечать символом ``шляпки''.

Микроскопические величины в уравнении Больцмана
\[
	\frac{\partial{f}}{\partial{t}} + \xi_i\frac{\partial{f}}{\partial X_i} = 
	\int (f'f'_1-ff_1)|\boldsymbol{\xi}-\boldsymbol{\xi}'|b\dd b \dd \varepsilon \boldsymbol{\dd\xi},
\]
такие как координата \(X_i\), скорость \(\xi_i\), время \(t\), прицельное расстояние \(b\)
и функция распределения \(f\) принимают безразмерный вид согласно формулам:
\[ f = \hat{f}f_0,\; \xi_i = \zeta_i\nu_0,\; X_i = x_i\ell,\;
	t = \hat{t}\frac{\ell}{\nu_0},\; b = \hat{b}d_m \]
Для макроскопических параметров, таких как плотность \( \rho = \hat{\rho}\rho_0\), макроскопическая скорость \(v_i = \hat{v}_iv_0\),
температура \(T = \hat{T}T_0\), тензор напряжений \(p_{ij} = \hat{p}_{ij}p_0\) и тепловой поток \(q_i = \hat{q}_iq_0\) 
безразмерные соотношения определим естественным образом:
\[ \rho_0 = f_0 \nu_0^3, \; v_0 = \nu_0, \; p_0 = \rho_0RT_0, \; q_0 = p_0\nu_0, \]
где \(R = k_B/m\) "--- удельная газовая постоянная, равная отношению постоянной Больцмана \(k_B\) к молекулярной массе \(m\).
Значения \(\rho_0\) и \(T_0\) выбирались как средние плотность и температура соответственно,
в частности для задачи переноса тепла \(T_0 = (T_1+T_2)/2\).

Наконец, свяжем две единицы измерения длины \(x\), \(b\) и единицы температуры \(T_0\), скорости \(\nu_0\):
\[ \ell = \frac{m}{\pi\sqrt2 \rho_0 d_m^2}, \; \nu_0 = 2RT_0, \]
так что \(\ell\) окажется длиной свободного пробега, \(d_m\) "--- эффективным диаметром молекул газа,
а \(\nu_0\) "--- средней тепловой скоростью газа.

Итак, в безразмерных переменных кинетическое уравнение Больцмана примет вид
\[ \frac{\partial\hat{f}}{\partial\hat{t}} + \zeta_i\frac{\partial\hat{f}}{\partial x_i} = \hat{J}(\hat{f},\hat{f}), \]
\[ 
	\hat{J}(\hat{f},\hat{f}) = \frac1{\pi\sqrt2}\int (\hat{f'}\hat{f'_1}-\hat{f}\hat{f_1})
	|\boldsymbol{\zeta}-\boldsymbol{\zeta}'| \hat{b}\dd \hat{b} \dd \varepsilon \boldsymbol{\dd\zeta},
\]
а макропараметры будут вычиляться по формулам:
\begin{alignat*}{2}
	\hat{\rho} &= \int \hat{f}\boldsymbol{\dd\zeta}, \\
	\hat{\rho}\hat{v}_i &= \int \zeta_i \hat{f}\boldsymbol{\dd\zeta}, \\
	\frac3{2}\hat{\rho}\hat{T} &= \int(\zeta_i-\hat{v}_i)^2\hat{f}\boldsymbol{\dd\zeta}, \\
	\frac1{2}\hat{p}_{ij} &= \int(\zeta_i-\hat{v}_i)(\zeta_j-\hat{v}_j)\hat{f}\boldsymbol{\dd\zeta}, \\
	\hat{q}_i &= \int(\zeta_i-\hat{v}_i)(\zeta_j-\hat{v}_j)^2\hat{f}\boldsymbol{\dd\zeta}.
\end{alignat*}

Давление вычисляется как \(\hat{p} \equiv \hat{p}_{ii}/3 = \hat{\rho}\hat{T}\), а число Кнудсена будем определять как \(\Kn=\ell/L\).
Аналогично можно записать безразмерные коэффициенты вязкости \(\mu\) и теплопроводности \(\lambda\):
\[ \mu = \hat{\mu}\rho_0\nu_0\ell, \; \lambda = \hat{\lambda}R\rho_0\nu_0\ell. \]
 
\subsection{Модельные уравнения}

\subsection{Разложения Гильберта и Чепмена"--~Энскога}

\subsection{Линейная теория}

\subsection{Нелинейная теория. Эффект призрака}

