\documentclass[11pt]{article}
\usepackage[utf8]{inputenc}
\usepackage[T1]{fontenc}
\usepackage[english]{babel}

\usepackage{amssymb, amsmath}
\usepackage{csquotes}
\usepackage[headings]{fullpage}
\usepackage{fancyhdr}

\setlength\headheight{13.6pt}
\allowdisplaybreaks[3]

\pagestyle{fancy}
    \lhead{Research Statement}
    \rhead{July 2017}
    \chead{{\large{\bf Oleg Rogozin}}}
    \lfoot{}
    \rfoot{\bf \thepage}
    \cfoot{}

\usepackage[
    pdfauthor={Oleg Rogozin},
    pdftitle={Research Statement},
    colorlinks,
    pdftex,
    unicode]{hyperref}

\usepackage[
    backend=biber,
    style=verbose-trad3,
    language=auto,
    doi=false,
    eprint=true,
    giveninits]{biblatex}
\bibliography{statement}

\renewcommand\cite\footcite

\begin{document}

\section{Past research accomplishments}

My CFD part of research mainly relates to the solution of the \emph{Kogan--Galkin--Friedlander} (KGF) equations~\cite{Kogan1976}.
These equations correctly describe flows driven by large temperature variations.
Due to some thermal-stress (non-Navier--Stokes) terms in the momentum equation,
the KGF equations predicted a new type of gravity-free convection.
I have adopted the widespread SIMPLE algorithm (for incompressible Navier--Stokes equations)
to implement the finite-volume KGF solver,
extensively used for the parametric numerical analysis of some slow nonisothermal flows~\cite{Rogozin2014}.

The main part of my research devotes to the \emph{kinetic theory of gases} with special interest
in the high-accuracy numerical analysis of the boundary-value problems on the basis of the Boltzmann equation.
The major difficulty of the kinetic description is associated with sharp variations of the distribution function.
The common \emph{DSMC method} is able to achieve the engineering accuracy in many cases,
mainly due to the modern high-performance computing; however, its resolution capability is strongly
limited by the inherited stochastic noise.
I have extended the \emph{Tcheremissine's projection discrete-velocity method}~\cite{Tcheremissine1998, Tcheremissine2006}
for nonuniform velocity grids to obtain high-accuracy solutions of
some classical molecular gas dynamics problems~\cite{Rogozin2016, Rogozin2017}.
Sone's asymptotic theory is actively employed to check the obtained accuracy for small Knudsen numbers~\cite{Sone2007}.

My experience principally connected with \verb_C++_-based solvers;
\verb+python+ is actively used for the data analysis.
I also employ special third-party software for pre- and postprocessing,
e.g. GMSH for mesh generation, ParaView for visualization.

\section{Past software development accomplishments}

Aforementioned problem-solving environments are designed and developed under my active participation.
The \emph{Boltzmann equation solver} has been firstly implemented from scratch by the team of students~\cite{Rogozin2010}.
C++11 with MPI are mainly used to achieve the high parallel performance.
The cross-platform open-source code, which is based on the object-orienteering and template paradigms,
is available on GitHub~\cite{github-kesolver}.
The \emph{KGF equations solver} \verb+snitSimpleFoam+ and supplementary toolkit
are developed within the CFD platform OpenFOAM.
The open-source code is also available on GitHub~\cite{github-openfoam}.

Additional software development experience was obtained in bio-informatics startup iBinom.
It connected with industry-level coding standards, peer-reviewed code, unit testing and so on.

\section{Research / software development interests}

My current research closely coupled with numerical methods for multiscale simulation of rarefied gas flows.
In collaboration with Prof. Vladimir Aristov and Dr. Oleg Ilyin\footnotemark, we develop approaches
for \emph{fluid-kinetic coupling} of the Boltzmann equation and lattice Boltzmann equation.
The former is able to describe the Knudsen layer correctly, the latter has a low computational cost.
\footnotetext{
    Federal Research Center "Computer Science and Control" of Russian Academy of Sciences, Moscow, Russia
}

In collaboration with Prof. Zhi-Hui Li\footnotemark, we develop the \emph{gas-kinetic schemes}.
These \emph{asymptotic-preserving} schemes aimed to avoid low-Knudsen-number stiffness,
but computationally efficient numerical schemes are proposed only for simplified model equations.
\footnotetext{
    Hypervelocity Aerodynamics Institute, China Aerodynamics Research and Development Center, Mianyang, China
}

My software development principles comprise extensive use of open-source frameworks and libraries,
cutting-edge technologies, and particular high level of abstraction.
All of them are partially inspired by sophisticated (but a little bit old-fashioned) CFD toolkit OpenFOAM.

\section{Long-term career objectives}

The fundamental challenge for the field of numerical methods in the kinetic theory (and many others) is
development of a \emph{universal} computational efficient high-accuracy multiscale solver.
Adaptive grids in the velocity space seem to be the only way to accomplish this objective.
Some progress has been approached for 2D problems recently,
but used algorithms are still far from optimal~\cite{Kolobov2013}.
The Galerkin--Fourier methods have been reached its maturity by providing fast and robust techniques~\cite{Pareschi2006, Reese2013},
while the promising wavelet-based methods are still in its infancy~\cite{Lemou2003, Tran2013}.
Development of the 3D adaptive wavelet-based Boltzmann solver is very tempting but ambitious objective.
Success in it, therefore, depends substantially on the proper splitting into intermediate stages.
Experience in related CFD approach, being exciting in itself at the same time, is one of such important stages in my opinion.
This is the primary motivation to deviate slightly from my main research area and submit the present application.

\end{document}
