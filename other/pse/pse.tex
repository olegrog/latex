%&pdflatex
\documentclass[landscape,a4paper,12pt]{article}

\usepackage[T2A]{fontenc}
\usepackage[utf8]{inputenc}
\usepackage[english,russian]{babel}
\usepackage{amssymb, amsmath, amsfonts}
\usepackage{graphicx}
\usepackage{tikz}
\usetikzlibrary{mindmap}
\usetikzlibrary{decorations.text}
\usetikzlibrary{arrows,fit,positioning,shapes.multipart}
\usepackage{fullpage}

%\usepackage[]{iwona}
%\usepackage{mathpazo}
%\usepackage{comicsans}
%\usepackage{arev}
%\usepackage{mathptmx}
%\usepackage{pxfonts}
%\usepackage{fouriernc}

\begin{document}
\thispagestyle{empty}

\section*{Схема статьи}
\begin{tikzpicture}[block/.style={shape=rectangle,rounded corners,draw=blue!50,fill=blue!20,thick,inner sep=5pt,
						minimum size=1cm,node distance=2cm,text badly centered},
					    miniblock/.style={shape=rectangle,draw=blue!50,fill=blue!30,thick,inner sep=5pt,
						minimum size=1cm,node distance=5mm,text badly centered},>=latex',thick]
	\node[block,minimum width=4cm,minimum height=2.9cm] (in) at (2,1.55)			{};
	\node[block,minimum width=4cm,minimum height=2.9cm] (mesh) at (2,-1.55)			{};
	\node[block,minimum width=5.5cm,minimum height=6cm] (solver) at (7.5,0)			{};
	\node[block,minimum width=3.5cm,minimum height=6cm] (out) at (12.75,0)	        {};
	\draw [<-] (in.east)+(.75cm-1pt,0) to (in.east);
	\draw [<-] (mesh.east)+(.75cm-1pt,0) to (mesh.east);
	\draw [->] (solver) to (out);

	\node[below] at (in.north) {\textbf{Входные данные}};
	\node[miniblock,text width=3.5cm,inner sep=0pt] at (2,2) {Конфигурационный XML файл};
	\node[miniblock,text width=3.5cm,inner sep=0pt] at (2,.8) {Интерактивная оболочка};

	\node[below] at (mesh.north) {\textbf{Расчетная сетка}};
	\node[miniblock,text width=3.5cm,inner sep=0pt] (rgen) at (2,-1.1) {Генератор сеток прямоугольных};
	\node[miniblock,text width=3.5cm,inner sep=0pt] (gmsh) at (2,-2.3) {GSMH};

	\node[below] at (solver.north) {\textbf{Солвер}};
	\node[miniblock,rotate=90,text width=4cm] (int) at (9.4,0) {Интеграл столкновений};
	\node[miniblock,text width=2.5cm] (rect)  at (6.5,1.5) {RectSolv};
	\node[miniblock,text width=2.5cm] (gpu)   at (6.5,0) {GPUSolv};
	\node[miniblock,text width=2.5cm] (unstr) at (6.5,-1.5) {UnstructSolv};
	\draw [<->] (int.north)+(0,1.5) to (rect.east);
	\draw [<->] (int.north) to (gpu.east);
	\draw [<->] (int.north)+(0,-1.5) to (unstr.east);

	\node[below,rotate=90] (visual) at (out.west) {\textit{Визуализация}};
	\node[above,rotate=90] at (out.east) {\textbf{Выходные данные}};
	\node[miniblock,text width=1.5cm] (gnu)  at (13,2.25) 	{Gnuplot};
	\node[miniblock,text width=1.5cm] (para) at (13,0.75) 	{Paraview};
	\node[miniblock,text width=1.5cm] (bk)   at (13,-0.75) {Bkviewer};
	\node[miniblock,text width=1.5cm] (ncl)  at (13,-2.25) {NCL};
	\draw [->] (visual.east) to[in=180,out=90] (gnu.west);
	\draw [->] (visual.south)+(0,.75) to (para.west);
	\draw [->] (visual.south)+(0,-.75) to (bk.west);
	\draw [->] (visual.west) to[in=180,out=-90] (ncl.west);

	\draw [->,dashed,thin] (rgen.east)+(0,.2) to[out=0,in=135] (rect.north);
	\draw [->,dashed,thin] (rgen.east)+(0,-.2) to[out=0,in=135] (gpu.north);
	\draw [->,dashed,thin] (gmsh.east) to[out=0,in=-135] (unstr.south);

\end{tikzpicture}

\section*{Структурная схема NSolver}
\begin{tikzpicture}[every node/.style={shape=rectangle,rounded corners,draw=blue!50,fill=blue!20,thick,inner sep=5pt,
 					minimum size=1cm,node distance=1cm,text width=3cm,text badly centered}, >=latex',thick]
	\node (solver) {\Large\textbf{NSolver}};
	\node[shape=rectangle split,rectangle split parts=2,text width=6cm] (listeners) [above=of solver]
		{\textit{Starter, Stopper, Saver, Logger, ...}\nodepart{second}\textbf{Listeners}};
	\node (scheme)    [left=of solver]        {\textbf{Difference scheme}};
	\node (integral)  [right=of solver]       {\textbf{Collision integral}};
	\node (grid)      [below left=of solver]  {\textbf{Grid \\ Boundaries}};
	\node (gases)     [below right=of solver] {\textbf{Gas parameters}};
	\node (decomp)    [below=of solver]       {\textbf{Domain decomposition}};
	\draw [<->] (listeners) to (solver);
	\draw [<-]  (scheme)    to (solver);
	\draw [<-]  (integral)  to (solver);
	\draw [->]  (grid)      to (solver);
	\draw [<->] (decomp)    to (solver);
	\draw [->]  (gases)     to (solver);
 \end{tikzpicture}

\section*{Структурная схема NSolver (2 вариант)}
\begin{tikzpicture}[level 1 concept/.append style={sibling angle=60,level distance=4.5cm}]
 
	\path[mindmap,concept color=black,text=white,minimum size=2cm]
	node[concept] {\LARGE\textbf{NSolver}}
	[clockwise from=90]
	child[concept color=green!50!black] {
		node[concept] {\large{Listeners}}
		[clockwise from=180]
		child { node[concept] {Starter} }
		child { node[concept] {Stopper} }
		child { node[concept] {Saver} }
		child { node[concept] {Logger} }
	}
	child[concept color=red,minimum size=2.5cm] { node[concept] {\large\textbf{Collision integral}} }
	child[concept color=blue] { node[concept] {Gas parameters} }
	child[concept color=orange] { node[concept] {Domain decomposition} }
	child[concept color=blue] { node[concept] {Grid \\ Boundaries} }
	child[concept color=red,minimum size=2.5cm] { node[concept] {\large\textbf{Difference scheme}} };
\end{tikzpicture}
 
 \section*{Блок-схема NSolver}
 \begin{tikzpicture}[node distance=2cm,every path/.style={draw,>=latex',very thick},thick,
					block/.style={rectangle,text width=3cm,text=white,text badly centered,rounded corners,minimum height=4em}]
	\node[block,fill=blue] (init) {Считывание начальных данных};
	\node[block,fill=orange,below of=init] (decomp) {Разбиение расчетной сетки на домены};
	\node[block,fill=green!50!black,below of=decomp] (listeners) {Листенеры};
	\node[below of=listeners] (dummy) {};
	\node[block,fill=green!50!black,node distance=5cm,above left of=listeners] (save) {Сохранение результатов};
	\node[block,fill=green!50!black,node distance=5cm,above right of=listeners] (stop) {Остановка расчёта};

	\node[block,fill=red,left of=dummy, node distance=3cm] (ldiff) {Расчет уравнения переноса для \(\frac\tau2\)};
	\node[block,fill=red,right of=dummy, node distance=3cm] (rdiff) {Расчет уравнения переноса для \(\frac\tau2\)};
	\node[block,fill=red,below of=dummy] (integr) {Расчет интеграла столкновений \\ для \(\tau\)};

	\draw[->] (init) to (decomp);
	\draw[->] (decomp) to (listeners);
	\draw[->] (listeners.east) to [bend left=45] (rdiff.north);
	\draw[<-] (listeners.west) to [bend right=45] (ldiff.north);
	\draw[<-] (integr.east) to [bend right=45] (rdiff.south);
	\draw[->] (integr.west) to [bend left=45] (ldiff.south);
	\draw[->] (listeners) to [bend right=30] node[right,text width=1.2cm,text centered] {если \\ нужно} (stop.south);
	\draw[->] (listeners) to [bend left=30] node[left,text width=1.2cm,text centered] {если \\ нужно} (save.south) ;
\end{tikzpicture}


Перевод на русский считаю лучше не делать))

Солвер хранит всю информацию в массиве ячеек (\textit{cells}) и границ (\textit{boundaries}).
Набор ячеек считывается из сетки (\textit{grid}). Набор границ --- из геометрии (\textit{geometry}).
Все ячейки знают про своих соседей, в т.ч. границ.
Модуль \textit{Domain decomposition} нумерует все ячейки MPI\_rank-ом. Параметры газа (\textit{gas parameters}) --- это в т.ч. информация о смеси.
\textit{Listeners} --- широкий класс объектов, которые вызываются с каждой итерацией.
\textit{Starter} на нулевой итерации задаёт начальные условия.
\textit{Stopper} останавливает итерационный процесс.
\textit{Saver} периодически сохраняет необходимые данные.
\textit{Logger} соотвественно регистрирует совершенные действия в журнале.
Стрелки указывают на информационный поток.

Жду жесткой критики и предложений))


\end{document}
