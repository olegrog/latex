\documentclass[english,russian,a4paper,12pt]{article}
\usepackage[utf8x]{inputenc}
\usepackage[T2A]{fontenc}
\usepackage{babel}
\usepackage{csquotes}
%\usepackage{type1ec} 
%\usepackage{literat} %mathptmx
%\IfFileExists{literat.sty}{\usepackage{literat}}{}

\usepackage{fullpage}
\usepackage{indentfirst}
\usepackage[font=small,labelfont=bf,labelsep=period]{caption}

\usepackage{amssymb, amsmath}
\usepackage{wrapfig}

% \usepackage[]{iwona}
% \usepackage{mathpazo}
% \usepackage{comicsans}
% \usepackage{arev}
% \usepackage{mathptmx}
% \usepackage{pxfonts}
% \usepackage{fouriernc}

\usepackage[
	pdfauthor={Oleg Rogozin},
	pdftitle={On Sport Injuries and Food},
	colorlinks, pdftex, unicode]{hyperref}

\title{О спортивных травмах и питании}
\author{Рогозин Олег}

\begin{document}
\maketitle
\tableofcontents

\section{Первая помощь при травме}
Зачастую можно наблюдать, как не только начинающие спортсмены, но и заядлые любители при получении травмы в силу незнания
основ физиологии совершают множество действий, которые наоборот приводят к ухудшению состояния и даже более серьёзным последствиям.
Известно, что от компетентности и своевременности оказания первой помощи существенно зависит как длительность лечения,
так и его сложность. Поэтому так важно спортсмену любого уровня быть знакомым с основными правилами,
рекомендуемыми на сегодняшний день медицинским сообществом.
В англоязычных странах известен удобный для запоминания акроним \href{http://en.wikipedia.org/wiki/RICE_(medicine)}{\textbf{RICE}}.
\begin{itemize}
	\item \textbf{Rest}. Первым делом необходимо \textbf{обеспечить покой}, дабы избежать дальнейшего повреждения и
	усиления внутреннего кровоизлияния. Кроме крови, это может быть и суставная жидкость, например.
	\item \textbf{Ice}. \textbf{Местное охлаждение} является \textbf{лучшим средством} против отёка мягких тканей.
	Большое значение имеет его \textbf{регулярность}, особенно в первые часы после получения травмы (вплоть до каждых 20 минут).
	Прикладывать лёд желательно прикладывать до тех пор, пока есть локальное повышение температуры.
	\item \textbf{Compression}. Наложение \textbf{давящей повязки} снижает кровоток в области повреждения,
	чем значительно замедляет развитие воспаления. Лучше всего для этого подходит эластичный бинт,
	поскольку позволяет более равномерно распределить сжимающую силу, исключая передавливания.
	\item \textbf{Elevation}. Наконец, рекомендуется расположить повреждённое место выше уровня сердца.
	Это особенно важно при травмах нижних конечностей, поскольку возникающее гидростатическое давление
	также способствует увеличению отёчности.
\end{itemize}
В течение первых 2-3 дней эта формула является \textbf{универсальной} при любых повреждениях в качестве \textbf{самолечения}.
Цель её "--- максимально быстрое заживление повреждённых тканей при минимальном воспалительном отклике,
Другими словами, следование этим простым советам позволяет 
\begin{itemize}
	\item сократить период полного восстановления, определяемый зачастую размером образовавшейся опухоли;
	\item предупредить развитие болевого синдрома (в противном случае часто на следующий день состояние ухудшается).
\end{itemize}

Следует понимать, что \textbf{боль и воспаление} "--- это защитная реакция организма, направленная
на сообщение особи о том, что данную часть тела необходимо поберечь во избежании \textbf{серьёзных последствий}.
Поэтому надо прислушаться к умному организму и одновременно уменьшить негативные последствия его реакции.

Если травма носит серьёзный характер, то не стоит злоупотреблять услугами \textbf{травматологов} (в т.\,ч. платными консультациями),
поскольку недолеченные повреждения часто переходят в хроническую стадию или, что хуже, к другим тяжёлым заболеваниям.
В этом случае стоимость лечения может вырасти в десятки и даже сотни раз.

Существует также <<продвинутая>> формула \textbf{HI-RICE}, рекомендуемая при особо остром и продолжительном болевом синдроме.
\begin{itemize}
	\item \textbf{Hydration}. При сильной и длительной боли,
	как и во всех стрессовых ситуациях, организму необходимо много жидкости.
	Обильное питьё позволяет \textbf{восстановить водный баланс}.
	\item \textbf{Ibuprofen}. Среди медикаментозных средств на сегодняшний день при травмах обычно используются
	так называемые \textbf{нестероидные противоспалительные препараты} (\href{http://ru.wikipedia.org/wiki/NSAID}{НПВП}).
	На западе самым распространённым из них является \textbf{ибупрофен}, чем объясняется его включение в аббревиатуру.
	Возможно как местное использование (втирание геля или мази), так и ввиде таблеток.
\end{itemize}

\subsection{Нестероидные противоспалительные препараты}

Так называют большую группу лекарственных средств, обладающих следующими эффектами: 
\begin{itemize}
	\item обезболивающим,
	\item жаропонижающим,
	\item противовоспалительным.
\end{itemize}
Использование в названии термина «нестероидные» подчеркивает их отличие от гормональных препаратов,
которые кроме сильного воздействия имеют значительно больше нежелательных последствий.

История противовоспалительных средств восходит ещё к 1829 году, когда была выделена салициловая кислота из коры ивы.
До сих пор мы пользуемся её производной в виде \textbf{аспирина} (ацетилсалициловая кислота).
Классическими препаратами сегодня считаются \textbf{ибупрофен} (из пропионовой кислоты) и \textbf{диклофенак} (из уксусной кислоты),
впервые синтезированные в 1962 и 1966 годах соотвественно.
Вполне очевидно, что развитие НПВП идёт по пути снижения токсичности и побочных эффектов при одновременном росте эффективности.

%\section*{Схема статьи}
% \begin{tikzpicture}[block/.style={shape=rectangle,rounded corners,draw=blue!50,fill=blue!20,thick,inner sep=5pt,
% 						minimum size=1cm,node distance=2cm,text badly centered},
% 					miniblock/.style={shape=rectangle,draw=blue!50,fill=blue!30,thick,inner sep=5pt,
% 					minimum size=1cm,node distance=5mm,text badly centered},>=latex',thick]
% 	\node[block,minimum width=4cm,minimum height=2.9cm] (in) at (2,1.55)			{};
% 	\node[block,minimum width=4cm,minimum height=2.9cm] (mesh) at (2,-1.55)			{};
% 	\node[block,minimum width=5.5cm,minimum height=6cm] (solver) at (7.5,0)				{};
% 	\node[block,minimum width=3.5cm,minimum height=6cm] (out) at (12.75,0)	{};
% 	\draw [<-] (in.east)+(.75cm-1pt,0) to (in.east);
% 	\draw [<-] (mesh.east)+(.75cm-1pt,0) to (mesh.east);
% 	\draw [->] (solver) to (out);
% 
% 	\node[below] at (in.north) {\textbf{Входные данные}};
% 	\node[miniblock,text width=3.5cm,inner sep=0pt] at (2,2) {Конфигурационный XML файл};
% 	\node[miniblock,text width=3.5cm,inner sep=0pt] at (2,.8) {Интерактивная оболочка};
% 
% 	\node[below] at (mesh.north) {\textbf{Расчетная сетка}};
% 	\node[miniblock,text width=3.5cm,inner sep=0pt] (rgen) at (2,-1.1) {Генератор сеток прямоугольных};
% 	\node[miniblock,text width=3.5cm,inner sep=0pt] (gmsh) at (2,-2.3) {GSMH};
% 
% 	\node[below] at (solver.north) {\textbf{Солвер}};
% 	\node[miniblock,rotate=90,text width=4cm] (int) at (9.4,0) {Интеграл столкновений};
% 	\node[miniblock,text width=2.5cm] (rect)  at (6.5,1.5) {RectSolv};
% 	\node[miniblock,text width=2.5cm] (gpu)   at (6.5,0) {GPUSolv};
% 	\node[miniblock,text width=2.5cm] (unstr) at (6.5,-1.5) {UnstructSolv};
% 	\draw [<->] (int.north)+(0,1.5) to (rect.east);
% 	\draw [<->] (int.north) to (gpu.east);
% 	\draw [<->] (int.north)+(0,-1.5) to (unstr.east);
% 
% 	\node[below,rotate=90] (visual) at (out.west) {\textit{Визуализация}};
% 	\node[above,rotate=90] at (out.east) {\textbf{Выходные данные}};
% 	\node[miniblock,text width=1.5cm] (gnu)  at (13,2.25) 	{Gnuplot};
% 	\node[miniblock,text width=1.5cm] (para) at (13,0.75) 	{Paraview};
% 	\node[miniblock,text width=1.5cm] (bk)   at (13,-0.75) {Bkviewer};
% 	\node[miniblock,text width=1.5cm] (ncl)  at (13,-2.25) {NCL};
% 	\draw [->] (visual.east) to[in=180,out=90] (gnu.west);
% 	\draw [->] (visual.south)+(0,.75) to (para.west);
% 	\draw [->] (visual.south)+(0,-.75) to (bk.west);
% 	\draw [->] (visual.west) to[in=180,out=-90] (ncl.west);
% 
% 	\draw [->,dashed,thin] (rgen.east)+(0,.2) to[out=0,in=135] (rect.north);
% 	\draw [->,dashed,thin] (rgen.east)+(0,-.2) to[out=0,in=135] (gpu.north);
% 	\draw [->,dashed,thin] (gmsh.east) to[out=0,in=-135] (unstr.south);
% 
% \end{tikzpicture}

% \section*{Структурная схема NSolver}
% \begin{tikzpicture}[every node/.style={shape=rectangle,rounded corners,draw=blue!50,fill=blue!20,thick,inner sep=5pt,
% 					minimum size=1cm,node distance=1cm,text width=3cm,text badly centered}, >=latex',thick]
% 	\node (solver) {\Large\textbf{NSolver}};
% 	\node[shape=rectangle split,rectangle split parts=2,text width=6cm] (listeners) [above=of solver]
% 		{\textit{Starter, Stopper, Saver, Logger, ...}\nodepart{second}\textbf{Listeners}};
% 	\node (scheme)    [left=of solver]        {\textbf{Difference scheme}};
% 	\node (integral)  [right=of solver]       {\textbf{Collision integral}};
% 	\node (grid)      [below left=of solver]  {\textbf{Grid \\ Boundaries}};
% 	\node (gases)     [below right=of solver] {\textbf{Gas parameters}};
% 	\node (decomp)    [below=of solver]       {\textbf{Domain decomposition}};
% 	\draw [<->] (listeners) to (solver);
% 	\draw [<-]  (scheme)    to (solver);
% 	\draw [<-]  (integral)  to (solver);
% 	\draw [->]  (grid)      to (solver);
% 	\draw [<->] (decomp)    to (solver);
% 	\draw [->]  (gases)     to (solver);
% \end{tikzpicture}

% \section*{Структурная схема NSolver (2 вариант)}
% \begin{tikzpicture}[level 1 concept/.append style={sibling angle=60,level distance=4.5cm}]
% 
% 	\path[mindmap,concept color=black,text=white,minimum size=2cm]
% 	node[concept] {\LARGE\textbf{NSolver}}
% 	[clockwise from=90]
% 	child[concept color=green!50!black] {
% 		node[concept] {\large{Listeners}}
% 		[clockwise from=180]
% 		child { node[concept] {Starter} }
% 		child { node[concept] {Stopper} }
% 		child { node[concept] {Saver} }
% 		child { node[concept] {Logger} }
% 	}
% 	child[concept color=red,minimum size=2.5cm] { node[concept] {\large\textbf{Collision integral}} }
% 	child[concept color=blue] { node[concept] {Gas parameters} }
% 	child[concept color=orange] { node[concept] {Domain decomposition} }
% 	child[concept color=blue] { node[concept] {Grid \\ Boundaries} }
% 	child[concept color=red,minimum size=2.5cm] { node[concept] {\large\textbf{Difference scheme}} };
% \end{tikzpicture}
% 
% \section*{Блок-схема NSolver}
% \begin{tikzpicture}[node distance=2cm,every path/.style={draw,>=latex',very thick},thick,
% 					block/.style={rectangle,text width=3cm,text=white,text badly centered,rounded corners,minimum height=4em}]
% 	\node[block,fill=blue] (init) {Считывание начальных данных};
% 	\node[block,fill=orange,below of=init] (decomp) {Разбиение расчетной сетки на домены};
% 	\node[block,fill=green!50!black,below of=decomp] (listeners) {Листенеры};
% 	\node[below of=listeners] (dummy) {};
% 	\node[block,fill=green!50!black,node distance=5cm,above left of=listeners] (save) {Сохранение результатов};
% 	\node[block,fill=green!50!black,node distance=5cm,above right of=listeners] (stop) {Остановка расчёта};
% 
% 	\node[block,fill=red,left of=dummy, node distance=3cm] (ldiff) {Расчет уравнения переноса для \(\frac\tau2\)};
% 	\node[block,fill=red,right of=dummy, node distance=3cm] (rdiff) {Расчет уравнения переноса для \(\frac\tau2\)};
% 	\node[block,fill=red,below of=dummy] (integr) {Расчет интеграла столкновений \\ для \(\tau\)};
% 
% 	\draw[->] (init) to (decomp);
% 	\draw[->] (decomp) to (listeners);
% 	\draw[->] (listeners.east) to [bend left=45] (rdiff.north);
% 	\draw[<-] (listeners.west) to [bend right=45] (ldiff.north);
% 	\draw[<-] (integr.east) to [bend right=45] (rdiff.south);
% 	\draw[->] (integr.west) to [bend left=45] (ldiff.south);
% 	\draw[->] (listeners) to [bend right=30] node[right,text width=1.2cm,text centered] {если \\ нужно} (stop.south);
% 	\draw[->] (listeners) to [bend left=30] node[left,text width=1.2cm,text centered] {если \\ нужно} (save.south) ;
% \end{tikzpicture}
% 

% Перевод на русский считаю лучше не делать))
% 
% Солвер хранит всю информацию в массиве ячеек (\textit{cells}) и границ (\textit{boundaries}).
% Набор ячеек считывается из сетки (\textit{grid}). Набор границ --- из геометрии (\textit{geometry}).
% Все ячейки знают про своих соседей, в т.ч. границ.
% Модуль \textit{Domain decomposition} нумерует все ячейки MPI\_rank-ом. Параметры газа (\textit{gas parameters}) --- это в т.ч. информация о смеси.
% \textit{Listeners} --- широкий класс объектов, которые вызываются с каждой итерацией.
% \textit{Starter} на нулевой итерации задаёт начальные условия.
% \textit{Stopper} останавливает итерационный процесс.
% \textit{Saver} периодически сохраняет необходимые данные.
% \textit{Logger} соотвественно регистрирует совершенные действия в журнале.
% Стрелки указывают на информационный поток.
% 
% Жду жесткой критики и предложений))


\end{document}
