%&pdflatex
\documentclass[10pt,a4paper,final]{moderncv}
\usepackage[T2A]{fontenc}
\usepackage[utf8]{inputenc}
\usepackage[english,russian]{babel}

%% ModernCV themes
\moderncvstyle{classic}                    % style options are 'casual' (default), 'classic', 'oldstyle' and 'banking'
\moderncvcolor{orange}
\nopagenumbers{}

%% Adjust the page margins
\usepackage[scale=0.75]{geometry}

%% Personal data
\firstname{Олег}
\familyname{Рогозин}
\title{}
%\address{ул. Расплетина, д. 21, кв. 176}{123060 Москва, Россия}
\mobile{+7~(916)~660~09~52}
\email{oleg.rogozin@phystech.edu}
\extrainfo{17 ноября 1987}
\photo[64pt][0.4pt]{../cv/photo}

%\usepackage{mathpazo}
\usepackage{csquotes}
\usepackage{graphicx}

\author{Олег Рогозин}
\title{Резюме}
\date{\today}

\usepackage[
    backend=biber,
    style=gost-numeric,
    autolang=other,
    maxbibnames=99, minbibnames=99,
    natbib=true,
    sorting=ydnt,
    url=false,
    eprint=false,
    pagetracker,
    defernumbers=true,
    firstinits,
]{biblatex}
\bibliography{main, other}

\DeclareSourcemap{
  \maps[datatype=bibtex, overwrite]{
    \map{
      \perdatasource{main.bib}
      \step[fieldset=keywords, fieldvalue=primary, append]
    }
    \map{
      \perdatasource{other.bib}
      \step[fieldset=keywords, fieldvalue=secondary, append]
    }
  }
}

\newcommand{\Skoltech}{Сколковский институт науки и технологий}
\newcommand{\Mipt}{Московский физико-технический институт}
\newcommand{\Kurchatov}{Hациональный исследовательский центр <<Курчатовский Институт>>}
\newcommand{\Fic}{Федеральный исследовательский центр <<Информатика и управление>> Российской академии наук}
\newcommand{\Dolgopa}{Долгопрудный}
\newcommand{\Moscow}{Москва}

\begin{document}
\makecvtitle

\section{Опыт}

\subsection{Исследовательский}
\cventry{2017--н.в.}{Научное исследование}{\Skoltech}{\Moscow}{}{
\begin{itemize}
    \item Многомасштабное моделирование и симуляция физических процессов селективного лазерного плавления
    \item Макроскопическое моделирование технолических процессов при аддитивном производстве керамических изделий
%    \item Методы фазового поля для задач кристаллизации многокомпонентных сплавов
\end{itemize}
}
\cventry{2017--н.в.}{Научное исследование}{\Fic}{\Moscow}{}{
\begin{itemize}
    \item Многомасштабные численные методы для задач динамики разреженного газа на основе сращивания кинетических схем
        для решётчатых уравнений Больцмана и методов дискретных скоростей
    \item Высокоточное моделирование сверхзвуковых разреженных течений на основе уравнения Больцмана
\end{itemize}
}
\cventry{2009--2015}{Диссертационное исследование}{\Mipt}{\Dolgopa}{}{
\begin{itemize}
    \item Разработка проблемно-моделирующей среды для высокопроизводительных вычислений
    \item Развитие и реализация численных методов для молекулярной газовой динамики
    \item Численный анализ медленных неизотермических течений в рамках кинетического и гидродинамического описания
\end{itemize}
}

%\subsection{Индустриальный}
%\cventry{2013--2017}{Инженер-программист}{ООО <<Атомстрой>>}{\Moscow}{}{
%\begin{itemize}
%    \item Поддержка и обслуживание программного обеспечения
%\end{itemize}
%}
\cventry{2013--2015}{Старший программист}{ООО <<Бином>>}{\Moscow}{}{
\begin{itemize}
    \item Программная разработка на стороне сервера на основе NodeJS и Amazon Web Services
    \item Проектирование и реализация файловой системы специального назначения на основе Amazon S3
\end{itemize}
}
\cventry{2011--2012}{Инженер второй категории}{\Kurchatov}{\Moscow}{}{
\begin{itemize}
    \item Поиск и устранение неисправностей в сетях InfiniBand на суперкомпьютере
    \item Разработка диагностических инструментов для сети InfiniBand
    \item Настройка программного обеспечения для суперкомпьютера
    \item Стерео-визуализация научных презентаций
\end{itemize}
}
\cventry{2006--2011}{Лаборант шестого разряда}{\Kurchatov}{\Moscow}{}{
\begin{itemize}
    \item Администрирование Linux-станций
\end{itemize}
}

\subsection{Преподавательский}
\cventry{2018--н.в.}{Лектор}{\Skoltech}{\Moscow}{}{
\begin{itemize}
    \item Лекции по теме <<Численные методы в науке и технике>> для магистерских студентов
\end{itemize}
}
\cventry{2012--2012}{Лектор}{\Mipt}{\Dolgopa}{}{
\begin{itemize}
    \item Лекции по теме <<Архитектура компьютеров>> для студентов второго курса
\end{itemize}
}
\cventry{2005--2006}{Преподаватель}{\Mipt}{\Dolgopa}{}{
\begin{itemize}
    \item Проверка работ учеников Заочной физико-технической школы
\end{itemize}
}

\section{Образование}
\cventry{2011--2014}{Кандидат физико-математических наук}{\Mipt}{\Dolgopa}
{}{
<<Численный и асимптотический анализ некоторых классических задач молекулярной газодинамики>>
\begin{itemize}
    \item Научный руководитель: д.ф.-м.н., проф. Феликс Григорьевич Черемисин
    \item Специальность: 01.02.05 <<Механика жидкости, газа и плазмы>>
    \item Защита \httplink[состоялась]{http://www.frccsc.ru/diss-council/00207303/diss/list/rogozin_oa}
        в совете Д 002.073.03 Вычислительного центра РАН 24 мая 2018 г.
\end{itemize}
}
\cventry{2009--2011}{Магистр прикладной математики и физики}{\Mipt}{\Dolgopa}{\textit{4.88/5.00}}
{<<Решение классических задач динамики разреженного газа проекционным методом дискретных ординат>>}
\cventry{2005--2009}{Бакалавр прикладной математики и физики}{\Mipt}{\Dolgopa}{\textit{4.97/5.00}}
{<<Моделирование неравновесных газовых течений в каналах различной геометрии на основе уравнения Больцмана>>}

\section{Языки}
\cvitem{Русский}{Родной}
\cvitem{Английский}{Выше среднего}
\cvitem{Немецкий}{Базовый}
\cvitem{Французский}{Базовый}

\section{Компьютерные навыки}
\cvitem{Операционные системы}{Linux (администрирование), OS X, Windows}
\cvitem{Языки}{C++98/11/14/17/20, Python, Matlab/Octave, Fortran, Javascript, Shell, SQL, Assembler}
\cvitem{Программное обеспечение}{OpenFOAM, Amazon Web Services}
\cvitem{Параллельные технологии}{MPI, OpenMP, Hadoop, HPX}
\cvitem{Сети}{Ethernet, InfiniBand}

\nocite{*}
\printbibliography[title=Основные публикации, keyword=primary, resetnumbers=true]
\printbibliography[title=Другие публикации, keyword=secondary, resetnumbers=true]

\end{document}
