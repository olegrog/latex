%&pdflatex
\documentclass{article}
\usepackage[utf8]{inputenc}
\usepackage[T2A]{fontenc}
\usepackage[russian]{babel}

\usepackage{amssymb, amsmath, amsthm}
\usepackage{fullpage}
\usepackage{indentfirst}

%\overfullrule=2cm

\usepackage[
    pdfauthor={Rogozin Oleg},
    pdftitle={On the projection-interpolation method},
    colorlinks, pdftex, unicode
]{hyperref}

\newtheorem{remark}{Remark}

\title{Проекционно-интерполяционный метод}
\author{Олег Рогозин}

\DeclareMathOperator{\supp}{supp}

\newcommand{\dd}{\mathrm{d}}
\newcommand{\pder}[2][]{\frac{\partial#1}{\partial#2}}
\newcommand{\dzeta}{\boldsymbol{\dd\zeta}}
\newcommand{\bzeta}{\boldsymbol{\zeta}}
\newcommand{\bxi}{\boldsymbol{\xi}}
\newcommand{\bomega}{\boldsymbol{\omega}}
\newcommand{\Nu}{\mathcal{N}}
\newcommand{\Set}[2]{\{\,{#1}:{#2}\,\}}
\newcommand{\eqdef}{\overset{\mathrm{def}}{=\joinrel=}}

\begin{document}

\maketitle
\tableofcontents

\section{Метод дискретных скоростей}

Пространственно-однородное уравнение Больцмана
\begin{equation}\label{eq:Boltzmann}
    \pder[f]{t} = \int_{\mathbb{R}^3\times\mathbb{S}^2}
        (f'f'_*-ff_*)B(\bzeta,\bzeta_*,\bomega)
        \dd\Omega(\boldsymbol\omega)\dzeta_*
\end{equation}
может быть записано в слабой форме
\begin{equation}\label{eq:weak_form}
    \pder{t}\int_{\mathbb{R}^3} \varphi_\gamma(\bzeta) f\dzeta =
        \int_{\mathbb{R}^3\times\mathbb{R}^3\times\mathbb{S}^2}
        \varphi_\gamma(\bzeta) (f'f'_*-ff_*)B(\bzeta,\bzeta_*,\bomega)
        \dd\Omega(\bomega)\dzeta\dzeta_* \quad (\gamma\in\mathbb{N}).
\end{equation}
Предполагая, что \(f\in L^2(\mathbb{R}^3)\), уравнения~\eqref{eq:Boltzmann} и~\eqref{eq:weak_form} эквивалентны
тогда и только тогда, когда \(\{\varphi_\gamma(\bzeta)\}_{\gamma=1}^\infty\)
"--- базис в гильбертовом пространстве \(L^2(\mathbb{R}^3)\).

\emph{Метод дискретных скоростей} является приближённым методом решения интегро-дифференциального уравнения
путём сведения его к системе дифференциальных уравнений.
Например, подставляя \(\varphi_\gamma = \delta(\bzeta - \bzeta_\gamma)\) в~\eqref{eq:weak_form},
получаем для \(\{f_\gamma\}_{\gamma=1}^n\) систему нелинейных уравнений
\begin{equation}\label{eq:discrete_velocities}
    \pder[f_\gamma]{t} =
        \int_{\mathbb{R}^3\times\mathbb{R}^3\times\mathbb{S}^2}
        \delta(\bzeta-\bzeta_\gamma)(f'f'_*-ff_*)B(\bzeta,\bzeta_*,\bomega)
        \dd\Omega(\bomega)\dzeta\dzeta_* \quad (\gamma=1,2,\dots,n).
\end{equation}
Функция распределения определяется по дискретному набору значений, т.\,е. \(f = f(\bzeta_1, \bzeta_2, \dots, \bzeta_n)\).

Далее будем использовать симметричную слабую форму пространственно-однородного уравнения Больцмана
\begin{equation}\label{eq:symm_weak_form}
    \pder{t}\int_{\mathbb{R}^3} \varphi_\gamma f\dzeta =
        \frac14\int_{\mathbb{R}^3\times\mathbb{R}^3\times\mathbb{S}^2}
        \left(\varphi_\gamma+\varphi_{\gamma*}-\varphi_\gamma'-\varphi_{\gamma*}'\right)
        (f'f'_*-ff_*)B(\bzeta,\bzeta_*,\bomega)
        \dd\Omega(\bomega)\dzeta\dzeta_* \quad (\gamma\in\mathbb{N}).
\end{equation}

Формально можно было бы разложить функцию распределения в ряд Фурье по системе дельта-функций Дир\'{а}ка
\begin{equation}\label{eq:Dirac_basis_L2}
    f = \sum_{\gamma=1}^\infty f_\gamma\delta(\bzeta - \bzeta_\gamma),
\end{equation}
однако \(\delta(\bzeta)\notin L^2(\mathbb{R}^3)\), поэтому дельта-функции не могут служить базисными функциями в \(L^2\).
В более широком пространстве, например, в пространстве обобщённых функций \(\mathcal{D}'(\mathbb{R}^3)\),
дельта-функция вместе со всеми своими производными могут служить базисом для функционалов с точечным носителем.
Формально говоря, если \(f\in\mathcal{D}'(\mathbb{R}^3)\) и \(\supp{f}=\{0\}\),
то для мультииндекса \(j\in\mathbb{Z}_{\geqslant0}^3\) найдутся такие \({c_j}\), что
\begin{equation}\label{eq:Dirac_basis_D'}
    f(\bzeta) = \sum_{j\in\mathbb{Z}^3} c_j \delta^{(j)}(\bzeta), \quad
    \int_{\mathbb{R}^3} \delta^{(j)}(\bzeta) f(\bzeta) \dzeta \eqdef
        \frac{\partial^{j_1+j_2+j_3}f(0)}{\partial x^{j_1}\partial x^{j_2}\partial x^{j_3}}.
\end{equation}

\section{Решение в пространстве кусочно-постоянных функций}

Рассмотрим конечномерное пространство \(L^2_n\) кусочно-постоянных функций внутри \(L^2\).
Определим ортогональный (ненормированный) базис в \(L^2_n\) через финитные функции вида
\begin{equation}\label{eq:delta_definition}
    \delta^{(n)}_\gamma(\bzeta) \eqdef \frac{\mathbf{1}_{\Omega^{(n)}_\gamma}(\bzeta)}{w^{(n)}_\gamma} =
    \begin{cases}
        (w^{(n)}_\gamma)^{-1}, & \bzeta\in\Omega^{(n)}_\gamma, \\
        0, & \bzeta\notin\Omega^{(n)}_\gamma. \\
    \end{cases}
\end{equation}
где \(\{\Omega^{(n)}_\gamma\}\) "--- некоторое разбиение компакта \(\Omega^{(n)}\subset\mathbb{R}^3\),
а мера \(\mu(\Omega^{(n)}_\gamma) = w^{(n)}_\gamma\). Везде в этом параграфе будем подразумевать \(\gamma=1,2,\dots,n\).
Таким образом,
\begin{gather}
    \int_{\mathbb{R}^3} \delta^{(n)}_\gamma(\bzeta) \dzeta = 1, \label{eq:L2n_norm}\\
    \int_{\mathbb{R}^3} \delta^{(n)}_\alpha(\bzeta) \delta^{(n)}_\beta(\bzeta) \dzeta =
    \frac{\delta_{\alpha\beta}}{w_\alpha} = \frac{\delta_{\alpha\beta}}{w_\beta}. \label{eq:L2n_orth}
\end{gather}
Потребуем также
\begin{gather}
    \lim_{n\to\infty} \Omega^{(n)} = \mathbb{R}^3, \label{eq:Omega_limit}\\
    \lim_{n\to\infty} \sup\Set{|\bzeta-\bxi|}{\bzeta,\bxi\in\Omega^{(n)}_\gamma} = 0 \label{eq:diameter_limit}
\end{gather}
тогда
\begin{gather}
    \lim_{n\to\infty} \delta^{(n)}_\gamma(\bzeta) = \delta(\bzeta-\bzeta_\gamma), \label{eq:delta_limit}\\
    \lim_{n\to\infty} L^2_n = L^2. \label{eq:subspace_limit}
\end{gather}

Произвольная функция \(\phi(\bzeta) \in L^2\) может быть \emph{спроецирована} на пространство \(L^2_n\),
так что координаты \(\phi\) по базису \(\{\delta^{(n)}_\gamma(\bzeta)\}\) равны
\begin{equation}\label{eq:L2n_coords}
    \phi^{(n)}_\gamma = \int_{\mathbb{R}^3} \phi(\bzeta) \delta^{(n)}_\gamma(\bzeta) \dzeta.
\end{equation}
Проекция \(\phi^{(n)}(\bzeta) \in L^2_n\) определяется однозначно как
\begin{equation}\label{eq:L2n_expansion}
    \phi^{(n)}(\bzeta) = \sum_{\gamma=1}^n \phi^{(n)}_\gamma w^{(n)}_\gamma \delta^{(n)}_\gamma(\bzeta).
\end{equation}
В силу~\eqref{eq:L2n_norm}, интеграл от \(\phi^{(n)}(\bzeta)\) в точности равен своей конечной сумме Римана, т.\,е.
\begin{equation}\label{eq:Riemann_sum}
    \int_{\mathbb{R}^3} \phi^{(n)}(\bzeta) \dzeta = \sum_{\gamma=1}^n \phi^{(n)}_\gamma w^{(n)}_\gamma.
\end{equation}
В пределе
\begin{gather}
    \lim_{n\to\infty} \phi^{(n)}(\bzeta) = \phi(\bzeta), \label{function_limit}\\
    \lim_{n\to\infty} \sum_{\gamma=1}^n \phi^{(n)}_\gamma w^{(n)}_\gamma =
        \int_{\mathbb{R}^3} \phi(\bzeta) \dzeta. \label{eq:Riemann_sum_limit}
\end{gather}

Подставляя \(\varphi_\gamma = \delta^{(n)}_\gamma(\bzeta)\) в~\eqref{eq:symm_weak_form}, ищем решение в пространстве \(L^2_n\):
\begin{multline}\label{eq:L2n_Boltzmann}
    \pder[f^{(n)}_\gamma]{t} =
        \frac14\int_{\mathbb{R}^3\times\mathbb{R}^3\times\mathbb{S}^2}
        \left(\delta^{(n)}_\gamma(\bzeta)+\delta^{(n)}_\gamma(\bzeta_*)-\delta^{(n)}_\gamma(\bzeta')-\delta^{(n)}_\gamma(\bzeta'_*)\right) \\
        \times\left(f^{(n)}(\bzeta')f^{(n)}(\bzeta'_*)-f^{(n)}(\bzeta)f^{(n)}(\bzeta_*)\right) B(\bzeta,\bzeta_*,\bomega)
        \dd\Omega(\bomega)\dzeta\dzeta_*.
\end{multline}
Последовательность решений \(\{f^{(n)}(\bzeta)\}\) при \(n\to\infty\) стремится к точному решению~\eqref{eq:symm_weak_form} в \(L^2\).
Если аппроксимировать подинтегральное выражение некоторыми функциями из \(L^2_n(\mathbb{R}^6)\):
\begin{gather}
    w^{(n)}_\gamma \delta^{(n)}_\gamma(\bzeta') = \sum_{\alpha=1}^n \sum_{\beta=1}^n
        \Phi^{(n)}_{\alpha\beta\gamma}(\bomega) w^{(n)}_\alpha \delta^{(n)}_\alpha(\bzeta)
        w^{(n)}_\beta \delta^{(n)}_\beta(\bzeta_*), \label{eq:Phi_abc}\\
    w^{(n)}_\gamma \delta^{(n)}_\gamma(\bzeta'_*) = \sum_{\alpha=1}^n \sum_{\beta=1}^n
        \Psi^{(n)}_{\alpha\beta\gamma}(\bomega) w^{(n)}_\alpha \delta^{(n)}_\alpha(\bzeta)
        w^{(n)}_\beta \delta^{(n)}_\beta(\bzeta_*), \label{eq:Psi_abc}\\
    f^{(n)}(\bzeta') = \sum_{\alpha=1}^n \sum_{\beta=1}^n
        F^{(n)}_{\alpha\beta}(\bomega) w^{(n)}_\alpha \delta^{(n)}_\alpha(\bzeta)
        w^{(n)}_\beta \delta^{(n)}_\beta(\bzeta_*), \label{eq:F_ab}\\
    f^{(n)}(\bzeta'_*) = \sum_{\alpha=1}^n \sum_{\beta=1}^n
        G^{(n)}_{\alpha\beta}(\bomega) w^{(n)}_\alpha \delta^{(n)}_\alpha(\bzeta)
        w^{(n)}_\beta \delta^{(n)}_\beta(\bzeta_*), \label{eq:G_ab}
\end{gather}
то интеграл в~\eqref{eq:L2n_Boltzmann} по переменным \(\bzeta\) и \(\bzeta_*\)
можно записать через суммы вида~\eqref{eq:Riemann_sum}:
\begin{multline}\label{eq:L2n_Boltzmann_sum}
    \pder[f^{(n)}_\gamma]{t} =
        \frac14\int_{\mathbb{S}^2}\frac{\dd\Omega(\bomega)}{w^{(n)}_\gamma} \sum_{\alpha=1}^n \sum_{\beta=1}^n
        \left(\delta_{\alpha\gamma} + \delta_{\beta\gamma}
            - \Phi^{(n)}_{\alpha\beta\gamma}(\bomega) - \Psi^{(n)}_{\alpha\beta\gamma}(\bomega)\right) \\
        \times\left(F^{(n)}_{\alpha\beta}(\bomega)G^{(n)}_{\alpha\beta}(\bomega) - f^{(n)}_\alpha f^{(n)}_\beta\right)
        B^{(n)}_{\alpha\beta}(\bomega) w^{(n)}_\alpha w^{(n)}_\beta.
\end{multline}
где \(\delta_{\alpha\beta}\) "--- символ Кр\'{о}некера. Уравнение~\eqref{eq:L2n_Boltzmann_sum}
удобно переписать для \(\hat{f}^{(n)}_\gamma \eqdef f^{(n)}_\gamma w^{(n)}_\gamma\):
\begin{multline}\label{eq:L2n_Boltzmann_sum_hat}
    \pder[\hat{f}^{(n)}_\gamma]{t} =
        \frac14\int_{\mathbb{S}^2}\dd\Omega(\bomega) \sum_{\alpha=1}^n \sum_{\beta=1}^n
        \left(\delta_{\alpha\gamma} + \delta_{\beta\gamma}
            - \Phi^{(n)}_{\alpha\beta\gamma}(\bomega) - \Psi^{(n)}_{\alpha\beta\gamma}(\bomega)\right) \\
        \times\left(w^{(n)}_\alpha w^{(n)}_\beta F^{(n)}_{\alpha\beta}(\bomega)G^{(n)}_{\alpha\beta}(\bomega)
            - \hat{f}^{(n)}_\alpha \hat{f}^{(n)}_\beta\right)
        B^{(n)}_{\alpha\beta}(\bomega).
\end{multline}
Для того чтобы решение уравнения~\eqref{eq:L2n_Boltzmann_sum_hat} сходилось к решению~\eqref{eq:symm_weak_form},
необходимо для всех \(\alpha,\beta,\gamma\) потребовать
\begin{gather}
    \lim_{n\to\infty} \Phi^{(n)}_{\alpha\beta\gamma}(\bomega) =
        \delta(\bzeta'(\bzeta_\alpha,\bzeta_\beta,\bomega)-\bzeta_\gamma), \quad
    \lim_{n\to\infty} \Psi^{(n)}_{\alpha\beta\gamma}(\bomega) =
        \delta(\bzeta'_*(\bzeta_\alpha,\bzeta_\beta,\bomega)-\bzeta_\gamma), \label{eq:PhiPsi_limit}\\
    \lim_{n\to\infty} F^{(n)}_{\alpha\beta}(\bomega) =
        f(\bzeta'(\bzeta_\alpha,\bzeta_\beta,\bomega)), \quad
    \lim_{n\to\infty} G^{(n)}_{\alpha\beta}(\bomega) =
        f(\bzeta'_*(\bzeta_\alpha,\bzeta_\beta,\bomega)), \label{eq:FG_limit}\\
    \lim_{n\to\infty} B^{(n)}_{\alpha\beta}(\bomega) =
        B(\bzeta_\alpha,\bzeta_\beta,\bomega). \label{eq:B_limit}
\end{gather}
Имеющийся произвол в определении операторов \(\Phi^{(n)}_{\alpha\beta\gamma}(\bomega)\) и \(\Psi^{(n)}_{\alpha\beta\gamma}(\bomega)\)
может быть использован для построения консервативной численной схемы.
Для этого достаточно использовать линейные формы вида
\begin{equation}\label{conservative_scheme}
    \Phi^{(n)}_{\alpha\beta\gamma} = \sum_{i=1}^N r_{\lambda^{(n)}_i} \delta_{\lambda^{(n)}_i\gamma}, \quad
    \Psi^{(n)}_{\alpha\beta\gamma} = \sum_{i=1}^N r_{\mu^{(n)}_i} \delta_{\mu^{(n)}_i\gamma},
\end{equation}
где индексы \(\lambda^{(n)}_i=\lambda^{(n)}_i(\bzeta_\alpha,\bzeta_\beta,\bomega)\in\mathbb{N}\),
\(\mu^{(n)}_i=\mu^{(n)}_i(\bzeta_\alpha,\bzeta_\beta,\bomega)\in\mathbb{N}\),
а коэффициенты \(r_{\lambda^{(n)}_i}\) и \(r_{\mu^{(n)}_i}\) выбираются так, чтобы сумма в~\eqref{eq:L2n_Boltzmann_sum}
не изменяла столкновительные инварианты для любых \(\gamma\).
Такая же форма для логарифмов \(F^{(n)}_{\alpha\beta}\) и \(G^{(n)}_{\alpha\beta}\),
\begin{equation}\label{htheorem_scheme}
    \ln F^{(n)}_{\alpha\beta} = \sum_{i=1}^N r_{\lambda^{(n)}_i} \ln f^{(n)}_{\lambda^{(n)}_i}, \quad
    \ln G^{(n)}_{\alpha\beta} = \sum_{i=1}^N r_{\mu^{(n)}_i} \ln f^{(n)}_{\mu^{(n)}_i},
\end{equation}
обеспечивает справедливость дискретного аналога H-теоремы.
Наконец, в качестве \(B^{(n)}_{\alpha\beta}\) может быть выбрана проекция на \(L^2_n(\mathbb{R}^6)\), т.\,е.
\begin{equation}\label{B_projection}
    B^{(n)}_{\alpha\beta}(\bomega) = \int_{\mathbb{R}^3\times\mathbb{R}^3} B(\bzeta,\bzeta_*,\bomega)
        \delta^{(n)}_\alpha(\bzeta) \delta^{(n)}_\beta(\bzeta_*) \dzeta\dzeta_*.
\end{equation}
\begin{remark}
    \(B^{(n)}_{\alpha\beta}\) "--- линейный оператор,
    \(\Delta^{(n)}_{\alpha\beta\gamma}\), \(F^{(n)}_{\alpha\beta}\), \(G^{(n)}_{\alpha\beta}\) "--- нелинейные.
    В линейной алгебре и функциональном анализе проекцией принято называть линейный идемпотентный (\(P^2=P\)) оператор,
    поэтому из всех идемпотентных операторов
    \(\Delta^{(n)}_{\alpha\beta\gamma}\), \(F^{(n)}_{\alpha\beta}\), \(G^{(n)}_{\alpha\beta}\), \(B^{(n)}_{\alpha\beta}\),
    отображающих \(L^2 \mapsto L^2_n\), только \(B^{(n)}_{\alpha\beta}\) можно называть проекционным.
\end{remark}

\section{Решение в пространстве дельта-функций}

Выражения, полученные в предыдущем разделе, можно сделать более наглядными (теряя при этом в строгости),
если базисные функции \(\delta^{(n)}_\gamma(\bzeta)\) заменить на дельта-функции \(\delta(\bzeta-\bzeta_\gamma)\).
Другими словами, рассмотрим аппроксимацию
\begin{gather}
    \phi(\bzeta) = \sum_\gamma \hat{\phi}_\gamma \delta(\bzeta-\bzeta_\gamma),
    \quad \hat{\phi}_\gamma = \phi_\gamma w_\gamma, \label{eq:deltas_expansion} \\
    \phi_\gamma = \int_{\mathbb{R}^3} \phi(\bzeta) \delta(\bzeta-\bzeta_\gamma) \dzeta = \phi(\bzeta_\gamma), \label{eq:deltas_coords} \\
    \int_{\mathbb{R}^3} \phi(\bzeta)\dzeta = \sum_\gamma \hat{\phi}_\gamma. \label{eq:deltas_integral}
\end{gather}
Уравнение~\eqref{eq:L2n_Boltzmann_sum_hat} принимает такую же дискретную форму
\begin{multline}\label{eq:deltas_Boltzmann_sum_hat}
    \pder[\hat{f}_\gamma]{t} =
        \frac14\int_{\mathbb{S}^2}\dd\Omega(\bomega) \sum_\alpha \sum_\beta
        \left(\delta_{\alpha\gamma} + \delta_{\beta\gamma}
            - \Phi_{\alpha\beta\gamma}(\bomega) - \Psi_{\alpha\beta\gamma}(\bomega)\right) \\
        \times\left(w_\alpha w_\beta F_{\alpha\beta}(\bomega)G_{\alpha\beta}(\bomega)
            - \hat{f}_\alpha \hat{f}_\beta\right)
        B_{\alpha\beta}(\bomega).
\end{multline}
В отличие от пространства \(L^2_n\) разложение~\eqref{eq:deltas_expansion} содержит дискретный набор значений \(\phi(\bzeta_\gamma)\),
а не функционалы от \(\phi(\bzeta)\). Таким образом, по сравнению с~\eqref{B_projection} вычисление \(B_{\alpha\beta}\)
не требует интегрирования:
\begin{equation}\label{B_exact}
    B_{\alpha\beta}(\bomega) = B(\bzeta_\alpha,\bzeta_\beta,\bomega).
\end{equation}
Существенно упрощается частный вид предельных (с точки зрения последовательности \(\{L^2_n\}\))
аппроксимаций~\eqref{eq:Phi_abc}--\eqref{eq:G_ab} при~\eqref{conservative_scheme} и~\eqref{htheorem_scheme}:
\begin{gather}
    \delta(\bzeta'-\bzeta_\gamma) = \sum_{i=1}^N r_{\lambda_i} \delta(\bzeta_{\lambda_i}-\bzeta_\gamma), \quad
    \delta(\bzeta'_*-\bzeta_\gamma) = \sum_{i=1}^N r_{\mu_i}\delta(\bzeta_{\mu_i}-\bzeta_\gamma), \label{deltas_conservative}\\
    \ln f(\bzeta') = \sum_{i=1}^N r_{\lambda_i} \ln f(\bzeta_{\lambda_i}), \quad
    \ln f(\bzeta'_*) = \sum_{i=1}^N r_{\mu_i} \ln f(\bzeta_{\mu_i}). \label{deltas_htheorem}
\end{gather}
Выражения~\eqref{deltas_conservative} можно формально рассматривать как приближённое решение уравнений
\(\phi=\delta(\bzeta'-\bzeta_\gamma)\) и \(\phi=\delta(\bzeta'_*-\bzeta_\gamma)\)
в пространстве функций вида~\eqref{eq:deltas_expansion}
\emph{проекционным} методом (Петрова"--~Галёркина) для весовых функций \(\psi_s(\bzeta)\):
\begin{gather}
    \int_{\mathbb{R}^3} \psi_s(\bzeta_\gamma) \left( \delta(\bzeta'-\bzeta_\gamma)
        - \sum_{i=1}^N r_{\lambda_i} \delta(\bzeta_{\lambda_i}-\bzeta_\gamma) \right) \dzeta_\gamma = 0, \label{lambda_galerkin}\\
    \int_{\mathbb{R}^3} \psi_s(\bzeta_\gamma) \left( \delta(\bzeta'_*-\bzeta_\gamma)
        - \sum_{i=1}^N r_{\mu_i} \delta(\bzeta_{\mu_i}-\bzeta_\gamma) \right) \dzeta_\gamma = 0, \label{mu_galerkin}\\
\end{gather}
Если множество \(\{\psi_s\}\) содержит все столкновительные инварианты, например
\begin{equation}\label{collision_invariants}
    \psi_0 = 1, \quad \psi_i = \zeta_i, \quad \psi_4 = \zeta_i^2,
\end{equation}
то при найденных \(\bzeta_{\lambda_i}\), \(r_{\lambda_i}\), \(\bzeta_{\mu_i}\), \(r_{\mu_i}\)
кубатура~\eqref{eq:deltas_Boltzmann_sum_hat} консервативна.
В~\eqref{deltas_htheorem} значения функции распределения в промежуточных точках (\(\bzeta'\) и \(\bzeta'_*\))
выражаются через определённый набор значений \(\{f(\bzeta_\gamma)\}\).
Это есть не что иное, как точечная \emph{интерполяция}.
Таким образом, изложенный метод вычисления интеграла столкновений может быть назван \emph{проекционно-интерполяционным}.

\end{document}
