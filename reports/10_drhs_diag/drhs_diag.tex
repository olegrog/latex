\documentclass{article}
\usepackage[utf8]{inputenc}
\usepackage{amsmath,amssymb,amsthm}
\usepackage{fullpage}

\title{\texttt{user\_case} functions}
\date{\today}
\author{Oleg Rogozin, Oleg Vasilyev}

%%% General commands
\newcommand{\pert}[1]{\delta#1}
\newcommand{\pder}[2][]{\frac{\partial#1}{\partial#2}}      % partial derivative
\newcommand{\dder}[1]{\Delta_{#1}}                          % discrete derivative
\newcommand{\vder}[2][]{\frac{\pert#1}{\pert#2}}            % functional derivative
\newcommand{\pderder}[3][]{\frac{\partial^2#1}{\partial#2\partial#3}}
\newcommand{\pderdual}[2][]{\frac{\partial^2#1}{\partial#2^2}}
\newcommand{\LL}{\mathcal{L}}
\newcommand{\FF}{\mathcal{F}}
\newcommand{\bx}{\mathbf x}
\newcommand{\diag}[1]{\left(#1\right)^\mathrm{diag}}
\newcommand{\diagA}[1]{\left(#1\right)^{m=n}}
\newcommand{\Set}[2]{\{\,{#1}:{#2}\,\}}
\newcommand{\eqdef}{\mathrel{\overset{\makebox[0pt]{\mbox{\normalfont\tiny\sffamily def}}}{=}}}

%%% Special commands
\newcommand{\matr}[4]{\begin{pmatrix} #1 & #2\\#3 & #4\end{pmatrix}}
\newcommand{\vect}[2]{\begin{pmatrix} #1\\#2\end{pmatrix}}
\newcommand{\funcUser}[4]{\texttt{user\_#1\(_{#3}\)(\(#2\))}\(\eqdef\displaystyle#4\)}
\newcommand{\funcDiff}[3]{\texttt{c\_diff\_#1(\(#2\))}\(\;\longrightarrow\;#3\)}

\theoremstyle{definition}
\newtheorem{example}{Example}
\newtheorem{remark}{Remark}
\newtheorem{exercise}{Exercise}

\begin{document}

\maketitle

Consider a Cauchy problem for the set of equations:
\begin{equation}\label{eq:problem}
    \pder[u_i]{t} = \FF_i, \quad
    u_i(t,\bx): \mathbb{R}_+\times\mathbb{R}^d\mapsto\mathbb{R}^N \quad (i=1,\dots,N),
\end{equation}
where \(\Set{\FF_i}{i=1,\dots,N}\) is the set of differential operators (nonlinear in general).
\(\FF_i\) is called the \emph{RHS} (\emph{right hand side}).
\(\pert{\FF_i}\) is the \emph{perturbation} of the RHS.
\begin{example}
Perturbation can be expressed in terms of functional derivative:
\begin{itemize}
    \item if \(\FF_i\) is linear, i.e., \(\FF_i = \LL_{ij}u_j\), then
        \(\pert{\FF_i} = \LL_{ij}\pert{u}_j\);
    \item if \(\FF_i = \LL_{ij}(u_k)u_j\), then
        \(\pert{\FF_i} = \left( \pder[\LL_{ij}]{u_k}u_j + \LL_{ij} \right)\pert{u}_j\);
    \item if \(\FF = \FF(u, \pder[u]{\bx})\), then
        \(\pert{\FF} = \pert{u}\left(\pder[\FF]{u} - \pder{\bx}\pder[\FF]{\pder[u]{\bx}}\right)\).
\end{itemize}
\end{example}
Next, we introduce the discretization of the \(\bx\) space with the set of points \(\Set{\bx^n}{n\in\mathbb{Z}^d}\).
Let \(u_j^n\) and \(\FF_i^n\) be approximations of the solution and RHS, respectively, at the point \(\bx^n\).
The \emph{discrete RHS} and its perturbation can be represented in the matrix form:
\begin{equation}\label{eq:matrix_representation}
    \FF_i^n = A_{ij}^{nm}u_j^m + b_i^n, \quad \pert{\FF_i^n} = B_{ij}^{nm}\pert{u_j^m},
    %\quad B_{ij}^{nm} = A_{ij}^{nm} + \vder[A_{ij}^{nm}]{u_k^l}u_k^l,
\end{equation}
then
\begin{equation}\label{eq:diagonal_term}
     \diag{\FF_i^n} \eqdef \left( A_{ij}^{nm} \right)_{j=i}^{m=n}, \quad
     \diag{\pert{\FF_i^n}} \eqdef \left( B_{ij}^{nm} \right)_{j=i}^{m=n}
\end{equation}
are the \emph{diagonal terms} of the corresponding discrete operators.
Let \(\dder{x_i}^{mn}\) and \(\dder{x_i,x_j}^{mn}\) denote the discrete approximation
of the partial derivatives \(\pder{x_i}\) and \(\pderder{x_i}{x_j}\), respectively.
For simplicity, let
\begin{equation}\label{eq:dder}
     \dder{x_i}u^n\eqdef\dder{x_i}^{mn}u^m, \quad \dder{x_i,x_j}u^n\eqdef\dder{x_i,x_j}^{mn}u^m.
\end{equation}
\begin{example}
    For the forward difference approximation
    \[ \dder{x}u^n = \frac{u^n-u^{n-1}}{h}, \quad \diag{\dder{x}u^n} = \diagA{\dder{x}^{mn}} = \frac1h\diagA{\delta^{mn}}, \]
    where \(\delta^{mn}\) is the Kronecker delta.
    For the central difference approximation
    \[ \dder{x,x}u^n = \frac{u^{n-1}-2u^n-u^{n+1}}{h}, \quad \diag{\dder{x,x}u^n} = -\frac2h\diagA{\delta^{mn}}. \]
\end{example}
In general, we need to calculate the following functions:
\begin{itemize}
    \item \funcUser{rhs}{u_j^n}{i}{ \FF_i^n },
    \item \funcUser{Drhs}{\pert{u_j^n},u_k^n}{i}{ \pert{\FF_i^n} },
    \item \funcUser{Drhs\_diag}{u_k^n}{i}{ \diag{\pert{\FF_i^n}} }.
\end{itemize}
Some discrete operators can be calculated using the following functions:
\begin{itemize}
    \item \funcDiff{fast}{u^n}{\dder{x_i}u^n, (\dder{x_i,x_j}u^n)_{j=i}},
    \item \funcDiff{diag}{}{\diagA{\dder{x_i}^{mn}}, \diagA{\dder{x_i,x_j}^{mn}}_{j=i}}.
\end{itemize}
\begin{example}
    For \(\FF = u\), \[
        \FF^n = u^n, \quad \pert{\FF^n} = \pert{u^n}, \quad \diag{\pert{\FF^n}} = \diagA{\delta^{mn}}.
    \]
\end{example}
\begin{example}
    For \(\FF = \pderdual[u]{x_i}\), \[
        \FF^n = \dder{x_i,x_i}u^n, \quad
        \pert{\FF^n} = \dder{x_i,x_i}\pert{u^n}, \quad
        \diag{\pert{\FF^n}} = \diagA{\dder{x_i,x_i}^{mn}}.
    \]
\end{example}
\begin{example}
    For \(\FF = u\pder[u]{x}\), \[
        \begin{gathered}
        \FF^n = \diagA{u^m\dder{x}u^n}, \quad
        \pert{\FF^n} = \diagA{\pert{u^m}\dder{x}u^n} + \diagA{u^m\dder{x}\pert{u^n}}, \\
        \diag{\pert{\FF^n}} = \dder{x}u^n + \diagA{u^m\diagA{\dder{x}^{mn}}}.
        \end{gathered}
    \]
\end{example}
\begin{example}
    For \(\FF = \frac12\pder[u^2]{x}\), \[
        \FF^n = \frac12\dder{x}\diagA{u^mu^n}, \quad
        \pert{\FF^n} = \dder{x}\diagA{u^m\pert{u^n}}, \quad
        \diag{\pert{\FF^n}} = \dder{x}u^n.
    \]
\end{example}
\begin{remark}
    Despite \(u\pder[u]{x} = \frac12\pder[u^2]{x}\), \(\diagA{u^m\dder{x}u^n} \neq \frac12\dder{x}\diagA{u^mu^n}\).
    The latter formulation corresponds to the conservative scheme.
\end{remark}

\begin{example}
    For \(\FF = \frac12\left(\pder[u]{x}\right)^2\), \[
        \FF^n = \frac12\diagA{\dder{x}u^m\dder{x}u^n}, \quad
        \pert{\FF^n} = \diagA{\dder{x}u^m\dder{x}\pert{u^n}}, \quad
        \diag{\pert{\FF^n}} = \diagA{\dder{x}u^m\diagA{\dder{x}^{mn}}}.
    \]
\end{example}
\begin{example}
    For \(\FF_i = \matr1102\vect{u_1}{u_2}\), \[
        \FF_i^n = \vect{u_1^n+u_2^n}{2u_2^n}, \quad
        \pert{\FF_i^n} = \vect{\pert{u_1^n}+\pert{u_2^n}}{2\pert{u_2^n}}, \quad
        \diag{\pert{\FF_i^n}} = \vect{1}{2}\diagA{\delta^{mn}}.
    \]
\end{example}
\begin{example}
    For \(\FF = \pder{x}\left( k(x)\pder[u]{x} \right)\), \[
        \FF^n = \dder{x}\diagA{k(x^m)\dder{x}u^n}, \quad
        \pert{\FF^n} = \dder{x}\diagA{k(x^m)\dder{x}\pert{u^n}}, \quad
        \diag{\pert{\FF^n}} = \dder{x}\diagA{k(x^m)\diagA{\dder{x}^{mn}}}.
    \]
    If we suppose that \(k(x)\) is slowly varying function, then \[
        \diag{\pert{\FF^n}} \approx \diagA{k(x^m)\diagA{\dder{x,x}^{mn}}},
    \]
    otherwise the diagonal term depends on the form of \(\dder{x}^{mn}\).
\end{example}

\begin{exercise}
    Calculate the desired functions for \[
        \pder[u]{t} = u\left(\pder[u]{x} + \pder[v]{x}\right), \quad
        \pder[v]{t} = \pder{x}(vu) + \nu\left(\pderdual[u]{x}+\pderdual[v]{y}\right).
    \]
\end{exercise}
\begin{exercise}
    Calculate the desired functions for \[
        \pder[u]{t} = \left(\pder[u^2]{x} + \pder[uv]{y}\right), \quad
        \pder[v]{t} = \pder{x}(uv) + v \pder[v]{y} + \nu\left(\pderdual[u]{x}+\pderdual[v]{y}\right).
    \]
\end{exercise}
\end{document}
