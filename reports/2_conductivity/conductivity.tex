\documentclass[a4paper,10pt]{article}
\usepackage{fullpage}
\usepackage[utf8x]{inputenc}
\usepackage{amssymb, amsmath, amsfonts}
\usepackage[english,russian]{babel}
\usepackage{array}

\usepackage[pdfauthor={Rogozin Oleg},%
pdftitle={Heat Conductivity Test},%
colorlinks,%
pagebackref,pdftex, unicode]{hyperref}

\newcommand{\dd}{\:\mathrm{d}}
\newcommand{\Kn}{\mathrm{Kn}}
\title{Тест на теплопроводность}
\author{Рогозин Олег}
\date{}
\begin{document}
\maketitle
Тест на теплопроводность является обязательным для верификации работы любого солвера.
Ниже описываются формулы для вычисления разностных аналогов макропараметров в безразмерных величинах,
а также соотношения между ними, которые должны выполняться в задаче одномерной теплопроводности при различных числах Кнудсена.

\section{Безразмерные переменные}
Переход к безразмерным переменным осуществляем по следующим формулам:
\[ f = \tilde{f}f_0,\; \xi_\alpha = \tilde{\xi_\alpha}\nu_0,\; x = \tilde{x}\lambda,\;
	t = \tilde{t}\frac{\lambda}{\nu_0},\; b = \tilde{b}\sigma\]
Для макропараметров \( n = \tilde{n}n_0, \; T = \tilde{T}T_0, \; q = \tilde{q}q_0\)
условие инвариантности формул относительно единиц измерения требует выполнения следующих условий:
\begin{alignat*}{2}
	n = \int f\dd\boldsymbol\xi &\Longrightarrow n_0 = f_0 \nu_0^3, \\
	T = \frac1{n}\int\frac{m\xi^2}{3}f\dd\boldsymbol\xi &\Longrightarrow T_0 = m\nu_0^2, \\
	q_\alpha = \int\xi_\alpha\frac{m\xi^2}{2}f\dd\boldsymbol\xi &\Longrightarrow q_0 = n_0m\nu_0^3.
\end{alignat*}
Наконец, свяжем две единицы измерения длины "--- \(x\) и \(b\):
\[ \lambda = \frac1{\pi\sqrt2 n_0 \sigma^2}, \nu_0 = \sqrt{2RT}.\]
так что \(\lambda\) оказывается длиной свободного пробега, а \(\sigma\) "--- эффективным диаметром молекул газа.
Кинетическое уравнение Больцмана в таком случае принимает вид (тильды опущены)
\[ f_t + \boldsymbol\xi\cdot\nabla f = \frac1{\pi\sqrt2}\int (f'f'_1-ff_1)gb\dd b \dd \varphi \dd\boldsymbol\xi \]
\section{Разностные формулы}
Безразмерные макропараметры вычисляются следующим образом:
\[ n = \sum_i f_i \Delta V, \]
\[ T = \frac1{3n}\sum_i \xi_i^2 f_i \Delta V, \]
\[ q_\alpha = \frac1{2}\sum_i \left(\xi_\alpha\right)_i\xi_i^2 f_i \Delta V, \]
где объём ячейки в скоростном пространстве равен
\[\Delta V = \left(\frac{\xi_\mathrm{cut}}{R}\right)^3\]

Уравнение Больцмана решается методом расщепления. Пространственная разностная сетка определяет размеры ячеек.
При решении уравнения переноса условие Куранта определяет шаг по времени \(\tau\),
который используется для моделирования интеграла столкновений.
\section{Постановка задачи}
Рассматривается одномерная задача теплопроводности.
Газ ограничен двумя плоскостями, удалёнными друг от друга на расстояние \(L\).
От них происходит диффузное отражение молекул газа.
С одной стороны температура \(T_1\), с другой "--- \(T_2<T_1\).
Необходимо найти значение потока тепла в стационарном режиме.
В зависимости от числа Кнудсена \(\Kn=\lambda/L\) решение задачи существенно меняет свой характер.
Отдельно рассмотрены случаи \(\Kn\gg1\) и \(\Kn\ll1\), для которых существует теоретическая оценка. 

\section{Свободномолекулярный режим течения \(\Kn\gg1\)}
В первую очередь необходимо проверить решение только левой части кинетического уравнения.
При \(\Kn\gg1\) можно отбросить интеграл столкновения.
В стационарном режиме поток тепла и температура постоянны и равны
\footnote{см. \textit{Коган М.Н.} Динамика разреженного газа, стр. 257, форм. (2.17) и (2.19).}
\[ q = 2n \sqrt{\frac{2T_1 T_2}\pi} \left(\sqrt{T_1}-\sqrt{T_2}\right), \quad T=\sqrt{T_1 T_2}. \]
\section{Модель сплошной среды \(\Kn\ll1\)}
Согласно первому приближению теории Чепмена"--~Энскога поток тепла будет равен
\[ \mathbf{q} = -\kappa\nabla T. \]
Для модели твердых сфер во втором приближении Чэпмена"--~Энскога коэффициент теплопроводности равен
\footnote{см. \textit{Ландау Л.Д., Лифшиц Е.М.} Теоретическая физика. Т. 10. Физическая кинетика, стр. 53, форм. (5).}
\[ \kappa = \frac{75}{64\sqrt\pi\sigma^2}\sqrt{\frac{T}{m}}. \]
Для соблюдения большой точности, скажем, что третье и четвертое приближения дают поправку \((1+0.015+0.001)\).
Переходя к безразмерным переменным
\[ \tilde{\mathbf{q}}n_0m\nu_0^3 = -\frac{75}{64\sqrt\pi\sigma^2}\sqrt{\tilde{T}} \nu_0\:\tilde{\nabla}\pi\sqrt2 n_0 \sigma^2\: \tilde{T} m\nu_0^2, \]
формулы принимают следующий вид (тильды опущены):
\[ \mathbf{q} = -\kappa \nabla T,\quad \kappa = \frac{75}{64}\sqrt{2\pi T}. \]
В стационарном режиме распределение температуры будет линейным:
\[ T = T_1 -\frac{x}{L}(T_1-T_2). \]
А поскольку в безразмерных переменных \(\Kn=1/L\), то поток тепла равен
\[ q = \frac{75}{64}\nabla T\sqrt{2\pi T} = \frac{75}{64}(T_1-T_2)\Kn\sqrt{2\pi T}. \]

\section{Конкретный пример}
Положим \(T_1=1.25\), \(T_2=1\), тогда для свободномолекулярного случая:
\[q(\Kn\rightarrow\infty)\simeq0.21\cdot n. \]
Для приближения к модели сплошной среды необходимо уменьшение числа Кнудсена,
что чревато большими требованиями к вычислительным ресурсам:
для получения хорошей точности нужно также уменьшать размеры пространственных ячеек и шаг по времени.
В связи с этим вполне разумно взять \(\Kn=0.02\), но тогда наблюдается скачок температуры у стенок,
поэтому поток тепла и градиент температуры необходимо вычислять локально:
\[q(\Kn=0.02, x=L/2)\simeq1.55\cdot10^{-2}, \]
Кроме того, можно проводить оценку, вообще не дожидаясь наступления стационарного состояния,
ввиду опять же локальности свойства теплопроводности.
\end{document}

