%&pdflatex
\documentclass{article}
\usepackage[utf8]{inputenc}
\usepackage[T2A]{fontenc}
\usepackage[russian]{babel}

\usepackage{amssymb, amsmath, amsthm}
\usepackage{fullpage}
\usepackage{indentfirst}

\usepackage[
    pdfauthor={Rogozin Oleg},
    pdftitle={Problem},
    colorlinks, pdftex, unicode
]{hyperref}

\newtheorem{proposition}{Proposition}
\newtheorem{remark}{Remark}

\title{Задача}
\author{Олег Рогозин}

\newcommand{\bx}{\boldsymbol{x}}
\newcommand{\by}{\boldsymbol{y}}

\begin{document}

\section{Задача}

Рассмотрим некоторую область \(\Omega\subset\mathbb{R}^3\), ограниченную поверхностью \(S\),
состоящей из \emph{замкнутых} поверхностей:
\begin{equation}\label{eq:domain_boundary}
    S \equiv \partial\Omega = \bigcup_a S_a, \quad \partial{S_a} = 0.
\end{equation}
Определим внутри \(\Omega\) функции \(\psi\), \(p\) и \(u_i\):
\begin{gather}
    \partial_i \left( \psi^c u_i \right) = 0, \label{eq:p_def}\\
    u_i\psi_i = \frac{\gamma_2}2\psi\psi_{ii}, \label{eq:psi_def}\\
    p_k = f_\psi \psi_k + f_{\psi_k} \psi_{jj} - 2\psi^c u_j\partial_j u_k, \label{eq:u_i_def}
\end{gather}
где \(\psi_i\equiv\partial_i\psi\), \(\psi_{ii}\) "--- лаплассиан,
\begin{equation}\label{eq:f_def}
    f = \frac{\gamma_7}2 c^2 \psi^c\psi_j^2, \quad
    f_\psi\equiv\partial_\psi f, \quad
    f_{\psi_j}\equiv\partial_{\psi_j} f.
\end{equation}
Граничные условия:
\begin{equation}\label{eq:boundary_conditions}
    \psi|_{S_a} = \psi_a, \quad u_i|_S = 0.
\end{equation}
где \(\psi_a\) "--- константы.
Функция \(p\) входит в систему уравнений только своей производной,
поэтому при известных \(\psi\) и \(u_i\) может быть найдена из решения задачи Неймана:
\begin{equation}\label{eq:p_Neumann}
    p_{kk} = \partial_k\left( f_\psi \psi_k + f_{\psi_k} \psi_{jj} - 2\psi^c u_j\partial_j u_k \right), \quad
    p_k n_k|_S = \left. f_\psi \psi_k n_k \right|_S,
\end{equation}
Решение задачи Неймана определяется с точностью до константы, поэтому для определённости можно положить
\begin{equation}\label{eq:p_over_domain}
    \int_\Omega p = 0.
\end{equation}

Определим дополнительно следующие функционалы:
\begin{equation}\label{eq:force_energy}
    F_i^a = \oint_{S_a} (f-p) n_i, \quad U = \int_\Omega (f-p).
\end{equation}
Известно, что при сдвиге поверхности \(S_a\) на вектор \(\delta r_i^a\) вариация равна
\begin{equation}\label{eq:variation_U}
    \delta\int_\Omega f = F_i^a + \int_\Omega \delta(f-p).
\end{equation}
При \(\delta\psi|_S = 0\)
\begin{equation}\label{eq:variation_f}
    \int_\Omega \delta f = \int_\Omega \left( f_\psi - \partial_k f_{\psi_k} \right)\delta\psi =
    -\int_\Omega \left( g_\psi\psi_k^2 + 2g\psi_{kk} \right)\delta\psi,
\end{equation}
где \(f = g\psi_k^2\).

Необходимо вычислить
\begin{equation}\label{eq:problem}
    \int_\Omega \delta p.
\end{equation}

\section{Известные утверждения}

\begin{proposition}\label{prop:integral_mixed}
Для произвольной функции \(h(\psi)\)
\begin{equation}\label{eq:integral_mixed}
    \int_\Omega h \psi_{ij} \psi_j = \int_\Omega h \psi_i \psi_{jj}.
\end{equation}
\end{proposition}

\begin{proof}
Запишем два дифференциальных тождества:
\begin{gather}
    \partial_i\left( h\psi_j^2 \right) = h_\psi\psi_i\psi_j^2 + 2h\psi_{ij}\psi_j, \label{eq:prop1-1}\\
    \partial_j\left( h\psi_i\psi_j \right) = h_\psi\psi_i\psi_j^2 + h\psi_{ij}\psi_j + h\psi_i\psi_{jj}. \label{eq:prop1-2}
\end{gather}
В силу~\eqref{eq:boundary_conditions}, \(\psi_i|_S = \psi_k n_k n_i\), следовательно,
интегралы от~\eqref{eq:prop1-1} и~\eqref{eq:prop1-2} по \(\Omega\) равны друг другу.
\end{proof}

\begin{proposition}\label{prop:sum_forces}
Справедливо
\begin{equation}\label{eq:sum_forces}
    \sum_a F_i^a = 0.
\end{equation}
\end{proposition}

\begin{proof}
Используя~\eqref{eq:u_i_def},~\eqref{eq:boundary_conditions} и~\eqref{eq:integral_mixed}, получаем
\[
    \sum_a F_i^a = \int_\Omega \left( f_\psi\psi_i + f_{\psi_j}\psi_{ij} - p_i \right) =
    2\oint_S \psi^c u_i u_j n_j = 0.
\]
\end{proof}

\begin{proposition}\label{prop:energy_psi}
Справедливо
\begin{equation}\label{eq:energy_psi}
    U = \frac{\gamma_7}{2}c^2 \sum_a \frac{\psi_a^{c+1}}{c+1} \oint_{S_a} \psi_j n_j.
\end{equation}
\end{proposition}

\begin{proof}
Утверждение следует из определения~\eqref{eq:f_def} и равенств
\begin{gather*}
    \partial_j\left( \psi^{c+1} \psi_j \right) = (c+1)\psi^c\psi_j^2 + \psi^{c+1}\psi_{jj} =
        (c+1)\psi^c\psi_j^2 + \frac2{\gamma_2}\psi^c\psi_j u_j, \\
    \partial_j\left( \psi^{c+1} u_j \right) = (c+1)\psi^c\psi_j u_j + \psi^{c+1}\partial_j u_j =
        \psi^c \psi_j u_j.
\end{gather*}
\end{proof}

\begin{proposition}\label{prop:capacity}
Если пренебречь полем \(u_i\) по сравнению с \(\psi_i\), то существует константная матрица \(C_{ab}\), такая что
\begin{equation}\label{eq:capacity}
    e_a\equiv\oint_{S_a} \psi_j n_j = C_{ab}\psi_b, \quad
    U = \frac{\gamma_7c^2}{2(c+1)} C_{ab} \psi_a^{c+1}\psi_b.
\end{equation}
\end{proposition}

\begin{proof}
Из~\eqref{eq:p_def} и~\eqref{eq:psi_def} вытекает уравнение
\[ \psi_{kk} = -\frac2{c\gamma_2}\partial_k u_k, \]
общее решение которого
\begin{equation}\label{eq:psi_Green}
    \psi = \frac2{c\gamma_2} \int_{\Omega'} u_k G_k + \sum_b \psi_b \oint_{S_b'} G_k n_k
\end{equation}
записывается через функцию Грина \(G(\bx, \bx')\):
\begin{equation}\label{eq:Green_function_psi}
    G_{kk} = \delta(\bx-\bx'), \quad G|_S = 0.
\end{equation}
Приравнивая \(u_k=0\) в~\eqref{eq:psi_Green}, получаем искомое утверждение, причём
\[ C_{ab} = \oint_{S_a}\oint_{S_b'} \partial_{x_i x_k'} G n_i n_k'. \]
\end{proof}


\section{Функция Грина}

Определим функцию Грина \(G(\bx, \bx')\) для задачи Неймана~\eqref{eq:p_Neumann}:
\begin{equation}\label{eq:Green_function_p}
    G_{kk} = \delta(\bx-\bx') - \frac1V, \quad \left. G_k n_k \right|_S = 0,
\end{equation}
где \(V = \int_\Omega\). Здесь и далее подразумеваем, что производные от \(G\)
берутся по переменной \(\bx'\), а выражения под интегралом по \(\Omega'\) также
зависят от \(\bx'\) вместо \(\bx\).
Тогда решение~\eqref{eq:p_Neumann} может быть записано явно:
\begin{equation}\label{eq:pressure_solution}
    p = \int_{\Omega'} p G_{kk} = -\int_{\Omega'} p_k G_k.
\end{equation}

Функция \(G(\bx, \bx')\) определяется с точностью до функции от \(x\).
При дополнительном условии \(\int_{\Omega'} G = 0\)
обеспечивается её симметричность \(G(\bx, \bx') = G(\bx', \bx)\).

Кстати, в силу~\eqref{eq:Green_function_p}, для произвольной дифференцируемой функции \(h(\bx)\) справедливы равенства
\begin{equation}\label{eq:Green_double_integral}
    \int_\Omega \int_{\Omega'} h G_{kk} = 0, \quad
    \int_\Omega \int_{\Omega'} h_k G_k = 0.
\end{equation}


\end{document}
