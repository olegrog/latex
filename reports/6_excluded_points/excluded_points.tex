%&pdflatex
\documentclass[a4paper,12pt]{article}
\usepackage{amssymb, amsmath}
\usepackage[utf8]{inputenc}
\usepackage[T2A,T1]{fontenc}
\usepackage[english,russian]{babel}
\usepackage{csquotes}
\usepackage{mathrsfs}

\usepackage{fullpage}
\usepackage{indentfirst}
\usepackage[font=small,labelsep=period,tableposition=top]{caption}

\usepackage{graphicx}

\usepackage[
    pdfauthor={Rogozin Oleg},
    pdftitle={On points that are excluded from the integration lattice},
    colorlinks, pdftex, unicode
]{hyperref}

\title{О точках, исключаемых из кубатурной решётки}
\author{Рогозин Олег}

\newcommand{\Kn}{\mathrm{Kn}}
\newcommand{\Ma}{\mathrm{Ma}}
\newcommand{\dd}{\:\mathrm{d}}
\newcommand{\pder}[2][]{\frac{\partial#1}{\partial#2}}
\newcommand{\pderder}[2][]{\frac{\partial^2 #1}{\partial #2^2}}
\newcommand{\Pder}[2][]{\partial#1/\partial#2}
\newcommand{\dzeta}{\boldsymbol{\dd\zeta}}
\newcommand{\bzeta}{\boldsymbol{\zeta}}
\newcommand{\bh}{\boldsymbol{h}}
\newcommand{\be}{\boldsymbol{e}}
\newcommand{\Nu}{\mathcal{N}}
\newcommand{\OO}[1]{O(#1)}
\newcommand{\Set}[2]{\{\,{#1}:{#2}\,\}}

\begin{document}

\maketitle
\tableofcontents

\section{Основные формулы}

Рассмотрим пространственно-однородное уравнение Больцмана
\begin{equation}\label{eq:Boltzmann}
    \pder[f]{t} = J(f,f)
\end{equation}
со столкновительным членом, записанном в виде
\begin{equation}\label{eq:ci}
    J(f,f) = \int (f'f'_*-ff_*)B\dd\Omega(\boldsymbol\alpha)\dzeta_*.
\end{equation}

%%% Velocity grid and functions on it
Пусть регулярная скоростная сетка построена таким образом,
что кубатура в пространстве \(\bzeta\) выражается в виде взвешенной суммы
\begin{equation}\label{eq:zeta_cubature}
    \int F(\bzeta) \dzeta \approx \sum_{\gamma\in\Gamma} F_\gamma w_\gamma =
        \sum_{\gamma\in\Gamma} \hat{F_\gamma},
        \quad \sum_{\gamma\in\Gamma} w_\gamma = V_\Gamma,
        \quad F_\gamma = F(\bzeta_\gamma),
\end{equation}
где \(\mathcal{V} = \Set{\zeta_\gamma}{\gamma\in\Gamma}\) "--- множество дискретных скоростей,
\(F\) "--- произвольная функция от \(\bzeta\) (\(\hat{F_\gamma}\) используется для удобства),
а \(V_\Gamma\) "--- общий объём скоростной сетки.
Тогда интеграл столкновений~\eqref{eq:ci}, записанный в симметризованной форме
\begin{equation}\label{eq:symm_ci}
    \begin{aligned}
    J(f_\gamma, f_\gamma) = \frac14\int &\left[
        \delta(\bzeta-\bzeta_\gamma) + \delta(\bzeta_*-\bzeta_\gamma)
        - \delta(\bzeta'-\bzeta_\gamma) - \delta(\bzeta'_*-\bzeta_\gamma)\right] \\
        &\times(f'f'_* - ff_*)B \dd\Omega(\boldsymbol{\alpha}) \dzeta\dzeta_*,
    \end{aligned}
\end{equation}
где \(\delta(\bzeta)\) "--- дельта-функция Дирака в \(\mathbb{R}^3\),
имеет следующий дискретный аналог:
\begin{equation}\label{eq:discrete_symm_ci}
    \hat{J}_\gamma = \frac{\pi V_\Gamma^2}{\sum_{\nu\in\Nu} w_{\alpha_\nu}w_{\beta_\nu}}
        \sum_{\nu\in\Nu} \left(
            \delta_{\alpha_\nu\gamma} + \delta_{\beta_\nu\gamma}
            - \delta_{\alpha'_\nu\gamma} - \delta_{\beta'_\nu\gamma}
        \right)\left(
            \frac{w_{\alpha_\nu}w_{\beta_\nu}}{w_{\alpha'_\nu}w_{\beta'_\nu}}
            \hat{f}_{\alpha'_\nu}\hat{f}_{\beta'_\nu} - \hat{f}_{\alpha_\nu}\hat{f}_{\beta_\nu}
        \right)B_\nu.
\end{equation}
Здесь \(\alpha_\nu\in\Gamma\), \(\beta_\nu\in\Gamma\) и \(\boldsymbol{\alpha}_\nu\)
выбираются по некоторому правилу для восьмимерной кубатуры.
Дискретный аналог распределения Максвелла может быть записан в следующем виде:
\begin{equation}\label{eq:discrete_Maxwell}
    \hat{f}_{M\gamma} = \rho\left[\sum_{\alpha\in\Gamma}w_\alpha\exp
            \left(-\frac{(\bzeta_\alpha - \boldsymbol{v})^2}{T}\right)
        \right]^{-1}
        w_\gamma\exp\left(-\frac{(\bzeta_\gamma - \boldsymbol{v})^2}{T}\right).
\end{equation}

%%% Projection and interpolation
В общем случае, скорости после столкновения,
\(\bzeta_{\alpha'_\nu}\) и \(\bzeta_{\beta'_\nu}\), не попадают в \(\mathcal{V}\).
Если они просто заменяются ближайшими сеточными скоростями,
\(\bzeta_{\lambda_\nu}\in\mathcal{V}\) и \(\bzeta_{\mu_\nu}\in\mathcal{V}\),
дискретный интеграл столкновений~\eqref{eq:discrete_symm_ci} теряет свойство консервативности.
Более того, дискретное распределение Максвелла~\eqref{eq:discrete_Maxwell} перестаёт быть равновесным состоянием.
Для решения первой проблемы применяется проецирование \(\bzeta_{\alpha'_\nu}\)
на множество узлов \(\Set{\bzeta_{\lambda_\nu+s_a}}{a\in\Lambda}\subset\mathcal{V}\):
\begin{equation}\label{eq:ci_projection}
    \delta_{\alpha'_\nu\gamma} = \sum_{a\in\Lambda} r_{\lambda,a}\delta_{\lambda_\nu+s_a,\gamma}.
\end{equation}
Для определённости будем считать, что \(\Lambda = \Set{a}{r_{\lambda,a}\neq0}\).
\emph{Проекционные веса} \(r_{\lambda,a}\) подбираются таким образом, чтобы обеспечить сохранение массы, импульса и энергии, т.е.
\begin{equation}\label{eq:stencil_conservation}
    \sum_{a\in\Lambda} r_{\lambda,a} = 1, \quad
    \sum_{a\in\Lambda} r_{\lambda,a} \bzeta_{\lambda_\nu+s_a} = \bzeta_{\alpha'_\nu}, \quad
    \sum_{a\in\Lambda} r_{\lambda,a} \bzeta_{\lambda_\nu+s_a}^2 = \bzeta_{\alpha'_\nu}^2.
\end{equation}
Набор правил смещения \(\mathcal{S} = \Set{s_a}{a\in\Lambda}\)
называется \emph{проекционным шаблоном}.
Для решения второй проблемы, а именно выполнения
\begin{equation}\label{eq:strict_interpolation}
    \hat{J}_\gamma(\hat{f}_{M\gamma}, \hat{f}_{M\gamma}) = 0,
\end{equation}
подбирается необходимая интерполяция \(\hat{f}_{\alpha'_\nu}\).
Аналогичные процедуры применяются к \(\bzeta_{\beta'_\nu}\) и \(\hat{f}_{\beta'_\nu}\).

%%% Time-evolution scheme
Интегрируя уравнение Больцмана~\eqref{eq:Boltzmann} вдоль интервала времени~\(\Delta\tau\), получаем
\begin{equation}\label{eq:time_scheme}
    \hat{f}_\gamma(t+\Delta\tau) = \hat{f}_\gamma(t) + \int_t^{t+\Delta\tau} \hat{J}_\gamma \dd{t}.
\end{equation}
Если переписать~\eqref{eq:discrete_symm_ci} как
\begin{equation}\label{eq:discrete_short_ci}
    \hat{J}_\gamma = \frac1{|\Nu|}\sum_{\nu\in\Nu} \hat{\mathscr{J}}_{\gamma\nu}
\end{equation}
и подставить в~\eqref{eq:time_scheme}, то получается схема непрерывного счёта
\begin{equation}\label{eq:continuous_scheme}
   \hat{f}_\gamma^{\nu+1} = \hat{f}_\gamma^\nu + \frac{\Delta\tau}{|\Nu|}\hat{\mathscr{J}}_{\gamma\nu},
\end{equation}
обеспечивающая второй порядок аппроксимации по времени.

%%% Positive distribution function
Схема~\eqref{eq:continuous_scheme} допускает отрицательные значения функции распределения,
что противоречит её физической природе. Чтобы обеспечить её положительность,
достаточно потребовать выполнения неравенства
\begin{equation}\label{eq:positive_f}
   \hat{f}_\gamma^\nu + \frac{\Delta\tau}{|\Nu|}\hat{\mathscr{J}}_{\gamma\nu} > 0
\end{equation}
для всех \(\gamma\in\Gamma\) и \(\nu\in\Nu\).
Если \(\gamma = \alpha_\nu\), то получаем оценку
\begin{equation}\label{eq:positive_f_alpha}
   \hat{f}_{\alpha_\nu} - \frac{\Delta\tau \hat{A}}{|\Nu|}\hat{f}_{\alpha_\nu}\hat{f}_{\beta_\nu} > 0, \quad
   \hat{A} = \frac{\pi V_\gamma|\Nu|}{\sum_{\nu\in\Nu} w_{\alpha_\nu}w_{\beta_\nu}}
\end{equation}
или
\begin{equation}\label{eq:positive_f_alpha2}
   |\Nu| > \Delta\tau \hat{A} \hat{f}_{\max},
\end{equation}
где
\begin{equation}\label{eq:hat_f_max}
   \hat{f}_{\max} = \max_{\gamma\in\Gamma} \hat{f}_\gamma.
\end{equation}
Аналогичная оценка справедлива для \(\gamma = \beta_\nu\).
Рассмотрение проекционных скоростей \(\gamma = \lambda_\nu+s_a\)
зависит от выбранной интерполяционной стратегии.

\section{Равномерная сетка}

Возьмём равномерную прямоугольную сетку с шагом \(h\), тогда,
полагая \(w_\gamma = V_\Gamma/|\mathcal{V}| = h^3\) в~\eqref{eq:discrete_symm_ci}, получаем
\begin{equation}\label{eq:discrete_symm_ci_uniform}
    J_\gamma = \frac{\pi V_\Gamma|\mathcal{V}|}{|\Nu|}
        \sum_{\nu\in\Nu} \left(
            \delta_{\alpha_\nu\gamma} + \delta_{\beta_\nu\gamma}
            - \delta_{\alpha'_\nu\gamma} - \delta_{\beta'_\nu\gamma}
        \right)\left( f_{\alpha'_\nu} f_{\beta'_\nu} - f_{\alpha_\nu} f_{\beta_\nu} \right)B_\nu.
\end{equation}
Ввиду симметричности сетки, для достижения консервативности при проецировании на неё,
достаточно использовать два ближайших узла, т.\,е.
\begin{equation}\label{eq:uniform_projection}
    \delta_{\alpha'_\nu\gamma} = (1-r)\delta_{\lambda_\nu\gamma} + r\delta_{\lambda_\nu+s,\gamma}, \quad
    \delta_{\beta'_\nu\gamma} = (1-r)\delta_{\mu_\nu\gamma} + r\delta_{\mu_\nu-s,\gamma},
\end{equation}
где
\begin{equation}\label{eq:r_uniform}
    r = \frac{E_0-E_1}{E_2-E_1}, \quad
    E_0 = \bzeta_{\alpha'_\nu}^2 + \bzeta_{\beta'_\nu}^2, \quad
    E_1 = \bzeta_{\lambda_\nu}^2 + \bzeta_{\mu_\nu}^2, \quad
    E_2 = \bzeta_{\lambda_\nu+s}^2 + \bzeta_{\mu_\nu-s}^2.
\end{equation}
При этом справедливо неравенство \(0\leq r < 1\).

Для того чтобы оценить минимальное значение \(|\mathcal{V}|\), при которых выполняется~\eqref{eq:positive_f},
введём ещё одну (кроме \(f_{\max}\)) характеристику функции распределения "---
максимальное отношение между соседними дискретными скоростями:
\begin{equation}\label{eq:max_f_ratio_uniform}
    \left(\frac{f_{\gamma+s}}{f_\gamma}\right)_{\max} =
       \max_{(\bzeta_{\gamma+s}-\bzeta_{\gamma})\be_i = \{0,\pm{h}\}} \frac{f_{\gamma+s}}{f_\gamma},
\end{equation}
где \(\be_i\) "--- базис прямоугольной сетки.
Очевидно, что для гладких функций это значение невелико,
в то время как для разрывных функций оно может достигать нескольких порядков.

Для равномерной сетки неравенство~\eqref{eq:positive_f_alpha2} перепишется как
\begin{equation}\label{eq:positive_f_alpha_uniform}
   |\Nu| > \Delta\tau A f_{\max}, \quad A = \pi V_\Gamma.
\end{equation}
Ради удобства введём также сокращённые обозначения
\begin{equation}\label{eq:uniform_deltas}
    \Delta_1 = f_{\lambda_\nu\gamma}f_{\mu_\nu\gamma}, \quad
    \Delta_2 = f_{\lambda_\nu+s,\gamma}f_{\mu_\nu-s,\gamma}.
\end{equation}

\subsection{Степенная интерполяция}

Степенная интерполяция
\begin{equation}\label{eq:uniform_pim}
    f_{\alpha'_\nu} = \left(f_{\lambda_\nu\gamma}\right)^{1-r} \left(f_{\lambda_\nu+s,\gamma}\right)^r, \quad
    f_{\beta'_\nu} = \left(f_{\mu_\nu\gamma}\right)^{1-r} \left(f_{\mu_\nu-s,\gamma}\right)^r,
\end{equation}
или короче
\begin{equation}\label{eq:uniform_pim2}
    f_{\alpha'_\nu}f_{\beta'_\nu} = \Delta_1^{1-r} \Delta_2^r,
\end{equation}
точно обеспечивает~\eqref{eq:strict_interpolation}.
Для выполнения~\eqref{eq:positive_f} при \(\gamma = \lambda_\nu\) справедлива оценка
\begin{equation}\label{eq:positive_f_lambda_uniform_pim}
   f_{\lambda_\nu} - \frac{\Delta\tau A}{|\Nu|} \Delta_1^{1-r} \Delta_2^r > 0
\end{equation}
или
\begin{equation}\label{eq:positive_f_lambda_uniform_pim2}
   |\Nu| > \Delta\tau A f_{\max} \left(\frac{f_{\gamma+s}}{f_\gamma}\right)_{\max}^2.
\end{equation}
Аналогичные выкладки проводятся для \(\gamma = \lambda_\nu+s, \mu_\nu, \mu_\nu-s\).
Таким образом, общая оценка совпадает с~\eqref{eq:positive_f_lambda_uniform_pim2},
поскольку она включает~\eqref{eq:positive_f_alpha_uniform}.

\subsection{Симметричная интерполяция}

Симметричная интерполяция
\begin{equation}\label{eq:uniform_spm}
    f_{\alpha'_\nu}f_{\beta'_\nu} = \frac{\Delta_1\Delta_2}{r\Delta_1+(1-r)\Delta_2}
\end{equation}
обеспечивает~\eqref{eq:strict_interpolation} с точностью \(\OO{h}\).
Для выполнения~\eqref{eq:positive_f} при \(\gamma = \lambda_\nu\) справедлива оценка
\begin{equation}\label{eq:positive_f_lambda_uniform_spm}
   f_{\lambda_\nu} - \frac{\Delta\tau A}{|\Nu|} \frac{\Delta_1\Delta_2}{r\Delta_1+(1-r)\Delta_2} > 0
\end{equation}
или
\begin{equation}\label{eq:positive_f_lambda_uniform_spm2}
   |\Nu| > \Delta\tau A f_{\max} \left(\frac{f_{\gamma+s}}{f_\gamma}\right)_{\max}.
\end{equation}
Аналогичные выкладки проводятся для \(\gamma = \lambda_\nu+s, \mu_\nu, \mu_\nu-s\).
Видно, что симметричная интерполяция позволяет обойтись меньшей кубатурной решёткой
по сравнению со степенной интерполяцией, особенно если функция распределения претерпевает
резкие изменения на сетке.

\section{Неравномерная сетка}

Рассмотрим наиболее компактный, пятиточечный (\(|\mathcal{S}|=5\)), шаблон на прямоугольной сетке.
Пусть \(\boldsymbol{\eta} = \bzeta_{\alpha'_\nu} - \bzeta_{\lambda_\nu}\),
\(\bh_+\) и \(\bh_-\) "--- минимальные диагональные смещения на сетке,
такие что \(\bh_+\) направлен в тот же октант, что и \(\boldsymbol{\eta}\),
а \(\bh_-\) лежит в противоположном.
Тогда \emph{компактная 5-точечная схема} строится на узлах
\begin{equation}\label{eq:stencil_nodes}
    \bzeta_{\lambda_\nu+s_0} = \bzeta_{\lambda_\nu}, \quad
    \bzeta_{\lambda_\nu+s_i} = \bzeta_{\lambda_\nu} + (\bh_+\cdot \be_i)\be_i, \quad
    \bzeta_{\lambda_\nu+s_4} = \bzeta_{\lambda_\nu} + \bh_-,
\end{equation}
где \(\be_i\) "--- базис прямоугольной сетки.
Для выполнения~\eqref{eq:stencil_conservation}, выбираются проекционные веса
\begin{equation}\label{eq:stencil_weights}
    r_{\lambda,0} = 1 - \sum_{l=1}^4 r_{\lambda,l}, \quad
    r_{\lambda,i} = \frac{(\boldsymbol{\eta} - r_{\lambda,4}\bh_-)\cdot\be_i}
        {\bh_+\cdot\be_i}, \quad
    r_{\lambda,4} = \frac{\boldsymbol{\eta}^2 - \boldsymbol{\eta}\cdot\bh_+}
        {\bh_-^2 - \bh_-\cdot\bh_+}.
\end{equation}
При этом справедливо неравенство \(0\leq r < 1\).



\end{document}
