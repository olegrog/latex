\documentclass{article}
\usepackage[english]{babel}
\usepackage{csquotes}

%%% Functional packages
\usepackage{amsmath, amssymb, amsthm}
\usepackage{physics, siunitx}
\usepackage{subcaption}
\usepackage{graphicx}
\usepackage[subfolder]{gnuplottex}

%%% Configuration packages
\usepackage{fullpage}

%%% Problem-specific aliases
\newcommand{\Kn}{\mathrm{Kn}}
\newcommand{\Ma}{\mathrm{Ma}}
\newcommand{\bxi}{\boldsymbol{\xi}}
\newcommand{\bzeta}{\boldsymbol{\zeta}}

\usepackage[
    pdfusetitle,
    colorlinks
]{hyperref}

\usepackage[
    backend=biber,
    style=numeric,
    giveninits=true,
    sorting=none,
]{biblatex}
\addbibresource{shakhov_model.bib}

\title{Shakhov relaxation model for Boltzmann collisional operator}
\author{Oleg Rogozin}

\begin{document}
\maketitle
\tableofcontents

\section{Kinetic equation}

%%% Boltzmann equation
The Boltzmann kinetic equation have the following dimensionless form:
\begin{equation}\label{eq:Boltzmann}
    \pdv{f}{t} + \xi_i\pdv{f}{x_i} = \frac1kJ(f),
\end{equation}
where $f(t,x_i,\xi_i)$ is the velocity distribution function, $k=\Kn\sqrt\pi/2$,
and $\Kn$ is the Knudsen number.
The collision operator $J(f)$ can be simplified by a relaxation model
\begin{equation}\label{eq:model}
    J(f) = \frac{\rho T^{1-\omega}}{\gamma_1}\qty(F - f),
\end{equation}
which reproduces the viscosity coefficient $\gamma_1T^\omega$,
preserving the following conservation properties:
\begin{equation}\label{eq:F}
    \int \begin{pmatrix} 1 \\ \xi_i \\ \xi^2 \end{pmatrix} J(f)\dd\bxi = 0.
\end{equation}
It is also crucial that $J(f_M) = 0$, where
\begin{equation}\label{eq:Maxwell}
    f_M = \frac{\rho}{(\pi T)^{D/2}}\exp(-\frac{c^2}T)
\end{equation}
is the Maxwell distribution, $D$ is the dimensionality of the velocity space.
For brevity, we use \(c_i = \xi_i - v_i\) and \(c = |\boldsymbol{c}|\).

%%% Macroscopic variables
The macroscopic variables are calculated as follows:
\begin{equation}\label{eq:macro}
    \rho = \int f \dd\bxi, \quad
    \rho v_i = \int \xi_i f \dd\bxi, \quad
    p = \rho T = \frac2D\int c^2 f \dd\bxi, \quad
    q_i = \int c_i c^2 f \dd\bxi,
\end{equation}
where \(\rho\) is the density, \(v_i\) is the velocity, \(T\) is the temperature,
\(p\) is the pressure, and \(q_i\) is the heat-flux vector.

%%% Specific relaxation models
When $F=f_M$, the relaxation model is called BGK~\cite{bhatnagar1954model}.
\textcite{shakhov1968generalization} proposed its modification
\begin{equation}\label{eq:Shakhov}
    F = f_S = f_M\qty(1 + \frac{1-\Pr}{D+2}\frac{4q_ic_i}{pT}\qty(\frac{c^2}{T} - \frac{D+2}2))
\end{equation}
to ensure the given Prandtl number $\Pr$.

\section{Space-homogeneous problem}

%%% Space-homogeneous solution
In the space-homogeneous case, we have
\begin{equation}\label{eq:relax}
    \pdv{f}{t} = \frac1\tau\qty(f_S - f),
\end{equation}
where the relaxation time $\tau$ is time-independent.
This equation can be solved analytically (see, e.g., \cite{titarev2004numerical}).
The heat flux turns out to relax exponentially:
\begin{equation}\label{eq:relax:q}
    q_i(t) = q_i(0)\exp(-\Pr\frac{t}\tau).
\end{equation}
In contrast, the distribution function evolves non-monotonically:
\begin{equation}\label{eq:relax:f}
    f(t) = f(0)\exp(-\frac{t}\tau) + f_M\qty(1 + A\exp(-\Pr\frac{t}\tau) - (1+A)\exp(-\frac{t}\tau)),
\end{equation}
where
\begin{equation}\label{eq:relax:A}
    A(\xi) = \frac{4}{D+2}\frac{q_i(0)c_i}{pT}\qty(\frac{c^2}{T} - \frac{D+2}2).
\end{equation}

\subsection{$H$ theorem}

%%% H function
For the Boltzmann equation, it is known that the so-called $H$ function defined as
\begin{equation}\label{eq:H}
    H[f](t) = \int f\ln{f}\dd\bxi
\end{equation}
never increases, i.e.,
\begin{equation}\label{eq:dHdt}
    \dv{H}{t} = \int \pdv{f}{t}\ln{f}\dd\bxi \leq 0,
\end{equation}
and the equality is achieved at $f = f_M$ only.
This statement is referred to as the Boltzmann $H$ theorem.

%%% H theorem for relaxation models
For the BGK model, which preserves positiveness of the distribution function, the proof of the $H$ theorem is straightforward:
\begin{equation}\label{eq:dHdt_BGK}
    \tau\dv{H}{t} = \int(f_M-f)\ln{f}\dd\bxi = \int (f_M-f)(\ln{f}-\ln{f_M})\dd\bxi \leq 0.
\end{equation}
It is a particular case of the Shakhov model for $\Pr=1$. When $f$ is nonnegative and close to $f_M$ in some sense,
it is possible to show that (see Appendix~\ref{sec:H_theorem})
\begin{equation}\label{eq:dHdt_Shakhov}
    \tau\dv{H}{t} \approx \int(f_S-f)(\ln{f}-\ln{f_S})\dd\bxi - \frac{\Pr(1-\Pr)}{D+2}\frac{4q^2}{pT^2},
\end{equation}
which effectively means that the $H$ theorem holds at least for $0\leq\Pr\leq1$ under these assumptions.
However, in the general case, the solution~\eqref{eq:relax:f} admits negative values of the distribution function,
which makes the definition~\eqref{eq:H} not applicable.
Moreover, it is possible to find a positive distribution function for which $\dv*{H}{t}>0$ (see Appendix~\ref{sec:violateH}).

%%% Method of regularization
Adding an arbitrary positive constant to the distribution function in Eq.~\eqref{eq:H} is a typical way
of regularizing such a singularity, but it makes derivation~\eqref{eq:dHdt_BGK} invalid.
Alternatively, all the negative values of the distribution function can be discarded by using the functional
\begin{equation}\label{eq:H_discarded}
    \hat{H}[f] = H[\max(f,0)].
\end{equation}
The main advantage of this primitive approach is that it does not alter the definition of the $H$ function
when $f\geq0$; however, it deteriorates the smootheness of the functional.
The influence of this crude modification can be mollified by taking a convolution with a Maxwellian
(also called the Weierstrass transform):
\begin{equation}\label{eq:H_blurred}
    \tilde{H}[f] = H[\max(g,0)], \quad g(\bxi) = \int f(\bzeta)f_M(\bxi-\bzeta)\dd\bzeta.
\end{equation}
An analogous convolution operation, known as the Husimi Q representation in quantum mechanics,
is widely used for constructing quasiprobability distributions in phase space~\cite{husimi1940some}.

\section{Numerical tests}

\newcommand{\addfigure}[2]{
\begin{figure}
    \vspace{-20pt}
    \centering
    \begin{subfigure}[b]{0.5\textwidth}
        \includegraphics[width=\textwidth]{_#1/vdf-0.0}
        \caption{$f(\xi)$ at $t=0$}
        \label{fig:#1:0}
    \end{subfigure}%
    \begin{subfigure}[b]{0.5\textwidth}
        \includegraphics[width=\textwidth]{_#1/vdf-1.0}
        \caption{$f(\xi)$ at $t=1$}
        \label{fig:#1:1}
    \end{subfigure}\\
    \begin{subfigure}[b]{0.5\textwidth}
        \includegraphics[width=\textwidth]{_#1/vdf-2.0}
        \caption{$f(\xi)$ at $t=2$}
        \label{fig:#1:2}
    \end{subfigure}%
    \begin{subfigure}[b]{0.5\textwidth}
        \includegraphics[width=\textwidth]{_#1/vdf-4.0}
        \caption{$f(\xi)$ at $t=4$}
        \label{fig:#1:4}
    \end{subfigure}
    \begin{subfigure}[b]{0.6\textwidth}
        \includegraphics[width=\textwidth]{_#1/solution}
        \caption{$\hat{H}(t)$ and $\tilde{H}(t)$ rescaled to the interval from $\hat{H}[f_M]$ to $\hat{H}[f(0)]$.}
        \label{fig:#1:H}
    \end{subfigure}
    \caption{
        Relaxation of the #2.
        The blue solid lines are the solution at various time moments.
        The orange dashed lines correspond to the Maxwellian with the same macroscopic variables.
        The green line shows the rescaled convolution of the $H$ function with $f_M$.
    }\label{fig:#1}
\end{figure}
}
\addfigure{piecewise}{piecewise-constant function~\eqref{eq:piecewise} with $k=2$, $\xi_i=0$, and $\xi_o=2$}
\addfigure{piecewise_hole}{two-beam distribution function~\eqref{eq:piecewise} with $k=2$, $\xi_i=1$, and $\xi_o=2$}
\addfigure{pmaxwell}{piecewise-Maxwell distribution function~\eqref{eq:pmaxwell} with $q_0=0.1$}
\addfigure{pmaxwell_Q2}{piecewise-Maxwell distribution function~\eqref{eq:pmaxwell} with $q_0=2$}
\addfigure{grad13}{Grad's distribution function~\eqref{eq:grad13} with $q_0=0.1$}
\addfigure{grad13_Q2}{Grad's distribution function~\eqref{eq:grad13} with $q_0=2$}
\addfigure{grad13_N4}{Grad's distribution function~\eqref{eq:grad13} with $q_0=0.1$ on a coarse mesh}

In this section, we simulate the evolution of a distribution function
for the 1-D space-homogeneous case with $\Pr = 2/3$.
We consider several cases for initial-value problem with $\rho=1$, $v=0$.
The relaxation of the piecewise-constant distribution function
\begin{equation}\label{eq:piecewise}
    f_1(\xi) = \begin{cases}
        A &\text{if } -\xi_o < \xi < -\xi_i, \\
        Ak &\text{if } \xi_i/\sqrt{k} < \xi < \xi_o/\sqrt{k}
    \end{cases}
\end{equation}
is shown in Figs.~\ref{fig:piecewise} and~\ref{fig:piecewise_hole}.
The relaxation of the piecewise Maxwellian
\begin{equation}\label{eq:pmaxwell}
    f_2(\xi) = \begin{cases}
        f_M[\rho_1, 0, T_1](\xi) &\text{if } \xi < 0, \\
        f_M[\rho_2, 0, T_2](\xi) &\text{if } \xi > 0
    \end{cases}
\end{equation}
is shown in Fig.~\ref{fig:pmaxwell} and~\ref{fig:pmaxwell_Q2}, where parameters
\begin{equation}\label{eq:pmaxwell_params}
    \rho_1 = 1-\Delta, \quad \rho_2 = 1+\Delta, \quad T_1 = \rho_2/\rho_1, \quad T_2 = \rho_1/\rho_2
\end{equation}
guarantee that $\rho=1$, $v=0$, $T=1$, and
\begin{equation}\label{eq:pmaxwell_q}
    q = -\frac2{\sqrt\pi}\frac{\Delta}{\sqrt{1-\Delta^2}}.
\end{equation}
Finally, the relaxation of a smooth distribution function with a non-zero heat flux
\begin{equation}\label{eq:grad13}
    f_3(\xi) = f_M[1, 0, T](\xi)\qty(1 + \frac{2q_0\xi}{T^2}\qty(\frac{2\xi^2}{(D+2)T} - 1))
\end{equation}
is shown in Fig.~\ref{fig:grad13} and~\ref{fig:grad13_Q2}.

%%% Discussion
All the results obtained illustrate that the $H$ function decays monotonically.
Moreover, in the absence of negative values, the relaxation is almost exponential.
An example of a positive distribution function,
which takes noticably negative values during its relaxation, is shown in Fig.~\ref{fig:pmaxwell_Q2}.
A significantly nonpositive distribution function is considered in Fig.~\ref{fig:grad13_Q2},
where the second derivative of $\hat{H}(t)$ is not positive everywhere unlike $\tilde{H}(t)$,
which behavior is similar to the cases with a positive distribution function.

%%% Coarse mesh
Fig.~\ref{fig:grad13_N4} demonstrates that a crude discretization of the velocity space can lead
to loss of monotonicity of the $\hat{H}$. This fact is explained by a significant contribution
of the discarded negative values of the distribution function. It is also seen that $\tilde{H}(t)$
still relaxes monotonically.

\appendix

\section{Proof of the $H$ theorem for the linearized case}\label{sec:H_theorem}

From the definition of $f_S$ and $H$, we have
\begin{equation}\label{eq:dHdt_S1}
    \begin{aligned}
    \tau\dv{H}{t} &= \int(f_S-f)(\ln{f}-\ln{f_S})\dd\bxi + \int(f_S-f)\ln{f_M}\dd\bxi \\
    &+ \int(f_S-f)\ln(1+\frac{1-\Pr}{D+2}\frac{4q_ic_i}{pT}\qty(\frac{c^2}{T} - \frac{D+2}2))\dd\bxi.
    \end{aligned}
\end{equation}
The first integral in Eq.~\eqref{eq:dHdt_S1} is nonpositive if we assume that $f\geq0$.
The second integral is equal to zero, since $\ln{f_M}$ is a summational invariant (span of $1$, $\bxi$, and $\xi^2$).
When $f$ is close to $f_M$, we can expand the third integrand (say, $I_3$) to the power series, i.e.,
\begin{equation}\label{eq:dHdt_S2}
    I_3 \approx \int(f_S-f)\frac{1-\Pr}{D+2}\frac{4q_ic_i}{pT}\qty(\frac{c^2}{T} - \frac{D+2}2)\dd\bxi
    = \frac{1-\Pr}{D+2}\frac{4q_i}{pT^2}\int(f_S-f)c_ic^2\dd\bxi = -\frac{\Pr(1-\Pr)}{D+2}\frac{4q^2}{pT^2}.
\end{equation}
By substituting Eq.~\eqref{eq:dHdt_S2} into Eq.~\eqref{eq:dHdt_S1}, we obtain Eq.~\eqref{eq:dHdt_Shakhov}.

\section{Evaluation of $\dv*{H}{t}$ for a piecewise Maxwellian}\label{sec:violateH}

For the one-dimensional piecewise Maxwellian~\eqref{eq:pmaxwell},
one can find the analytical value of the $H$ function and its time derivative:
\begin{gather}
    H(f_2) = \frac12\ln\frac{1-\Delta^2}{\pi} + \Delta\ln\frac{1+\Delta}{1-\Delta} - \frac12, \label{eq:H_2}\\
    \begin{aligned}
    \tau\dv{H(f_2)}{t} &= \overbrace{\int(f_M - f_2)\ln f_2 \dd\xi}^\text{BGK term}
    + \overbrace{(1-\Pr)\int A(\xi)f_M\ln f_2\dd\xi}^\text{Shakhov correction} \\
    &= - \frac{\Delta^2}{1-\Delta^2} - \Delta\ln\frac{1+\Delta}{1-\Delta}
    + \frac{1-\Pr}{3\pi}\frac{4\Delta}{\sqrt{1-\Delta^2}}\qty(\frac{2\Delta}{1-\Delta^2} + \ln\frac{1+\Delta}{1-\Delta}).
    \end{aligned}\label{eq:dHdt_2}
\end{gather}
The entropy production as a function of $\Delta$ is shown in Fig.~\ref{fig:dHdt_2}.
It is seen that the entropy production changes the sign when $\Delta$ is close to unity for any $0\leq\Pr<1$,
but this fact is elusive numerically, since positive values of $\dv*{H}{t}$ exist on an extremely small time interval,
after which the distribution function loses its positivity.

\begin{figure}
    \centering
    \footnotesize
    \begin{gnuplot}[scale=1, terminal=epslatex, terminaloptions=color lw 3]
        set sample 200
        set xzeroaxis dt 0
        set xrange [0:1]
        set yrange [-4:4]
        set key top center
        f(x, Pr) = -x**2/(1-x**2) - x*log((1+x)/(1-x)) + (1-Pr)*4./3/pi*x/sqrt(1-x**2)*(2*x/(1-x**2) + log((1+x)/(1-x)))
        do for [i=1:3] {
            set style line i linewidth 3
        }
        plot f(x, 0) title '$\Pr=0$', f(x, 1./3) title '$\Pr=1/3$', f(x, 1./2) title '$\Pr=1/2$', \
            f(x, 2./3) title '$\Pr=2/3$', f(x, 1) title '$\Pr=1$', f(x, 2) title '$\Pr=2$'
    \end{gnuplot}
    \caption{
        The dependence of $\dv*{H}{t}$ on $\Delta$ for the normalized piecewise Maxwellian $f_2$,
        defined by Eq.~\eqref{eq:pmaxwell}.
        The analytical expression~\eqref{eq:dHdt_2} for various Prandtl numbers is shown.
    }
    \label{fig:dHdt_2}
\end{figure}

\printbibliography

\end{document}
