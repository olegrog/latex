\documentclass{article}
\usepackage[english]{babel}
\usepackage{csquotes}

%%% Functional packages
\usepackage{amsmath, amssymb, amsthm}
\usepackage{physics}
\usepackage{subcaption}
\usepackage{graphicx}

%%% Configuration packages
\usepackage{fullpage}

%%% Problem-specific aliases
\newcommand{\Kn}{\mathrm{Kn}}
\newcommand{\Ma}{\mathrm{Ma}}
\newcommand{\equil}[1]{#1^\mathrm{(eq)}}
\newcommand{\bxi}{\boldsymbol{\xi}}

\usepackage[
    pdfusetitle,
    colorlinks
]{hyperref}

\usepackage[
    backend=biber,
    style=numeric,
    giveninits=true,
    sorting=none,
    mincrossrefs=100, % do not add parent collections
]{biblatex}
\addbibresource{shakhov_model.bib}

\title{Shakhov relaxation model for Boltzmann collisional operator}
\author{Oleg Rogozin}

\begin{document}
\maketitle
\tableofcontents

\section{Kinetic equation}

%%% Boltzmann--Shakhov equation
The Boltzmann kinetic equation have the following dimensionless form:
\begin{equation}\label{eq:Boltzmann}
    \pdv{f}{t} + \xi_i\pdv{f}{x_i} = \frac1kJ(f),
\end{equation}
where $f(t,x_i,\xi_i)$ is the velocity distribution function, $k=\Kn\sqrt\pi/2$,
and $\Kn$ is the Knudsen number.
The collision operator $J(f)$ can be simplified by the relaxation model
\begin{equation}\label{eq:model}
    J_S(f) = \frac{\rho T^{1-\omega}}{\gamma_1}\qty(\equil{f} - f),
\end{equation}
which reproduces the viscosity coefficient $\gamma_1T^\omega$.
\textcite{shakhov1968generalization} proposed the equilibrium function in the form
\begin{equation}\label{eq:Shakhov}
    \equil{f} = f_M\qty(1 + \frac{1-\Pr}{\pi^{D/2}}\frac{2q_ic_i}{pT}\qty(\frac{2c^2}{(D+2)T} - 1))
\end{equation}
to ensure the given Prandtl number $\Pr$. Here
\begin{equation}\label{eq:Maxwell}
    f_M = \frac{\rho}{(\pi T)^{D/2}}\exp(-\frac{c^2}T)
\end{equation}
is the Maxwell distribution, $D$ is the dimensionality of the velocity space .
For brevity, we use \(c_i = \xi_i - v_i\) and \(c = |\boldsymbol{c}|\).

%%% Macroscopic variables
The macroscopic variables are calculated as follows:
\begin{equation}\label{eq:macro}
    \rho = \int f \dd\bxi, \quad
    \rho v_i = \int \xi_i f \dd\bxi, \quad
    p = \rho T = \frac2D\int c^2 f \dd\bxi, \quad
    q_i = \int c_i c^2 f \dd\bxi,
\end{equation}
where \(\rho\) is the density, \(v_i\) is the velocity, \(T\) is the temperature, \(p\) is the pressure,
and \(q_i\) is the heat-flux vector.

\section{Space-homogeneous relaxation problem}

%%% Space-homogeneous solution
In the space-homogeneous case, we have
\begin{equation}\label{eq:relax}
    \pdv{f}{t} = \frac1\tau\qty(\equil{f} - f),
\end{equation}
where the relaxation time $\tau$ is time-independent.
This equation can be solved analytically (see, e.g., \cite{titarev2004numerical}).
The heat flux relaxes uniformly:
\begin{equation}\label{eq:relax:q}
    q(t) = q(0)\exp(-\frac1D\frac{t}\tau).
\end{equation}
In contrast, the distribution function evolves non-monotonically:
\begin{equation}\label{eq:relax:f}
    f(t) = f(0)\exp(-\frac{t}\tau) + f_M\qty(1 - \exp(-\frac{t}\tau) + DA(t)\qty(1-\exp(-\frac1D\frac{t}\tau))),
\end{equation}
where
\begin{equation}\label{eq:relax:A}
    A(t) = \frac{1-\Pr}{\pi^{D/2}}\frac{2q(t)}{\rho T^2}(\xi - u)\qty(\frac{2c^2}{(D+2)T} - 1)
\end{equation}
for a radially symmetric distribution function $f(t,\xi_i) = f(t,\xi)$.

%%% H-function
It is known that so called $H$-function defined as
\begin{equation}\label{eq:relax:H}
    H(t) = \int f\ln{f}\dd\xi
\end{equation}
never increases, i.e., $\dv*{H}{t}\leq0$, for the Boltzmann equation.
In the general case, this is not true for the Shakhov relaxation model.
Nevertheless, it can be shown that
\begin{equation}\label{eq:relax:dHdt}
    \dv{H}{t} = -\frac{\Pr(1-\Pr)}{\tau(D+2)}\frac{4q^2}{pT^2} + \mathcal{O}(q^4).
\end{equation}

\section{Numerical tests}

\newcommand{\addfigure}[2]{
\begin{figure}
    \vspace{-20pt}
    \centering
    \begin{subfigure}[b]{0.5\textwidth}
        \includegraphics[width=\textwidth]{_#1/vdf-0.0}
        \caption{$f(\xi)$ at $t=0$}
        \label{fig:#1:0}
    \end{subfigure}%
    \begin{subfigure}[b]{0.5\textwidth}
        \includegraphics[width=\textwidth]{_#1/vdf-1.0}
        \caption{$f(\xi)$ at $t=1$}
        \label{fig:#1:1}
    \end{subfigure}\\
    \begin{subfigure}[b]{0.5\textwidth}
        \includegraphics[width=\textwidth]{_#1/vdf-2.0}
        \caption{$f(\xi)$ at $t=2$}
        \label{fig:#1:2}
    \end{subfigure}%
    \begin{subfigure}[b]{0.5\textwidth}
        \includegraphics[width=\textwidth]{_#1/vdf-4.0}
        \caption{$f(\xi)$ at $t=4$}
        \label{fig:#1:4}
    \end{subfigure}
    \begin{subfigure}[b]{0.6\textwidth}
        \includegraphics[width=\textwidth]{_#1/solution}
        \caption{$H$-function}
        \label{fig:#1:H}
    \end{subfigure}
    \caption{
        Relaxation of #2.
        The blue solid lines with markers are the solution at various time moments.
        The orange dashed lines correspond to the Maxwellian with the same macroscopic variables.
    }\label{fig:#1}
\end{figure}
}
\addfigure{piecewise}{a piecewise-constant function~\eqref{eq:piecewise} with $k=2$, $\xi_i=0$, and $\xi_o=2$}
\addfigure{piecewise_hole}{two-beam distribution function~\eqref{eq:piecewise} with $k=2$, $\xi_i=1$, and $\xi_o=2$}
\addfigure{bimaxwell}{bi-maxwell distribution function~\eqref{eq:bimaxwell} with $T=1$ and $q_0=0.1$}
\addfigure{grad13}{a Grad's distribution function~\eqref{eq:grad13} with $T=1$ and $q_0=0.1$}

Here, we simulate evolution of a distribution function for the 1-D space-homogeneous case with $\Pr = 2/3$.
We consider several cases for initial-value problem with $\rho=1$ and $u=0$.
Relaxation of a piecewise-constant VDF
\begin{equation}\label{eq:piecewise}
    f(\xi) = \begin{cases}
        A &\text{if } -\xi_o < \xi < -\xi_i, \\
        Ak &\text{if } \xi_i/\sqrt{k} < \xi < \xi_o/\sqrt{k}
    \end{cases}
\end{equation}
is shown in Figs.~\ref{fig:piecewise} and~\ref{fig:piecewise_hole}.
Relaxation of a bi-Maxwellian
\begin{equation}\label{eq:bimaxwell}
    f(\xi) = \begin{cases}
        f_M[\rho_1, 0, T_1](\xi) &\text{if } \xi < 0, \\
        f_M[\rho_2, 0, T_2](\xi) &\text{if } \xi > 0
    \end{cases}
\end{equation}
is shown in Fig.~\ref{fig:bimaxwell}.
Finally, relaxation of a smooth distribution function with a non-zero heat flux
\begin{equation}\label{eq:grad13}
    f(\xi) = f_M[1, 0, T](\xi)\qty(1 + \frac{2q_0\xi}{T^2}\qty(\frac{2\xi^2}{(D+2)T} - 1))
\end{equation}
is shown in Fig.~\ref{fig:grad13}.


\printbibliography

\end{document}
