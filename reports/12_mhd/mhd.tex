\documentclass{article}
\usepackage[utf8]{inputenc}
\usepackage[T2A, T1]{fontenc}
\usepackage[russian, english]{babel}
\usepackage{csquotes}

%%% functional packages
\usepackage{amssymb, mathtools, amsthm, mathrsfs}
\usepackage{graphicx}
\usepackage{physics, siunitx}
\usepackage[subfolder]{gnuplottex}

%%% configuration packages
\usepackage{fullpage}

\usepackage[
    pdfusetitle,
    colorlinks
]{hyperref}

\usepackage[
    backend=biber,
    style=numeric,
    autolang=other,
    sorting=none,
    giveninits,
    mincrossrefs=100, % do not add parent collections
]{biblatex}
\bibliography{mhd}

\title{Magnetohydrodynamics in selective laser melting}
\author{Oleg Rogozin}

%%% Special symbols
\newcommand{\tran}{\mathsf{T}}

%%% Bold symbols
\newcommand{\bv}{\vb{v}}
\newcommand{\bn}{\vu{n}}
\newcommand{\bB}{\vb{B}}
\newcommand{\bE}{\vb{E}}

\begin{document}
\maketitle
\tableofcontents

\section{Mathematical model}

%%% Governing equations
The behavior of the incompressible viscous electrically conducting fluid in the presence of constant magnetic field is described by the following equations (see Eqs.~(66.7)--(66.9) in~\cite{Landavshits8})\footnote{
    To convert the Gaussian units to the SI ones, use \href{https://en.wikipedia.org/wiki/Gaussian_units\#Electromagnetic_unit_names}{formulas} $\bB_\text{G} = (4\pi/\mu_0)^{1/2}\bB_\text{SI}$, $\sigma_\text{G} = \sigma_\text{SI}/(4\pi\epsilon_0)$, and $c = (\epsilon_0\mu_0)^{-1/2}$.
}:
\begin{gather}
    \div\bB = 0, \quad \div\bv = 0, \label{eq:continuity}\\
    \pdv{\bB}{t} + (\bv\vdot\grad)\bB = (\bB\vdot\grad)\bv + \frac1{\mu_0\sigma}\laplacian\bB, \label{eq:B}\\
    \rho\pdv{\bv}{t} + \rho(\bv\vdot\grad)\bv = -\grad(p+\frac{B^2}{2\mu_0})
        + \frac1{\mu_0}(\bB\vdot\grad)\bB + \eta\laplacian\bv, \label{eq:momentum}
\end{gather}
where $\bB$ is the magnetic field (\si{\tesla} = \si{J/A.m^2}),
$\bv$ is the velocity (\si{m/s}),
$\sigma$ is the electrical conductivity (\si{\ohm^{-1}.m^{-1} = A^2.s/m.J}),
$\rho$ is the density (\si{kg/m^3}), $p$ is the pressure (\si{Pa = J/m^3}),
$\eta$ is the dynamic viscosity (\si{kg/m.s}),
$\mu_0 = \SI{1.2576e-6}{J/A^2.m}$ is the vacuum permeability.

%%% Boundary conditions
The following boundary conditions are valid at a gas--liquid interface (see Eq.~(61.14) in~\cite{Landavshits6}):
\begin{equation}\label{eq:bc1}
    \qty(p_g - p_l + \gamma\div\bn)\bn = \qty(\vb*\tau_g - \vb*\tau_l)\vdot\bn + \grad\gamma,
\end{equation}
where subscripts $g$ and $l$ correspond to the gas and liquid quantities, respectively,
$\gamma$ is the surface tension (\si{J/m^2}), $\bn$ is the unit normal directed into the gas,
and
\begin{equation}\label{eq:tau}
    \vb*\tau = \eta\qty(\grad\bv + (\grad\bv)^\tran)
\end{equation}
is the viscous stress tensor.
In case of temperature-dependent surface tension $\gamma=\gamma(T)$, $\grad\gamma = \gamma'\grad{T}$.
Additionally, conditions
\begin{equation}\label{eq:bc2}
    \bv_l = \bv_g, \quad \bB_l = \bB_g
\end{equation}
are imposed in the case $\eta>0$ and $\sigma<\infty$ to ensure that there is no jumps of $\bv$ and $\bB$ across the interface.

\section{Uni-directional flow}

\begin{figure}
    \centering
    \begin{gnuplot}[scale=0.8, terminal=epslatex, terminaloptions=color lw 2]
        set xrange [-1e-5:1e1]
        set yrange [0:1]
        set grid
        plot 2*(cosh(x)-1)/x/sinh(x) title '$f(x)$' lw 3, \
        1-x**2/12 title '$1-x^2/12$' dt 4, 2/x title '$2/x$' dt 4
    \end{gnuplot}
    \caption{Function $f(x) = 2(\cosh{x}-1)/(x\sinh{x})$.}
    \label{fig:f}
\end{figure}

%%% General solution
Let us consider a uni-directional flow along the $x$ axis between two infinite parallel surfaces:
the upper one ($z=0$) is a gas--liquid interface with uniform temperature gradient $T_x$,
and the fluid is resting at the lower one ($z=-\lambda$).
Moreover, a constant magnetic field $\mathcal{B}$ along the $z$ axis is applied to the system.
Therefore, under assumptions $\dv*{t}=0$, $\dv*{x}=0$, $\dv*{y}=0$, $v_y=0$, $v_z=0$, and $B_y=0$ the set of equations~\eqref{eq:continuity}--\eqref{eq:momentum} reduces to
\begin{equation}\label{eq:governing}
    B_z' = 0, \quad \mu_0\sigma B_z v_x' = -B_x'', \quad \mu_0\eta v_x'' = -B_z B_x',
\end{equation}
where prime denotes the first derivative with respect to $z$.
The general solution of~\eqref{eq:governing} can be written as
\begin{equation}\label{eq:general_solution}
    v_x = \frac1\omega(C_1\cosh\omega z + C_2\sinh\omega z + C_3), \quad
    B_x = \frac{\mu_0\eta}{B_z}(C_1\sinh\omega z + C_2\cosh\omega z + C_4),
\end{equation}
where $\omega^2 = \sigma B_z^2/\eta$ and $C_i$ are some constants.

%%% Boundary conditions
The boundary conditions~\eqref{eq:bc1} and~\eqref{eq:bc2} take the form
\begin{equation}\label{eq:bc}
    \eta v_x' = \gamma' T_x, \quad B_z = \mathcal{B}, \quad B_x = 0
\end{equation}
at $z=0$.
Then, we immediately obtain that $B_z=\mathcal{B}$ in the whole domain $-\lambda<z<0$.
However, the remaining condition $v=0$ at $z=-\lambda$ is insufficient to find the unique solution of~\eqref{eq:governing}.
If we supplement the boundary conditions with $B_x = 0$ at $z=-\lambda$, then we have
\begin{equation}\label{eq:constants}
    C_1 = C_2\frac{\cosh\omega\lambda-1}{\sinh\omega\lambda}, \quad
    C_2 = \frac{\gamma'T_x}{\eta}, \quad
    C_3 = C_1, \quad C_4 = C_2.
\end{equation}
Thus, the velocity at $z=0$ is
\begin{equation}\label{eq:velocity}
    v_x = \frac{C_1+C_3}{\omega}
        = \frac{2\gamma'T_x}{\eta\omega}\frac{\cosh\omega\lambda-1}{\sinh\omega\lambda}
        = \frac{\gamma'T_x\lambda}{\eta} f(\omega\lambda).
\end{equation}
Function $f(x)$ is shown in Fig.~\ref{fig:f} and has asymptotics
\begin{equation}\label{eq:f_asym}
    f(x) = 1 - \frac{x^2}{12} + \order{x^4} \text{ as } x\to0, \quad
    f(x) \sim \frac2x \text{ as } x\to\infty.
\end{equation}
Since $f(x)$ is a decreasing function,
we conclude that the external static magnetic field, regardless of its strength, has a damping effect on the velocity field.
This fact is obtained from the qualitative analysis in~\cite{du2019influence}.

\appendix

\section{Index notation}

Eqs.~\eqref{eq:continuity}--\eqref{eq:momentum} can be written in the index notation:
\begin{gather}
    \pdv{v_i}{x_i} = 0, \quad \pdv{B_i}{x_i} = 0, \label{eq:continuityI}\\
    \pdv{B_i}{t} + v_j\pdv{B_i}{x_j} = B_j\pdv{v_i}{x_j} + \frac1{\mu_0\sigma} \pdv[2]{B_i}{x_j}{x_j}, \label{eq:BI}\\
    \rho\pdv{v_i}{t} + \rho v_j\pdv{v_i}{x_j}
    = -\pdv{p}{x_i} - \frac{B_j}{\mu_0}\pdv{B_j}{x_i} + \frac{B_j}{\mu_0}\pdv{B_i}{x_j} + \eta\pdv[2]{v_i}{x_j}{x_j}. \label{eq:momentumI}
\end{gather}
The boundary condition~\eqref{eq:bc1} with~\eqref{eq:tau} has the form
\begin{equation}\label{eq:bc1I}
    \qty(p_g - p_l + \gamma\pdv{n_j}{x_j})n_i
    = \qty[
        \eta_g\qty( \pdv{v_{gi}}{x_j} + \pdv{v_{gj}}{x_i} )
      - \eta_l\qty( \pdv{v_{li}}{x_j} + \pdv{v_{lj}}{x_i} )
    ]n_j + \pdv{\gamma}{x_i}.
\end{equation}

\printbibliography

\end{document}
