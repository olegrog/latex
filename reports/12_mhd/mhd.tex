\documentclass{article}
\usepackage[utf8]{inputenc}
\usepackage[T2A, T1]{fontenc}
\usepackage[russian, english]{babel}
\usepackage{csquotes}

%%% functional packages
\usepackage{amssymb, mathtools, amsthm, mathrsfs}
\usepackage{graphicx}
\usepackage{physics, siunitx}
\usepackage{multirow}
\usepackage[subfolder]{gnuplottex}
\usepackage[referable]{threeparttablex} % for \tnotex

%%% configuration packages
\usepackage{fullpage}

\usepackage[
    pdfusetitle,
    colorlinks
]{hyperref}

\usepackage[
    backend=biber,
    style=numeric,
    autolang=other,
    sorting=none,
    giveninits,
    mincrossrefs=100, % do not add parent collections
]{biblatex}
\bibliography{mhd}

\title{Magnetohydrodynamics in selective laser melting}
\author{Oleg Rogozin}

%%% Special symbols
\newcommand{\tran}{\mathsf{T}}
\DeclareSIUnit{\wtpercent}{wt\%}
\newcommand{\Ha}{\mathrm{Ha}}

%%% Bold symbols
\newcommand{\bv}{\vb{v}}
\newcommand{\bn}{\vu{n}}
\newcommand{\bB}{\vb{B}}
\newcommand{\bE}{\vb{E}}

\begin{document}
\maketitle
\tableofcontents

\section{Mathematical model}

%%% Governing equations
The behavior of the incompressible viscous electrically conducting fluid
in the presence of constant magnetic field is described by the following equations
(see Eqs.~(66.7)--(66.9) in~\cite{Landavshits8})\footnote{
    To convert the Gaussian units to the SI ones, use
    \href{https://en.wikipedia.org/wiki/Gaussian_units\#Electromagnetic_unit_names}{formulas}
    $\bB_\text{G} = (4\pi/\mu_0)^{1/2}\bB_\text{SI}$,
    $\sigma_\text{G} = \sigma_\text{SI}/(4\pi\epsilon_0)$,
    and $c = (\epsilon_0\mu_0)^{-1/2}$.
}:
\begin{gather}
    \div\bB = 0, \quad \div\bv = 0, \label{eq:continuity}\\
    \pdv{\bB}{t} + (\bv\vdot\grad)\bB = (\bB\vdot\grad)\bv + \frac1{\mu_0\sigma}\laplacian\bB, \label{eq:B}\\
    \rho\pdv{\bv}{t} + \rho(\bv\vdot\grad)\bv = -\grad(p+\frac{B^2}{2\mu_0})
        + \frac1{\mu_0}(\bB\vdot\grad)\bB + \eta\laplacian\bv, \label{eq:momentum}
\end{gather}
where $\bB$ is the magnetic field (\si{\tesla} = \si{J/A.m^2}),
$\bv$ is the velocity (\si{m/s}),
$\sigma$ is the electrical conductivity (\si{\ohm^{-1}.m^{-1} = A^2.s/J.m}),
$\rho$ is the density (\si{kg/m^3}), $p$ is the pressure (\si{Pa = J/m^3}),
$\eta$ is the dynamic viscosity (\si{Pa.s = J.s/m^3}),
$\mu_0 = \SI{1.2576e-6}{J/A^2.m}$ is the vacuum permeability.

%%% Boundary conditions
The following boundary conditions are valid at a gas--liquid interface (see Eq.~(61.14) in~\cite{Landavshits6}):
\begin{equation}\label{eq:bc1}
    \qty(p_g - p_l + \gamma\div\bn)\bn = \qty(\vb*\tau_g - \vb*\tau_l)\vdot\bn + \grad\gamma,
\end{equation}
where subscripts $g$ and $l$ correspond to the gas and liquid quantities, respectively,
$\gamma$ is the surface tension (\si{J/m^2}), $\bn$ is the unit normal directed into the gas,
and
\begin{equation}\label{eq:tau}
    \vb*\tau = \eta\qty(\grad\bv + (\grad\bv)^\tran)
\end{equation}
is the viscous stress tensor.
In case of temperature-dependent surface tension $\gamma=\gamma(T)$, $\grad\gamma = \gamma'\grad{T}$.
Additionally, conditions
\begin{equation}\label{eq:bc2}
    \bv_l = \bv_g, \quad \bB_l = \bB_g
\end{equation}
are imposed in the case $\eta>0$ and $\sigma<\infty$ to ensure that
there is no jumps of $\bv$ and $\bB$ across the interface.

\section{Uni-directional flow}

\begin{figure}
    \centering
    \begin{gnuplot}[scale=0.8, terminal=epslatex, terminaloptions=color lw 4]
        set xrange [-1:0]
        set key top center
        set grid
        xi=2
        plot (1+cosh(2*xi*x))/2/xi*(tanh(xi) + tanh(xi*x)) \
                title '$\hat{v}_x$ for $\xi=2$', \
            1+x title '$\hat{v}_x$ for $\xi=0$', \
            sinh(-2*xi*x)/2/xi*(tanh(xi) + tanh(xi*x)) \
                title '$\hat{B}_x$ for $\xi=2$', \
            -x*(1+x) title '$\hat{B}_x/\xi$ for $\xi=0$'
    \end{gnuplot}
    \caption{
        Dimensionless quantities $\hat{v}_x$ and $\hat{B}_x$,
        given by~\eqref{eq:hatv_x} and~\eqref{eq:hatB_x},
        as functions of $\hat{z}$.
        For small $\xi$, $\hat{v}_x = 1+\hat{z}$
        and $\hat{B}_x = -\xi\hat{z}(1+\hat{z})$.
    }
    \label{fig:solution}
\end{figure}

\begin{figure}
    \centering
    \begin{gnuplot}[scale=0.8, terminal=epslatex, terminaloptions=color lw 2]
        set xrange [1e-1:1e2]
        set log x
        set yrange [0:1]
        set grid
        plot tanh(x)/x title '$\tanh(\xi)/\xi$' lw 3, \
            1-x**2/3 title '$1-\xi^2/3$' dt 4, 1/x title '$1/\xi$' dt 4
    \end{gnuplot}
    \caption{
        Dimensionless velocity $\hat{v}_x(0) = \tanh(\xi)/\xi$ and its asymptotics:
        $\hat{v}_x(0) = 1 - \xi^2/3 + \order{\xi^4} \text{ as } \xi\to0$
        and $\hat{v}_x(0) \sim 1/\xi \text{ as } \xi\to\infty$.
    }\label{fig:v0}
\end{figure}

%%% General solution
Let us consider a uni-directional flow along the $x$ axis between two infinite parallel surfaces:
the upper one ($z=0$) is a gas--liquid interface with uniform temperature gradient $T_x$,
and the fluid is resting at the lower one ($z=-\lambda$).
Moreover, a constant magnetic field $\mathcal{B}$ along the $z$ axis is applied to the system.
Therefore, under assumptions $\dv*{t}=0$, $\dv*{x}=0$, $\dv*{y}=0$, $v_y=0$, $v_z=0$,
and $B_y=0$ the set of equations~\eqref{eq:continuity}--\eqref{eq:momentum} reduces to
\begin{equation}\label{eq:governing}
    B_z' = 0, \quad \mu_0\sigma B_z v_x' = -B_x'', \quad \mu_0\eta v_x'' = -B_z B_x',
\end{equation}
where prime denotes the derivative with respect to $z$.
The general solution of Eqs.~\eqref{eq:governing} can be written as
\begin{equation}\label{eq:general_solution}
    v_x = \frac1\omega(C_1\cosh(\omega z) + C_2\sinh(\omega z) + C_3), \quad
    B_x = -\frac{\mu_0\eta}{B_z}(C_1\sinh(\omega z) + C_2\cosh(\omega z) + C_4),
\end{equation}
where $\omega^2 = \sigma B_z^2/\eta$ and $C_i$ are some constants.

%%% Boundary conditions, particular solution, nondimensionalization
The boundary conditions~\eqref{eq:bc1} and~\eqref{eq:bc2} take the form
\begin{equation}\label{eq:bc}
    \eta v_x' = \gamma'T_x, \quad B_z = \mathcal{B}, \quad B_x = 0
\end{equation}
at $z=0$.
Then, we immediately obtain that $B_z=\mathcal{B}$ in the whole domain $-\lambda<z<0$.
With the remaining conditions $v_x=0$ and $B_x=0$ at $z=-\lambda$,
the constants are given by
\begin{equation}\label{eq:constants}
    C_1 = C_2\tanh(\frac{\omega\lambda}2), \quad
    C_2 = \frac{\gamma'T_x}{\eta}, \quad
    C_3 = C_1, \quad C_4 = -C_2.
\end{equation}
Thus, the general solution~\eqref{eq:general_solution} takes the form
\begin{gather}
    v_x = \frac{\gamma'T_x}{\eta\omega}(1+\cosh(\omega z))
        \qty(\tanh(\frac{\omega\lambda}2) + \tanh(\frac{\omega z}2)), \label{eq:v_x}\\
    B_x = \frac{\mu_0\gamma'T_x}{\mathcal{B}}\sinh(-\omega z)
        \qty(\tanh(\frac{\omega\lambda}2) + \tanh(\frac{\omega z}2)), \label{eq:B_x}
\end{gather}
which can be nondimensionalized as
\begin{gather}
    \hat{v}_x = \frac{1+\cosh(2\xi\hat{z})}{2\xi}
        \qty( \tanh(\xi) + \tanh(\xi\hat{z}) ), \label{eq:hatv_x}\\
    \hat{B}_x = \frac{\sinh(-2\xi\hat{z})}{2\xi}
        \qty( \tanh(\xi) + \tanh(\xi\hat{z}) ) \label{eq:hatB_x}
\end{gather}
according to the following relations:
\begin{equation}\label{eq:nondimensioned}
    v_x = U \hat{v}_x, \quad B_x = B_0 \hat{B}_x, \quad
    z = \lambda \hat{z}, \quad \xi = \omega\lambda/2,
\end{equation}
where $U$ and $B_0$ are the characteristic velocity and induced magnetic field,
respectively, given by
\begin{equation}\label{eq:characteristic}
    U = \frac{\gamma'T_x\lambda}{\eta}, \quad
    B_0 = \mu_0U\sqrt{\sigma\eta}.
\end{equation}

%%% Analysis of the solution
The obtained solution is illustrated in Fig.~\ref{fig:solution}.
Velocity $\hat{v}_x(\hat{z})$ reaches its maximum at $\hat{z}=0$.
Since the dependence of $\hat{v}_x(\hat{z})$ on $\xi$ (shown in Fig.~\ref{fig:v0})
is a monotonically decreasing function, we conclude that the external static magnetic field,
regardless of its strength, has a damping effect on the velocity field.
This fact is also mentioned in~\cite{du2019influence}.

\section{Dimensional analysis}

\begin{table}
    \centering
    \begin{threeparttable}[b]
    \caption{Physical properties of stainless steel 316L.}
    \label{table:properties}
    \footnotesize
    \begin{tabular}{lcccc}
        \hline\noalign{\smallskip}
        Physical property & Symbol & Value & Unit & Reference \\[3pt] \hline\noalign{\smallskip}
        Density & $\rho$ & \num{7500} & \si{\kg\per\cubic\m} & \cite{kim1975thermophysical}\tnotex{a} \\[3pt]
        \noalign{\smallskip}
        Viscosity & $\eta$ & $\num{2.54e-4}\exp(\SI{5490}{K}/T)$ & \si{\Pa\s} & \cite{kim1975thermophysical} \\[3pt]
        \noalign{\smallskip}
        Melting temperature & $T_M$ & \num{1700} & \multirow{2}*{\si{\K}} & \cite{kim1975thermophysical} \\
        Boiling temperature & $T_B$ & \num{3090} & & \cite{kim1975thermophysical} \\[3pt]
        \noalign{\smallskip}
        Surface tension & $\gamma$ & $\num{1.85} - \SI{8.9e-5}{\per\K}T$ & \si{\Pa\m} & \cite{schmidt2006surface} \\[3pt]
        \noalign{\smallskip}
        Electrical conductivity & $\sigma$ & $\num{6.58e5}$ & \si{\per\ohm\per\m} & \cite{chu1978electrical}\tnotex{a} \\[3pt]
        \hline
    \end{tabular}
    \begin{tablenotes}
        \item[a]\label{a} The values are taken at the melting temperature.
    \end{tablenotes}
    \end{threeparttable}
\end{table}

The material properties are tabulated in Table~\ref{table:properties}.
Note that viscosity of the liquid metal has a strong dependence on temperature.
Specifically, $\eta_L(T_M) = \SI{6.4}{mPa.s}$ and $\eta_L(T_B) = \SI{1.5}{mPa.s}$.
The temperature gradient can be estimated as $T_x=(T_B-T_M)/\lambda$.
The characteristic viscosity, velocity, and length in the melt pool are estimated as
\begin{equation}
    \eta = \eta_L\qty(\frac{T_M+T_B}{2}) = \SI{2.5}{mPa.s}, \quad
    U = \frac{\gamma'(T_B - T_M)}{\eta} = \SI{49}{m/s}, \quad
    \lambda = \SI{100}{\um}.
\end{equation}
The Reynolds and Hartmann numbers are given as
\begin{equation}
    \Re = \frac{\rho U L}{\eta} = \frac{\rho \gamma'(T_B - T_M) L}{\eta^2}, \quad
    \Ha = B\lambda\qty(\frac{\sigma}{\eta})^{1/2} = \omega\lambda = 2\xi.
\end{equation}
For the liquid metal and the ambient gas, we have
\begin{equation}
    \Re_\text{liquid}(L=\lambda) = 15000, \quad
    \Re_\text{gas}(L=10\lambda) = 180.
\end{equation}
For $\Ha = 4$, which corresponds to $B = \SI{3.5}{T}$,
the velocity $U$ reduces by approximately 2 times, which is seen in~Fig.~\ref{fig:v0}.

\appendix

\section{Index notation}

Eqs.~\eqref{eq:continuity}--\eqref{eq:momentum} can be written in the index notation:
\begin{gather}
    \pdv{v_i}{x_i} = 0, \quad \pdv{B_i}{x_i} = 0, \label{eq:continuityI}\\
    \pdv{B_i}{t} + v_j\pdv{B_i}{x_j} = B_j\pdv{v_i}{x_j} + \frac1{\mu_0\sigma} \pdv[2]{B_i}{x_j}{x_j}, \label{eq:BI}\\
    \rho\pdv{v_i}{t} + \rho v_j\pdv{v_i}{x_j}
    = -\pdv{p}{x_i} - \frac{B_j}{\mu_0}\pdv{B_j}{x_i} + \frac{B_j}{\mu_0}\pdv{B_i}{x_j} + \eta\pdv[2]{v_i}{x_j}{x_j}. \label{eq:momentumI}
\end{gather}
The boundary condition~\eqref{eq:bc1} with~\eqref{eq:tau} has the form
\begin{equation}\label{eq:bc1I}
    \qty(p_g - p_l + \gamma\pdv{n_j}{x_j})n_i
    = \qty[
        \eta_g\qty( \pdv{v_{gi}}{x_j} + \pdv{v_{gj}}{x_i} )
      - \eta_l\qty( \pdv{v_{li}}{x_j} + \pdv{v_{lj}}{x_i} )
    ]n_j + \pdv{\gamma}{x_i}.
\end{equation}

\printbibliography

\end{document}
