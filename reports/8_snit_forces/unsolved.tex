%&pdflatex
\documentclass{article}
\usepackage[utf8]{inputenc}
\usepackage[T2A]{fontenc}
\usepackage[russian]{babel}

\usepackage{amssymb, amsmath, amsthm}
\usepackage{fullpage}
\usepackage{indentfirst}

\usepackage[
    pdfauthor={Rogozin Oleg},
    pdftitle={Problem},
    colorlinks, pdftex, unicode
]{hyperref}

\newtheorem{proposition}{Proposition}
\newtheorem{remark}{Remark}

\title{Задача}
\author{Олег Рогозин}

\newcommand{\bx}{\boldsymbol{x}}
\newcommand{\by}{\boldsymbol{y}}

\begin{document}

\section{Задача}

Рассмотрим некоторую область \(\Omega\subset\mathbb{R}^3\), ограниченную поверхностью \(S\),
состоящей из \emph{замкнутых} поверхностей:
\begin{equation}\label{eq:domain_boundary}
    S \equiv \partial\Omega = \bigcup_a S_a, \quad \partial{S_a} = 0.
\end{equation}
Определим внутри \(\Omega\) гармоническую функцию \(\psi\) на основе единственного решения задачи Дирихле:
\begin{equation}\label{eq:psi_Dirichlet}
    \psi_{ii} = 0, \quad \psi|_{S_a} = \psi_a,
\end{equation}
где \(\psi_a\) "--- константы, \(\psi_i\equiv\partial_i\psi\), \(\psi_{ii}\) "--- лаплассиан.

Внутри \(\Omega\) определим также фукнцию \(p\) на основе решения задачи Неймана:
\begin{equation}\label{eq:p_Neumann}
    \partial_{ii}\left( p + \frac{g}2\psi_j^2 \right) = \partial_{ij}\left( g\psi_i\psi_j \right), \quad
    \left.\partial_i\left( p + \frac{g}2\psi_j^2 \right)n_i\right|_S =
    \left.\partial_j\left( g\psi_i\psi_j \right)n_i\right|_S,
\end{equation}
которое упрощается до
\begin{equation}\label{eq:p_Neumann2}
    p_{ii} = \partial_i\left( f_\psi \psi_i \right), \quad
    p_i n_i|_S = \left. f_\psi \psi_i n_i \right|_S,
\end{equation}
где \(g\) "--- заданная произвольная функция от \(\psi\), \(g_\psi\) "--- её производная, \(f=g\psi_j^2/2\).
Решение задачи Неймана определяется с точностью до константы, поэтому для определённости можно положить
\begin{equation}\label{eq:p_over_domain}
    \int_\Omega p = 0.
\end{equation}

Необходимо доказать
\begin{proposition}\label{prop:main}
\begin{equation}\label{eq:forces_sum}
    \int_S\left( p + \frac{g}2\psi_j^2\right)n_i - \int_S\left( g\psi_i\psi_j \right)n_j \equiv
    \int_S\left( p - f \right)n_i = 0.
\end{equation}
\end{proposition}

\begin{remark}
    Утверждение~\ref{prop:main} проверено численно для области, заключённой между двумя некоаксиальными цилиндрами,
    и функций \(g=\psi^s\).
\end{remark}

\section{Известные утверждения}

\begin{proposition}\label{prop:integral_mixed}
Для произвольной функции \(h(\psi)\)
\begin{equation}\label{eq:integral_mixed}
    \int_\Omega h \psi_{ij} \psi_j = 0.
\end{equation}
\end{proposition}

\begin{proof}
Запишем два дифференциальных тождества:
\begin{gather}
    \partial_i\left( h\psi_j^2 \right) = h_\psi\psi_i\psi_j^2 + 2h\psi_{ij}\psi_j, \label{eq:prop1-1}\\
    \partial_j\left( h\psi_i\psi_j \right) = h_\psi\psi_i\psi_j^2 + h\psi_{ij}\psi_j + h\psi_i\psi_{jj}. \label{eq:prop1-2}
\end{gather}
Внутри \(\Omega\) последний член в~\eqref{eq:prop1-2} равен нулю.
Интегралы от~\eqref{eq:prop1-1} и~\eqref{eq:prop1-2} по \(\Omega\) равны друг другу,
поскольку все поверхности \(S_a\) эквипотенциальны, а значит \(\psi_i|_S = \psi_k n_k n_i\).
\end{proof}

В частности,~\eqref{eq:forces_sum} можно записать как
\begin{equation}\label{eq:forces_sum2}
    \int_S\left( p - f \right)n_i = \int_\Omega \left( p_i - f_\psi\psi_i \right) = 0.
\end{equation}

\begin{remark}
    \[ p_i \neq f_\psi\psi_i. \]
\end{remark}

\section{Функция Грина}

Определим функцию Грина \(G(\bx, \bx')\) для задачи Неймана~\eqref{eq:p_Neumann2}:
\begin{equation}\label{eq:Green_function}
    G_{ii} = \delta(\bx-\bx') - \frac1V, \quad
    \left. G_k n_k \right|_S = 0,
\end{equation}
где \(V = \int_\Omega\). Здесь и далее подразумеваем, что производные от \(G\)
берутся по переменной \(\bx'\), а выражения под интегралом по \(\Omega'\) также
зависят от \(\bx'\) вместо \(\bx\).
Тогда решение~\eqref{eq:p_Neumann2} может быть записано явно:
\begin{equation}\label{eq:pressure_solution}
    p = -\int_{\Omega'} f_\psi \psi_k G_k.
\end{equation}

Функция \(G(\bx, \bx')\) определяется с точностью до функции от \(x\).
При дополнительном условии \(\int_{\Omega'} G = 0\)
обеспечивается её симметричность \(G(\bx, \bx') = G(\bx', \bx)\)

Поскольку \(f_k = f_\psi \psi_k + f_{\psi_j}\psi_{jk}\) и
\(\int_{\Omega'} f_k G_k = -\int_{\Omega'} f G_{kk} = \frac1V\int_\Omega f - f\),
то~\eqref{eq:pressure_solution} можно переписать как
\begin{equation}\label{eq:pressure_solution2}
    p - f + \frac1V\int_\Omega f = \int_{\Omega'} f_{\psi_j} \psi_{jk} G_k =
    \int_{\Omega'} f_{\psi_j} \left( \psi_k G_{jk} - \psi_j G_{kk} \right).
\end{equation}
Последнее равенство вытекает из следующих тождеств:
\begin{gather*}
    \partial_k\left( g\psi_j^2 G_k \right) = g_\psi \psi_j^2\psi_k G_k + 2g\psi_j\psi_{jk}G_k + g\psi_j^2 G_{kk}, \\
    \partial_j\left( g\psi_j\psi_k G_k\right) = g_\psi \psi_j^2\psi_k G_k + g\psi_j\psi_{jk}G_k + g\psi_j\psi_k G_{jk}.
\end{gather*}

Таким образом, утверждение~\ref{prop:main} равносильно
\begin{proposition}\label{prop:main2}
\begin{equation}\label{eq:forces_sum3}
    \int_S n_i \int_{\Omega'} g \psi_j \psi_{jk} G_k = 0.
\end{equation}
\end{proposition}

Кстати, в силу~\eqref{eq:Green_function}, для произвольной дифференцируемой функции \(h(\bx)\) справедливы равенства
\begin{equation}\label{eq:Green_double_integral}
    \int_\Omega \int_{\Omega'} h G_{kk} = 0, \quad
    \int_\Omega \int_{\Omega'} h_k G_k = 0.
\end{equation}


\end{document}
