%&pdflatex
\documentclass{article}
\usepackage[utf8]{inputenc}
\usepackage[T2A]{fontenc}
\usepackage[russian]{babel}

\usepackage{amssymb, amsmath, amsthm}
\usepackage{fullpage}
\usepackage{indentfirst}

\usepackage[
    pdfauthor={Rogozin Oleg},
    pdftitle={Problem},
    colorlinks, pdftex, unicode
]{hyperref}

\theoremstyle{plain}
\newtheorem{lemma}{Lemma}
\newtheorem{proposition}{Proposition}
\newtheorem{exercise}{Exercise}

%\theoremstyle{remark}
\newtheorem{remark}{Remark}

\title{Задача}
\author{Олег Рогозин}

\DeclareMathOperator{\supp}{supp}

\newcommand{\dd}{\mathrm{d}}
\newcommand{\pder}[2][]{\frac{\partial#1}{\partial#2}}
\newcommand{\pderdual}[2][]{\frac{\partial^2#1}{\partial#2^2}}
\newcommand{\Set}[2]{\{\,{#1}:{#2}\,\}}
%\newcommand{\eqdef}{\overset{\mathrm{def}}{=\joinrel=}}
\newcommand{\eqdef}{\equiv}

\begin{document}

\section{Известный метод решения}

Рассмотрим классическую задачу электростатики о нахождении электрического поля,
возникающего между идеально проводящими телами. На основании законов Максвелла,
существует потенциал электрического поля \(\varphi\), для которого справедливо
уравнение Лапласа
\begin{equation}\label{eq:Laplace}
    \pderdual[\varphi]{x_i} = 0.
\end{equation}
Будем считать, что на поверхности проводников \(\varphi\) постоянен.
В силу линейности~\eqref{eq:Laplace} электрический заряд линейно связан с граничными условиями:
\begin{equation}\label{eq:charge}
    e_a \eqdef \oint_{S_a} \pder[\varphi]{x_i}n_i\dd{S} = C_{ab} \varphi_b,
\end{equation}
где \(n_i\) "--- единичный вектор нормали к поверхности, направленный наружу от тела,
\(\varphi_a\) "--- потенциал тела \(a\), \(S_a\) "--- его поверхность.
Коэффициенты ёмкости \(C_{aa}\) и коэффициенты электростатической индукции \(C_{ab}\,(a\neq b)\)
зависят только от геометрии задачи.
Энергия электрического поля
\begin{equation}\label{eq:energy}
    U \eqdef \int \left(\pder[\varphi]{x_i}\right)^2\dd{V} =
    \sum_a \varphi_a \oint_{S_a} \pder[\varphi]{x_i}n_i\dd{S} = C_{ab} \varphi_a \varphi_b.
\end{equation}
Сила, действующая на проводник \(a\),
\begin{equation}\label{eq:force}
    F^a_i = \oint_{S_a} \left(\pder[\varphi]{x_j}\right)^2 n_i\dd{S}
\end{equation}
может быть найдена как вариационная производная функционала~\eqref{eq:energy}
при смещении проводника \(a\) на вектор \(\delta r^a_i\):
\begin{equation}\label{eq:variation}
    \delta \int \left(\pder[\varphi]{x_j}\right)^2\dd{V} =
    \oint_{S_a} \left(\pder[\varphi]{x_j}\right)^2 n_i \delta r^a_i \dd{S} +
    2\sum_b \oint_{S_b} \pder[\varphi]{x_j} n_j\delta\varphi - 2\int \pderdual[\varphi]{x_j} \delta\varphi.
\end{equation}
Последние два интеграла равны нулю благодаря~\eqref{eq:Laplace} и неизменности потенциалов тел:
\begin{equation}\label{eq:delta_phi}
    \forall a \quad \delta\varphi|_{S_a} = 0.
\end{equation}

Для случая двух тел
\begin{equation}\label{eq:capacity}
    e_2 = -e_1 \eqdef e = C\Delta\varphi \eqdef C(\varphi_2 - \varphi_1).
\end{equation}
Эти тела притягиваются друг другу с силой
\begin{equation}\label{eq:force2}
    F_i = \left(\frac{\delta U}{\delta r^a_i}\right)_\varphi = \pder[C]{r^a_i}(\Delta\varphi)^2.
\end{equation}

\section{Задача}

Вместо~\eqref{eq:Laplace} рассмотрим поле, подчиняющееся эллиптическому уравнению вида
\begin{equation}\label{eq:elliptic}
    \pder{x_i}\left( \Gamma_2(T) \pder[T]{x_i} \right) = 0
\end{equation}
и дополним его уравнением сохранения импульса
\begin{equation}\label{eq:momentum}
    \pder{x_i}\left( p + \frac{\Gamma_7(T)}{2}\left(\pder[T]{x_j}\right)^2 \right) =
    \pder{x_j}\left( \Gamma_7(T)\pder[T]{x_i}\pder[T]{x_j} \right).
\end{equation}
Давление \(p\) определяется с точностью до константы, поэтому для удобства положим
\begin{equation}\label{eq:pressure}
    \int p\dd{V} = 0.
\end{equation}
На основе уравнения импульса~\eqref{eq:momentum} можно определить силу, действующую на тело \(a\),
\begin{equation}\label{eq:force_ell}
    F^a_i \eqdef \oint_{S_a} \left( p - \frac{\Gamma_7(T)}{2}\left(\pder[T]{x_j}\right)^2 \right) n_i\dd{S}.
\end{equation}
Коэффициенты \(\Gamma_2\) и \(\Gamma_7\) определим в виде
\begin{equation}\label{eq:gammas}
    \Gamma_n(T) = \gamma_n T^{p_n}.
\end{equation}
При \(T=1+o(1)\) задача сводится к \emph{линейной} электростатической,
где два равномерно нагретых тела с температурами \(T_1\) и \(T_2\) притягиваются с силой \(F \propto (T_2-T_1)^2\).
Требуется доказать следующую
\begin{lemma}\label{lem:force}
    Два равномерно нагретых тела притягиваются с силой
    \begin{equation}\label{eq:force_temp}
        F \propto \left( T_2^{1+p_7-p_2} - T_1^{1+p_7-p_2} \right)\left( T_2^{1+p_2} - T_1^{1+p_2} \right).
    \end{equation}
\end{lemma}

\begin{remark}
    Справедливость утверждения проверена на численных примерах:
    \begin{itemize}
        \item при \(p_2=1/2\) и \(p_7=0\) (газ твёрдых сфер),
        \item при \(p_2=1\) и \(p_7=1\) (модель БГК).
    \end{itemize}
\end{remark}

\subsection{Решение}

В отличие от~\eqref{eq:Laplace} уравнение~\eqref{eq:elliptic} нелинейно, однако его бездивергентная форма
позволяет по аналогии с~\eqref{eq:charge} сделать
\begin{proposition}
    В области, заключённой между телами с площадью \(S_a\) и температурой \(T_a\),
    \begin{equation}\label{eq:charge_ell}
        e_a \eqdef T_a^{p_2} \oint_{S_a} \pder[T]{x_i}n_i\dd{S} = C_{ab} T_b^{1+p_2}.
    \end{equation}
\end{proposition}
\begin{proof}
     Нелинейное уравнение~\eqref{eq:elliptic} для \(T\) является линейным для поля \(\psi=T^{1+p_2}\).
     Подставляя это выражение в~\eqref{eq:charge} вместо \(\varphi\), получаем искомое утверждение.
\end{proof}
Отметим, что в силу~\eqref{eq:elliptic} и~\eqref{eq:momentum}
\begin{equation}\label{eq:closed}
    \sum_a F^a_i = 0, \quad \sum_a e_a = 0.
\end{equation}
Несложно построить аналог энергии
\begin{equation}\label{eq:energy_ell}
    U \eqdef \int \left[ p - \frac{\Gamma_7(T)}2\left(\pder[T]{x_i}\right)^2 \right]\dd{V} =
    \gamma_7 \sum_a T_a^{1+p_7} \oint_{S_a} \pder[T]{x_i}n_i\dd{S} =
    C_{ab} T_a^{1+p_7-p_2} T_b^{1+p_2},
\end{equation}
Для двух тел с температурами \(T_1\) и \(T_2\) можно записать:
\begin{equation}\label{eq:two_bodies}
    e = C \left( T_2^{1+p_2} - T_1^{1+p_2} \right), \quad
    U = C \left( T_2^{1+p_7-p_2} - T_1^{1+p_7-p_2} \right)\left( T_2^{1+p_2} - T_1^{1+p_2} \right).
\end{equation}
Выпишем вариацию энергии \(\delta{U}\) при смещении тела \(a\) на вектор \(\delta r^a_i\):
\begin{equation}\label{eq:variation_ell}
    \delta U = F^a_i\delta r^a_i +
        \int \delta\left[ p - \frac{\Gamma_7(T)}2\left(\pder[T]{x_j}\right)^2 \right]\dd{V}.
\end{equation}
Последний интеграл равен нулю, поскольку
\begin{equation}\label{eq:sum_forces}
    \int \pder{x_i} \left[ p - \frac{\Gamma_7(T)}2\left(\pder[T]{x_j}\right)^2 \right]\dd{V} = 0
\end{equation}
и справедливо
\begin{proposition}
    Если \(\delta T|_{S_a} = 0\) для всех тел \(a\), то
    \begin{equation}\label{eq:delta_to_partial}
        \int \frac{\delta}{\delta r^a_i} \left[ p - \frac{\Gamma_7(T)}2\left(\pder[T]{x_j}\right)^2 \right]\dd{V} =
        \int \pder{x_i} \left[ p - \frac{\Gamma_7(T)}2\left(\pder[T]{x_j}\right)^2 \right]\dd{V}.
    \end{equation}
\end{proposition}
Лемма~\ref{lem:force} доказана.

\end{document}
