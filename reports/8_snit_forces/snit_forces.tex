%&pdflatex
\documentclass{article}
\usepackage[utf8]{inputenc}
\usepackage[T2A]{fontenc}
\usepackage[russian]{babel}

\usepackage{amssymb, amsmath, amsthm}
\usepackage{fullpage}
\usepackage{indentfirst}
\usepackage{subcaption}
\usepackage{tikz}
\usepackage{pgfplots}

\usepackage[
    pdfauthor={Rogozin Oleg},
    pdftitle={Problem},
    colorlinks, pdftex, unicode
]{hyperref}

\usepackage[
    backend=biber,
    style=gost-numeric,
    autolang=other,
    maxbibnames=99, minbibnames=99,
    natbib=true,
    sorting=ydnt,
    url=false,
    eprint=false,
    pagetracker,
    firstinits]{biblatex}
\bibliography{snit_forces}

\theoremstyle{plain}
\newtheorem*{lemma}{Lemma}
%\theoremstyle{remark}
\newtheorem*{remark}{Remark}

\title{Задача}
\author{Олег Рогозин}

\DeclareMathOperator{\supp}{supp}

\newcommand{\dd}{\mathrm{d}}
\newcommand{\pder}[2][]{\frac{\partial#1}{\partial#2}}
\newcommand{\pderdual}[2][]{\frac{\partial^2#1}{\partial#2^2}}
\newcommand{\pderder}[3][]{\frac{\partial^2#1}{\partial#2\partial#3}}
\newcommand{\Pder}[2][]{\partial#1/\partial#2}
\newcommand{\Pderdual}[2][]{\partial^2#1/\partial#2^2}
\newcommand{\Pderder}[3][]{\partial^2#1/\partial#2\partial#3}
\newcommand{\Set}[2]{\{\,{#1}:{#2}\,\}}
%\newcommand{\eqdef}{\overset{\mathrm{def}}{=\joinrel=}}
\newcommand{\eqdef}{\equiv}

\begin{document}

\section{Известный метод решения}

Рассмотрим классическую задачу электростатики о нахождении электрического поля,
возникающего между идеально проводящими телами. На основании законов Максвелла,
существует потенциал электрического поля \(\varphi\), для которого справедливо
уравнение Лапласа
\begin{equation}\label{eq:Laplace}
    \pderdual[\varphi]{x_i} = 0.
\end{equation}
Будем считать, что на поверхности проводников \(\varphi\) постоянен.
В силу линейности~\eqref{eq:Laplace} электрический заряд линейно связан с граничными условиями:
\begin{equation}\label{eq:charge}
    e_a \eqdef \oint_{S_a} \pder[\varphi]{x_i}n_i\dd{S} = C_{ab} \varphi_b,
\end{equation}
где \(n_i\) "--- единичный вектор нормали к поверхности, направленный наружу от тела,
\(\varphi_a\) "--- потенциал тела \(a\), \(S_a\) "--- его поверхность.
Коэффициенты ёмкости \(C_{aa}\) и коэффициенты электростатической индукции \(C_{ab}\,(a\neq b)\)
зависят только от геометрии задачи.
Энергия электрического поля
\begin{equation}\label{eq:energy}
    U \eqdef \int \left(\pder[\varphi]{x_i}\right)^2\dd{V} =
    \sum_a \varphi_a \oint_{S_a} \pder[\varphi]{x_i}n_i\dd{S} = C_{ab} \varphi_a \varphi_b.
\end{equation}
Сила, действующая на проводник \(a\),
\begin{equation}\label{eq:force}
    F^a_i = \oint_{S_a} \left(\pder[\varphi]{x_j}\right)^2 n_i\dd{S}
\end{equation}
может быть найдена как вариационная производная функционала~\eqref{eq:energy}
при смещении проводника \(a\) на вектор \(\delta r^a_i\):
\begin{equation}\label{eq:variation}
    \delta \int \left(\pder[\varphi]{x_j}\right)^2\dd{V} =
    \oint_{S_a} \left(\pder[\varphi]{x_j}\right)^2 n_i \delta r^a_i \dd{S} +
    2\sum_b \oint_{S_b} \pder[\varphi]{x_j} n_j\delta\varphi - 2\int \pderdual[\varphi]{x_j} \delta\varphi.
\end{equation}
Последние два интеграла равны нулю благодаря~\eqref{eq:Laplace} и неизменности потенциалов тел:
\begin{equation}\label{eq:delta_phi}
    \forall a \quad \delta\varphi|_{S_a} = 0.
\end{equation}

Для случая двух тел
\begin{equation}\label{eq:capacity}
    e_2 = -e_1 \eqdef e = C\Delta\varphi \eqdef C(\varphi_2 - \varphi_1).
\end{equation}
Эти тела притягиваются друг другу с силой
\begin{equation}\label{eq:force2}
    F_i = \left(\frac{\delta U}{\delta r^a_i}\right)_\varphi = \pder[C]{r^a_i}(\Delta\varphi)^2.
\end{equation}

\section{Задача}

Рассмотрим температурное поле, подчиняющееся эллиптическому уравнению вида
\begin{equation}\label{eq:elliptic}
    \pder{x_i}\left( \Gamma_2(T) \pder[T]{x_i} \right) = 0
\end{equation}
и дополним его уравнением сохранения импульса
\begin{equation}\label{eq:momentum}
    \pder{x_i}\left( p + \frac{\Gamma_7(T)}{2}\left(\pder[T]{x_j}\right)^2 \right) =
    \pder{x_j}\left( \Gamma_7(T)\pder[T]{x_i}\pder[T]{x_j} \right).
\end{equation}
Давление \(p\) определяется с точностью до константы, поэтому для удобства положим
\begin{equation}\label{eq:pressure}
    \int p\dd{V} = 0.
\end{equation}
На основе уравнения импульса~\eqref{eq:momentum} можно определить силу, действующую на тело \(a\),
\begin{equation}\label{eq:force_ell}
    F^a_i \eqdef \oint_{S_a} \left( p - \frac{\Gamma_7(T)}{2}\left(\pder[T]{x_j}\right)^2 \right) n_i\dd{S}.
\end{equation}
Коэффициенты \(\Gamma_2\) и \(\Gamma_7\) определим в виде
\begin{equation}\label{eq:gammas}
    \Gamma_n(T) = \gamma_n T^{s_n}.
\end{equation}
При \(T=1+o(1)\) задача сводится к \emph{линейной} электростатической,
где два равномерно нагретых тела с температурами \(T_1\) и \(T_2\) притягиваются с силой \(F \propto (T_2-T_1)^2\).
Требуется доказать следующую
\begin{lemma}\label{lem:force}
    Два равномерно нагретых тела притягиваются с силой
    \begin{equation}\label{eq:force_temp}
        F_i \propto \left( T_2^{1+s_7-s_2} - T_1^{1+s_7-s_2} \right)\left( T_2^{1+s_2} - T_1^{1+s_2} \right).
    \end{equation}
\end{lemma}

\begin{proof}

В отличие от~\eqref{eq:Laplace} уравнение~\eqref{eq:elliptic} нелинейно,
однако при помощи замены \(\psi=T^{1+s_2}\) получаем уравнение Лапласа для \(\psi\).
Таким образом, в соответствии с~\eqref{eq:charge} имеем
\begin{equation}\label{eq:charge_linear}
    \oint_{S_a} \pder[\psi]{x_i}n_i\dd{S} = C'_{ab} \psi,
\end{equation}
на основе чего можно ввести аналог заряда
\begin{equation}\label{eq:charge_ell}
    e_a \eqdef T_a^{s_2} \oint_{S_a} \pder[T]{x_i}n_i\dd{S} = C_{ab} T_b^{1+s_2}.
\end{equation}
Отметим, что в силу бездивергентных уравнений~\eqref{eq:elliptic} и~\eqref{eq:momentum},
\begin{equation}\label{eq:closed}
    \sum_a F^a_i = 0, \quad \sum_a e_a = 0.
\end{equation}
Несложно построить аналог энергии
\begin{equation}\label{eq:energy_ell}
    U \eqdef \int \left[ p - \frac{\Gamma_7(T)}2\left(\pder[T]{x_i}\right)^2 \right]\dd{V} =
    \gamma_7 \sum_a T_a^{1+s_7} \oint_{S_a} \pder[T]{x_i}n_i\dd{S} =
    C_{ab} T_a^{1+s_7-s_2} T_b^{1+s_2},
\end{equation}
Для двух тел с температурами \(T_1\) и \(T_2\) можно записать:
\begin{equation}\label{eq:two_bodies}
    e = C \left( T_2^{1+s_2} - T_1^{1+s_2} \right), \quad
    U = C \left( T_2^{1+s_7-s_2} - T_1^{1+s_7-s_2} \right)\left( T_2^{1+s_2} - T_1^{1+s_2} \right).
\end{equation}

Используем основную формулу для вариации в случае переменной области\cite{Gelfand1961}
\begin{equation}\label{eq:variation_ell}
    \delta U = \int \left[
        \left( \pder[\Phi]{\psi} - \pderder[\Phi]{x_i}{\pder[\psi]{x_i}} \right) \delta\psi +
        \pder{x_i}\left( \pder[\Phi]{\pder[\psi]{x_i}} \delta\psi \right) +
        \pder{x_i}\left( \Phi \delta x_i \right)
    \right]\dd{V}
\end{equation}
для
\begin{equation}\label{eq:psi_Phi_definition}
    \psi = T^{\frac{2+s_7}{3}}, \quad
    \Phi = p - \frac{\gamma_7}2 T^{s_7}\left(\pder[T]{x_j}\right)^2 =
    p - \frac{\gamma_7}2 \left(\frac3{2+s_7}\right)^2 \psi \left(\pder[\psi]{x_j}\right)^2.
\end{equation}
Для поля \(\psi\) вместо~\eqref{eq:elliptic} и~\eqref{eq:momentum} получаем уравнения
\begin{equation}\label{eq:psi_equations}
    \pder[p]{x_i} = \frac{\gamma_7}2 (s_7-2s_2)\left(\frac3{2+s_7}\right)^3
        \left(\pder[\psi]{x_j}\right)^2 \pder[\psi]{x_i}, \quad
    \pder{x_i}\left( \psi^\frac{1-s_7+3s_2}{2+s_7} \pder[\psi]{x_i} \right) = 0.
\end{equation}
Замечаем, что
\begin{equation}\label{eq:pressure_psi}
    p = p\left(x_i, \pder[\psi]{x_i}\right),
\end{equation}
поэтому
\begin{equation}\label{eq:pressure_variations}
    \pder[p]{\psi} = 0, \quad \pderder[p]{x_i}{\pder[\psi]{x_i}} =
    \frac{3\gamma_7}2(s_7-2s_2)\left(\frac3{2+s_7}\right)^3\left(\pder[\psi]{x_i}\right)^2.
\end{equation}
Таким образом, оказывается, что
\begin{equation}\label{eq:eq:variation_ell1}
    \pder[\Phi]{\psi} - \pderder[\Phi]{x_i}{\pder[\psi]{x_i}} = 0.
\end{equation}
С физической точки зрения, это отражение закона сохранения энергии.
Для вариации энергии \(\delta{U}\) при смещении тела \(a\) на вектор \(\delta r^a_i\) и
при постоянных температурах тел (\(\delta \psi_b=0\)) получаем
\begin{gather}
    \int \pder{x_i}\left( \pder[\Phi]{\pder[\psi]{x_i}} \delta\psi \right) \dd{V} =
    \sum_b \delta \psi_b \oint_{S_b} \pder[\Phi]{\pder[\psi]{x_i}} n_i \dd{S} = 0, \label{eq:variation_ell2}\\
    \int \pder{x_i}\left( \Phi \delta x_i \right) \dd{V} =
    \oint_{S_a} \Phi n_i \delta r^a_i \dd{S} = F^a_i \delta r^a_i. \label{eq:variation_ell3}
\end{gather}
Итак, для двух тел
\begin{equation}\label{eq:answer}
    F_i = \frac{\delta U}{\delta r^a_i}
        \propto \left( T_2^{1+s_7-s_2} - T_1^{1+s_7-s_2} \right)\left( T_2^{1+s_2} - T_1^{1+s_2} \right).
\end{equation}

\end{proof}
\begin{remark}
    При \(s_2 = 2s_7\) оказывается, что \(\Pder[p]{x_i} = 0\).
\end{remark}

\begin{figure}[ht]
    \centering
    \begin{subfigure}[b]{0.49\textwidth}
        \centering
        \begin{tikzpicture}
            \begin{loglogaxis}[
                xlabel=\(\tau\),
                ylabel=\(F\)]
                \addplot gnuplot [raw gnuplot, mark=none, color=blue, thick]{
                    set logscale xy;
                    set xrange [1e-2:1e3];
                    plot ((x+1)**.5-1)*((x+1)**1.5-1);
                };
                \addplot gnuplot [raw gnuplot, mark=none, color=black, very thin, dashed]{
                    set xrange [1e0:1e3];
                    plot x**2;
                };
                \addplot gnuplot [raw gnuplot, mark=none, color=black, very thin, dashed]{
                    set xrange [1e-2:1e+2];
                    plot 4./5*x**2;
                };
            \end{loglogaxis}
        \end{tikzpicture}
        \caption{газ твёрдых сфер: \(s_2=1/2\) и \(s_7=0\).
            Асимптоты: \(\frac45\tau^2\),~\(\tau\to0\) и \(\tau^2\),~\(\tau\to\infty\)}
        \label{fig:gull}
    \end{subfigure}
    ~
    \begin{subfigure}[b]{0.49\textwidth}
        \centering
        \begin{tikzpicture}
            \begin{loglogaxis}[
                xlabel=\(\tau\),
                ylabel=\(F\)]
                \addplot gnuplot [raw gnuplot, mark=none, color=blue, thick]{
                    set logscale xy;
                    set xrange [1e-2:1e3];
                    plot ((x+1)**1-1)*((x+1)**2-1);
                };
                \addplot gnuplot [raw gnuplot, mark=none, color=black, very thin, dashed]{
                    set samples 2;
                    set xrange [1e0:1e3];
                    plot x**3;
                };
                \addplot gnuplot [raw gnuplot, mark=none, color=black, very thin, dashed]{
                    set samples 2;
                    set xrange [1e-2:1e+1];
                    plot 2*x**2;
                };
            \end{loglogaxis}
        \end{tikzpicture}
        \caption{модель БГК: \(s_2=1\) и \(s_7=1\).
            Асимптоты: \(2\tau^2\),~\(\tau\to0\) и \(\tau^3\),~\(\tau\to\infty\)}
        \label{fig:mouse}
    \end{subfigure}
    \caption{Зависимость силы притяжения двух тел \(F\) от разности температур \(\tau=T_2-T_1\) при \(T_1=1\).
        Тонкие пунктирные линии соответствуют асимптотам.}
    \label{fig:temperature}
\end{figure}

На рис.~\ref{fig:temperature} показаны соответствующие зависимости для некоторых частных случаев.

\printbibliography

\end{document}
