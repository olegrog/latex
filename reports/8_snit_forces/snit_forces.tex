%&pdflatex
\documentclass{article}
\usepackage[utf8]{inputenc}
\usepackage[T2A]{fontenc}
\usepackage[english,french,russian]{babel}

\usepackage{amssymb, amsmath, amsthm}
\usepackage{fullpage}
\usepackage{indentfirst}
\usepackage{subcaption}
\usepackage{tikz}
\usepackage{pgfplots}

\usepackage[
    pdfauthor={Rogozin Oleg},
    pdftitle={On forces acting on the uniform heated bodies},
    colorlinks, pdftex, unicode
]{hyperref}

\usepackage[
    backend=biber,
    style=gost-numeric,
    autolang=other,
    maxbibnames=99, minbibnames=99,
    natbib=true,
    sorting=ydnt,
    url=false,
    eprint=true,
    pagetracker,
    firstinits]{biblatex}
\bibliography{snit_forces}

\theoremstyle{plain}
\newtheorem*{lemma}{Lemma}
\newtheorem{proposition}{Proposition}
%\theoremstyle{remark}
\newtheorem*{remark}{Remark}

\title{О силах, действующих на равномерно нагретые тела}
\author{Олег Рогозин}

\newcommand{\dd}{\mathrm{d}}
\newcommand{\pder}[2][]{\frac{\partial#1}{\partial#2}}
\newcommand{\pderdual}[2][]{\frac{\partial^2#1}{\partial#2^2}}
\newcommand{\pderder}[3][]{\frac{\partial^2#1}{\partial#2\partial#3}}
\newcommand{\Pder}[2][]{\partial#1/\partial#2}
\newcommand{\Pderdual}[2][]{\partial^2#1/\partial#2^2}
\newcommand{\Pderder}[3][]{\partial^2#1/\partial#2\partial#3}
\newcommand{\Set}[2]{\{\,{#1}:{#2}\,\}}
\newcommand{\bx}{\boldsymbol{x}}
\newcommand{\by}{\boldsymbol{y}}
%\newcommand{\eqdef}{\overset{\mathrm{def}}{=\joinrel=}}
\newcommand{\eqdef}{\equiv}

\begin{document}

\section{Известный метод решения}

Рассмотрим классическую задачу электростатики о нахождении электрического поля,
возникающего между идеально проводящими телами. На основании законов Максвелла,
существует потенциал электрического поля \(\varphi\), для которого справедливо
уравнение Лапласа
\begin{equation}\label{eq:Laplace}
    \pderdual[\varphi]{x_i} = 0.
\end{equation}
Будем считать, что на поверхности проводников \(\varphi\) постоянен.
В силу линейности~\eqref{eq:Laplace} электрический заряд линейно связан с граничными условиями,
\begin{equation}\label{eq:charge}
    e_a \eqdef \oint_{S_a} \pder[\varphi]{x_i}n_i\dd{S} = C_{ab} \varphi_b,
\end{equation}
где \(n_i\) "--- единичный вектор нормали к поверхности, направленный наружу от тела,
\(\varphi_a\) "--- потенциал тела \(a\), \(S_a\) "--- его поверхность.
Коэффициенты ёмкости \(C_{aa}\) и коэффициенты электростатической индукции \(C_{ab}\,(a\neq b)\)
зависят только от геометрии задачи.
Энергия электрического поля
\begin{equation}\label{eq:energy}
    U \eqdef \int \left(\pder[\varphi]{x_i}\right)^2\dd{V} =
    \sum_a \varphi_a \oint_{S_a} \pder[\varphi]{x_i}n_i\dd{S} = C_{ab} \varphi_a \varphi_b.
\end{equation}
Сила, действующая на проводник \(a\),
\begin{equation}\label{eq:force}
    F^a_i = \oint_{S_a} \left(\pder[\varphi]{x_j}\right)^2 n_i\dd{S}
\end{equation}
может быть найдена как вариационная производная функционала~\eqref{eq:energy}
при смещении проводника \(a\) на вектор \(\delta r^a_i\).
Для этого воспользуемся основной формулой для вариации функционала от \(\Phi(\varphi,\Pder[\varphi]{x_i})\)
в случае переменной области~\cite{Gelfand1961},
\begin{equation}\label{eq:variation_general}
    \delta \int \Phi \dd{V} = \int \left[
        \left( \Phi_\varphi - \pder[\Phi_{\varphi_i}]{x_i} \right) \delta\varphi +
        \pder{x_i}\left( \pder[\Phi_{\varphi_i}]{x_i} \delta\varphi \right) +
        \pder{x_i}\left( \Phi \delta x_i \right)
    \right]\dd{V},
\end{equation}
где введены следующие обозначения:
\begin{equation}\label{eq:Phi_derivatives}
    \Phi_\varphi = \pder[\Phi]{\varphi}, \quad \Phi_{\varphi_i} = \pder[\Phi]{\pder[\varphi]{x_i}}.
\end{equation}
При \(\Phi=(\Pder[\varphi]{x_j})^2\) получается
\begin{equation}\label{eq:variation}
    \delta U =
    - 2\int \pderdual[\varphi]{x_j} \delta\varphi
    + 2\sum_b \oint_{S_b} \pder[\varphi]{x_j} n_j\delta\varphi
    + \oint_{S_a} \left(\pder[\varphi]{x_j}\right)^2 n_i \delta r^a_i \dd{S}.
\end{equation}
Первые два интеграла в правой части~\eqref{eq:variation} равны нулю
благодаря~\eqref{eq:Laplace} и неизменности потенциалов тел,
\begin{equation}\label{eq:delta_phi}
    \forall a \quad \delta\varphi|_{S_a} = 0,
\end{equation}
поэтому
\begin{equation}\label{eq:force_n}
    F^a_i = \left(\frac{\delta U}{\delta r^a_i}\right)_\varphi = \pder[C_{ab}]{r^a_i} \varphi_a \varphi_b.
\end{equation}

Для случая двух тел
\begin{equation}\label{eq:capacity}
    e_2 = -e_1 \eqdef e = C\Delta\varphi \eqdef C(\varphi_2 - \varphi_1).
\end{equation}
Эти тела притягиваются друг другу с силой
\begin{equation}\label{eq:force2}
    F_i = \left(\frac{\delta U}{\delta r_i}\right)_\varphi = \pder[C]{r_i}(\Delta\varphi)^2.
\end{equation}

\section{Задача}

Рассмотрим температурное поле \(T\), подчиняющееся эллиптическому уравнению вида
\begin{equation}\label{eq:elliptic}
    \pder{x_i}\left( \Gamma_2(T) \pder[T]{x_i} \right) = 0,
\end{equation}
и дополним его полем давления \(p\), которое вычисляется из краевой задачи Неймана
\begin{gather}
    \pderdual{x_i}\left( p + \frac{\Gamma_7(T)}{2}\left(\pder[T]{x_j}\right)^2 \right) =
    \pderder{x_i}{x_j}\left( \Gamma_7(T)\pder[T]{x_i}\pder[T]{x_j} \right), \label{eq:pressure_equation}\\
    \pder{x_i}\left.\left( p + \frac{\Gamma_7(T)}{2}\left(\pder[T]{x_j}\right)^2 \right)n_i\right|_S =
    \pder{x_j}\left.\left( \Gamma_7(T)\pder[T]{x_i}\pder[T]{x_j} \right)n_i\right|_S. \label{eq:pressure_boundary}
\end{gather}
Давление \(p\) определяется с точностью до константы, поэтому для определённости положим
\begin{equation}\label{eq:pressure_integral}
    \int p\dd{V} = 0.
\end{equation}
Для задачи~\eqref{eq:pressure_equation} можно определить силу, действующую на тело \(a\),
\begin{equation}\label{eq:force_ell}
    F^a_i \eqdef \oint_{S_a} \left( \frac{\Gamma_7(T)}{2}\left(\pder[T]{x_j}\right)^2 - p \right) n_i\dd{S} =
    \oint_{S_a} \Gamma_7(T)\pder[T]{x_i}\pder[T]{x_j} n_j\dd{S} -
    \oint_{S_a} \left( \frac{\Gamma_7(T)}{2}\left(\pder[T]{x_j}\right)^2 + p \right) n_i\dd{S}.
\end{equation}
Транспортные коэффициенты для степенных потенциалов:
\begin{equation}\label{eq:gammas}
    \Gamma_2(T) = \gamma_2 T^s, \quad \Gamma_7(T) = \gamma_7 T^{2s-1}.
\end{equation}
При \(T=1+o(1)\) задача сводится к \emph{линейной} электростатической,
где два равномерно нагретых тела с температурами \(T_1\) и \(T_2\) притягиваются с силой \(F \propto (T_2-T_1)^2\).
Требуется доказать следующую
\begin{lemma}\label{lem:force}
    Два равномерно нагретых тела притягиваются с силой
    \begin{equation}\label{eq:force_temp_0}
        F_i \propto \left( \ln{T_2}-\ln{T_1} \right)\left( T_2 - T_1 \right)
    \end{equation}
    при \(s=0\),
    \begin{equation}\label{eq:force_temp_-1}
        F_i \propto \left( T_2^{-1} - T_1^{-1} \right)\left( \ln{T_2}-\ln{T_1} \right)
    \end{equation}
    при \(s=-1\) и
    \begin{equation}\label{eq:force_temp}
        F_i \propto \left( T_2^s - T_1^s \right)\left( T_2^{1+s} - T_1^{1+s} \right)
    \end{equation}
    в остальных случаях.
\end{lemma}

\begin{figure}[ht]
    \centering
    \begin{subfigure}[b]{0.49\textwidth}
        \centering
        \begin{tikzpicture}
            \begin{loglogaxis}[
                xlabel=\(\tau\),
                ylabel=\(F\)]
                \addplot gnuplot [raw gnuplot, mark=none, color=blue, thick]{
                    set logscale xy;
                    set xrange [1e-2:1e3];
                    plot ((x+1)**.5-1)*((x+1)**1.5-1);
                };
                \addplot gnuplot [raw gnuplot, mark=none, color=black, very thin, dashed]{
                    set xrange [1e0:1e3];
                    plot x**2;
                };
                \addplot gnuplot [raw gnuplot, mark=none, color=black, very thin, dashed]{
                    set xrange [1e-2:1e+2];
                    plot 3./4*x**2;
                };
            \end{loglogaxis}
        \end{tikzpicture}
        \caption{газ твёрдых сфер: \(s_2=1/2\) и \(s_7=0\).
            Асимптоты: \(\frac34\tau^2\),~\(\tau\to0\) и \(\tau^2\),~\(\tau\to\infty\)}
        \label{fig:gull}
    \end{subfigure}
    ~
    \begin{subfigure}[b]{0.49\textwidth}
        \centering
        \begin{tikzpicture}
            \begin{loglogaxis}[
                xlabel=\(\tau\),
                ylabel=\(F\)]
                \addplot gnuplot [raw gnuplot, mark=none, color=blue, thick]{
                    set logscale xy;
                    set xrange [1e-2:1e3];
                    plot ((x+1)**1-1)*((x+1)**2-1);
                };
                \addplot gnuplot [raw gnuplot, mark=none, color=black, very thin, dashed]{
                    set samples 2;
                    set xrange [1e0:1e3];
                    plot x**3;
                };
                \addplot gnuplot [raw gnuplot, mark=none, color=black, very thin, dashed]{
                    set samples 2;
                    set xrange [1e-2:1e+1];
                    plot 2*x**2;
                };
            \end{loglogaxis}
        \end{tikzpicture}
        \caption{максвелловские молекулы или модель БГК: \(s_2=1\) и \(s_7=1\).
            Асимптоты: \(2\tau^2\),~\(\tau\to0\) и \(\tau^3\),~\(\tau\to\infty\)}
        \label{fig:mouse}
    \end{subfigure}
    \caption{Зависимость силы притяжения двух тел \(F\) от разности температур \(\tau=T_2-T_1\) при \(T_1=1\).
        Тонкие пунктирные линии соответствуют асимптотам.}
    \label{fig:temperature}
\end{figure}

На рис.~\ref{fig:temperature} показаны соответствующие зависимости для некоторых частных случаев.

\section{Обоснование леммы}

В отличие от~\eqref{eq:Laplace} уравнение~\eqref{eq:elliptic} нелинейно,
однако при помощи замены \(\psi=T^{1+s}\) получается уравнение Лапласа для \(\psi\).
При \(s=-1\) используется замена \(\psi=\ln{T}\).
Таким образом, по аналогии с~\eqref{eq:charge} можно ввести аналог заряда
\begin{equation}\label{eq:charge_ell}
    e_a \eqdef T_a^s \oint_{S_a} \pder[T]{x_i}n_i\dd{S} = C'_{ab} T_b^{1+s}, \quad \sum_a e_a = 0.
\end{equation}
При \(s=-1\) получается \(e_a = C_{ab} \ln{T_b}\).
Несложно также построить аналог энергии
\begin{equation}\label{eq:energy_ell}
    U \eqdef \int \left[ \frac{\Gamma_7(T)}2\left(\pder[T]{x_i}\right)^2 - p \right]\dd{V} =
    \gamma_7 \sum_a T_a^{2s} \oint_{S_a} \pder[T]{x_i}n_i\dd{S} =
    C_{ab} T_a^s T_b^{1+s}.
\end{equation}
При \(s=0\)
\begin{equation}\label{eq:energy_ell0}
    U = \gamma_7 \sum_a \ln{T_a} \oint_{S_a} \pder[T]{x_i}n_i\dd{S} = C_{ab} T_b \ln{T_a}.
\end{equation}
Для двух тел с температурами \(T_1\) и \(T_2\) можно записать
\begin{equation}\label{eq:two_bodies}
    e = C \left( T_2^{1+s} - T_1^{1+s} \right), \quad
    U = C \left( T_2^s - T_1^s \right)\left( T_2^{1+s} - T_1^{1+s} \right),
\end{equation}
поскольку \(e_1+e_2=0\) и \(e_1=e_2=0\) при \(T_1=T_2\).
Для частных случаев \(s=-1\) и \(s=0\) выражения аналогичны.

\section{Функция Грина}

Для удобства обозначим
\begin{equation}\label{eq:g7_function}
    f\left(\psi,\pder[\psi]{x_i}\right) = \frac{\Gamma_7(T)}2\left(\pder[T]{x_j}\right)^2 =
    \frac{\gamma_7}2 c^2 \psi^c\left(\pder[\psi]{x_j}\right)^2, \quad c=-\frac1{1+s},
\end{equation}
тогда краевая задача~\eqref{eq:pressure_equation},~\eqref{eq:pressure_boundary} примет вид
\begin{equation}\label{eq:pressure_short}
    \pderdual[p]{x_k} = \pder[f_\psi]{x_k}\pder[\psi]{x_k}, \quad
    \left.\pder[p]{x_k}n_k\right|_S = \left.f_\psi\pder[\psi]{x_k}n_k\right|_S,
\end{equation}
где использованы обозначения частных производных~\eqref{eq:Phi_derivatives}.
Через функцию Грина \(G(\bx, \bx')\) для задачи Неймана~\cite{Franklin2012},
\begin{equation}\label{eq:Green_function}
    \pderdual[G]{x_k'} = \delta(\bx-\bx') - \frac1V, \quad
    \left.\pder[G]{x_k'}n_k\right|_{S'} = 0,
\end{equation}
можно выразить решение~\eqref{eq:pressure_short}:
\begin{equation}\label{eq:pressure_solution}
    p = -\int f_\psi(\bx')\pder[\psi(\bx')]{x_k'}\pder[G(\bx,\bx')]{x_k'} \dd{V'}.
\end{equation}
Функция \(G(\bx, \bx')\) определяется с точностью до функции от \(x\).
При дополнительном условии
\begin{equation}\label{eq:Green_normalization}
    \int G(\bx, \bx') \dd{V'} = 0
\end{equation}
обеспечивается её симметричность
\begin{equation}\label{eq:Green_symmetry}
    G(\bx, \bx') = G(\bx', \bx).
\end{equation}
Этот факт легко подтверждается подстановкой \(v=G(\bx,\by)\), \(u=G(\bx',\by)\)
во вторую формулу Грина
\begin{equation}\label{eq:Green_second_identity}
    \oint \left( u\pder[v]{x_k} - v\pder[u]{x_k} \right)n_k\dd{S} =
    \int \left( u\pderdual[v]{x_k} - v\pderdual[u]{x_k} \right)\dd{V}.
\end{equation}

Используем основную формулу для вариации в случае переменной области~\eqref{eq:variation_general}
для \(\Phi = f-p\) при сдвиге тела \(a\) на вектор \(\delta r^a_i\):
\begin{equation}\label{eq:variation_energy}
    \delta\int(f-p)\dd{V} = F^a_i \delta r^a_i + \int\delta(f-p)\dd{V}.
\end{equation}
При подстановке~\eqref{eq:energy_ell} и~\eqref{eq:pressure_solution} в~\eqref{eq:variation_energy} получаем
\begin{equation}\label{eq:variation_energy2}
    \pder[C_{ab}]{r^a_i} \psi^{c+1}_a \psi_b - F^a_i \delta r^a_i =
    \int\dd{V}\left( \delta{f} + \delta\int f'_\psi\pder[\psi']{x_k'}\pder[G]{x_k'} \dd{V'} \right) =
    \int\dd{V}\left[ \delta{f} + \int\delta\left( f'_\psi\pder[\psi']{x_k'}\pder[G]{x_k'}\right) \dd{V'} \right].
\end{equation}
Последнее равенство справедливо, поскольку \(\Pder[\psi]{x_k}\) и \(\Pder[G]{x_k}\) перпендикулярны на границе.

Из определения функции Грина~\eqref{eq:Green_function} следует,
что для произвольной дифференцируемой функции \(g(\bx)\) справедливы равенства
\begin{equation}\label{eq:Green_double_integral}
    \iint g\pderdual[G]{x_k'}\dd{V'}\dd{V} = 0, \quad
    \iint \pder[g]{x_k'}\pder[G]{x_k'}\dd{V'}\dd{V} = 0.
\end{equation}
Используя постоянство температур тел (\(\delta\psi|_S=0\)) и~\eqref{eq:Green_double_integral},
получаем цепочку тождественных преобразований
\begin{multline}\label{eq:variation_energy3}
    \iint \pder[G]{x_k'}\delta\left( f'_\psi\pder[\psi']{x_k'} \right) \dd{V'}\dd{V} =
    -\iint \pder{x_j'}\left( f'_{\psi\psi_j}\pder[\psi']{x_k'}\pder[G]{x_k'} \right)\delta\psi' \dd{V'}\dd{V} = \\
    \iint f'_{\psi\psi_j}\pder[\psi']{x_k'}\pder[G]{x_k'} \delta\left(\pder[\psi']{x_j'}\right) \dd{V'}\dd{V} =
    -\frac12\iint \psi'^c\pder[G]{x_k'} \delta\left[\pder{x_k}\left(\pder[\psi']{x_j}\right)^2\right] \dd{V'}\dd{V}.
\end{multline}

Для вариации функции Грина справедлива формула Адамара~\cite{Hadamard1908, Schiffer1958}
\begin{equation}\label{eq:Hadamard_identity}
    \delta{G}(\bx,\bx') = \oint\pder[G(\bx,\by)]{y_k}\pder[G(\bx',\by)]{y_k} n_i\delta{x_i}.
\end{equation}
Доказательство~\eqref{eq:Hadamard_identity} основывается на второй формуле Грина~\eqref{eq:Green_second_identity}
при \(v=G(\bx,\by)\), \(u=G(\bx',\by)+\delta{G}(\bx',\by)\), поскольку \(\delta{G}\) является гармонической функцией.
Кроме того, очевидно \(\delta(\Pderdual[G]{x_k'}) = 0\).

\section{Утверждения, требующие доказательства}

Для гармонической функции \(\psi\) (\(\Pderdual[\psi]{x_j} = 0\)) с граничными условиями
\(\psi|_{S_a} = \psi_a = \mathrm{const}\), где полная поверхность \(S=\bigcup_a S_a\), справедливо
\begin{equation}\label{eq:integral_mixed}
    \int g(\psi)\pderder[\psi]{x_i}{x_j}\pder[\psi]{x_j}\dd{V} = \int g(\psi)\pderdual[\psi]{x_j}\pder[\psi]{x_i}\dd{V} = 0,
\end{equation}
где \(g\) "--- произвольная функция от \(\psi\). Доказательство следует из формул
\begin{gather}
    \pder{x_i}\left( g \left(\pder[\psi]{x_j}\right)^2 \right) =
        g_\psi\pder[\psi]{x_i}\left(\pder[\psi]{x_j}\right)^2 + 2g\pderder[\psi]{x_i}{x_j}\pder[\psi]{x_j}, \\
    \pder{x_j}\left( g \pder[\psi]{x_i}\pder[\psi]{x_j} \right) =
        g_\psi\pder[\psi]{x_i}\left(\pder[\psi]{x_j}\right)^2 + g\pderder[\psi]{x_i}{x_j}\pder[\psi]{x_j} +
        g\pder[\psi]{x_i}\pderdual[\psi]{x_j}.
\end{gather}

\begin{proposition}
    Сумма сил, действующих на тела, равна нулю:
    \begin{equation}\label{eq:problem1}
        \sum_a F^a_i = \int \pder{x_i}(f-p) \dd{V} = 0
    \end{equation}
    или через функцию Грина
    \begin{equation}\label{eq:problem1_Green}
        \int \pder[f]{x_i} \dd{V} = -\iint f'_\psi\pder[\psi']{x_k'} \pderder[G]{x_i}{x_k'} \dd{V'}\dd{V}.
    \end{equation}
\end{proposition}

\begin{proposition}
    Функционал~\eqref{eq:energy_ell} экcтремален по отношению к вариации поля \(\psi\):
    \begin{equation}\label{eq:problem2}
        \int \delta(f-p)\dd{V} = 0
    \end{equation}
    или через функцию Грина
    \begin{equation}\label{eq:problem2_Green}
        \int\delta{f}\dd{V} = -\iint \delta\left( f'_\psi\pder[\psi']{x_k'}\pder[G]{x_k'}\right) \dd{V'}\dd{V}.
    \end{equation}
\end{proposition}


\printbibliography

\end{document}
