\documentclass{article}
\usepackage[english]{babel}
\usepackage{csquotes}

%%% Functional packages
\usepackage{amsmath, amssymb, amsthm}
\usepackage{physics, siunitx}
\usepackage{multirow}
\usepackage[subfolder]{gnuplottex}

%%% configuration packages
\usepackage{fullpage}

\usepackage[
    pdfusetitle,
    colorlinks
]{hyperref}

\usepackage[
    backend=biber,
    style=authoryear,
    autolang=other,
    sorting=none,
    mincrossrefs=100, % do not add parent collections
]{biblatex}
\addbibresource{moving_heat_source.bib}

\title{A moving point heat source}
\author{Oleg Rogozin}

\newcommand{\bx}{\vb*{x}}

\begin{document}
\maketitle

\section{General solution}

%%% Formulation of the problem
Let us consider a semi-infinite space ($\bx\in\mathbb{R}^3$, $z\geq0$) with a point heat source
moving with constant speed $v$ along the $x$ axis in plane $z=0$.
In the frame of this source, the temperature $T$ (\si{K}) is governed by the following heat equation:
\begin{equation}\label{eq:heat}
    \laplacian{T} + \frac{v}{a}\pdv{T}{x} = \frac{P}{k}\delta(\bx),
\end{equation}
where $a$ is thermal diffusivity (\si{\m\squared\per\s}),
$k$ is the thermal conductivity (\si{\W\per\K\per\m}),
$P$ is the heat source power (\si{\W}),
and $\delta$ is the Dirac delta function (\si{\per\m\cubed}).w

%%% Solutions
The quasi-steady-state solution of~\eqref{eq:heat} with boundary conditions
\begin{equation}\label{eq:bc}
    \begin{cases}
        T\to0 \text{ as } |\bx|\to\infty & \text{if } z>0, \\
        \pdv{T}{z}=0 & \text{if } z=0
    \end{cases}
\end{equation}
has the form~\autocite{levin2008general}
\begin{equation}\label{eq:solution1}
    T = \frac{Pv(1-\xi)}{4\pi ka}\qty( \frac1{\sqrt{\xi^2 +\rho^2}} - \frac1{\sqrt{(2-\xi)^2 +\rho^2}} ),
\end{equation}
where the following coordinates are introduced:
\begin{equation}
    \xi = 1 - \exp(-\frac{v}{a}|x|), \quad
    \rho = \frac{v}{a}\exp(-\frac{v}{a}|x|)\sqrt{y^2 + z^2}.
\end{equation}
The alternative solution
\begin{equation}\label{eq:solution2}
    T = \frac{P}{4\pi kr}\exp(-\frac{v(x+r)}{2a}),
\end{equation}
where $r = |\bx|$, is due to~\textcite{rosenthal1946theory} but is not symmetric.

%%% Nondimensionalization
The characteristic temperature and length are
\begin{equation}\label{eq:reference}
    T_0 = \frac{Pv}{4\pi ka}, \quad L = \frac{a}{v}.
\end{equation}
In the dimensionless form,
\begin{equation}\label{eq:hats}
    \hat{T} = T/T_0, \quad \hat{\bx} = \bx/L, \quad
    \xi = 1 - \exp(-|\hat{x}|), \quad \rho = \exp(-|\hat{x}|)\sqrt{\hat{y}^2 + \hat{z}^2},
\end{equation}
solutions~\eqref{eq:solution1} and~\eqref{eq:solution2} take the form
\begin{equation}\label{eq:hat_solutions}
    \hat{T} = (1-\xi)\qty( \frac1{\sqrt{\xi^2 +\rho^2}} - \frac1{\sqrt{(2-\xi)^2 +\rho^2}} ), \quad
    \hat{T} = \frac1{\hat{r}}\exp(-\frac{\hat{x}+\hat{r}}2).
\end{equation}
and shown in Figs.~\ref{fig:solutions_x} and~\ref{fig:solutions_r}.

\begin{figure}
    \centering
    \begin{gnuplot}[scale=1, terminal=epslatex, terminaloptions=color lw 4]
        set xrange [-3:3]
        set yrange [0:3]
        #set log y; set yrange [1e-3: 1e3]
        set grid
        set key center top spacing 1.5 outside
        xi(x) = 1 - exp(-abs(x))
        plot (1-xi(x))*(1/xi(x) - 1/(2-xi(x))) \
                title '$\hat{T} = (1-\xi)^2/\xi(2-\xi)$ \autocite{levin2008general}', \
            exp(-(abs(x)+x)/2)/abs(x) \
                title '$\hat{T} = \exp(-(|x|+x)/2)/|x|$ \autocite{rosenthal1946theory}'
    \end{gnuplot}
    \caption{
        Solutions~\eqref{eq:hat_solutions} for $y=z=0$.
    }
    \label{fig:solutions_x}
\end{figure}

\begin{figure}
    \centering
    \begin{gnuplot}[scale=1, terminal=epslatex, terminaloptions=color lw 4]
        set xrange [0:3]
        set yrange [0:3]
        #set log y; set yrange [1e-3: 1e3]
        set grid
        set key center top spacing 1.5 outside
        plot 1/x - 1/sqrt(4+x**2) \
                title '$\hat{T} = 1/\rho - 1/\sqrt{4+\rho^2}$ \autocite{levin2008general}', \
            exp(-x/2)/x \
                title '$\hat{T} = \exp(-r/2)/r$ \autocite{rosenthal1946theory}'
    \end{gnuplot}
    \caption{
        Solutions~\eqref{eq:hat_solutions} for $x=0$.
    }
    \label{fig:solutions_r}
\end{figure}

\section{Dimensional estimations}

\begin{table}
    \centering
    \caption{
        Dimensional parameters used in estimations taken as typical values
        for SS316L and Trumpf TruPrint 1000.
    }
    %\footnotesize
    \sisetup{per-mode=repeated-symbol}
    \label{table:parameters}
    \begin{tabular}{lccccc}
        \hline\noalign{\smallskip}
        Quantity & Symbol & Value & Unit & Formula \\[3pt]
        \hline\noalign{\smallskip}
        Laser power & $P$ & \num{113} & \si{\W} & \multirow{5}*{---}\\
        Thermal conductivity & $k$ & \num{10} & \si{\W\per\K\per\m} & \\
        Heat capacity & $c_p$ & \num{700} & \si{\J\per\kg\per\K} & \\
        Density & $\rho_0$ & \num{7500} & \si{\kg\per\m\cubed} & \\
        Melting temperature & $T_M$ & \num{1700} & \si{\K} & \\[3pt]
        \hline\noalign{\smallskip}
        Thermal diffusivity & $a$ & \num{1.90} & \si{\mm\squared\per\s} & $=k/(\rho_0 c_p)$ \\
        Reference temperature & $T_0$ & \num{330e3} & \si{\K} & \multirow{2}*{Eq.~\eqref{eq:reference}} \\
        Reference length & $L$ & \num{2.72} & \si{\um} & \\
        Melt-pool depth & $d$ & \num{20} & \si{\um} & Eq.~\eqref{eq:depth} \\
        \hline
    \end{tabular}
\end{table}

The melt-pool depth can be roughly estimated as a $r$-coordinate
that corresponds to the melting temperature $T_M$ at $x=0$.
For solution~\eqref{eq:solution1}, we have
\begin{equation}\label{eq:solution1_r}
    T_M = T_0\qty( \frac1{\rho} - \frac1{\sqrt{4 +\rho^2}} ),
\end{equation}
which can be simplified to 
\begin{equation}\label{eq:solution1_r2}
    T_M = \frac{2T_0}{\rho^3},
\end{equation}
under assumption $\rho^2 \gg 1$.
Finally, replacing here $\rho$ by $d/L$, we obtain the estimation for the melt-pool depth
\begin{equation}\label{eq:depth}
    d = L\qty(\frac{2T_0}{T_M})^{1/3} = \qty(\frac{kP}{2\pi T_M})^{1/3}\qty(\rho_0 c_p v)^{-2/3}.
\end{equation}
The typical values are presented in Table~\ref{table:parameters}.

\printbibliography

\end{document}
