\documentclass{article}
\usepackage[english]{babel}
\usepackage{csquotes}

%%% Functional packages
\usepackage{amsmath, amssymb, amsthm}
\usepackage{physics, siunitx}
\usepackage{graphicx}
    \graphicspath{{plots/}}
\usepackage{multirow}
\usepackage[subfolder]{gnuplottex}
\usepackage[referable]{threeparttablex} % for \tnotex

%%% configuration packages
%\usepackage{fullpage}

\usepackage[
    pdfusetitle,
    colorlinks
]{hyperref}

\usepackage[
    backend=biber,
    style=numeric,
    autolang=other,
    sorting=none,
    mincrossrefs=100, % do not add parent collections
]{biblatex}
\addbibresource{moving_heat_source.bib}

\title{A moving point heat source}
\author{Oleg Rogozin}

%%% Problem-specific aliases
\newcommand{\liq}{\text{L}}
\newcommand{\sol}{\text{S}}

%%% Bold symbols
\newcommand{\bx}{\vb*{x}}

\begin{document}
\maketitle

\section{General solution}

%%% Formulation of the problem
Let us consider a semi-infinite space ($\bx\in\mathbb{R}^3$, $z\geq0$) with a point heat source
moving with constant speed $v$ along the $x$ axis in plane $z=0$.
In the frame of this source, the temperature $T$ (\si{K}) is governed by the following heat equation:
\begin{equation}\label{eq:heat}
    \laplacian{T} + \frac{v}{a}\pdv{T}{x} = \frac{P}{k}\delta(\bx),
\end{equation}
where $a$ is thermal diffusivity (\si{\m\squared\per\s}),
$k$ is the thermal conductivity (\si{\W\per\K\per\m}),
$P$ is the heat source power (\si{\W}),
and $\delta$ is the Dirac delta function (\si{\per\m\cubed}).

%%% Solutions
For the natural boundary conditions
\begin{equation}\label{eq:bc}
    \begin{cases}
        T\to0 \text{ as } |\bx|\to\infty & \text{if } z>0, \\
        \pdv{T}{z}=0 & \text{if } z=0,
    \end{cases}
\end{equation}
the well-known solution
\begin{equation}\label{eq:TR}
    T_R = \frac{P}{4\pi kr}\exp(-\frac{v(x+r)}{2a}),
\end{equation}
where $r = |\bx|$, is due to~\textcite{rosenthal1946theory}, but it decays too slowly as $x\to-\infty$.
\textcite{levin2008general} eliminated this drawback and found an alternative solution
\begin{equation}\label{eq:TL}
    T_L = \frac{Pv(1-\xi)}{4\pi ka}\qty( \frac1{\sqrt{\xi^2 +\rho^2}} - \frac1{\sqrt{(2-\xi)^2 +\rho^2}} ),
\end{equation}
where the following coordinates are introduced:
\begin{equation}
    \xi = 1 - \exp(-\frac{v}{a}x), \quad
    \rho = \frac{v}{a}\exp(-\frac{v}{a}x)\sqrt{y^2 + z^2}.
\end{equation}

%%% Nondimensionalization
The characteristic temperature and length are
\begin{equation}\label{eq:reference}
    T_0 = \frac{Pv}{4\pi ka}, \quad L = \frac{a}{v}.
\end{equation}
In the nondimensional variables given by $\hat{T} = T/T_0$ and $\hat{\bx} = \bx/L$,
\begin{equation}\label{eq:xi-rho}
    \xi = 1 - \exp(-\hat{x}), \quad \rho = \exp(-\hat{x})\sqrt{\hat{y}^2 + \hat{z}^2},
\end{equation}
and solutions~\eqref{eq:TR},~\eqref{eq:TL} take the form
\begin{gather}
    \hat{T}_R = \frac1{\hat{r}}\exp(-\frac{\hat{x}+\hat{r}}2), \label{eq:hatTR}\\
    \hat{T}_L = (1-\xi)\qty( \frac1{\sqrt{\xi^2 +\rho^2}} - \frac1{\sqrt{(2-\xi)^2 +\rho^2}} ). \label{eq:hatTL}
\end{gather}
They are shown in Figs.~\ref{fig:solutions_x} and~\ref{fig:solutions_r}.

\begin{figure}
    \centering
    \begin{gnuplot}[scale=1, terminal=epslatex, terminaloptions=color lw 4]
        set xrange [-3:3]
        set yrange [0:3]
        #set log y; set yrange [1e-3: 1e3]
        set grid
        set key center top spacing 1.5 outside
        set xlabel '$\hat{x}$'
        set ylabel '$\hat{T}$' rotate by 0
        xi(x) = 1 - exp(-x)
        plot exp(-(abs(x)+x)/2)/abs(x) \
                title '$\hat{T}_R = \exp(-(|x|+x)/2)/|x|$ \textcite{rosenthal1946theory}', \
            (1-xi(x))*(1/abs(xi(x)) - 1/(2-xi(x))) \
                title '$\hat{T}_L = (1-\xi)(|\xi|^{-1}-(2-\xi)^{-1})$ \textcite{levin2008general}'
    \end{gnuplot}
    \vspace{-10pt}
    \caption{
        Solutions~\eqref{eq:hatTR} and~\eqref{eq:hatTL} at $y=z=0$.
    }
    \label{fig:solutions_x}
\end{figure}

\begin{figure}
    \centering
    \begin{gnuplot}[scale=1, terminal=epslatex, terminaloptions=color lw 4]
        set xrange [0:3]
        set yrange [0:3]
        #set log y; set yrange [1e-3: 1e3]
        set grid
        set key center top spacing 1.5 outside
        set xlabel '$\hat{z}$'
        set ylabel '$\hat{T}$' rotate by 0
        plot exp(-x/2)/x \
                title '$\hat{T}_R = \exp(-r/2)/r$ \textcite{rosenthal1946theory}', \
            1/x - 1/sqrt(4+x**2) \
                title '$\hat{T}_L = 1/\rho - 1/\sqrt{4+\rho^2}$ \textcite{levin2008general}'
    \end{gnuplot}
    \vspace{-10pt}
    \caption{
        Solutions~\eqref{eq:hatTR} and~\eqref{eq:hatTL} at $x=0$.
    }
    \label{fig:solutions_r}
\end{figure}

\section{Example and discussion}

\begin{figure}
    \centering
    \includegraphics[width=\textwidth]{boundary}
    \caption{
        The melt-pool boundary obtained as the temperature contour line for $\hat{T}=T_M/T_0=0.0378$,
        where $T_M$ and $T_0$ are taken from Table~\ref{table:parameters}.
        Solutions~\eqref{eq:hatTR} and~\eqref{eq:hatTL} are shown.
        The dimensions of the melt pool are as follows:
        $\hat{l} = 28.9$ and $\hat{d} = 5.96$ for Rosenthal's solution,
        $\hat{l} = 5.96$ and $\hat{d} = 4.46$ for Levin's solution,
        where $\hat{l} = \hat{x}_\text{max} - \hat{x}_\text{min}$ is the length and $\hat{d}$ is the depth.
    }\label{fig:boundary}
\end{figure}

%%% Melt-pool boundary
Although the difference between the two solutions seems small
in Figs.~\ref{fig:solutions_x} and~\ref{fig:solutions_r},
the predicted melt-pool boundaries differ considerably as seen in Fig.~\ref{fig:boundary}.
The parameters of the material and process are specified in Table~\ref{table:parameters}.
The actual shape of the melt pool under the considered conditions is closer to that
given by Levin's solution~\eqref{eq:hatTL}.

\begin{table}
    \centering
    \begin{threeparttable}[b]
    \caption{
        Dimensional parameters taken as typical values for stainless steel 316L printed on Trumpf TruPrint 1000
        and estimations based on them.
        The tabular values are taken from~\cite{kim1975thermophysical,pichler2020measurements}.
    }
    %\footnotesize
    \sisetup{per-mode=symbol}
    \label{table:parameters}
    \begin{tabular}{lccccc}
        \hline\noalign{\smallskip}
        Quantity & Symbol & Value & Unit & Formula \\[2pt]
        \hline\noalign{\smallskip}
        Laser power & $P$ & \num{113} & \si{\W} & \multirow{2}*{---} \\
        Scanning speed & $v$ & \num{0.7} & \si{\m\per\s} & \\[2pt]
        \hline\noalign{\smallskip}
        Thermal conductivity & $k$ & \num{27.0} & \si{\W\per\K\per\m} & $=(35.96+17.98)/2$\tnotex{a} \\
        Heat capacity & $c_p$ & \num{727} & \si{\J\per\kg\per\K} & $=(684.8+769.9)/2$\tnotex{a} \\
        Density & $\rho_0$ & \num{7120} & \si{\kg\per\m\cubed} & $=(7270+6975)/2$\tnotex{a} \\
        Melting temperature & $T_M$ & \num{1690} & \si{\K} & $=(1675+1708)/2$\tnotex{a} \\
        Boiling temperature & $T_B$ & \num{3090} & \si{\K} & --- \\[2pt]
        \hline\noalign{\smallskip}
        Thermal diffusivity & $a$ & \num{5.22} & \si{\mm\squared\per\s} & $=k/(\rho_0 c_p)$ \\
        Reference temperature & $T_0$ & \num{44700} & \si{\K} & \multirow{2}*{Eq.~\eqref{eq:reference}} \\
        Reference length & $L$ & \num{7.46} & \si{\um} & \\[2pt]
        \hline\noalign{\smallskip}
        \multirow{2}*{Melt-pool depth} & \multirow{2}*{$d$} & \num{28.0} & \multirow{2}*{\si{\um}} & Eq.~\eqref{eq:depth} \\
        & & \num{33.6} & & Fig.~\ref{fig:boundary}\tnotex{b} \\[2pt]
        \multirow{2}*{Melt-pool length} & \multirow{2}*{$l$} & \num{44.4} & \multirow{2}*{\si{\um}} & Eq.~\eqref{eq:length} \\
        & & \num{44.7} & & Fig.~\ref{fig:boundary}\tnotex{b} \\[2pt]
        \multirow{3}*{Temperature gradient} & \multirow{3}*{$\nabla T$} & \num{41.6} & \multirow{3}*{\si{\K\per\um}} & $=(T_B-T_M)/d$ \\
        & & \num{94.7} & & $\min(\nabla T)$ in Fig.~\ref{fig:gradient}\tnotex{b} \\
        & & \num{400} & & $\max(\nabla T)$ in Fig.~\ref{fig:gradient}\tnotex{b} \\[2pt]
        Cooling rate & $\dot{T}$ & \num{155} & \si{\K\per\us} & $\max(\dot{T})$ in Fig.~\ref{fig:rate}\tnotex{b} \\[2pt]
        \hline
    \end{tabular}
    \begin{tablenotes}
        \item[a]\label{a} Values are taken as half-sums of the solidus and liquidus values.
        \item[b]\label{b} Estimations are based on Levin's solution.
    \end{tablenotes}
    \end{threeparttable}
\end{table}

%%% Estimations
The melt-pool depth can be roughly estimated as a $r$-coordinate
that corresponds to the melting temperature $T_M$ at $x=0$.
For solution~\eqref{eq:TL}, we have
\begin{equation}\label{eq:TL_r}
    T_M = T_0\qty( \frac1{\rho} - \frac1{\sqrt{4 +\rho^2}} ),
\end{equation}
which can be simplified to
\begin{equation}\label{eq:TL_r2}
    T_M = \frac{2T_0}{\rho^3},
\end{equation}
under the assumption $\rho^2 \gg 1$.
Finally, replacing here $\rho$ by $d/L$ and $a$ by $k/(\rho_0 c_p)$, we obtain the estimation for the melt-pool depth
\begin{equation}\label{eq:depth}
    d = L\qty(\frac{2T_0}{T_M})^{1/3} = \qty(\frac{kP}{2\pi T_M})^{1/3}\qty(\rho_0 c_p v)^{-2/3}.
\end{equation}
The similar estimations can be obtained for the melt-pool length
\begin{equation}\label{eq:length}
    l = \frac32L\ln\frac{2T_0}{T_M} = \frac32\frac{k}{\rho_0 c_p v}\ln\frac{P\rho_0 c_p v}{2\pi k^2 T_M},
\end{equation}
which follows from the asymptotic relations
\begin{equation}\label{eq:TL_x}
    T_L \to \begin{cases}
        2(1-\xi)^2 &\text{ as } \xi\to1-0, \\
        -2/\xi &\text{ as } \xi\to-\infty.
    \end{cases}
\end{equation}
The resulting estimations are included in Table~\ref{table:parameters}.
Comparing with the exact values presented in Fig.~\ref{fig:boundary},
it is seen that Eq.~\eqref{eq:length} are quite accurate,
while Eq.~\eqref{eq:depth} gives an error of more than 15\% error
mostly because the deepest point does not lie on the $x$ axis.

\begin{figure}
    \centering
    \includegraphics[width=\textwidth]{gradient}
    \caption{
        Nondimensional temperature gradient along the melt-pool boundary for $\hat{T}=T_M/T_0=0.0378$,
        where $T_M$ and $T_0$ are taken from Table~\ref{table:parameters}.
        Solutions~\eqref{eq:hatTR} and~\eqref{eq:hatTL} are shown.
        The coordinates $s=0$ and $s=1$ correspond to points $\hat{x}_\text{min}$ and $\hat{x}_\text{max}$, respectively.
        The middle black points are the deepest points of the melt pool and divide the profiles into two parts,
        where solidification (solid line) and fusion (dashed line) occur.
        The values on the solidification part of the curves vary in the following ranges:
        $0.00143 \leq \hat{\nabla}\hat{T} \leq 0.0113$ for Rosenthal's solution,
        $0.0168 \leq \hat{\nabla}\hat{T} \leq 0.068$ for Levin's solution,
        and $\hat{\nabla}\hat{T} = ((\partial\hat{T}/\partial\hat{x})^2 + (\partial\hat{T}/\partial\hat{z})^2)^{1/2}$.
    }\label{fig:gradient}
\end{figure}

\begin{figure}
    \centering
    \includegraphics[width=\textwidth]{speed}
    \caption{
        Nondimensional growth rate along the melt-pool boundary.
        Solidification occurs at positive values of $\cos\phi$.
    }\label{fig:speed}
\end{figure}

\begin{figure}
    \centering
    \includegraphics[width=\textwidth]{rate}
    \caption{
        Nondimensional cooling rate along the melt-pool boundary.
        The values are limited as follows:
        $-0.054\leq \cos\phi\hat{\nabla}\hat{T} \leq 0.00143$ for Rosenthal's solution,
        $-0.077\leq \cos\phi\hat{\nabla}\hat{T} \leq 0.038$ for Levin's solution,
        where the negative values correspond to the heating rate at the fusion line.
    }\label{fig:rate}
\end{figure}

%%% Temperature gradients
The temperature gradient can be evaluated using the following formulas:
\begin{gather}
    \pdv{\hat{T}_R}{\hat{x}} =
        -\hat{T}_R\qty( \frac{\hat{x}}{\hat{r}^2} + \frac12\qty(1 + \frac{\hat{x}}{\hat{r}}) ), \label{eq:dTRdx}\\
    \pdv{\hat{T}_R}{\hat{x}} =
        -\hat{T}_R\frac{\hat{x}}{\hat{r}}\qty( \frac1{\hat{r}} + \frac12 ), \label{eq:dTRdz}\\
    \pdv{\hat{T}_L}{\hat{x}} =
        (1-\xi)\qty( \frac{2-\xi}{((2-\xi)^2 + \rho^2)^{3/2}} - \frac{\xi}{(\xi^2 + \rho^2)^{3/2}} ), \label{eq:dTLdx}\\
    \pdv{\hat{T}_L}{\hat{z}} =
        \rho(1-\xi)^2\qty( \frac1{((2-\xi)^2 + \rho^2)^{3/2}} - \frac1{(\xi^2 + \rho^2)^{3/2}} ). \label{eq:dTLdz}
\end{gather}
Its magnitude along the melt-pool boundary is shown in Fig.~\ref{fig:gradient},
where $s$ in the normalized arc-length coordinate defined as
\begin{equation}\label{eq:s}
    s(\hat{x}) = \frac1{s_0}\int_{\hat{x}_\text{min}}^{\hat{x}}\sqrt{1+\tan^{-1}\phi}\dd{x}',
        \quad s_0 = s(\hat{x}_\text{max}),
\end{equation}
and $\phi$ is the angle between the velocity vector and temperature gradient, i.e.,
\begin{equation}\label{eq:phi}
    \phi = \arctan\qty(\frac{\partial\hat{T}/\partial\hat{z}}{\partial\hat{T}/\partial\hat{x}}).
\end{equation}
It can be seen that Rosenthal's solution predicts a much smaller temperature gradient
on the solidification section of the curve.

%%% Growth and cooling rate
The growth rate at the melt-pool boundary is equal to $v\cos\phi$.
It varies from $v$ at the left end to $-v$ at the right one as seen in Fig.~\ref{fig:speed}.
The negative values correspond to the fusion speed.
The cooling rate of the melt-pool boundary
\begin{equation}\label{eq:dotT}
    \dot{T} = v\cos\phi\nabla T,
\end{equation}
which profile is shown in Fig.~\ref{fig:rate}.

\begin{figure}
    \centering
    \includegraphics[width=\textwidth]{bifurcation}
    \caption{
        The bifurcation diagram for the linear stability of a planar solidification front
        in the dimensionless $(\hat{V},\hat{G})$ coordinates for the binary mixture
        with properties specified in Table~\ref{table:microstructure}.
        The temperature gradient and growth rate profiles along the melt-pool boundary,
        which are shown in Figs.~\ref{fig:gradient} and~\ref{fig:speed}, are projected on the diagram.
        The segments of the curves lying inside the hatched region of instability correspond to the case
        of microstructure formation. In the stability region, solidification proceeds homogeneously.
        Thin dashed lines correspond to the constitutional supercooling ($\hat{G} = \hat{V}$)
        and absolute stability ($\hat{V} = 1/K$) limits.
    }\label{fig:bifurcation}
\end{figure}

\begin{figure}
    \centering
    \includegraphics[width=\textwidth]{wavelength}
    \caption{
        The nondimensional wavelength $\hat\lambda=\lambda/d_0$ predicted by the linear stability theory
        along the solidification part of the melt-pool boundary.
        The values are in the following ranges:
        $17.8\leq \hat\lambda \leq 103.5$ for Rosenthal's solution,
        $0\leq \hat\lambda \leq 33.8$ for Levin's solution.
        Zero values correspond to a stable planar solidification front.
    }\label{fig:wavelength}
\end{figure}

\begin{table}
    \centering
    \caption{
        Parameters of stainless steel 316L used for microstructure predictions.
        The estimations of $\lambda$ are based on Levin's solution.
    }
    %\footnotesize
    \sisetup{per-mode=symbol}
    \label{table:microstructure}
    \begin{tabular}{lccccc}
        \hline\noalign{\smallskip}
        Quantity & Symbol & Value & Unit & Formula/Reference \\[2pt]
        \hline\noalign{\smallskip}
        Enthalpy of fusion & $H$ & \num{290} & \si{\kJ\per\kg} & \cite{pichler2020measurements} \\
        Freezing range & $\Delta{T}_0$ & \num{33.0} & \si{\K} & \cite{pichler2020measurements} \\
        Surface energy & $\sigma$ & \num{0.26} & \si{\J\per\m\squared} & \cite{bobadilla1988influence} \\
        Solute diffusion in liquid & $D_\liq$ & \num{2e-9} & \si{\m\squared\per\s} & --- \\
        Partition coefficient & $K$ & \num{0.4} & --- & --- \\[2pt]
        \hline\noalign{\smallskip}
        Gibbs--Thomson coefficient & $\Gamma$ & \num{21.3} & \si{\K\um} & $=\sigma T_M/(\rho H)$ \\
        Capillary length & $d_0$ & \num{6.45} & \si{\nm} & $=\Gamma/\Delta{T}_0$ \\[2pt]
        \multirow{2}*{Coupling coefficients} & $\alpha_V$ & \num{2.26} & \multirow{2}*{---} & \multirow{2}*{Eq.~\eqref{eq:alphaVG}} \\
        & $\alpha_G$ & \num{1.17} & & \\[2pt]
        \hline\noalign{\smallskip}
        \multirow{2}*{Most unstable wavelength} & \multirow{2}*{$\lambda$} & \num{0} & \multirow{2}*{\si{\um}} & $\min(\lambda)$ in Fig.~\ref{fig:wavelength} \\
        & & \num{0.22} & & $\max(\lambda)$ in Fig.~\ref{fig:wavelength} \\[2pt]
        \hline
    \end{tabular}
\end{table}

%%% Microstructure predictions
Based on the obtained profiles of temperature gradient and cooling rate,
one can predict some properties of the microstructure in the solidified material.
In particular, the analysis of the linear stability of a flat solidification front allows us
to determine the conditions under which a cellular structure forms and to estimate the resulting cell spacing.
Such conditions and estimations based on the binary-mixture assumption are presented
in Figs.~\ref{fig:bifurcation} and~\ref{fig:wavelength}, where
\begin{equation}\label{eq:hatVG}
    \hat{V} = \alpha_V\cos\phi, \quad \hat{G} = \alpha_G\nabla{T},
\end{equation}
and
\begin{equation}\label{eq:alphaVG}
    \alpha_V = \frac{vd_0}{D_\liq}, \quad \alpha_G = \frac{d_0T_0}{L\Delta{T}_0}
\end{equation}
are the dimensionless coupling coefficients.
Here, $d_0$ is the capillary length (\si{\m}),
$D_\liq$ is the solute diffusion coefficient in liquid (\si{\m\squared\per\s}),
and $\Delta{T}_0$ is the freezing range of the alloy (\si{\K}).

\printbibliography

\end{document}
