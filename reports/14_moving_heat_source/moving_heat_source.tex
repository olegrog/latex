\documentclass{article}
\usepackage[english]{babel}
\usepackage{csquotes}

%%% Functional packages
\usepackage{amsmath, amssymb, amsthm}
\usepackage{physics, siunitx}
\usepackage{graphicx}
    \graphicspath{{plots/}}
\usepackage{multirow}
\usepackage[subfolder]{gnuplottex}
\usepackage[referable]{threeparttablex} % for \tnotex

%%% configuration packages
%\usepackage{fullpage}

\usepackage[
    pdfusetitle,
    colorlinks
]{hyperref}

\usepackage[
    backend=biber,
    style=authoryear,
    autolang=other,
    sorting=none,
    mincrossrefs=100, % do not add parent collections
]{biblatex}
\addbibresource{moving_heat_source.bib}

\title{A moving point heat source}
\author{Oleg Rogozin}

\newcommand{\bx}{\vb*{x}}

\begin{document}
\maketitle

\section{General solution}

%%% Formulation of the problem
Let us consider a semi-infinite space ($\bx\in\mathbb{R}^3$, $z\geq0$) with a point heat source
moving with constant speed $v$ along the $x$ axis in plane $z=0$.
In the frame of this source, the temperature $T$ (\si{K}) is governed by the following heat equation:
\begin{equation}\label{eq:heat}
    \laplacian{T} + \frac{v}{a}\pdv{T}{x} = \frac{P}{k}\delta(\bx),
\end{equation}
where $a$ is thermal diffusivity (\si{\m\squared\per\s}),
$k$ is the thermal conductivity (\si{\W\per\K\per\m}),
$P$ is the heat source power (\si{\W}),
and $\delta$ is the Dirac delta function (\si{\per\m\cubed}).

%%% Solutions
The quasi-steady-state solution of~\eqref{eq:heat} with the boundary conditions
\begin{equation}\label{eq:bc}
    \begin{cases}
        T\to0 \text{ as } |\bx|\to\infty & \text{if } z>0, \\
        \pdv{T}{z}=0 & \text{if } z=0,
    \end{cases}
\end{equation}
has the form~\autocite{levin2008general}
\begin{equation}\label{eq:TL}
    T_L = \frac{Pv(1-\xi)}{4\pi ka}\qty( \frac1{\sqrt{\xi^2 +\rho^2}} - \frac1{\sqrt{(2-\xi)^2 +\rho^2}} ),
\end{equation}
where the following coordinates are introduced:
\begin{equation}
    \xi = 1 - \exp(-\frac{v}{a}x), \quad
    \rho = \frac{v}{a}\exp(-\frac{v}{a}x)\sqrt{y^2 + z^2}.
\end{equation}
The alternative solution
\begin{equation}\label{eq:TR}
    T_R = \frac{P}{4\pi kr}\exp(-\frac{v(x+r)}{2a}),
\end{equation}
where $r = |\bx|$, is due to~\textcite{rosenthal1946theory}, but it decays too slowly as $x\to-\infty$.

%%% Nondimensionalization
The characteristic temperature and length are
\begin{equation}\label{eq:reference}
    T_0 = \frac{Pv}{4\pi ka}, \quad L = \frac{a}{v}.
\end{equation}
In the nondimensional variables given by $\hat{T} = T/T_0$ and $\hat{\bx} = \bx/L$,
\begin{equation}\label{eq:xi-rho}
    \xi = 1 - \exp(-\hat{x}), \quad \rho = \exp(-\hat{x})\sqrt{\hat{y}^2 + \hat{z}^2},
\end{equation}
and solutions~\eqref{eq:TL},~\eqref{eq:TR} take the form
\begin{gather}
    \hat{T}_L = (1-\xi)\qty( \frac1{\sqrt{\xi^2 +\rho^2}} - \frac1{\sqrt{(2-\xi)^2 +\rho^2}} ), \label{eq:hatTL}\\
    \hat{T}_R = \frac1{\hat{r}}\exp(-\frac{\hat{x}+\hat{r}}2). \label{eq:hatTR}
\end{gather}
They are shown in Figs.~\ref{fig:solutions_x} and~\ref{fig:solutions_r}.

\begin{figure}
    \centering
    \begin{gnuplot}[scale=1, terminal=epslatex, terminaloptions=color lw 4]
        set xrange [-3:3]
        set yrange [0:3]
        #set log y; set yrange [1e-3: 1e3]
        set grid
        set key center top spacing 1.5 outside
        set xlabel '$\hat{x}$'
        set ylabel '$\hat{T}$' rotate by 0
        xi(x) = 1 - exp(-x)
        plot (1-xi(x))*(1/abs(xi(x)) - 1/(2-xi(x))) \
                title '$\hat{T}_L = (1-\xi)(|\xi|^{-1}-(2-\xi)^{-1})$ \autocite{levin2008general}', \
            exp(-(abs(x)+x)/2)/abs(x) \
                title '$\hat{T}_R = \exp(-(|x|+x)/2)/|x|$ \autocite{rosenthal1946theory}'
    \end{gnuplot}
    \vspace{-10pt}
    \caption{
        Solutions~\eqref{eq:hatTL} and~\eqref{eq:hatTR} at $y=z=0$.
    }
    \label{fig:solutions_x}
\end{figure}

\begin{figure}
    \centering
    \begin{gnuplot}[scale=1, terminal=epslatex, terminaloptions=color lw 4]
        set xrange [0:3]
        set yrange [0:3]
        #set log y; set yrange [1e-3: 1e3]
        set grid
        set key center top spacing 1.5 outside
        set xlabel '$\hat{z}$'
        set ylabel '$\hat{T}$' rotate by 0
        plot 1/x - 1/sqrt(4+x**2) \
                title '$\hat{T}_L = 1/\rho - 1/\sqrt{4+\rho^2}$ \autocite{levin2008general}', \
            exp(-x/2)/x \
                title '$\hat{T}_R = \exp(-r/2)/r$ \autocite{rosenthal1946theory}'
    \end{gnuplot}
    \vspace{-10pt}
    \caption{
        Solutions~\eqref{eq:hatTL} and~\eqref{eq:hatTR} at $x=0$.
    }
    \label{fig:solutions_r}
\end{figure}

\section{Example and discussion}

\begin{figure}
    \centering
    \includegraphics[width=\textwidth]{boundary}
    \caption{
        The melt-pool boundary obtained as the temperature contour line for $\hat{T}=T_M/T_0=0.0378$,
        where $T_M$ and $T_0$ are taken from Table~\ref{table:parameters}.
        Solutions~\eqref{eq:hatTL} and~\eqref{eq:hatTR} are shown.
        The dimensions of the melt pool are as follows:
        $\hat{l} = 6.0$ and $\hat{d} = 4.5$ for Levin's model,
        $\hat{l} = 29$ and $\hat{d} = 6.0$ for Rosenthal's model,
        where $\hat{l} = \hat{x}_\text{max} - \hat{x}_\text{min}$ is the length and $\hat{d}$ is the depth.
    }\label{fig:boundary}
\end{figure}

%%% Melt-pool boundary
Despite the fact that the difference between the two solutions seems small
in Figs.~\ref{fig:solutions_x} and~\ref{fig:solutions_r},
the predicted melt-pool boundaries differ considerably as seen in Fig.~\ref{fig:boundary}.
The parameters of the material and process are presented in Table~\ref{table:parameters}.
The actual shape of the melt pool under the specified conditions is closer to that
given by Levin's solution~\eqref{eq:hatTL}.

\begin{table}
    \centering
    \begin{threeparttable}[b]
    \caption{
        Dimensional parameters taken as typical values for stainless steel 316L printed on Trumpf TruPrint 1000
        and estimations based on them.
    }
    %\footnotesize
    \sisetup{per-mode=repeated-symbol}
    \label{table:parameters}
    \begin{tabular}{lccccc}
        \hline\noalign{\smallskip}
        Quantity & Symbol & Value & Unit & Formula \\[2pt]
        \hline\noalign{\smallskip}
        Laser power & $P$ & \num{113} & \si{\W} & \multirow{2}*{---} \\
        Scanning speed & $v$ & \num{0.7} & \si{\m\per\s} & \\[2pt]
        \hline\noalign{\smallskip}
        Thermal conductivity & $k$ & \num{27.0} & \si{\W\per\K\per\m} & $=(35.96+17.98)/2$\tnotex{a} \\
        Heat capacity & $c_p$ & \num{727} & \si{\J\per\kg\per\K} & $=(684.8+769.9)/2$\tnotex{a} \\
        Density & $\rho_0$ & \num{7120} & \si{\kg\per\m\cubed} & $=(7270+6975)/2$\tnotex{a} \\
        Melting temperature & $T_M$ & \num{1690} & \si{\K} & $=(1675+1708)/2$\tnotex{a} \\
        Boiling temperature & $T_B$ & \num{3090} & \si{\K} & --- \\[2pt]
        \hline\noalign{\smallskip}
        Thermal diffusivity & $a$ & \num{5.22} & \si{\mm\squared\per\s} & $=k/(\rho_0 c_p)$ \\
        Reference temperature & $T_0$ & \num{44700} & \si{\K} & \multirow{2}*{Eq.~\eqref{eq:reference}} \\
        Reference length & $L$ & \num{7.46} & \si{\um} & \\[2pt]
        \hline\noalign{\smallskip}
        \multirow{2}*{Melt-pool depth} & \multirow{2}*{$d$} & \num{28.0} & \multirow{2}*{\si{\um}} & Eq.~\eqref{eq:depth} \\
        & & \num{33.6} & & Fig.~\ref{fig:boundary} \\[2pt]
        \multirow{2}*{Melt-pool length} & \multirow{2}*{$l$} & \num{44.4} & \multirow{2}*{\si{\um}} & Eq.~\eqref{eq:length} \\
        & & \num{44.7} & & Fig.~\ref{fig:boundary} \\[2pt]
        \multirow{3}*{Temperature gradient} & \multirow{3}*{$\nabla T$} & \num{41.6} & \multirow{3}*{\si{\K\per\um}} & $=(T_B-T_M)/d$ \\
        & & \num{94.7} & & $\min(\nabla T)$ in Fig.~\ref{fig:gradient} \\
        & & \num{400} & & $\max(\nabla T)$ in Fig.~\ref{fig:gradient} \\[2pt]
        Cooling rate & $\dot{T}$ & \num{155} & \si{\K\per\us} & $\max(\dot{T})$ in Fig.~\ref{fig:rate} \\[2pt]
        \hline
    \end{tabular}
    \begin{tablenotes}
        \item[a]\label{a} Values are taken as half-sums of the solidus and liquidus values.
    \end{tablenotes}
    \end{threeparttable}
\end{table}

%%% Estimations
The melt-pool depth can be roughly estimated as a $r$-coordinate
that corresponds to the melting temperature $T_M$ at $x=0$.
For solution~\eqref{eq:TL}, we have
\begin{equation}\label{eq:TL_r}
    T_M = T_0\qty( \frac1{\rho} - \frac1{\sqrt{4 +\rho^2}} ),
\end{equation}
which can be simplified to
\begin{equation}\label{eq:TL_r2}
    T_M = \frac{2T_0}{\rho^3},
\end{equation}
under the assumption $\rho^2 \gg 1$.
Finally, replacing here $\rho$ by $d/L$ and $a$ by $k/(\rho_0 c_p)$, we obtain the estimation for the melt-pool depth
\begin{equation}\label{eq:depth}
    d = L\qty(\frac{2T_0}{T_M})^{1/3} = \qty(\frac{kP}{2\pi T_M})^{1/3}\qty(\rho_0 c_p v)^{-2/3}.
\end{equation}
The similar estimations can be obtained for the melt-pool length
\begin{equation}\label{eq:length}
    l = \frac32L\ln\frac{2T_0}{T_M} = \frac32\frac{k}{\rho_0 c_p v}\ln\frac{P\rho_0 c_p v}{2\pi k^2 T_M},
\end{equation}
which follows from the asymptotic relations
\begin{equation}\label{eq:TL_x}
    T_L \to \begin{cases}
        2(1-\xi)^2 &\text{ as } \xi\to1-0, \\
        -2/\xi &\text{ as } \xi\to-\infty.
    \end{cases}
\end{equation}
The resulting estimations are included in Table~\ref{table:parameters}.
Comparing with the exact values presented in Fig.~\ref{fig:boundary},
it is seen that Eq.~\eqref{eq:length} are quite accurate,
while Eq.~\eqref{eq:depth} gives an error of more than 15\% error
mostly because the deepest point does not lie on the $x$ axis.

\begin{figure}
    \centering
    \includegraphics[width=\textwidth]{gradient}
    \caption{
        The temperature gradient along the melt-pool boundary for $\hat{T}=T_M/T_0=0.0378$,
        where $T_M$ and $T_0$ are taken from Table~\ref{table:parameters}.
        Solutions~\eqref{eq:hatTL} and~\eqref{eq:hatTR} are shown.
        The coordinates $s=0$ and $s=1$ correspond to points $\hat{x}_\text{min}$ and $\hat{x}_\text{max}$, respectively.
        The middle black points are the deepest points of the melt pool and divide the profiles into two parts,
        where solidification (solid line) and fusion (dashed line) occur.
        The values on the solidification part of the curves vary in the following ranges:
        $0.016 \leq \hat{\nabla}\hat{T} \leq 0.067$ for Levin's model,
        $0.0014 \leq \hat{\nabla}\hat{T} \leq 0.011$ for Rosenthal's model,
        where $\hat{\nabla}\hat{T} = ((\partial\hat{T}/\partial\hat{x})^2 + (\partial\hat{T}/\partial\hat{z})^2)^{1/2}$.
    }\label{fig:gradient}
\end{figure}

\begin{figure}
    \centering
    \includegraphics[width=\textwidth]{speed}
    \caption{
        The nondimensional solidification speed along the melt-pool boundary.
        Solidification occurs at positive values.
    }\label{fig:speed}
\end{figure}

\begin{figure}
    \centering
    \includegraphics[width=\textwidth]{rate}
    \caption{
        The nondimensional cooling rate along the melt-pool boundary.
        The values on the solidification part of the curves are limited as follows:
        $\hat{\nabla}\hat{T}\cos\phi \leq 0.037$ for Levin's model,
        $\hat{\nabla}\hat{T}\cos\phi \leq 0.0013$ for Rosenthal's model.
    }\label{fig:rate}
\end{figure}

%%% Temperature gradients
The temperature gradient can be evaluated using the following formulas:
\begin{gather}
    \pdv{\hat{T}_L}{\hat{x}} =
        (1-\xi)\qty( \frac{2-\xi}{((2-\xi)^2 + \rho^2)^{3/2}} - \frac{\xi}{(\xi^2 + \rho^2)^{3/2}} ), \label{eq:dTLdx}\\
    \pdv{\hat{T}_L}{\hat{z}} =
        \rho(1-\xi)^2\qty( \frac1{((2-\xi)^2 + \rho^2)^{3/2}} - \frac1{(\xi^2 + \rho^2)^{3/2}} ), \label{eq:dTLdz}\\
    \pdv{\hat{T}_R}{\hat{x}} =
        -\hat{T}_R\qty( \frac{\hat{x}}{\hat{r}^2} + \frac12\qty(1 + \frac{\hat{x}}{\hat{r}}) ), \label{eq:dTRdx}\\
    \pdv{\hat{T}_R}{\hat{x}} =
        -\hat{T}_R\frac{\hat{x}}{\hat{r}}\qty( \frac1{\hat{r}} + \frac12 ). \label{eq:dTRdz}
\end{gather}
Its magnitude along the melt-pool boundary is shown in Fig.~\ref{fig:gradient}.
It can be seen that Rosenthal's model predicts a much smaller temperature gradient
on the solidification section of the curve.

%%% Cooling rate
The solidification speed of the melt-pool boundary is equal to $v\cos\phi$ (see Fig.~\ref{fig:speed}),
where $\phi$ is the angle between the velocity vector and temperature gradient:
\begin{equation}\label{eq:phi}
    \phi = \arctan\qty(\frac{\partial\hat{T}/\partial\hat{z}}{\partial\hat{T}/\partial\hat{x}}).
\end{equation}
It varies from $v$ at the left end to $-v$ at the right one.
The negative values correspond to the fusion speed.
The cooling rate of the melt-pool boundary
\begin{equation}\label{eq:dotT}
    \dot{T} = v\cos\phi\nabla T,
\end{equation}
which profile is shown in Fig.~\ref{fig:rate}.

\printbibliography

\end{document}
