%&pdflatex
\documentclass{article}
\usepackage[utf8]{inputenc}
\usepackage[T2A]{fontenc}
\usepackage[english,french,russian]{babel}

\usepackage{amssymb, amsmath, amsthm}
\usepackage{fullpage}
\usepackage{indentfirst}
\usepackage{subcaption}
\usepackage{tikz}
\usepackage{pgfplots}

\usepackage[
    pdfauthor={Rogozin Oleg},
    pdftitle={Fluid-kinetic coupling between LBE and BGK},
    colorlinks, pdftex, unicode
]{hyperref}

\usepackage[
    backend=biber,
    style=gost-numeric,
    autolang=other,
    maxbibnames=99, minbibnames=99,
    natbib=true,
    sorting=ydnt,
    url=false,
    eprint=true,
    pagetracker,
    firstinits]{biblatex}
\bibliography{bgk_lattice}

\theoremstyle{plain}
\newtheorem*{lemma}{Lemma}
\newtheorem{proposition}{Proposition}
%\theoremstyle{remark}
\newtheorem*{remark}{Remark}

\title{Стыковка моделей LB и BGK}
\author{Олег Рогозин}

\newcommand{\Kn}{\mathrm{Kn}}
\newcommand{\dd}{\mathrm{d}}
\newcommand{\pder}[2][]{\frac{\partial#1}{\partial#2}}
\newcommand{\pderdual}[2][]{\frac{\partial^2#1}{\partial#2^2}}
\newcommand{\pderder}[3][]{\frac{\partial^2#1}{\partial#2\partial#3}}
\newcommand{\Pder}[2][]{\partial#1/\partial#2}
\newcommand{\Pderdual}[2][]{\partial^2#1/\partial#2^2}
\newcommand{\Pderder}[3][]{\partial^2#1/\partial#2\partial#3}
\newcommand{\Set}[2]{\{\,{#1}:{#2}\,\}}
\newcommand{\dzeta}{\boldsymbol{\dd\zeta}}
\newcommand{\OO}[1]{O(#1)}

\begin{document}

\section{Решение одномерного уравнения БГК}

В безразмерных переменных кинетическое уравнение БГК
\begin{equation}\label{eq:bgk}
    \pder[f]{t} + \zeta_y\pder[f]{y} = \nu(M[f]-f), \quad \nu = \frac{\rho}k
\end{equation}
где максвеллиан
\begin{equation}\label{eq:Maxwell}
    M[f] = \frac{\rho}{(\pi T)^{3/2}}\exp\left(-\frac{(\zeta_i - v_i)^2}T\right),
\end{equation}
число Кнудсена
\begin{equation}\label{eq:Knudsen}
    \Kn = \frac2{\sqrt\pi}k.
\end{equation}

Для численного решения~\eqref{eq:bgk} используется явно-неявная схема первого порядка
\begin{equation}\label{eq:imex1}
    f^{n+1} = f^n + \frac{\nu^{n+1}(M[f^{n+1}]-f^n) - \zeta_y\pder[f^{n}]{y}}{1+\nu^{n+1}\Delta{t}}\Delta{t},
\end{equation}
где \(M[f^{n+1}]\) и \(\nu^{n+1}\) вычисляется через \(\rho^{n+1}\), \(\zeta^{n+1}_i\) и \(T^{n+1}\) из
\begin{equation}\label{eq:imex2}
    \int\psi f^{n+1}\dzeta = \int\psi f^n\dzeta - \Delta{t}\int\zeta_y\pder[f^{n}]{y}\dzeta, \quad \psi = 1,\zeta_i,\zeta^2.
\end{equation}
Схема~\eqref{eq:imex1}--\eqref{eq:imex2} сохраняет устойчивость при \(k\to0\) (asymptotic preserving).
Консервативная схема в физическом пространстве строится методом конечных объёмов
\begin{equation}\label{eq:discrete_imex}
    f^{n+1}_m = f^n_m + \frac{
        \nu^{n+1}(M[f^{n+1}_m]-f^n_m) - \frac1{\Delta{y}}\left(F^n_{m+\frac12}-F^n_{m-\frac12}\right)
    }{1+\nu^{n+1}_m\Delta{t}}\Delta{t},
\end{equation}
где
\begin{equation}\label{eq:fluxes}
    F_{m+\frac12} = \zeta_y\left( f_m + \frac{1-\gamma}2\Delta{f_m}\right) \quad(\zeta_y>0),
    \quad \gamma = \frac{\zeta_y\Delta{t}}{\Delta{y}}.
\end{equation}
При \(\Delta{f_m} = 0\) достигается первый порядок точности.
TVD-схема второго порядка \(\OO{\Delta{t}^2+\Delta{y}^2}\) для уравнения переноса
может быть построена, например, с помощью MC-лимитера
\begin{equation}\label{eq:MC_limiter}
    \overline{\Delta{f_m}} = \frac12\min\left(
        |\Delta f_{m+\frac12} + \Delta f_{m-\frac12}|, 4|\Delta f_{m+\frac12}|, 4|\Delta f_{m-\frac12}|
    \right)\mathrm{sign}(\Delta f_{m+\frac12}),
\end{equation}
где \(\Delta f_{m+\frac12} = f_{m+1} - f_m\).

\section{Плоское течение Куэтта}

Рассмотрим газ между двумя параллельными пластинами со скоростями \(v_B = (\pm{U}/2,0,0)\),
расположенными в \(y=\pm1/2\), тогда расчётная область \(y\in(0,1/2)\) с граничными условиями
диффузного отражения в \(y=1/2\)
\begin{equation}\label{eq:diffuse}
    f(\zeta_x,\zeta_y,\zeta_z) = \frac2{\pi} \exp\left[-\left(\zeta_x-\frac{U}{2}\right)^2-\zeta_y^2-\zeta_z^2\right]
        \int_{\zeta_{*y}>0}\zeta_{*y} f_* \dzeta_*, \quad \zeta_y<0
\end{equation}
и антисиммитричным условием в \(y=0\):
\begin{equation}\label{eq:antisymmetry}
    f(\zeta_x,\zeta_y,\zeta_z) = f(-\zeta_x,-\zeta_y,\zeta_z), \quad \zeta_y>0.
\end{equation}
Решение задачи Куэтта для линеаризованного уравнения БГК сводится к одномерному интегральному уравнению,
высокоточные решения которого представлены в~\cite{Luo2015, Luo2016}.

\subsection{Discrete-velocity method}

Equation~\eqref{eq:bgk} is solved by the second-order splitting scheme into the transport equation
\begin{equation}\label{eq:split_transport}
    \pder[f]{t} + \zeta_i\pder[f]{x_i} = 0,
\end{equation}
for which a finite-volume method with an explicit second-order TVD scheme is used,
and the space-homogeneous BGK equation
\begin{equation}\label{eq:homogeneous}
    \pder[f]{t} = \nu (\Phi-f),
\end{equation}
which has the exact solution
\begin{equation}\label{eq:exact_bgk}
    f_t = \Phi + (f_0-\Phi)\exp(-\nu t).
\end{equation}
The three-dimensional velocity space is discretized by the uniform lattice confined with the sphere of radius \(\zeta_{\max}=4\).
\(M=16\) nodes are placed along each axis. The total amount of nodes is equal to \(2176\).

\section{Numerical example}

The proposed hybrid method is tested on the linearized Couette-flow problem.
Gas is placed between two infinite parallel plates at \(y=\pm1/2\) with constant temperature \(T=1\)
and velocities \((v = \pm\Delta v/2, 0, 0)\).
Complete diffuse reflection are assumed at the plates.
The average density is equal to unity:
\begin{equation}\label{eq:total_mass}
    \int_{-1/2}^{1/2}\rho\dd{y} = 1.
\end{equation}

The physical space \(0<y<1/2\) is divided into \(N=40\) nonuniform cells refined near \(y=1/2\).
The obtained results for \(k=0.1\) and \(\Delta v = 0.02\) are presented in Fig.*.
The buffer layer is located at a distance of \(1.2\) mean free path from the plate.




\printbibliography

\end{document}
