%&pdflatex
\documentclass{article}
\usepackage[utf8]{inputenc}
\usepackage[T2A]{fontenc}
\usepackage[english,french,russian]{babel}

\usepackage{amssymb, amsmath, amsthm}
\usepackage{fullpage}
\usepackage{indentfirst}
\usepackage{subcaption}
\usepackage{tikz}
\usepackage{pgfplots}

\usepackage[
    pdfusetitle,
    colorlinks,
    unicode
]{hyperref}

\usepackage[
    backend=biber,
    style=gost-numeric,
    sorting=none,
    autolang=other,
    maxbibnames=99, minbibnames=99,
    natbib=true,
    url=false,
    eprint=true,
    pagetracker,
    firstinits]{biblatex}
\bibliography{bgk_lattice}

\theoremstyle{plain}
\newtheorem*{lemma}{Lemma}
\newtheorem{proposition}{Proposition}
%\theoremstyle{remark}
\newtheorem*{remark}{Remark}

\title{Fluid-kinetic scheme based on the coupling of discrete-velocity models}
\author{Oleg Rogozin}

\newcommand{\Kn}{\mathrm{Kn}}
\newcommand{\dd}{\mathrm{d}}
\newcommand{\pder}[2][]{\frac{\partial#1}{\partial#2}}
\newcommand{\pderdual}[2][]{\frac{\partial^2#1}{\partial#2^2}}
\newcommand{\pderder}[3][]{\frac{\partial^2#1}{\partial#2\partial#3}}
\newcommand{\Pder}[2][]{\partial#1/\partial#2}
\newcommand{\Pderdual}[2][]{\partial^2#1/\partial#2^2}
\newcommand{\Pderder}[3][]{\partial^2#1/\partial#2\partial#3}
\newcommand{\Set}[2]{\{\,{#1}:{#2}\,\}}
\newcommand{\dzeta}{\boldsymbol{\dd\zeta}}
\newcommand{\OO}[1]{O(#1)}

\begin{document}
\maketitle
\tableofcontents

\section{Basic equations}

Let \(L\), \(T^{(0)}\), and \(p^{(0)}\) be, respectively, the reference length, temperature, and pressure.
Then the macroscopic variables have the following form:
\(TT^{(0)}\) is the temperature, \(p_{ij}p^{(0)}\) is the stress tensor, \(\rho p^{(0)}/RT^{(0)}\) is the density,
\(v_i(2RT^{(0)})^{1/2}\) is the velocity, \(q_ip^{(0)}(2RT^{(0)})^{1/2}\) is the heat-flow vector,
and \(f(2p^{(0)})/(2RT^{(0)})^{5/2}\) is the distribution function
in space \((tL/(2RT^{(0)})^{1/2}, x_iL, \zeta_i(2RT^{(0)})^{1/2})\).
The specific gas constant \(R = k_B/m\), where \(k_B\) is the Boltzmann constant and \(m\) is the molar mass.
For this nondimensional notation,
\begin{equation}\label{eq:bgkw}
    \pder[f]{t} + \zeta_i\pder[f]{x_i} = \frac{\rho}k(f_e-f)
\end{equation}
is the BGKW equation~\cite{Krook1954, Welander1954}.
The equilibrium function is a local Maxwellian
\begin{equation}\label{eq:Maxwellian}
    f_e = \frac{\rho}{(\pi T)^{3/2}}\exp\left(-\frac{(\zeta_i - v_i)^2}T\right),
\end{equation}
and the Knudsen number can be expressed through the reference gas viscosity \(\mu^{(0)}\):
\begin{equation}\label{eq:Knudsen_number}
    k = \frac{\sqrt\pi}2\Kn = \mu^{(0)} \frac{\sqrt{2RT^{(0)}}}{p^{(0)}L}.
\end{equation}

\section{Numerical method}

\subsection{Discrete-velocity models}
% uniform + correction
% lattices see at Gatignol

\subsection{Conservative scheme}
% simple explicit scheme for demonstration
% split, fvm, exact

\subsection{Coupling algorithm}
% \emph{conservative}

\section{Results and discussions}

\subsection{Couette-flow problem}
%\subsection{Heat-transfer problem}
% Numerical example -- Couette-flow problem
% different U, large U and transverse heat flow

The proposed hybrid method is tested on the linearized Couette-flow problem.
Gas is placed between two infinite parallel plates at \(y=\pm1/2\) with constant temperature \(T=1\)
and velocities \((v = \pm\Delta v/2, 0, 0)\).
Complete diffuse reflection are assumed at the plates.
The average density is equal to unity:
\begin{equation}\label{eq:total_mass}
    \int_{-1/2}^{1/2}\rho\dd{y} = 1.
\end{equation}

The physical space \(0<y<1/2\) is divided into \(N=40\) nonuniform cells refined near \(y=1/2\).
The obtained results for \(k=0.1\) and \(\Delta v = 0.02\) are presented in Fig.*.
The buffer layer is located at a distance of \(1.2\) mean free path from the plate.

The solution of the Couette-flow problem for the linearized BGK equation is reduced to one-dimensional integral equation,
high-precision solutions of which are presented in~\cite{Luo2015, Luo2016}.


\subsection{Discrete-velocity method}

The velocity space is discretized to the uniform grid at half-integer points
\begin{equation}\label{eq:velocity_grid}
    \zeta_i = h\left(k_i+\frac12\right), \quad k_i\in\mathbb{Z}^3.
\end{equation}
Equation~\eqref{eq:bgkw} is solved by the second-order splitting scheme.
The transport equation
\begin{equation}\label{eq:split_transport}
    \pder[f]{t} + \zeta_i\pder[f]{x_i} = 0,
\end{equation}
is solved within the finite-volume representation
\begin{equation}\label{eq:finite_volume}
    f^{n+1}_m = f^n_m - \frac{\Delta{t}}{\Delta{y}}\left(F^n_{m+\frac12}-F^n_{m-\frac12}\right),
\end{equation}
where fluxes
\begin{equation}\label{eq:fluxes}
    F_{m+\frac12} = \zeta_y\left( f_m + \frac{1-\gamma}2\Delta{f_m}\right) \quad(\zeta_y>0),
    \quad \gamma = \frac{\zeta_y\Delta{t}}{\Delta{y}}.
\end{equation}
can be calculated by the second-order TVD scheme, e.g. with the MC limiter
\begin{equation}\label{eq:MC_limiter}
    \overline{\Delta{f_m}} = \frac12\min\left(
        |\Delta f_{m+\frac12} + \Delta f_{m-\frac12}|, 4|\Delta f_{m+\frac12}|, 4|\Delta f_{m-\frac12}|
    \right)\mathrm{sign}(\Delta f_{m+\frac12}),
\end{equation}
where \(\Delta f_{m+\frac12} = f_{m+1} - f_m\).
The space-homogeneous BGK equation
\begin{equation}\label{eq:homogeneous}
    \pder[f]{t} = \frac{\rho}k (M[f]-f),
\end{equation}
has the exact solution
\begin{equation}\label{eq:exact_bgk}
    f(t) = M[f] + (f(t_0)-M[f])\exp\left(-\frac{\rho}k (t-t_0)\right).
\end{equation}
The three-dimensional velocity space is discretized by the uniform lattice confined with the sphere of radius \(\zeta_{\max}=4\).
\(M=16\) nodes are placed along each axis. The total amount of nodes is equal to \(2176\).

\subsection{Boundary conditions}

The diffuse-reflection boundary conditions are used at \(y=1/2\)
\begin{equation}\label{eq:diffuse}
    f(\zeta_x,\zeta_y,\zeta_z) = \frac2{\pi} \exp\left[-\left(\zeta_x-\frac{U}{2}\right)^2-\zeta_y^2-\zeta_z^2\right]
        \int_{\zeta_{*y}>0}\zeta_{*y} f_* \dzeta_*, \quad \zeta_y<0.
\end{equation}
The solution is antisymmetric at \(y=0\):
\begin{equation}\label{eq:antisymmetry}
    f(\zeta_x,\zeta_y,\zeta_z) = f(-\zeta_x,-\zeta_y,\zeta_z), \quad \zeta_y>0.
\end{equation}

\appendix

\section{Benchmark solution of the linearized BGKW equation}

\printbibliography

\end{document}
