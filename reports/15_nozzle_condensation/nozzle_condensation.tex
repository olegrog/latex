\documentclass{article}
\usepackage[english]{babel}
\usepackage{csquotes}

%%% Functional packages
\usepackage{amsmath, amssymb, amsthm}
\usepackage{physics, siunitx}
\usepackage{multirow}
\usepackage{chemformula}        % for \ch

%%% Configuration packages
%\usepackage{fullpage}

\usepackage[
    pdfusetitle,
    colorlinks
]{hyperref}

\usepackage[
    backend=biber,
    style=numeric,
    giveninits=true,
    sorting=none,
    mincrossrefs=100, % do not add parent collections
]{biblatex}
\addbibresource{nozzle_condensation.bib}

\title{A condensation flow in supersonic nozzles}
\author{Oleg Rogozin}

%%% Problem-specific aliases
\newcommand{\vap}{\text{v}}
\newcommand{\ine}{\text{i}}
\newcommand{\con}{\text{c}}
\newcommand{\eq}{\text{eq}}
\newcommand{\crit}[2][]{#2_\text{cr#1}}
\newcommand{\Hill}{\text{H}}

%%% Bold symbols
\newcommand{\bv}{\vb*{v}}
\newcommand{\bn}{\vu*{n}}
\newcommand{\bx}{\vb*{x}}

\begin{document}
\maketitle
\tableofcontents

\section{Gas dynamics}

%%% Governing equations
The steady-state behavior of an inviscid gas in the presence of phase transition
can be described by the diabatic flow equations:
\begin{gather}
    \div(\rho\bv) = 0, \label{eq:rho}\\
    \div(\rho\bv\bv) + \grad{p} = 0, \label{eq:v}\\
    \div(\rho\bv h_0) = \dot{q}, \label{eq:h}
\end{gather}
where $\rho$ is the density (\si{\kg\per\cubic\m}), $\bv$ is the velocity (\si{\m\per\s}),
$p$ is the pressure (\si{\Pa}), $\dot{q}$ is the volumetric heat source (\si{\J\per\cubic\m\per\s}),
\begin{equation}\label{eq:h0}
    h_0 = c_pT + \frac{|\bv|^2}{2}
\end{equation}
is the stagnation enthalpy (\si{\J\per\kg}),
$c_p$ is the temperature-independent specific heat capacity at constant pressure (\si{\J\per\kg\per\K}),
$T$ is the temperature (\si{\K}).

%%% 1D geometry
Let us consider a one-dimensional compressible flow in a nozzle with a specified cross-sectional area $A(x)$.
Integrating Eqs.~\eqref{eq:rho}--\eqref{eq:h} over the cross section, we obtain
\begin{gather}
    \dv{\dot{m}}{x} = 0, \label{eq:continuity}\\
    \dv{x}(p + \rho v^2) = -\frac{\rho v^2}{A}\dv{A}{x}, \label{eq:momentum}\\
    \dv{x}(c_pT + \frac{v^2}{2}) = \frac{A\dot{q}}{\dot{m}}, \label{eq:energy}
\end{gather}
where $\dot{m} = \rho Av$ is the constant mass flow rate through the nozzle (\si{\kg\per\s})
and $v$ is the velocity component normal to the cross section.

%%% Equation of state
The system of governing equations is closed by the equation of state
\begin{equation}\label{eq:eos}
    p = \frac{\rho RT}{\mathcal{M}},
\end{equation}
where $\mathcal{M}$ is molar mass of the mixture of ideal gases (\si{\kg\per\mol}),
$R$ is the universal gas constant ($=\SI{8.3145}{\J\per\mol\per\K}$).

%%% Mach number
Next, it is convenient to deal with the Mach number
\begin{equation}\label{eq:Mach}
    M = \qty(\frac{\mathcal{M}v^2}{\gamma RT})^{1/2},
\end{equation}
where
\begin{equation}\label{eq:gamma}
    \gamma = \qty(1-\frac{R}{c_p\mathcal{M}})^{-1}
\end{equation}
is the heat capacity ratio.
Using the following relations:
\begin{equation*}
    \rho v^2 = \gamma p M^2, \quad v^2 = (\gamma-1)c_pT M^2,
\end{equation*}
Eqs.~\eqref{eq:momentum} and~\eqref{eq:energy} take the form
\begin{gather}
    \dv{x}(p\qty(1 + \gamma M^2)) = -\frac{\gamma p M^2}{A}\dv{A}{x}, \label{eq:momentum2}\\
    \dv{x}(c_pT\qty(1 + \frac{\gamma-1}{2}M^2)) = \frac{A\dot{q}}{\dot{m}}, \label{eq:energy2}
\end{gather}
and mass flow rate
\begin{equation}\label{eq:dotm}
    \dot{m} = \frac{\gamma pMA}{((\gamma-1)c_pT)^{1/2}}.
\end{equation}

\subsection{Isentropic solution}

In the absence of heat source ($\dot{q}=0$) and constant $\gamma$,
which corresponds to a steady isentropic flow of a calorically perfect gas,
the differential equations~\eqref{eq:momentum2}--\eqref{eq:dotm}
can be reduced to the algebraic ones~\cite[Sec.~5.4]{anderson1990modern}:
\begin{gather}
    \qty(\frac{A}{A_*})^2 = \frac1{M^2}
        \qty(\frac{2}{\gamma+1}\qty(1+\frac{\gamma-1}{2}M^2))^{(\gamma+1)/(\gamma-1)}, \label{eq:isentropic:Ma}\\
    \frac{T}{T_*} = \frac{\gamma+1}{2 + (\gamma-1)M^2}, \label{eq:isentropic:T}\\
    p = \frac{\dot{m}}{\gamma MA}\qty((\gamma-1)c_pT)^{1/2}, \label{eq:isentropic:P}
\end{gather}
where $A_*$ and $T_*$ are the cross-sectional area and temperature at the nozzle throat.
Eq.~\eqref{eq:isentropic:Ma} is called the area--Mach number relation.
It requires that $A\geq A_*$ and gives both supersonic and subsonic solutions.

\section{Gas properties}

%%% Mixture
Let us consider a binary mixture of two gases.
One of them is a vapor that can condense, while the other is inert.
Thus, $\rho = \rho_\ine + \rho_\vap + \rho_\con$.
Let $w_0$ be the initial vapor mass fraction, then
\begin{equation}\label{eq:rhoM}
    \frac{\rho_\ine}{\rho} = 1-w_0, \quad \frac{\rho_\vap}{\rho} = w_0-g, \quad \frac{\rho_\con}{\rho} = g,
\end{equation}
where $g$ is the condensate mass fraction.
The specific heat capacity of the mixture is expressed according to the mass fractions as
\begin{equation}\label{eq:c_p}
    c_p = (1-w_0)c_{p\ine} + (w_0-g)c_{p\vap} + gc_{p\con}.
\end{equation}
Similarly,
\begin{equation}\label{eq:M}
    \frac1{\mathcal{M}} = \frac{1-w_0}{\mathcal{M}_\ine} + \frac{w_0-g}{\mathcal{M}_\vap},
\end{equation}
where the contribution of the condensate phase is zero.

%%% Heat source
The heat source in the energy equation~\eqref{eq:energy2} can be written in terms of $g$ as
\begin{equation}\label{eq:dotq}
    \frac{A\dot{q}}{\dot{m}} = \Delta{h}\dv{g}{x},
\end{equation}
where the change in the specific enthalpy due to the phase transition,
\begin{equation}\label{eq:deltah}
    \Delta{h} = \qty(c_{p\con} - c_{p\vap})T + H(T),
\end{equation}
and $H$ is the latent heat of condensation (\si{\J\per\kg}).

\subsection{Derivation of the heat-source term}

Substituting the condensate continuity equation
\begin{equation}\label{eq:rho_c}
    \div(\rho_c\bv) = \dot\rho_c
\end{equation}
into the energy equation
\begin{equation}\label{eq:h-H}
    \div(\rho\bv h_0) = \Delta{h}\dot\rho_c
\end{equation}
and integrating the latter over the cross section, we obtain
\begin{equation}\label{eq:energy-H}
    \dv{x}(c_pT + \frac{v^2}{2}) = \Delta{h}\dv{g}{x}.
\end{equation}
Comparing Eq.~\eqref{eq:energy-H} with Eq.~\eqref{eq:energy} gives us Eq.~\eqref{eq:dotq}.
In turn, rewriting Eq.~\eqref{eq:energy-H} as
\begin{equation}\label{eq:energy-H2}
    c_p\dv{T}{x} + v\dv{v}{x} = \qty(\Delta{h} - \dv{c_p}{g})\dv{g}{x},
\end{equation}
we obtain Eq.~\eqref{eq:deltah} using the definition of $H$.

\section{Condensation}

%%% Moment equations
The gas dynamic equations are augmented by the following moment equations~\cite{hill1966condensation}:
\begin{equation}\label{eq:mu_k}
    \div(\mu_k\bv) = \dot{\mu}_k \quad (k=0,1,2,\dots),
\end{equation}
where $\mu_k$ is the $k$th moment of the particle distribution function ($\si{\m}^{k-3}$), and
\begin{equation}\label{eq:dotmu}
    \dot\mu_k = \begin{cases}
        J &\text{if } k=0,\\
        \crit{r}^{k}J + k\dot{r}(r_\Hill)\mu_{k-1} &\text{if } k>0.
    \end{cases}
\end{equation}
Here, $J$ is the nucleation rate (\si{\per\cubic\m\per\s}),
$\dot{r}$ is the growth rate function (\si{\m\per\s}) of droplet radius $r$,
and $r_\Hill=\sqrt{\mu_2/\mu_0}$ is the surface-averaged radius (\si{\m}).
The nucleation terms in the source~\eqref{eq:dotmu} vanish, when oversaturation $S = p_\vap/p_\eq$,
which is the ratio of the vapor pressure $p_\vap = (w_0 - g)p\mathcal{M}/\mathcal{M}_\vap$
to the saturated pressure $p_\eq$ at local flow temperature, is less or equal to one.
When $S>1$, new particles nucleate at a given size,
called the critical radius, $\crit{r} = \lambda_K/\ln(S)$, where
\begin{equation}\label{eq:Kelvin}
    \lambda_K = \frac{2\sigma \mathcal{M}_\vap}{\rho_L R T}
\end{equation}
is the Kelvin length (\si{m}) and $\sigma$ is the surface tension coefficient (\si{\J\per\square\m}).

%%% Condensate mass fraction
The condensate mass fraction is expressed in terms of $\mu_3$ as
\begin{equation}\label{eq:g}
     g = \frac{4\pi\rho_L}{3} \frac{\mu_3}{\rho},
\end{equation}
where $\rho_L$ is the density of the condensed phase (\si{\kg\per\cubic\m}).

%%% Nucleation rate
According to the Hale model~\cite{hale2004scaling}, nucleation rate
\begin{equation}\label{eq:J}
    J = J_0\exp(-\frac{16\pi}{3}\Omega^3\frac{(\crit{T}/T-1)^3}{\ln^2{S}}) \quad (S > 1),
\end{equation}
where $J_0$ is the nucleation prefactor (\si{\per\cubic\m\per\s}),
$\Omega$ is the excess surface entropy per molecule divided by the Boltzmann constant,
and $\crit{T}$ is the critical temperature (\si{\K}).

%%% Growth rate
The isothermal growth law~\cite{sinha2009modeling} states that
\begin{equation}\label{eq:dotr}
    \dot{r}(r) = \frac{5\pi\alpha}{16\rho_L}
        \qty(\frac{\mathcal{M}_\vap}{2\pi RT})^{1/2} \qty(p_\vap-p_\eq\exp(\frac{\lambda_K}{r})),
\end{equation}
where $\alpha$ is the condensation coefficient.

%%% 1D geometry
Finally, let us note that in the considered 1-D case the moment equations take the form
\begin{equation}\label{eq:moments}
    \dv{x}(\frac{\mu_k}{\rho}) = \frac{\dot{\mu}_kA}{\dot{m}}.
\end{equation}

\subsection{Derivation of the equations of moments}

%%% GDE
The evolution of the particle distribution function $f(\bx,r)$,
which is the number density of particles with radius $r$ at point $\bx$,
is governed by the general dynamic equation~\cite{seinfeld2006atmospheric, hagmeijer2005solution}
\begin{equation}\label{eq:gde}
    \div(f\bv) + \pdv{r}\qty(\dot{r}f) = \delta(r-\crit{r})J,
\end{equation}
where $\delta(r)$ is the Dirac delta function.
Multiplying Eq.~\eqref{eq:gde} by $r^k$ and integrating the result over all possible $r$,
we obtain
\begin{equation}\label{eq:int_gde}
    \div(\mu_k\bv) + \int_0^\infty r^k\pdv{r}\qty(\dot{r}f)\dd{r} = \crit{r}^k J,
\end{equation}
where
\begin{equation}\label{eq:int_mu_k}
    \mu_k = \int_0^\infty r^k f \dd{r}
\end{equation}
is the $k$th moment of the particle distribution function.

%%% Hill's closure
The growth term in Eq.~\eqref{eq:int_gde} vanishes when $k=0$.
For $k>0$, putting $\dot{r}(r) = \dot{r}(r_H)$, we have
\begin{equation}\label{eq:Hill}
    \int_0^\infty r^k\pdv{r}\qty(\dot{r}f)\dd{r} = -k\dot{r}(r_H)\mu_{k-1}.
\end{equation}
Substituting Eq.~\eqref{eq:int_mu_k} into Eq.~\eqref{eq:int_gde},
we come to Eqs.~\eqref{eq:mu_k} with sources~\eqref{eq:dotmu}.

\section{Dimensional analysis}

Let $A_*$, $p_*$, $T_*$, and $g_*$ be, respectively, the cross-sectional area, pressure, temperature,
and condensate mass fraction at the nozzle throat, where $M=1$ is assumed, and put
\begin{equation}\label{eq:reference}
    \rho_* = \frac{p_*\mathcal{M}(g_*)}{RT_*}, \quad
    v_* = \qty(\gamma(g_*)\frac{p_*}{\rho_*})^{1/2}, \quad
    L = \sqrt{A_*}, \quad
    n_0 = \frac{3\rho_*}{4\pi\lambda_K^3\rho_L},
\end{equation}
where $\rho_*$ and $v_*$ are the density and velocity at the nozzle throat,
$L$ and $n_0$ are the reference length and particle concentration.
Then, the nondimensional variables are defined as follows:
\begin{equation}\label{eq:nondimensional}
    \left.\begin{aligned}
        \hat{x} &= x/L,             &\hat{A} &= A/A_*,  &\hat{r} &= r/\lambda_K, \\
        \hat{p} &= p/p_*,           &\hat{T} &= T/T_*,  &\hat{\mu}_k &= \mu_k/n_0\lambda_K^k, \\
        \hat{\rho} &= \rho/\rho_*,  &\hat{v} &= v/v_*,  &\hat{\dot{\mu}}_k &= \mu_kL/n_0\lambda_K^kv_*.
    \end{aligned}\quad\right\}
\end{equation}
Since $\dot{m} = \rho_*v_*A_*$, the governing equations take the form
\begin{gather}
    \dv{\hat{x}}(\hat{p}\qty(1 + \gamma M^2))
        = -\frac{\gamma \hat{p} M^2}{\hat{A}}\dv{\hat{A}}{\hat{x}}, \label{eq:momentum3}\\
    \dv{\hat{x}}(\hat{c}_p\hat{T}\qty(1 + \frac{\gamma-1}{2}M^2))
        = \qty(\hat{c}_p'\hat{T} + \frac{H}{c_p(g_*)T_*})\hat{A}\hat{\dot{\mu}}_3, \label{eq:energy3}\\
    \dv{\hat{x}}(\frac{\hat{\mu}_k}{\hat{\rho}}) = \hat{A}\hat{\dot{\mu}}_k, \label{eq:moments3}
\end{gather}
where $g = \hat{\mu}_3/\hat{\rho}$ and
\begin{equation}\label{eq:dot_mu_k}
    \hat{\dot{\mu}}_k = \qty( \frac{J_0L}{n_0v_*} \crit{\hat{r}}^{k}\hat{J}
        + \qty(\frac{\mathcal{M}_\vap}{\gamma(g_*)\mathcal{M}(g_*)})^{1/2}\frac{L\rho_*}{\lambda_K\rho_L}
            k\hat{\dot{r}}(\hat{r}_\Hill)\hat{\mu}_{k-1}).
\end{equation}
The nucleation and growth models takes the following dimensionless form:
\begin{gather}
    \hat{J} \equiv \frac{J}{J_0}
        = \exp(-\frac{16\pi}{3}\Omega^3\frac{(\crit{T}/T_*\hat{T}-1)^3}{\ln^2{S}}), \label{eq:nucleation3}\\
    \hat{\dot{r}} \equiv \frac{\dot{r}\rho_L}{p_0}\qty(\frac{RT_0}{\mathcal{M}_\vap})^{1/2}
        = \frac{5\pi\alpha}{16(2\pi\hat{T})^{1/2}}
            \qty(\hat{p}_\vap - \hat{p}_\eq\exp(\frac1{\hat{r}})). \label{eq:growth3}
\end{gather}
Four dimensionless quantities are identified in the above formulas:
\begin{equation}\label{eq:dimensionless}
    \frac{\crit{T}}{T_*}, \quad \frac{H}{c_p(g_*)T_*}, \quad
    \frac{J_0L}{n_0v_*}, \quad \qty(\frac{\mathcal{M}_v}{\gamma(g_*)\mathcal{M}(g_*)})^{1/2}\frac{L\rho_*}{\lambda_K\rho_L}.
\end{equation}

\section{Numerical analysis}

\subsection{Direct solution}

The governing equations~\eqref{eq:momentum2}--\eqref{eq:dotm} can be solved directly, which yields three quantities:
\begin{equation}\label{eq:sol1:y}
    y_c = \frac{(\gamma pM)^2}{(\gamma-1)c_pT}, \quad
    y_m = p\qty(1 + \gamma M^2), \quad
    y_e = c_pT\qty(1 + \frac{\gamma-1}{2}M^2).
\end{equation}
Then, the Mach number is found from the algebraic equation
\begin{equation}\label{eq:sol1:Ma}
    \qty(y_c y_e - \frac12y_m^2)\gamma^2M^4 + \qty(2y_c y_e - \frac{\gamma}{\gamma-1}y_m^2)\gamma M^2 + y_cy_e = 0.
\end{equation}
However, we have to choose the physically correct root of Eq.~\eqref{eq:sol1:Ma} manually,
which is not convenient for numerical analysis in the general case.

\subsection{Transformed equations}

Instead, the governing equations~\eqref{eq:momentum2}--\eqref{eq:dotm} can be transformed to the following:
\begin{gather}
    \frac1p\dv{p}{x} = \frac{\gamma M^2}{M^2 - 1}
        \qty( -\frac1A\dv{A}{x} + \qty(\frac{H}{c_pT} - \frac{\mathcal{M}}{\mathcal{M}_\vap})\dv{g}{x} ), \label{eq:sol2:p}\\
    \dv{\gamma M^2}{x} =
        -\frac{\gamma M^2}{A}\dv{A}{x} - \qty(1 + \gamma M^2)\frac1p\dv{p}{x}, \label{eq:sol2:Ma}\\
    \frac1T\dv{T}{x} = \frac1{A}\dv{A}{x} + \frac{\gamma M^2-1}{\gamma p M^2}\dv{p}{x}
        + \frac{\mathcal{M}}{\mathcal{M}_\vap}\dv{g}{x}, \label{eq:sol2:T}
\end{gather}
where
\begin{equation}\label{eq:g_prime}
    \dv{g}{x} = \frac{4\pi\rho_L}{3}\dv{x}(\frac{\mu_3}{\rho}).
\end{equation}
The following relations are used to derive these equations:
\begin{gather}
    \frac1\gamma\dv{\gamma}{x}
        = -(\gamma-1)\qty(\frac1{c_p}\dv{c_p}{x} + \frac1{\mathcal{M}}\dv{\mathcal{M}}{x}), \label{eq:gamma_prime}\\
    \frac1{c_p}\dv{c_p}{x} = \frac{c_{p\con} - c_{p\vap}}{c_p}\dv{g}{x}, \label{eq:c_p_prime}\\
    \frac1{\mathcal{M}}\dv{\mathcal{M}}{x} = \frac{\mathcal{M}}{\mathcal{M}_\vap}\dv{g}{x}. \label{eq:barM_prime}
\end{gather}
They are obtained by taking the derivative of~\eqref{eq:gamma}, \eqref{eq:c_p}, and~\eqref{eq:M}.
\printbibliography

\end{document}
