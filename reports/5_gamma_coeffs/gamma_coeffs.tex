%&pdflatex
\documentclass[a4paper,12pt]{article}
\usepackage{amssymb, amsmath}
\usepackage[utf8]{inputenc}
\usepackage[T2A,T1]{fontenc}
\usepackage[english,russian]{babel}
\usepackage{csquotes}

\usepackage{fullpage}
\usepackage{indentfirst}
\usepackage[font=small,labelsep=period,tableposition=top]{caption}
\usepackage[skip=0pt]{subcaption}
\captionsetup[subtable]{position=top}
%\captionsetup{subrefformat=parens}

\usepackage{graphicx}

\usepackage[
    pdfauthor={Rogozin Oleg},
    pdftitle={Evaluation of the transport coefficients for hard-sphere molecules},
    colorlinks, pdftex, unicode
]{hyperref}

\title{Вычисление транспортных коэффициентов \(\gamma_8\), \(\gamma_9\) и \(\gamma_{10}\) для модели твёрдых сфер}
\author{Рогозин Олег}

%use russian alphabet for subfigure and subtable counter
\makeatletter
  \def\thesubfigure{\textit{\asbuk{subfigure}}}
  \providecommand\thefigsubsep{,~}
  \def\p@subfigure{\@nameuse{thefigure}\thefigsubsep}
  \def\thesubtable{\textit{\asbuk{subtable}}}
  \providecommand\thefigsubsep{,~}
  \def\p@subtable{\@nameuse{thetable}\thefigsubsep}
\makeatother

\newcommand{\Kn}{\mathrm{Kn}}
\newcommand{\dd}{\mathrm{d}}
\newcommand{\der}[2][]{\frac{\dd#1}{\dd#2}}
\newcommand{\pder}[2][]{\frac{\partial#1}{\partial#2}}
\newcommand{\pderdual}[2][]{\frac{\partial^2#1}{\partial#2^2}}
\newcommand{\pderder}[3][]{\frac{\partial^2#1}{\partial#2\partial#3}}
\newcommand{\Pder}[2][]{\partial#1/\partial#2}

\newcommand{\B}{\ensuremath{\mathcal{B}^{(4)}}}
\newcommand{\Q}{\ensuremath{\mathcal{Q}^{(0)}}}
\newcommand{\T}[1]{\ensuremath{\mathcal{T}^{(#1)}}}
\newcommand{\TT}{\ensuremath{\tilde{\mathcal{T}}^{(0)}}}
\newcommand{\QQ}{\ensuremath{\tilde{\mathcal{Q}}^{(0)}}}
\newcommand{\ZD}[2]{\zeta_{#1}\delta_{#2}}
\newcommand{\ZZD}[3]{\zeta_{#1}\zeta_{#2}\delta_{#3}}
\newcommand{\ZZZ}{\zeta_i\zeta_j\zeta_k}
\newcommand{\ZZZZ}{\zeta_i\zeta_j\zeta_k\zeta_l}
\newcommand{\DD}[2]{\delta_{#1}\delta_{#2}}

\DeclareMathOperator{\sgn}{sgn}

\begin{document}

\maketitle
\tableofcontents

\section{Основные формулы}

В~\cite{Sone2000, Sone2002} представлены общие формулы для вычисления \(\gamma_8\) и \(\gamma_9\):
\begin{gather}
    \gamma_8 = I_6\left(\Q_2 - \QQ_{22}\right) + \frac17 I_8\left(\Q_3 - \QQ_3\right), \label{eq:gamma_8}\\
    \gamma_9 = -I_6\left(\B\right), \label{eq:gamma_9}\\
    \gamma_{10} = \frac58 I_6\left(\T{1}_1 + \T{2}_1 - 2\TT_{12}\right)
        + \frac18 I_8\left(\T{1}_2 + \T{2}_2 - 2\TT_2\right), \label{eq:gamma_10}
\end{gather}
где
\begin{equation}\label{eq:I_n}
    I_n[Z(\zeta)] = \frac{8}{15\sqrt\pi} \int_0^\infty \zeta^n Z(\zeta) \exp(-\zeta^2) \dd\zeta,
\end{equation}
а подынтегральные функции выражаются как линейные комбинации решений соответствующих интегральных уравнений:
\begin{gather}
    \mathcal{L}\left[\zeta_i\mathcal{A}(\zeta)\right] = -\zeta_i\left(\zeta^2-\frac52\right), \label{eq:A}\\[6pt]
    \mathcal{L}\left[\left(\zeta_i\zeta_j-\frac13\zeta^2\delta_{ij}\right)\mathcal{B}(\zeta)\right] =
        -2\left(\zeta_i\zeta_j-\frac13\zeta^2\delta_{ij}\right), \label{eq:B}\\[6pt]
    \mathcal{L}\left[\left(\zeta_i\zeta_j-\frac13\zeta^2\delta_{ij}\right)\B(\zeta)\right] =
        \left(\zeta_i\zeta_j-\frac13\zeta^2\delta_{ij}\right)\mathcal{B}(\zeta), \label{eq:B_4}
\end{gather}
\begin{gather}
    \begin{aligned}
    \mathcal{L}&\left[(\ZD{i}{jk}+\ZD{j}{ki}+\ZD{k}{ij})\T{m}_1(\zeta) + \ZZZ\T{m}_2(\zeta)\right] = IT^{(m)}_{ijk}, \\
        &\quad IT^{(1)}_{ijk} = -\ZZZ\left( 2\mathcal{A}(\zeta) - \frac1\zeta \der[\mathcal{A}(\zeta)]{\zeta} \right), \\
        &\quad IT^{(2)}_{ijk} = -\ZZZ\left( (\zeta^2-3)\mathcal{B}(\zeta) - \frac\zeta2 \der[\mathcal{B}(\zeta)]{\zeta} \right)
        + \frac{\gamma_1}{2} (\ZD{i}{jk}+\ZD{j}{ki}+\ZD{k}{ij}),
    \end{aligned}\label{eq:T}\\
    \mathcal{L}\left[\ZD{i}{jk}\TT_{11}(\zeta)+(\ZD{j}{ki}+\ZD{k}{ij})\TT_{12}(\zeta) + \ZZZ\TT_2(\zeta)\right]
        = \mathcal{J}(\zeta_i\mathcal{A}(\zeta), \zeta_j\zeta_k\mathcal{B}(\zeta)), \label{eq:TT}
\end{gather}
\begin{gather}
    \begin{aligned}
    \mathcal{L}&\left[(\DD{ij}{kl}+\DD{ik}{jl}+\DD{il}{jk}) \Q_1(\zeta) \right. \\
        &\quad + \left.(\ZZD{i}{j}{kl}+\ZZD{i}{k}{jl}+\ZZD{i}{l}{jk}+\ZZD{j}{k}{il}+\ZZD{j}{l}{ik}+\ZZD{k}{l}{ij}) \Q_2(\zeta)\right. \\
        &\quad + \left.\ZZZZ \Q_3(\zeta)\right] = \frac13\left[\zeta^2\mathcal{B}(\zeta)+2\gamma_1\left(\zeta^2-\frac32\right)\right](\DD{ij}{kl}+\DD{ik}{jl}+\DD{il}{jk}) \\
        &\quad - \ZZZZ\left(2\mathcal{B}(\zeta) - \frac1\zeta\der[\mathcal{B}(\zeta)]{\zeta}\right),
    \end{aligned}\label{eq:Q}\\[6pt]
    \begin{aligned}
    \mathcal{L}&\left[\DD{ij}{kl}\QQ_{11}(\zeta) + (\DD{ik}{jl}+\DD{il}{jk})\QQ_{12}(\zeta) + (\ZZD{i}{j}{kl}+\ZZD{k}{l}{ij})\QQ_{21}(\zeta) \right. \\
        &\quad + \left.(\ZZD{i}{k}{jl}+\ZZD{i}{l}{jk}+\ZZD{j}{k}{il}+\ZZD{j}{l}{ik}) \QQ_{22}(\zeta) + \ZZZZ \QQ_3(\zeta)\right] \\
        &\quad = \mathcal{J}(\zeta_i\zeta_j\mathcal{B}(\zeta), \zeta_k\zeta_l\mathcal{B}(\zeta))
    \end{aligned}\label{eq:QQ}
\end{gather}
с дополнительными условиями для \(\mathcal{A}\), \(\T{m}_1\), \(\TT_{11}\), \(\TT_{12}\), \(\Q_1\), \(\QQ_{11}\) и \(\QQ_{12}\):
\begin{gather}
    \int_0^\infty \zeta^4 \mathcal{A} E(\zeta) \dd\zeta = 0, \label{eq:A_constraint}\\
    \int_0^\infty \left( 5\zeta^4\T{m}_1 + \zeta^6\T{m}_2 \right) E(\zeta) \dd\zeta = 0, \label{eq:Tm_constraint}\\
    \int_0^\infty \left( 5\zeta^4\TT_{11} + \zeta^6\TT_2 \right) E(\zeta) \dd\zeta = 0, \label{eq:T11_constraint}\\
    \int_0^\infty \left( 5\zeta^4\TT_{12} + \zeta^6\TT_2 \right) E(\zeta) \dd\zeta = 0, \label{eq:T12_constraint}\\
    \int_0^\infty (1,\zeta^2)\left( 15\zeta^2\Q_1 + 10\zeta^4\Q_2 + \zeta^6\Q_3 \right) E(\zeta) \dd\zeta = 0, \\
    \int_0^\infty (1,\zeta^2)\left( 15\zeta^2\QQ_{11} + 10\zeta^4\QQ_{21} + \zeta^6\QQ_3 \right) E(\zeta) \dd\zeta = 0, \\
    \int_0^\infty (1,\zeta^2)\left( 15\zeta^2\QQ_{12} + 10\zeta^4\QQ_{22} + \zeta^6\QQ_3 \right) E(\zeta) \dd\zeta = 0,
\end{gather}
где
\begin{equation}\label{eq:E}
    E(\zeta) = \frac1{\pi^{3/2}}\exp\left(-\zeta^2\right).
\end{equation}
Коэффициент вязкости был вычислен с высокой точностью ещё на первом израильском компьютере WEIZAC~\cite{Pekeris1957}\footnote{
    Столь высокая точность была получена при решении обыкновенного дифференциального уравнения четвёртого порядка,
    к которому может быть сведено интегральное уравнение~\eqref{eq:B}.
}:
\begin{equation}\label{eq:gamma1}
    \gamma_1 \equiv I_6(\mathcal{B}) = 1.270042427,
\end{equation}

Операторы \(\mathcal{L}\) и \(\mathcal{J}\) связаны с интегралом столкновения \(J\) следующим образом:
\begin{equation}\label{eq:mathcalLJ}
    E\mathcal{L}(\phi) = 2J(1, E\phi), \quad E\mathcal{J}(\phi, \psi) = J(E\phi, E\psi).
\end{equation}
Пятимерный интеграл \(\mathcal{J}\) вычисляется как
\begin{equation}\label{eq:mathcalJ}
    \mathcal{J}(\phi,\psi) = \frac12 \int E_*(\phi'\psi'_* + \phi'_*\psi' - \phi\psi_* - \phi_*\psi) B
    \dd \Omega(\boldsymbol{\alpha}) \boldsymbol{\dd \zeta_*},
\end{equation}
где \(\dd \Omega(\boldsymbol{\alpha})\) "--- элемент телесного угла в направлении единичного вектора \(\alpha_i\),
определяющего направление разлётных скоростей:
\begin{equation}\label{scatter_velocities}
    \zeta'_i = \zeta_i + \alpha_i\alpha_j(\zeta_{j*}-\zeta_j), \quad
    \zeta'_{i*} = \zeta_{i*} - \alpha_i\alpha_j(\zeta_{j*}-\zeta_j).
\end{equation}
Для модели твёрдых сфер
\begin{equation}
    B = \frac{|\alpha_i(\zeta_{i*}-\zeta_i)|}{4\sqrt{2\pi}}.
\end{equation}
Линеаризованный интеграл столкновения может быть сведён к трёхмерному интегралу.
Для модели твёрдых сфер \(\mathcal{L}\) принимает следующий вид:
\begin{gather}\label{eq:linear_canonic}
    \mathcal{L}(\phi) = \mathcal{L}_1(\phi) - \mathcal{L}_2(\phi) - \nu(\zeta)\phi, \\[6pt]
    \mathcal{L}_1(\phi) = \frac1{\sqrt2\pi} \int \frac1{|\zeta_i-\xi_i|}
        \exp\left(-\xi_i^2 + \frac{|\zeta_i\times\xi_i|^2}{|\zeta_i-\xi_i|^2}\right) \phi(\xi_i) \boldsymbol{\dd\xi}, \\
    \mathcal{L}_2(\phi) = \frac1{2\sqrt2\pi} \int |\zeta_i-\xi_i|\exp\left(-\xi_i^2\right) \phi(\xi_i) \boldsymbol{\dd\xi}, \\
    \nu(\zeta) = \frac1{2\sqrt2}\left[ \exp(-\zeta^2) + \left(2\zeta+\frac1\zeta\right) \int_0^\zeta\exp(-c^2)\dd{c} \right].
\end{gather}

\section{Численные методы}

Ниже приводится описание метода вычисления соответствующих многомерных интегралов
и решения интегральных уравнений.
Стоит сразу отметить, что точность результатов можно значительно повысить,
если свести многомерные интегралы к одномерным,
однако это приведёт к более громоздким выражениям.
При такой необходимости следует обратиться к работам~\cite{Pekeris1957, Ohwada1992}.

\subsection{Многомерное интегрирование}\label{sec:Korobov}

Для вычисления трёх- и пятимерных интегралов вида
\begin{gather}
    \int f(\zeta_i) \boldsymbol{\dd\zeta} = \frac{4\pi}{3}\frac{\zeta_\mathrm{cut}^3}{N}
        \sum_{k=0}^N f(\zeta_i^{(k)}) + R, \label{eq:dicrete_L}\\
    \int f(\zeta_i,\alpha_i) \dd\Omega(\boldsymbol{\alpha})\boldsymbol{\dd\zeta} = \frac{16\pi^2}{3}\frac{\zeta_\mathrm{cut}^3}{N}
        \sum_{k=0}^N f(\zeta_i^{(k)},\alpha_i^{(k)}) + R, \label{eq:dicrete_J}
\end{gather}
на дискретном пространстве скоростей, ограниченным сферой \(\zeta = \zeta_\mathrm{cut}\),
использовались следующие кубатурные сетки Коробова:
\begin{gather}
    \zeta_i^{(k)} = \zeta_\mathrm{cut}\left( \Big\{\frac{k}{N}\Big\}, \Big\{\frac{k a_2}{N}\Big\}, \Big\{\frac{k a_3}N\Big\} \right), \\
    \begin{split}
    \alpha_i^{(k)} = \left( \sin\theta^{(k)}\cos\phi^{(k)}, \sin\theta^{(k)}\sin\phi^{(k)}, \cos\theta^{(k)} \right), \\
    \quad \theta^{(k)} =  \pi\Big\{\frac{k a_4}{N}\Big\},
    \quad \phi^{(k)}   = 2\pi\Big\{\frac{k a_5}{N}\Big\},
    \end{split}
\end{gather}
где фигурные скобки соответствуют остатку от деления.
Некоторые оптимальные коэффициенты представлены в табл.~\ref{tab:Korobov}.

\begin{table}
    \caption{Таблица оптимальных коэффициентов для вычисления трёхмерных~(\subref{tab:Korobov-a})
        и пятимерных~(\subref{tab:Korobov-b}) интегралов.}
    \label{tab:Korobov}
    {\footnotesize
    \begin{subtable}[b]{.4\linewidth}
        \centering
    	\caption{}\label{tab:Korobov-a}
        \begin{tabular}{|c|c|c|}\hline
               \(N\) & \(a_2\) & \(a_3\)  \\\hline
               10007 &     544 &    5733  \\
               50021 &   12962 &   42926  \\
              100003 &   47283 &   15021  \\
              200003 &    9488 &   20794  \\
              500009 &   33606 &  342914  \\
             1000003 &  342972 &  439897  \\
             2000003 &  235672 & 1208274  \\
             5000011 &  889866 & 4755875  \\
            10000019 & 4341869 &  594760  \\\hline
    	\end{tabular}
    \end{subtable}
    \begin{subtable}[b]{.6\linewidth}
        \centering
    	\caption{}\label{tab:Korobov-b}
        \begin{tabular}{|c|c|c|c|c|}\hline
               \(N\) & \(a_2\) & \(a_3\) & \(a_4\) & \(a_5\)  \\\hline
               10007 &     198 &    9183 &    6967 &    8507  \\
               50021 &    7255 &   12933 &   39540 &   42286  \\
              100003 &   11729 &   65316 &   68384 &   51876  \\
              200003 &   62638 &   60193 &  112581 &  142904  \\
              500009 &  191775 &  488648 &  283447 &   69999  \\
             1000003 &  335440 &  656043 &  403734 &  126676  \\
             2000003 &  701679 &  680513 &  965077 & 1248525  \\
             5000011 & 1516505 & 2355509 & 3317359 &  442579  \\
            10000019 & 3669402 & 5455092 & 7462912 & 2188321  \\\hline
    	\end{tabular}
    \end{subtable}}
\end{table}


\subsection{Решение интегральных уравнений}\label{sec:int_equations}

Интегральные уравнения вида \(\mathcal{L}[\phi(\zeta_i)Z(\zeta)] = \Phi(\zeta_i)\)
могут быть решены итерационным методом релаксации
\begin{gather}\label{eq:relax_method}
    Z_k^{(n+1)} = (1-\tau_k)Z_k^{(n)} + \tau_k\mathcal{F}_k^{(n)}, \\
    \mathcal{F}_k^{(n)} = \frac{\mathcal{L}_1(\phi_k Z_k^{(n)}) - \mathcal{L}_2(\phi_k Z_k^{(n)}) - \Phi_k}{\nu_k\phi_k},
\end{gather}
где индекс \(k\) соответствует дискретному значению функции при \(\zeta_i=\zeta_i^{(k)}\).
Для выбранного модуля \(\zeta^{(k)}\) имеется свобода выбора вектора \(\zeta_i^{(k)}\).
В общем случае можно положить \(\zeta_i^{(k)} = (1,1,1)/\sqrt3\).
Если \(\Phi\) не зависит от \(\zeta_z\), то разумно использовать \(\zeta_i^{(k)} = (1,1,0)/\sqrt2\).
Аналогично, при \(\Phi=\Phi(\zeta_x)\) пусть \(\zeta_i^{(k)} = (1,0,0)\).

Значение \(\mathcal{F}_k^{(n)}\) вычисляется с помощью численного интегрирования, описанного в~\ref{sec:Korobov}.
Для достижения большей точности для каждого \(n\) сетка Коробова сдвигается на произвольный вектор.
Модуль \(\tau_k\) выбирается достаточно малым, чтобы, обеспечить устойчивость метода,
а также снизить флуктуационные колебания, вызванные сменой кубатурной сетки.
Знак \(\tau\) определяется как в методе секущей:
\begin{equation}\label{eq:secant_method}
    \frac{\tau_k}{|\tau_k|} = \sgn\left(1-\frac{\mathcal{F}_k^{(n)} - \mathcal{F}_k^{(n-1)}}{Z_k^{(n)}-Z_k^{(n-1)}}\right).
\end{equation}

Для нахождения решения интегрального уравнения с дополнительным условием типа~\eqref{eq:A_constraint}
достаточно после каждой итерации корректировать аппроксимацию с помощью соответсвующей константы.

\subsection{Оценка точности}

\begin{figure}
    \centering
    \begin{minipage}[b]{0.5\textwidth}
        \centering
        \includegraphics{pics/diffB}
        \caption{Погрешность вычисления функции \(\mathcal{B}(\zeta)\)
            по сравнению с референсной \(\mathcal{B}^*(\zeta)\)}
        \label{fig:diffB}
    \end{minipage}%
    \begin{minipage}[b]{0.5\textwidth}
        \centering
        \includegraphics{pics/A}
        \caption{Транспортная функция \(\mathcal{A}(\zeta)\)}
        \label{fig:A}
    \end{minipage}
\end{figure}

Для оценки точности сравним результат вычисления функции \(\mathcal{B}(\zeta)\)
с референсными данными, взятыми, например, из~\cite{Sone2002}.

Функция \(\mathcal{B}\) вычисляется из интегрального уравнения
\begin{equation}\label{eq:B_par}
    \mathcal{L}(\zeta_x\zeta_y\mathcal{B}) = -2\zeta_x\zeta_y.
\end{equation}
Значение \(\mathcal{B}(0)\) не может быть вычислено непосредственно из~\eqref{eq:B_par}
с помощью метода, изложенного в~\ref{sec:int_equations},
поэтому должно быть получено с помощью экстраполяции.

Погрешность решения для мощности сетки Коробова порядка \(10^7\) изображена на рис.~\ref{fig:diffB},
а сама функция \(\mathcal{B}\) на рис.~\ref{fig:B}.
Погрешность вычисления коэффициента вязкости \(\gamma_1\) равна
\begin{equation}\label{eq:gamma_error}
    \frac{|\gamma_1-\gamma_1^*|}{\gamma_1^*} < 5\cdot10^{-5}.
\end{equation}
Ввиду структурной схожести интегральных уравнений~\eqref{eq:B}--\eqref{eq:QQ},
можно ожидать аналогичной точности для всех полученных результатов.
При вычислении \(\Q_3\) и \(\QQ_3\) погрешность увеличивается для малых \(\zeta\),
что, однако, мало влияет на точность оценки интегралов вида~\eqref{eq:I_n}.


\section{Результаты}

В дальнейшем будем использовать функции \(\mathcal{A}\) (рис.~\ref{fig:A}) и \(\mathcal{B}\) (рис.~\ref{fig:B}),
которые подробно и с высокой точностью табулированы, например, в~\cite{Sone2002}.

\subsection{Вычисление \texorpdfstring{$\gamma_8$}{gamma8}}

\begin{figure}
    \centering
    \begin{minipage}[b]{0.5\textwidth}
        \centering
        \includegraphics{pics/Q_2}
        \caption{Транспортная функция \(\Q_2(\zeta)\)}
        \label{fig:Q2}
    \end{minipage}%
    \begin{minipage}[b]{0.5\textwidth}
        \centering
        \includegraphics{pics/Q_3}
        \caption{Транспортная функция \(\Q_3(\zeta)\)}
        \label{fig:Q3}
    \end{minipage}
\end{figure}

Для вычисления \(\gamma_8\) необходимо провести некоторые преобразования.
Подставляя в форм.~\eqref{eq:Q} \(i=y, j=k=l=x\), а также \(i=j=z, k=x, l=y\), получаем
\begin{align}
    \mathcal{L}\left[ \zeta_x\zeta_y \left( 3\Q_2+\zeta_x^2\Q_3 \right) \right]
        &= -\zeta_x^3\zeta_y\left(2\mathcal{B} - \frac1\zeta\der[\mathcal{B}]{\zeta}\right), \label{eq:Q2}\\
    \mathcal{L}\left[ \zeta_x\zeta_y \left( \Q_2+\zeta_z^2\Q_3 \right) \right]
        &= -\zeta_x\zeta_y\zeta_z^2\left(2\mathcal{B} - \frac1\zeta\der[\mathcal{B}]{\zeta}\right),
\end{align}
откуда в силу линейности \(\mathcal{L}\) имеем
\begin{equation}\label{eq:Q3}
    \mathcal{L}\left[ \zeta_x\zeta_y \left( 3\zeta_z^2-\zeta_x^2 \right)\Q_3 \right]
        = -\zeta_x\zeta_y\left( 3\zeta_z^2-\zeta_x^2 \right)\left(2\mathcal{B} - \frac1\zeta\der[\mathcal{B}]{\zeta}\right).
\end{equation}
Таким образом, вычислив сначала \(\Q_3\) с помощью~\eqref{eq:Q3}, можно найти \(\Q_2\) из~\eqref{eq:Q2}.
Полученные функции изображены на рис.~\ref{fig:Q2} и~\ref{fig:Q3}.
Соответствующие интегралы от них равны
\begin{equation}\label{eq:gamma8a}
    I_6\left(\Q_2\right) = 0.544(4), \quad \frac17 I_8\left(\Q_3\right) = 0.993(3).
\end{equation}

\begin{figure}
    \centering
    \begin{minipage}[b]{0.5\textwidth}
        \centering
        \includegraphics{pics/QQ_22}
        \caption{Транспортная функция \(\QQ_{22}(\zeta)\)}
        \label{fig:QQ22}
    \end{minipage}%
    \begin{minipage}[b]{0.5\textwidth}
        \centering
        \includegraphics{pics/QQ_3}
        \caption{Транспортная функция \(\QQ_3(\zeta)\)}
        \label{fig:QQ3}
    \end{minipage}
\end{figure}

Аналогичными подстановками в форм.~\eqref{eq:QQ} получаем
\begin{align}
    \mathcal{L}\left[ \QQ_{11} + 2\QQ_{12} + 2\zeta_x^2\QQ_{21} + 4\zeta_x^2\QQ_{22} + \zeta_x^4\QQ_3 \right]
        &= \mathcal{J}\left( \zeta_x^2\mathcal{B}, \zeta_x^2\mathcal{B} \right) \equiv \mathcal{J}_{x}^{(1)}, \label{eq:J1}\\
    \mathcal{L}\left[ \QQ_{11} + (\zeta_x^2+\zeta_y^2)\QQ_{21} + \zeta_x^2\zeta_y^2\QQ_3 \right]
        &= \mathcal{J}\left( \zeta_x^2\mathcal{B}, \zeta_y^2\mathcal{B} \right) \equiv \mathcal{J}_{xy}^{(2)}, \label{eq:J2}\\
    \mathcal{L}\left[ \QQ_{12} + (\zeta_x^2+\zeta_y^2)\QQ_{22} + \zeta_x^2\zeta_y^2\QQ_3 \right]
        &= \mathcal{J}\left( \zeta_x\zeta_y\mathcal{B}, \zeta_x\zeta_y\mathcal{B} \right) \equiv \mathcal{J}_{xy}^{(3)}, \label{eq:J3}\\
    \mathcal{L}\left[ \zeta_x\zeta_y \left( \QQ_{21} + 2\QQ_{22} + \zeta_x^2\QQ_3 \right) \right]
        &= \mathcal{J}\left( \zeta_x\zeta_y\mathcal{B}, \zeta_x^2\mathcal{B} \right) \equiv \mathcal{J}_{xy}^{(4)}, \label{eq:J4}\\
    \mathcal{L}\left[ \zeta_x\zeta_y \left( \QQ_{21} + \zeta_z^2\QQ_3 \right) \right]
        &= \mathcal{J}\left( \zeta_x\zeta_y\mathcal{B}, \zeta_z^2\mathcal{B} \right) \equiv \mathcal{J}_{xyz}^{(5)}, \label{eq:J5}\\
    \mathcal{L}\left[ \zeta_x\zeta_y \left( \QQ_{22} + \zeta_z^2\QQ_3 \right) \right]
        &= \mathcal{J}\left( \zeta_x\zeta_y\mathcal{B}, \zeta_x\zeta_z\mathcal{B} \right) \equiv \mathcal{J}_{xyz}^{(6)}. \label{eq:J6}
\end{align}
Неоднородные члены интегральных уравнений~\eqref{eq:J1}--\eqref{eq:J6} представлены на рис.~\ref{fig:J_allQ}.

\begin{figure}
    \centering
    \begin{subfigure}[b]{.5\linewidth}
        \centering
        \includegraphics{pics/J_1}
    	\caption{интеграл \(\mathcal{J}\left( \zeta_x^2\mathcal{B}, \zeta_x^2\mathcal{B} \right)\)}\label{fig:J_1}
    \end{subfigure}%
    \begin{subfigure}[b]{.5\linewidth}
        \centering
        \includegraphics{pics/J_2}
    	\caption{интеграл \(\mathcal{J}\left( \zeta_x^2\mathcal{B}, \zeta_y^2\mathcal{B} \right)\)}\label{fig:J_2}
    \end{subfigure}\\[6pt]
    \begin{subfigure}[b]{.5\linewidth}
        \centering
        \includegraphics{pics/J_3}
    	\caption{интеграл \(\mathcal{J}\left( \zeta_x\zeta_y\mathcal{B}, \zeta_x\zeta_y\mathcal{B} \right)\)}\label{fig:J_3}
    \end{subfigure}%
    \begin{subfigure}[b]{.5\linewidth}
        \centering
        \includegraphics{pics/J_4}
    	\caption{интеграл \(\mathcal{J}\left( \zeta_x\zeta_y\mathcal{B}, \zeta_x^2\mathcal{B} \right)\)}\label{fig:J_4}
    \end{subfigure}\\[6pt]
    \begin{subfigure}[b]{.5\linewidth}
        \centering
        \includegraphics{pics/J_5}
    	\caption{интеграл \(\mathcal{J}\left( \zeta_x\zeta_y\mathcal{B}, \zeta_z^2\mathcal{B} \right)\)}\label{fig:J_5}
    \end{subfigure}%
    \begin{subfigure}[b]{.5\linewidth}
        \centering
        \includegraphics{pics/J_6}
    	\caption{интеграл \(\mathcal{J}\left( \zeta_x\zeta_y\mathcal{B}, \zeta_x\zeta_z\mathcal{B} \right)\)}\label{fig:J_6}
    \end{subfigure}
    \caption{Интеграл столкновения \(\mathcal{J}(\phi,\psi)\),
        вычисленный для \(\zeta_x=\zeta_y=\zeta/\sqrt2\) (\subref{fig:J_1}, \subref{fig:J_2}, \subref{fig:J_3})
        и для \(\zeta_x=\zeta_y=\zeta_z=\zeta/\sqrt3\) (\subref{fig:J_4}, \subref{fig:J_5}, \subref{fig:J_6}) }
    \label{fig:J_allQ}
\end{figure}

С помощью линейных преобразований соответственно над~\eqref{eq:J1}--\eqref{eq:J3} и~\eqref{eq:J4}--\eqref{eq:J6}
получаем уравнения
\begin{align}
    \mathcal{L}\left[ \left( 6\zeta_x^2\zeta_y^2 - \zeta_x^4 - \zeta_y^4 \right)\QQ_3 \right]
        &= 2\mathcal{J}_{xy}^{(2)} + 4\mathcal{J}_{xy}^{(3)} - \mathcal{J}_{x}^{(1)} - \mathcal{J}_{y}^{(1)}, \label{eq:QQ3_a}\\
    \mathcal{L}\left[ \zeta_x\zeta_y\left( 3\zeta_z^2 - \zeta_x^2 \right)\QQ_3 \right]
        &= \mathcal{J}_{xyz}^{(5)} + 2\mathcal{J}_{xyz}^{(6)} - \mathcal{J}_{xy}^{(4)}, \label{eq:QQ3_b}
\end{align}
позволяющие вычислить \(\QQ_3\).
На практике уравнение~\eqref{eq:QQ3_b} позволяет получить большую точность.
\(\QQ_{22}\) находится из~\eqref{eq:J6}.
Полученные функции изображены на рис.~\ref{fig:QQ22} и~\ref{fig:QQ3}.
Соответствующие интегралы от них равны
\begin{equation}\label{eq:gamma8b}
    I_6\left(\QQ_{22}\right) = 0.121(2), \quad \frac17 I_8\left(\QQ_3\right) = -0.0787(7).
\end{equation}

Подставляя~\eqref{eq:gamma8a} и~\eqref{eq:gamma8b} в~\eqref{eq:gamma_8}, получаем
\begin{equation}\label{eq:gamma_8_result}
    \gamma_8 = 1.496(3).
\end{equation}

\subsection{Вычисление \texorpdfstring{$\gamma_9$}{gamma9}}

\begin{figure}
    \centering
    \begin{minipage}[b]{0.5\textwidth}
        \centering
        \includegraphics{pics/B}
        \caption{Транспортная функция \(\mathcal{B}(\zeta)\)}
        \label{fig:B}
    \end{minipage}%
    \begin{minipage}[b]{0.5\textwidth}
        \centering
        \includegraphics{pics/B_4}
        \caption{Транспортная функция \(\B(\zeta)\)}
        \label{fig:B_4}
    \end{minipage}
\end{figure}

Функция \(\B(\zeta)\) (рис.~\ref{fig:B_4}) вычисляется непосредственно из уравнения
\begin{equation}\label{eq:B_4_par}
    \mathcal{L}\left(\zeta_x\zeta_y\B\right) = \zeta_x\zeta_y\mathcal{B}.
\end{equation}
Интегрирование по формуле~\eqref{eq:gamma_9} даёт
\begin{equation}\label{eq:gamma_9_result}
    \gamma_9 = 1.635(7).
\end{equation}

\subsection{Вычисление \texorpdfstring{$\gamma_{10}$}{gamma10}}

\begin{figure}
    \centering
    \begin{minipage}[b]{0.5\textwidth}
        \centering
        \includegraphics{pics/T1_1}
        \caption{Транспортная функция \(\T{1}_1(\zeta)\)}
        \label{fig:T1_1}
    \end{minipage}%
    \begin{minipage}[b]{0.5\textwidth}
        \centering
        \includegraphics{pics/T1_2}
        \caption{Транспортная функция \(\T{1}_2(\zeta)\)}
        \label{fig:T1_2}
    \end{minipage}
\end{figure}

\begin{figure}
    \centering
    \begin{minipage}[b]{0.5\textwidth}
        \centering
        \includegraphics{pics/T2_1}
        \caption{Транспортная функция \(\T{2}_1(\zeta)\)}
        \label{fig:T2_1}
    \end{minipage}%
    \begin{minipage}[b]{0.5\textwidth}
        \centering
        \includegraphics{pics/T2_2}
        \caption{Транспортная функция \(\T{2}_2(\zeta)\)}
        \label{fig:T2_2}
    \end{minipage}
\end{figure}

\begin{figure}
    \centering
    \begin{minipage}[b]{0.5\textwidth}
        \centering
        \includegraphics{pics/TT_12}
        \caption{Транспортная функция \(\TT_{12}(\zeta)\)}
        \label{fig:TT12}
    \end{minipage}%
    \begin{minipage}[b]{0.5\textwidth}
        \centering
        \includegraphics{pics/TT_2}
        \caption{Транспортная функция \(\TT_2(\zeta)\)}
        \label{fig:TT2}
    \end{minipage}
\end{figure}

\begin{figure}
    \centering
    \begin{subfigure}[b]{.5\linewidth}
        \centering
        \includegraphics{pics/J_a}
    	\caption{интеграл \(\mathcal{J}\left( \zeta_x\mathcal{A}, \zeta_x\zeta_y\mathcal{B} \right)\)}\label{fig:J_a}
    \end{subfigure}%
    \begin{subfigure}[b]{.5\linewidth}
        \centering
        \includegraphics{pics/J_b}
    	\caption{интеграл \(\mathcal{J}\left( \zeta_x\mathcal{A}, \zeta_y\zeta_z\mathcal{B} \right)\)}\label{fig:J_b}
    \end{subfigure}
    \caption{Интеграл столкновения \(\mathcal{J}(\phi,\psi)\),
        вычисленный для \(\zeta_x=\zeta_y=\zeta/\sqrt2\) (\subref{fig:J_a})
        и для \(\zeta_x=\zeta_y=\zeta_z=\zeta/\sqrt3\) (\subref{fig:J_b}) }
    \label{fig:J_allT}
\end{figure}


Из~\eqref{eq:T} следует, что функции \(\T{m}_1\) (рис.~\ref{fig:T1_1},~\ref{fig:T2_1})
и \(\T{m}_2\) (рис.~\ref{fig:T1_2},~\ref{fig:T2_2}) можно вычислить по формулам
\begin{align}
    \mathcal{L}\left( \zeta_x\zeta_y\zeta_z\T{1}_2 \right)
        &= -\zeta_x\zeta_y\zeta_z\left(2\mathcal{A} - \frac1\zeta\der[\mathcal{A}]{\zeta}\right), \label{eq:T2a}\\
    \mathcal{L}\left[ \zeta_x\left(3\T{1}_1 + \zeta_x^2\T{1}_2\right) \right]
        &= -\zeta_x^3\left(2\mathcal{A} - \frac1\zeta\der[\mathcal{A}]{\zeta}\right), \label{eq:T1a}
\end{align}
\begin{align}
    \mathcal{L}\left( \zeta_x\zeta_y\zeta_z\T{2}_2 \right)
        &= -\zeta_x\zeta_y\zeta_z\left((\zeta^2-3)\mathcal{B} - \frac\zeta2\der[\mathcal{B}]{\zeta}\right), \label{eq:T2b}\\
    \mathcal{L}\left[ \zeta_x\left(3\T{2}_1 + \zeta_x^2\T{2}_2\right) \right]
        &= -\zeta_x^3\left((\zeta^2-3)\mathcal{B} - \frac\zeta2\der[\mathcal{B}]{\zeta}\right) + \frac{3\gamma_1}{2}\zeta_x \label{eq:T1b}
\end{align}
при дополнительном условии~\eqref{eq:Tm_constraint}.
Соответствующие интегралы в~\eqref{eq:gamma_10} равны
\begin{gather}
    \frac58 I_6\left(\T{1}_1\right) = 0.986(2), \quad \frac18 I_8\left(\T{1}_2\right) = 0.505(3), \label{eq:gamma10a}\\
    \frac58 I_6\left(\T{2}_1\right) = 0.508(4), \quad \frac18 I_8\left(\T{2}_2\right) = 0.540(7). \label{eq:gamma10b}
\end{gather}

Функции \(\TT_{12}\) (рис.~\ref{fig:TT12}) и \(\TT_2\) (рис.~\ref{fig:TT2}) находятся из уравнений
\begin{align}
    \mathcal{L}\left( \zeta_x\zeta_y\zeta_z\TT_2 \right)
    	&= \mathcal{J}\left( \zeta_x\mathcal{A}, \zeta_y\zeta_z\mathcal{B} \right) \equiv \mathcal{J}_{xyz}^{(\mathrm{II})}, \label{eq:TT2}\\
    \mathcal{L}\left[ \zeta_y\left(\TT_{12} + \zeta_x^2\TT_2\right) \right]
        &= \mathcal{J}\left( \zeta_x\mathcal{A}, \zeta_x\zeta_y\mathcal{B} \right) \equiv \mathcal{J}_{xy}^{(\mathrm{I})}. \label{eq:TT12}
\end{align}
Неоднородные члены представлены на рис.~\ref{fig:J_allT}.
Численное интегрирование приводит к
\begin{equation}\label{eq:gamma10c}
    \frac58 I_6\left(\TT_{12}\right) = 0.038(0), \quad \frac18 I_8\left(\TT_2\right) = 0.0068(2).
\end{equation}

Объединяя~\eqref{eq:gamma10a},~\eqref{eq:gamma10b} и~\eqref{eq:gamma10c}, находим
\begin{equation}\label{eq:gamma_10_result}
    \gamma_{10} = 2.463(3).
\end{equation}

\begin{thebibliography}{9}

\bibitem{Sone2000}
    Sone, Y., Bardos, C., Golse, F., Sugimoto, H.:
    Asymptotic theory of the Boltzmann system, for a steady flow of a slightly rarefied gas
    with a finite Mach number: General theory.
    \newblock European Journal of Mechanics, B/Fluids, \textbf{19}(3), 325-360 (2000)

\bibitem{Sone2002}
    Sone, Y.: Kinetic Theory and Fluid Dynamics.
    \newblock Birkh{\"a}user, Boston (2002)

\bibitem{Pekeris1957}
    Pekeris, C. L., Alterman, Z.: Solution of the Boltzmann-Hilbert integral equation II.
    The coefficients of viscosity and heat conduction.
    \newblock Proceedings of the National Academy of Sciences of the U.S.A. \textbf{43}(11), 998-1007 (1957)

\bibitem{Ohwada1992}
    Ohwada, T., Y. Sone.: Analysis of thermal stress slip flow and negative thermophoresis
    using the Boltzmann equation for hard-sphere molecules.
    \newblock European Journal of Mechanics, B/Fluids. \textbf{11}(4), 389-414 (1992)

\end{thebibliography}

\end{document}
