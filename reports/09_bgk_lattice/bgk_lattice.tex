%&pdflatex
\documentclass{article}
\usepackage[utf8]{inputenc}
\usepackage[T1]{fontenc}
\usepackage[english]{babel}

%%% functional packages
\usepackage{amssymb, mathtools, amsthm, mathrsfs}
\usepackage{subcaption}

%%% configuration packages
\usepackage{fullpage}
\usepackage{indentfirst}
\usepackage{comment}

\usepackage[
    pdfusetitle,
    colorlinks
]{hyperref}

\usepackage[
    backend=biber,
    style=numeric,
    sorting=none,
    maxbibnames=99, minbibnames=99,
    natbib=true,
    url=false,
    eprint=true,
    pagetracker,
    firstinits]{biblatex}
\bibliography{../../science/10_dvm-lbm/dvm-lbm}

\title{Kinetic--kinetic coupling based on the discrete-velocity and lattice-Boltzmann methods under the finite-volume formulation}
\author{Oleg Rogozin}

% general
\newcommand{\Kn}{\mathrm{Kn}}
\newcommand{\Ma}{\mathrm{Ma}}
\newcommand{\dd}{\mathrm{d}}
\newcommand{\pder}[2][]{\frac{\partial#1}{\partial#2}}
\newcommand{\pderdual}[2][]{\frac{\partial^2#1}{\partial#2^2}}
\newcommand{\pderder}[3][]{\frac{\partial^2#1}{\partial#2\partial#3}}
\newcommand{\pderderder}[4][]{\frac{\partial^3#1}{\partial#2\partial#3\partial#4}}
\newcommand{\Pder}[2][]{\partial#1/\partial#2}
\newcommand{\Pderdual}[2][]{\partial^2#1/\partial#2^2}
\newcommand{\Pderder}[3][]{\partial^2#1/\partial#2\partial#3}
\newcommand{\Set}[2]{\{\,{#1}:{#2}\,\}}
\newcommand{\OO}[1]{O(#1)}
\newcommand{\transpose}[1]{#1^\mathsf{T}}
\DeclarePairedDelimiter\autobracket()       % mathtools are needed
\newcommand{\br}[1]{\autobracket*{#1}}

% topic-specific
\newcommand{\dxi}{\boldsymbol{\dd\xi}}
\newcommand{\bxi}{{\boldsymbol{\xi}}}
\newcommand{\bxia}{\bxi_\alpha}
\newcommand{\dx}{\boldsymbol{\dd{x}}}
\newcommand{\bx}{\boldsymbol{x}}
\newcommand{\bm}{\boldsymbol{m}}
\newcommand{\bv}{\boldsymbol{v}}
\newcommand{\bdot}{\boldsymbol{\cdot}}
\newcommand{\xiai}{\xi_{\alpha i}}
\newcommand{\xiaj}{\xi_{\alpha j}}
\newcommand{\xiak}{\xi_{\alpha k}}
\newcommand{\cai}{c_{\alpha i}}
\newcommand{\caj}{c_{\alpha j}}
\newcommand{\equil}[1]{#1^\mathrm{(eq)}}
\newcommand{\hermite}[1]{#1^\mathrm{(H)}}
\newcommand{\refer}[1]{#1_0}
\newcommand{\LB}{\mathrm{LB}}
\newcommand{\DV}{\mathrm{DV}}

\newcommand{\Aa}{a_{\alpha}}
\newcommand{\Aab}{a_{\alpha\beta}}
\newcommand{\Aabg}{a_{\alpha\beta\gamma}}
\newcommand{\Ha}{H_{\alpha}}
\newcommand{\Hab}{H_{\alpha\beta}}
\newcommand{\Habg}{H_{\alpha\beta\gamma}}
\newcommand{\xia}{\xi_{\alpha}}
\newcommand{\xib}{\xi_{\beta}}
\newcommand{\xig}{\xi_{\gamma}}

\begin{document}
\maketitle
\tableofcontents

\section{Basic equations}\label{sec:equations}

%%% Nondimensional variables
Consider a monatomic ideal gas in \(D\)-dimensions.
Let \(L\), \(\refer\rho\), \(\refer{T}\), \(V = \sqrt{2R\refer{T}}\) and \(\refer{p} = \refer{\rho}R\refer{T}\) be
the reference length, density, temperature, velocity, and pressure, respectively.
The specific gas constant \(R = k_B/m\), where \(k_B\) is the Boltzmann constant and \(m\) is the molar mass.
Then, \(f\refer{\rho}/V^D\) is the one-particle velocity distribution function (VDF)
defined in \((2D+1)\)-dimensional space \((tL/V, x_iL, \xi_iV)\) and
the macroscopic variables take the following form:
\(\rho\refer{\rho}\) is the density, \(v_iV\) is the velocity, \(T\refer{T}\) is the temperature,
\(p_{ij}\refer{p}\) is the stress tensor, \(q_i\refer{p}V\) is the heat-flow vector.
In the dimensionless form, they are calculated as polynomial moments of the VDF:
\begin{equation}\label{eq:macro}
    \begin{gathered}
    \rho = \int f \dxi, \quad
    v_i = \frac1{\rho} \int \xi_i f \dxi, \quad
    T = \frac{2}{D\rho}\int(\xi_i-v_i)^2 f \dxi = \frac{p_{ii}}{D\rho}, \\
    p_{ij} = 2 \int(\xi_i-v_i)(\xi_j-v_j) f \dxi, \quad
    q_i = \int(\xi_i-v_i)(\xi_j-v_j)^2 f \dxi.
    \end{gathered}
\end{equation}
Integration with respect to \(\bxi\) is, hereafter, carried out over \(\mathbb{R}^D\) unless otherwise stated.

%%% Boltzmann equation
The VDF is governed by the Boltzmann equation
\begin{equation}\label{eq:Boltzmann}
    \pder[f]{t} + \xi_i\pder[f]{x_i} = \frac1kJ(f),
\end{equation}
where \(J(f)\) is the collisional operator with a local Maxwellian as the equilibrium function
\begin{equation}\label{eq:equilibrium}
    \equil{f}\br{\xi_i,\rho,v_i,T} = \frac{\rho}{(\pi T)^{D/2}}\exp\br{-\frac{(\xi_i - v_i)^2}T}.
\end{equation}
The Knudsen number \(\Kn\) can be expressed in terms of the reference gas viscosity \(\refer\mu\):
\begin{equation}\label{eq:Knudsen_number}
    k = \frac{\sqrt\pi}2\Kn = \frac{\refer\mu V}{\refer{p}L}.
\end{equation}

%%% BGK equation
In the present paper, we restrict ourselves to the simplest relaxation model~\cite{Krook1954, Welander1954}
\begin{equation}\label{eq:bgk_integral}
    J(f) = \frac1\tau\br{\equil{f}-f}, \quad \tau = \frac{T^{\omega-1}}{\rho},
\end{equation}
often referred as the Bhatnagar--Gross--Krook (BGK) model of the Boltzmann collisional operator.
\(\omega\) is the exponent of the viscosity law of the gas.
The nonlinearity in~\eqref{eq:bgk_integral} is more severe in comparison to the full Boltzmann equation
since \(\equil{f}\) depends on \(f\) via its moments,
but the BGK model is much simpler from the numerical point of view.

%%% Kinetic BC
The gas-surface interaction is described via the diffuse-reflection boundary conditions:
\begin{equation}\label{eq:diffuse}
    f(t,\xi_i) = \br{-2\sqrt\frac\pi{T_B} \int_{\xi'_jn_j<0}\xi'_jn_jf(t,\xi'_i)\dxi'}
        \equil{f}(\xi_i,1,v_{Bi},T_B) \quad (\xi_in_i>0),
\end{equation}
where \(n_i\) is the unit vector normal to the boundary, directed into gas.
\(T_B\) and \(v_{Bi}\) are the boundary temperature and velocity, respectively.
It is also assumed that \(v_{Bi}n_i = 0\).

\section{The lattice-Boltzmann method}

%%% LB method
The \emph{LB method} can be considered as a special discretization of the BGK model~\cite{Succi2001}.
The investigated gas flow is assumed to be near-isothermal and slow,
i.e., the Mach number \(\Ma = |\bv|\) is close to zero and \(T = 1 + \OO{\Ma^2}\).
Under this restriction, the local Maxwell state is expanded into the Taylor series
in the bulk velocity \(\bv\) and perturbed temperature \((T-1)\)
around the absolute Maxwellian
\begin{equation}\label{eq:omega}
    \omega(\bxi) = \frac{1}{(2\pi)^{3/2}}\exp\br{ -\frac{\xi^2}{2} }, \quad \xi = |\bxi|
\end{equation}
with \(\rho=1\), \(\bv=0\), \(T=1\).
The resulting series is truncated to some finite order of \(\Ma\).
Particularly, the third-order expansion yields the following local equilibrium LB state:
\begin{equation}\label{eq:lbgk}
    \begin{aligned}
    \equil{f}_{\LB,j}(\bm) = \rho w_{\LB,j}\bigg( 1
        &+ \bxi_{\LB,j}\bdot\bv
        + \frac12\left[ (\bxi_{\LB,j}\bdot\bv)^2 - v^2 + (T-1)(\xi_{\LB,j}^2 - 3) \right] \\
        &+ \frac{\bxi_{\LB,j}\bdot\bv}6\left[ (\bxi_{\LB,j}\bdot\bv)^2 - 3v^2 + 3(T-1)(\xi_{\LB,j}^2-5) \right]
    \bigg), \quad j = 1,\ldots,N_\LB,
    \end{aligned}
\end{equation}
where the discrete velocities \(\bxi_{\LB,j}\) and the corresponding weights \(w_{\LB,j}\)
satisfy the following normalization conditions:
\begin{equation}
    \sum_j w_{\LB,j} = 1, \quad \sum_j w_{\LB,j} \bxi_{\LB,j} = 0, \quad \sum_j w_{\LB,j} \xi_{\LB,j}^2 = 3.
\end{equation}
The LB models are designed to enforce conservation laws
and guarantee that \(\equil{f}_{\LB,j}\) is at least an attractor~\cite{Luo2000}.

%%% Notation and particular cases
Hereinafter, a quadrature rule, based on \(\bxi_{\LB,j}\) and \(w_{\LB,j}\),
together with the discrete operator in form~\eqref{eq:lbgk},
is referred as the LB model.
Several approaches can be applied for the construction of LB models
like Gauss--Hermite~\cite{He1997, Shan1998, Shan2006, Shan2010}
and the entropic method~\cite{Karlin1999, Chikatamarla2006, Chikatamarla2009}.
The common notation D3Q\(p\) means \(N_\LB=p\) for the three-dimensional LB model.
The terms proportional to \((T-1)\) in~\eqref{eq:lbgk} vanish for isothermal LB models.
For low-order lattices, like D3Q19, the third-order terms are truncated in~\eqref{eq:lbgk}.

%%% Advantages of the LB method over the DV method
When the VDF is a slightly perturbed equilibrium,
it can be efficiently approximated using quadratures with small number \(N_\LB\).
The LB models are capable of reproducing low-order polynomial moments of the VDF accurately and,
therefore, describing a fluid-dynamic behavior of a gas, including that beyond the NS level.
Since the local LB equilibrium takes a polynomial form on the bulk velocity,
the conservation of mass, momentum and energy is achieved
with much less computational effort than for the conventional DV method.

\section{Mapping method}\label{sec:mapping}

%%% Hermite expansion for transfering information between models
The mapping method between LB and DV models have to be introduced in the spatial overlapping zone
for the bidirectional transfer of information between subdomains with different types of discretization.
In the overlapping (or buffer) zone, the VDF is assumed to be close to~\eqref{eq:omega}
and, therefore, can be approximated by a three-dimensional Hermite expansion
truncated, in particular, to the third order:
\begin{equation}\label{eq:grad}
    \hermite{f}(t,\bx,\bxi) = \omega(\bxi)\br{ a(t,\bx)
        + \sum_{\alpha} \Aa(t,\bx) \Ha
        + \frac12 \sum_{\alpha,\beta} \Aab(t,\bx) \Hab
        + \frac16 \sum_{\alpha,\beta,\gamma} \Aabg(t,\bx) \Habg
    },
\end{equation}
where \(a, \Aa, \Aab, \Aabg\) are the first coefficients of the projection of the VDF onto the Hermite basis:
\begin{equation}\label{eq:a_coeffs}
    a = \int f\dxi, \quad \Aa = \int \Ha f\dxi, \quad \Aab = \int \Hab f\dxi, \quad \Aabg = \int \Habg f\dxi,
\end{equation}
and $\Ha, \Hab, \Habg$ are the Hermite polynomials of the first, second, and third order, respectively:
\begin{equation}\label{eq:Hermite_polynomials}
    \begin{gathered}
    \Ha(\bxi) = \frac{(-1)}{\omega(\bxi)}\pder{\xia}\omega(\bxi) = \xia, \quad
    \Hab(\bxi) = \frac{1}{\omega(\bxi)}\pderder{\xia}{\xib}\omega(\bxi) = \xia\xib - \delta_{\alpha\beta}, \\
    \Habg(\bxi) = \frac{(-1)}{\omega(\bxi)}\pderderder{\xia}{\xib}{\xig}\omega(\bxi)
        = \xia\xib\xig - \xia\delta_{\beta\gamma} - \xib\delta_{\alpha\gamma} - \xig\delta_{\alpha\beta},
    \end{gathered}
\end{equation}
where \(\delta_{\alpha\beta}\) is the Kronecker delta.

%%% Discretization
The discretization of the proposed mapping in the velocity space is straightforward:
the expansion coefficients~\eqref{eq:a_coeffs} are calculated as quadratures~\eqref{eq:cubature},
and the truncated Hermite expansion~\eqref{eq:grad} is approximated as
\begin{equation}\label{eq:discrete_grad}
    \hermite{f}_j = \equil{f}_j( \bm_0 ) \br{ a
        + \sum_{\alpha} \Aa \Ha(\bxi_j)
        + \frac12 \sum_{\alpha,\beta} \Aab \Hab(\bxi_j)
        + \frac16 \sum_{\alpha,\beta,\gamma} \Aabg \Habg(\bxi_j)
    },
\end{equation}
where \(\bm_0 = \transpose{\br{1, \boldsymbol{0}, 1}}\),
\(\equil{f}_{\LB,j}(\bm_0) = w_{\LB,j}\), which is seen from~\eqref{eq:lbgk},
and \(\equil{f}_{\DV,j}(\bm_0)\) is close to \(w_{\DV,j}\omega(\bxi_{\DV,j})\),
but not equal to it for the conservative DV method.

%%% Naturalness of the mapping method for Gauss--Hermite LB models
It is worth noting that the described mapping method is natural for the Gauss--Hermite LB models.
Specifically, if the order of truncation used in the Hermite expansion~\eqref{eq:grad}
coincides (or exceeds) with the order of the Gauss--Hermite LB model,
all nonequilibrium information that can be contained within this LB model
is transmitted unchanged through this Hermite-based mapping.
Nevertheless, this method can be used for the LB models
that are not derived on the basis of the Gauss--Hermite quadratures.
For instance, the nonequilibrium part of LB VDF can be projected onto the basis
spanned by Hermite polynomials after the regularization procedure~\cite{Latt2006, Chen2006}
and~\cite{Zhang2006, Mont2015, Mattila2017}.
Then the equivalence between the LB VDF and the expansion of type~\eqref{eq:grad} is achieved.

\section{Numerical method}\label{sec:numerics}

\subsection{Discrete-velocity approximation}\label{sec:dv}

%%% Quadratures in velocity space
Formally, the VDF can be represented as a weighted sum
\begin{equation}\label{eq:discrete_velocity}
    f(\bx,\bxi,t) = \sum_\alpha w_\alpha f_\alpha(\bx,t)\delta(\bxi-\bxia),
\end{equation}
where \(\delta(\bxi)\) is the Dirac delta function in the velocity space.
Then, an arbitrary moment \(\phi(\bxi)\) from \(f\) is calculated as
\begin{equation}\label{eq:cubature}
    \int \phi f\dxi = \sum_\alpha w_\alpha \phi(\bxia) f_\alpha.
\end{equation}
It is convenient to deal with weighted values \(\hat{f}_\alpha = w_\alpha f_\alpha\).
The evolution of \(\hat{f}_\alpha\) is governed by the system of partial differential equations
\begin{equation}\label{eq:dvm}
    \pder[\hat{f}_\alpha]{t} + \xiai\pder[\hat{f}_\alpha]{x_i} = \frac1kJ(\hat{f}_\alpha),
\end{equation}
which is called the \emph{discrete-velocity (DV) model} of~\eqref{eq:Boltzmann}~\cite{Cabannes1980}.

%%% Conservative and entropic models
It is important for a DV model~\eqref{eq:dvm} to preserve conservation and entropy properties
of the continuous kinetic equation~\eqref{eq:Boltzmann}.
For the BGK model
\begin{equation}\label{eq:dvm-bgk}
    J(\hat{f}_\alpha) = \frac1\tau\br{\equil{\hat{f}}_\alpha-\hat{f}_\alpha},
\end{equation}
it can be accomplished when the discrete equilibrium function \(\equil{\hat{f}}_\alpha\)
is defined from the minimum entropy principle~\cite{Mieussens2000}:
\begin{equation}\label{eq:equilibrium-bgk}
    \equil{f}_\alpha = \exp(\beta_r\psi_{\alpha r}), \quad
    \psi_{\alpha r} = \transpose{\br{1,\bxia, \xi_\alpha^2}},
\end{equation}
where \(\beta_r\in\mathbb{R}^{D+2}\) is unique solution of
\begin{equation}\label{eq:beta}
    \sum_\alpha\psi_{\alpha r}\br{\equil{\hat{f}}_\alpha - \hat{f}_\alpha} = 0.
\end{equation}

%%% DV method
Let the molecular velocities \(\bxia\) be restricted to a Cartesian lattice \(X\)
with uniform spacing \(c\), called the \emph{lattice speed}.
The classical \emph{DV method} of solving~\eqref{eq:Boltzmann} is based on such approximations \(J(\hat{f}_\alpha)\)
that are consistent with \(J(f)\) when \(c\) goes to zero~\cite{Aristov2001}.
Due to exponential decay of the VDF, the given accuracy can be achieved
by cutting off all velocities that satisfy condition \(|\xi_\alpha| > \xi^{(\mathrm{cut})}\) from \(X\).

%%% LB method
A set of velocities with the corresponding weights \(\Set{(\bxia,w_\alpha)}{\alpha=1,\dots,Q}\)
is called the \emph{quadrature} (\emph{cubature} if \(D>1\)) \emph{rule}~\cite{Stroud1971}.
A common notation D\(n\)Q\(m\) means \(D=n\) and \(Q=m\).
Hereinafter, a quadrature rule, together with the discrete operator in form~\eqref{eq:dvm-bgk},
is referred as the \emph{lattice-BGK (LBGK) model}.
When the VDF is close to equilibrium, it can be acceptably approximated using quadratures with small number \(Q\).
LBGK models are capable of reproducing low-order polynomial moments of the VDF accurately and,
therefore, describing a fluid-dynamic behavior of a gas, including that beyond the Navier--Stokes level.
They serve as a foundation for the \emph{lattice Boltzmann (LB) method}.

%%% Difference between DV and LB
Within the introduced terminology, the only difference between the DV and LB methods is an ultimate goal.
The former one strives to capture kinetic properties of highly nonequilibrium flows and, therefore,
forced to have a quite large dimension of the approximation space \(Q\).
The latter one, on the contrary, is based on the assumption of a slightly perturbed equilibrium and, therefore,
tends to describe it in the most efficient way.

%%% Gauss--Hermite
In the present paper, we employ LBGK models based on the projection of the VDF onto Hermite basis~\cite{Shan2006}:
\begin{multline}\label{eq:gauss-hermite}
    \equil{f}_\alpha = \rho\Bigg\{ 1 + 2\xiai v_i + \overbrace{
        2\br{\xiai v_i}^2 - v_i^2 + (T-1)\br{\xiai^2 - \frac{D}2}
    }^\text{second order} + \\ \underbrace{
        \frac{2\xiaj v_j}3\left[ 2\br{\xiai v_i}^2 - 3v_i^2 + 3(T-1)\br{\xiai^2 - 1 - \frac{D}2}\right]
    }_\text{third order} + \cdots \Bigg\}.
\end{multline}
Truncating of~\eqref{eq:gauss-hermite} compromises the positivity condition and,
therefore, these LBGK models are not entropic, but the conservation properties are preserved.

\subsection{Time-integration method}\label{sec:splitting}

%%% Splitting scheme
For the present study, we start from the simplest numerical algorithm providing the second-order accuracy
for both time and physical coordinates.
Equation~\eqref{eq:Boltzmann} is solved by the symmetric Strang's splitting scheme~\cite{Bobylev2001}
\begin{equation}\label{eq:Strang}
    S^{\Delta{t}}_{A+B}(f_0) = S^{\Delta{t}/2}_A \br{S^{\Delta{t}}_B \br{S^{\Delta{t}/2}_A(f_0)} } + \OO{\Delta{t}^3},
\end{equation}
where \(A(f) = -\xi_i\Pder[f]{x_i}\), \(B(f) = J(f)/k\), \(\Delta{t}\) is the time step.
\(S^t_P (f_0)\) denotes the solution of the Cauchy problem
\begin{equation}\label{eq:Cauchy}
    \pder[f]{t} = P(f), \quad f|_{t=0} = f_0.
\end{equation}

%%% Space-homogeneous
Important implication of the splitting procedure is that the space-homogeneous BGK equation
\begin{equation}\label{eq:bgk_homogeneous}
    \pder[f]{t} = \frac{\rho}k \br{\equil{f}-f}
\end{equation}
has the exact solution
\begin{equation}\label{eq:bgk_exact}
    f(t) = \equil{f} + \br{f(t_0)-\equil{f}}\exp\br{-\frac{\rho}k (t-t_0)}.
\end{equation}
Moreover, generalization of this algorithm to the original Boltzmann equation is straightforward.

%%% Steady-state solution
To find a steady-state solution of the boundary-value problem,
the time-marching process is started from some initial approximation
and continues until the convergence criterion is met.

\subsection{Finite-volume formulation}\label{sec:fv}

For brevity, we consider a one-dimensional physical space.
The transport equation
\begin{equation}\label{eq:transport}
    \pder[f]{t} + \xi_1\pder[f]{x_1} = 0
\end{equation}
is approximated by the \emph{finite-volume (FV)} method:
\begin{equation}\label{eq:finite_volume}
    f^{n+1}_m = f^n_m - \frac{\Delta{t}}{\Delta{x_m}}\br{F^n_{m+1/2}-F^n_{m-1/2}}, \quad
    m = 1,\dots,M, \quad n\in\mathbb{N},
\end{equation}
where \(\Delta{x_m}\) is the width of \(m\) cell in the physical space,
\begin{equation}\label{eq:vdf_fv}
    f^n_m(\bxi) = f\br{n\Delta{t}, \frac{\Delta{x_m}}2 + \sum_{k=1}^{m-1}\Delta{x_k}, \bxi}.
\end{equation}
For \(\xi_1>0\), the internal fluxes can be written in the following form:
\begin{equation}\label{eq:internal_fluxes}
    F_{m+1/2}^n = \xi_1\br{ f_m^n + \frac{1-\gamma}2\overline{\Delta{f_m^n}}},
    \quad \gamma = \frac{\xi_1\Delta{t}}{\Delta{x_m}}, \quad m = 1,\dots,M.
\end{equation}
They are calculated by the second-order total variation diminishing (TVD) scheme,
e.g., with the monotonized central (MC) slope limiter
\begin{equation}\label{eq:limiter}
    \overline{\Delta{f_m^n}} = \begin{cases}\min\br{
         2\frac{|D_-|}{h_-}, \frac12\frac{|D_-+D_+|}{h_-+h_+}, 2\frac{|D_+|}{h_+}
    }\Delta{x_m}, \quad D_+D_- > 0, \\
    0, \quad D_+D_- \leq 0,
    \end{cases}
\end{equation}
where
\begin{equation}\label{eq:differences}
    D_\pm = \pm\br{f_{m\pm1}^n - f_m^n}, \quad h_\pm = \frac{\Delta{x_{m\pm1}} + \Delta{x_m}}2.
\end{equation}
The last flux \(F_{M+1/2}^n\) is calculated based on the linear extrapolation of the solution for the ghost cell:
\begin{equation}\label{eq:last_ghost}
    f_{M+1}^n = 2f_M^n - f_{M-1}^n.
\end{equation}
Note that sharp variations (in physical space) of solution can occur even for nearly incompressible flow,
especially for large \(|\bxi|\).

%%% Boundary conditions
The diffuse-reflection boundary condition~\eqref{eq:diffuse}, e.g. at \(x=0\),
is introduced through the first flux and ghost cell:
\begin{gather}
    F_{1/2}^n(\bxia) = \displaystyle\xi_1\frac{\sum_{\xi'_{\alpha1}<0}F_{m+1/2}^n(\bxia')}
        {\sum_{\xi'_{\alpha1}<0}\xi'_{\alpha1}\equil{f}(\bxi'_\alpha,1,v_{Bi},T_B)}
        \equil{f}(\bxi_\alpha, 1, v_{Bi}, T_B) \quad (\xi_{\alpha1}>0), \label{eq:first_flux}\\
    f_0^n(\bxia) = \displaystyle\frac{\sum_{\xi'_{\alpha1}<0}\xi'_{\alpha1}f_1^n(\bxia')}
        {\sum_{\xi'_{\alpha1}<0}\xi'_{\alpha1}\equil{f}(\bxi'_\alpha,1,v_{Bi},T_B)}
        \equil{f}(\bxi_\alpha, 1, v_{Bi}, T_B) \quad (\xi_{\alpha1}>0). \label{eq:first_ghost}
\end{gather}
This implementation yields the second-order accuracy along with conservation of mass.
For \(\xi_1<0\), all expressions are analogous.

%%% Why half-integer lattice?
The boundary conditions also dictate a way of discretization in the velocity space.
With respect to the origin of the velocity coordinates, only two types of lattices are symmetric~\cite{Inamuro1990}:
\emph{integer} \(\br{\xiai/c} \in \mathbb{Z}^D\) and \emph{half-integer} \(\br{\xiai/c + e_i/2} \in \mathbb{Z}^D\),
where \(e_i\) is the corresponding orthonormal basis.
For the considered boundary condition at \(x=0\) and \(D>1\), there is a zero-measure set of velocities
\(\Set{\bxi\in\mathbb{R}^D}{\xi_1=0}\), called tangential.
These velocities are immune to the diffuse reflection.
In contrast, the integer lattice contains a substantial subset of tangential velocities.
Therefore, to avoid an additional discretization error, the half-integer lattice should be employed.

%%% Diffuse reflection for LB
In the same manner, LB cubatures without tangential velocities are preferable to the classical ones.
Moreover, LB models can be augmented by special groups of velocities to approximate
the diffuse-reflection boundary condition more accurately~\cite{Feuchter2016}.
These models ensure vanishing errors of the relevant half-space integrals.

\subsection{Coupling algorithm}\label{sec:coupling}

%%% Homogeneous domain decomposition problem
Staying within the FV framework, divide our computational domain in the physical space
into subdomains, each employing its own DV model.
The coupling conditions at the interface between subdomains can be considered as virtual boundary conditions.
They are symmetric due to unified formulation in the physical space.

%%% Projection and reconstruction procedures
The concept of ghost cells suggests the simplest (from the algorithmic point of view) coupling strategy.
If the interface between subdomains lies in the near-continuum region,
it is admissible to exchange information only within a Hilbert subspace \(\mathcal{H}\)
spanned by the first \(D\)-dimensional Hermite polynomials.
Then, all that we need is to supplement each DV model with a mapping to this subspace.
In other words, a projection and reconstruction procedures should be implemented.
While the former one is nothing else but weighted cubatures~\eqref{eq:cubature},
the latter one depends on the underlying DV model.

%%% Specify the reconstruction procedure
In the present paper, we confine \(\mathcal{H}\) to the first \(1+5D/2+D^2/2\) polynomials
corresponding to the macroscopic variables~\eqref{eq:macro}.
Note that pressure tensor \(p_{ij}\) contains \(D(D+1)/2\) independent values due to its symmetric nature.
For the DV model, the desired reconstruction coincides with the Grad's thirteen-moment method:
\begin{equation}\label{eq:interface_dvm}
    f^{\mathcal{H}}_\alpha = \equil{f}_\alpha\br{
        1 + \frac{\cai\caj P_{ij}}{\rho T^2} + \frac{4\cai q_i}{\rho T^2}\br{\frac{\caj^2}{(D+2)T} - \frac12} },
\end{equation}
where \(P_{ij} = p_{ij} - \rho T\delta_{ij}\) and \(\cai = \xiai - v_i\).
For the Gauss--Hermite LB model, the appropriate discrete VDF is recovered as a part of the third-order Hermite expansion:
\begin{equation}\label{eq:interface_lbm}
    f^{\mathcal{H}}_\alpha = \equil{f}_\alpha + \underbrace{ \vphantom{ \br{\frac{2\xiaj^2}{D+2}-1} }
        \xiai\xiaj P_{ij}
    }_\text{second order} + \underbrace{
        2\xiai q_i\br{\frac{2\xiaj^2}{D+2}-1} - 2\xiai P_{ij}\br{ v_j - \xiaj\xiak v_k }
    }_\text{third order}.
\end{equation}

%%% Conservative scheme
At first glance, the proposed reconstruction procedure does not violate the conservation properties of the FV scheme,
because all moments required for the equilibrium function are calculated exactly.
However, the reality is different, since the FV scheme actually deals separately
with velocities directed in the opposite half-spaces with respect to the interface.
For this reason, mass, momentum, and energy fluxes across the coupling interface
are turned out to differ slightly for each DV model.
To ensure conservation properties, it is sufficient to correct the total flux of the VDF for one of the DV models.
In the present paper, we employ the polynomial correction (like in~\cite{Aristov1980}):
\begin{equation}\label{eq:poly_correction}
    \bar{F}^{(1)}_\alpha = F^{(1)}_\alpha(1+\gamma_r\psi_{\alpha r}), \quad
    \sum_{\alpha=1}^{Q^{(1)}} \bar{F}^{(1)}_\alpha\psi_{\alpha r} = \sum_{\alpha=1}^{Q^{(2)}} F^{(2)}_\alpha\psi_{\alpha r},
\end{equation}
where \(F^{(s)}\) and \(Q^{(s)}\) are the initial flux and number of velocities of \(s\) model, respectively,
\(\bar{F}^{(1)}\) is the corrected flux, \(\psi_{\alpha r}\) is defined in~\eqref{eq:beta}.
In practice, each component of \(\gamma_r\in\mathbb{R}^{D+2}\) is much less than unity;
therefore, the positivity is also preserved.

%%% 1D example
Finally, let us return to the one-dimensional example outlined in sec.~\ref{sec:fv}
and suppose that \(x=0\) is our interface.
In order to use~\eqref{eq:internal_fluxes}, the VDF should be reconstructed in the ghost cells.
In case of the second-order TVD scheme, \(f_{-1}^n\) is used for all \(\bxia\) and, additionally,
\(f_{-2}^n\) is required when \(\xi_{\alpha1}>0\).
In case of the first-order scheme, \(f_{-1}^n\) and only for \(\xi_{\alpha1}>0\) is sufficient.

\section{Results and discussions}\label{sec:examples}

\begin{comment}
\subsection{Couette-flow problem}

The proposed hybrid method is validated on the linearized Couette-flow problem.
Gas is placed between two infinite parallel plates at \(y=\pm1/2\) with constant temperature \(T=1\)
and velocities \((v = \pm\Delta v/2, 0, 0)\).

The average density is equal to unity:
\begin{equation}\label{eq:total_mass}
    \int_{-1/2}^{1/2}\rho\dd{y} = 1.
\end{equation}

The physical space \(0<y<1/2\) is divided into \(N=40\) nonuniform cells refined near \(y=1/2\).
The obtained results for \(k=0.1\) and \(\Delta v = 0.02\) are presented in Fig.*.
The buffer layer is located at a distance of \(1.2\) mean free path from the plate.

The three-dimensional velocity space is discretized by the uniform lattice confined with the sphere of radius \(\xi_{\max}=4\).
\(M=16\) nodes are placed along each axis. The total amount of nodes is equal to \(2176\).

The Couette-flow problem for the linearized BGK equation can be reduced to a one-dimensional integral equation
and, therefore, can be solved with extremely high accuracy~\cite{Luo2015, Luo2016}.

% Couple BC velocity correction: mass 5.88361e-07, momentum -7.2453e-06, energy -4.8491e-06 % U=0.02
% Couple BC velocity correction: mass 6.85421e-05, momentum -5.63719e-05, energy -0.000436927 % U=0.2
\end{comment}

\printbibliography

\end{document}

In an alternative approach, Grad (1949b, 1952) proposed to approximate the Boltzmann equation by expanding the single-particle VDF on the basis of the Hermite orthogonal polynomials in velocity space.

ЛБ  это по словам Суччи   "nonlinearity is local, advection is linear"

[Succi, 2001] differential form of the LBE

in a coherent manner -- согласованно

infinite non-equilibrium in the Sobolev norms

As a proof of concept of a prospective LB-DSMC coupling
