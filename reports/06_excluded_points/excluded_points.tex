%&pdflatex
\documentclass[a4paper,12pt]{article}
\usepackage{amssymb, amsmath}
\usepackage[utf8]{inputenc}
\usepackage[T2A,T1]{fontenc}
\usepackage[english,russian]{babel}
\usepackage{csquotes}
\usepackage{mathrsfs}

\usepackage{fullpage}
\usepackage{indentfirst}
\usepackage[font=small,labelsep=period,tableposition=top]{caption}

\usepackage{graphicx}
\usepackage{comment}

\usepackage[
    pdfauthor={Rogozin Oleg},
    pdftitle={On a set of excluded cubature points},
    colorlinks, pdftex, unicode
]{hyperref}

\title{О множестве исключённых кубатурных точек}
\author{Рогозин Олег}

\newcommand{\Kn}{\mathrm{Kn}}
\newcommand{\Ma}{\mathrm{Ma}}
\newcommand{\dd}{\:\mathrm{d}}
\newcommand{\pder}[2][]{\frac{\partial#1}{\partial#2}}
\newcommand{\pderder}[2][]{\frac{\partial^2 #1}{\partial #2^2}}
\newcommand{\Pder}[2][]{\partial#1/\partial#2}
\newcommand{\dzeta}{\boldsymbol{\dd\zeta}}
\newcommand{\bzeta}{\boldsymbol{\zeta}}
\newcommand{\bh}{\boldsymbol{h}}
\newcommand{\be}{\boldsymbol{e}}
\newcommand{\Nu}{\mathcal{N}}
\newcommand{\OO}[1]{O(#1)}
\newcommand{\Set}[2]{\{\,{#1}:{#2}\,\}}

\begin{document}

\maketitle
\tableofcontents

\section{Основные формулы}

Рассмотрим пространственно-однородное уравнение Больцмана
\begin{equation}\label{eq:Boltzmann}
    \pder[f]{t} = J(f,f)
\end{equation}
со столкновительным членом, записанном в виде
\begin{equation}\label{eq:ci}
    J(f,f) = \int (f'f'_*-ff_*)B\dd\Omega(\boldsymbol\alpha)\dzeta_*.
\end{equation}

\subsection{Дискретизация в пространстве скоростей}

Пусть сетка \(\mathcal{V} = \Set{\zeta_\gamma}{\gamma\in\Gamma}\) в пространстве \(\bzeta\)
(скоростная сетка) построена таким образом, что интеграл произвольной функции \(F(\bzeta)\)
аппроксимируется на ней в виде взвешенной суммы
\begin{equation}\label{eq:zeta_cubature}
    \int F(\bzeta) \dzeta \approx \sum_{\gamma\in\Gamma} F_\gamma w_\gamma =
        \sum_{\gamma\in\Gamma} \hat{F_\gamma},
        \quad \sum_{\gamma\in\Gamma} w_\gamma = V_\Gamma,
        \quad F_\gamma = F(\bzeta_\gamma),
\end{equation}
где \(V_\Gamma\) "--- объём, покрываемый скоростной сеткой.
В дальнейшем будем использовать взвешенные значения функции на сетке \(\hat{F_\gamma}\).
Тогда интеграл столкновений~\eqref{eq:ci}, записанный в симметризованной форме
\begin{equation}\label{eq:symm_ci}
    \begin{aligned}
    J(f_\gamma, f_\gamma) = \frac14\int &\left[
        \delta(\bzeta-\bzeta_\gamma) + \delta(\bzeta_*-\bzeta_\gamma)
        - \delta(\bzeta'-\bzeta_\gamma) - \delta(\bzeta'_*-\bzeta_\gamma)\right] \\
        &\times(f'f'_* - ff_*)B \dd\Omega(\boldsymbol{\alpha}) \dzeta\dzeta_*,
    \end{aligned}
\end{equation}
где \(\delta(\bzeta)\) "--- дельта-функция Дирака в \(\mathbb{R}^3\),
имеет следующий дискретный аналог:
\begin{equation}\label{eq:discrete_symm_ci}
    \hat{J}_\gamma(\hat{f}_\gamma, \hat{f}_\gamma) =
        \frac{\pi V_\Gamma^2}{\sum_{\nu\in\Nu} w_{\alpha}w_{\beta}}
        \sum_{\nu\in\Nu} \left(
            \delta_{\alpha\gamma} + \delta_{\beta\gamma}
            - \delta_{\alpha'\gamma} - \delta_{\beta'\gamma}
        \right)\left(
            \frac{w_{\alpha}w_{\beta}}{w_{\alpha'}w_{\beta'}}
            \hat{f}_{\alpha'}\hat{f}_{\beta'} - \hat{f}_{\alpha}\hat{f}_{\beta}
        \right)B_\nu.
\end{equation}
Здесь \(\alpha=\alpha(\nu)\in\Gamma\), \(\beta=\beta(\nu)\in\Gamma\) и \(\boldsymbol{\alpha}_\nu\)
выбираются по некоторому правилу восьмимерной кубатуры.
\(\Nu\) "--- множество кубатурных точек.
Дискретный аналог распределения Максвелла может быть записан в следующем виде:
\begin{equation}\label{eq:discrete_Maxwell}
    \hat{f}_{M\gamma} = \rho\left[\sum_{\alpha\in\Gamma}w_\alpha\exp
            \left(-\frac{(\bzeta_\alpha - \boldsymbol{v})^2}{T}\right)
        \right]^{-1}
        w_\gamma\exp\left(-\frac{(\bzeta_\gamma - \boldsymbol{v})^2}{T}\right).
\end{equation}

\subsection{Проекционно-интерполяционный метод}

В общем случае, скорости после столкновения,
\(\bzeta_{\alpha'}\) и \(\bzeta_{\beta'}\), не попадают в \(\mathcal{V}\).
Если они просто заменяются ближайшими сеточными скоростями,
\(\bzeta_{\lambda}\in\mathcal{V}\) и \(\bzeta_{\mu}\in\mathcal{V}\),
дискретный интеграл столкновений~\eqref{eq:discrete_symm_ci} теряет свойство консервативности.
Более того, дискретное распределение Максвелла~\eqref{eq:discrete_Maxwell} перестаёт быть равновесным состоянием.
Для решения первой проблемы применяется проецирование \(\bzeta_{\alpha'}\)
на множество узлов \(\Set{\bzeta_{\lambda+s_a}}{a\in\Lambda}\subset\mathcal{V}\):
\begin{equation}\label{eq:ci_projection}
    \delta_{\alpha'\gamma} = \sum_{a\in\Lambda} r_{\lambda,a}\delta_{\lambda+s_a,\gamma}.
\end{equation}
Для определённости будем считать, что \(\Lambda = \Set{a}{r_{\lambda,a}\neq0}\).
\emph{Проекционные веса} \(r_{\lambda,a}\) подбираются таким образом, чтобы обеспечить сохранение массы, импульса и энергии, т.е.
\begin{equation}\label{eq:stencil_conservation}
    \sum_{a\in\Lambda} r_{\lambda,a} = 1, \quad
    \sum_{a\in\Lambda} r_{\lambda,a} \bzeta_{\lambda+s_a} = \bzeta_{\alpha'}, \quad
    \sum_{a\in\Lambda} r_{\lambda,a} \bzeta_{\lambda+s_a}^2 = \bzeta_{\alpha'}^2.
\end{equation}
Набор правил смещения \(\mathcal{S} = \Set{s_a}{a\in\Lambda}\)
называется \emph{проекционным шаблоном}.
Для решения второй проблемы, а именно выполнения
\begin{equation}\label{eq:strict_interpolation}
    \hat{J}_\gamma(\hat{f}_{M\gamma}, \hat{f}_{M\gamma}) = 0,
\end{equation}
подбирается необходимая интерполяция \(\hat{f}_{\alpha'}\).
Рассмотрим её в виде среднего взвешенного по Колмогорову
\begin{equation}\label{eq:Kolmogorov_mean}
    \hat{f}_{\alpha'} = \phi_f^{-1}\left(\sum_{a\in\Lambda} q_{\lambda,a} \phi_f\left(\hat{f}_{\lambda+s_a}\right)\right), \quad
    w_{\alpha'} = \phi_w^{-1}\left(\sum_{a\in\Lambda} p_{\lambda,a} \phi_w\left(w_{\lambda+s_a}\right)\right),
\end{equation}
где соответствующие \emph{интерполяционные веса} нормированы:
\begin{equation}\label{eq:normalized_pq}
    \sum_{a\in\Lambda} q_{\lambda,a} = 1, \quad
    \sum_{a\in\Lambda} p_{\lambda,a} = 1,
\end{equation}
а \(\phi_f\) и \(\phi_w\) "--- непрерывные строго монотонные функции,
\(\phi_f^{-1}\) и \(\phi_w^{-1}\) "--- обратные к ним функции.
Если положить
\begin{equation}\label{eq:geometric_mean}
    \phi_f(x) = \phi_w(x) = \ln(x), \quad \phi_f^{-1}(x) = \phi_w^{-1}(x) = \exp(x), \quad
    p_{\lambda,a} = q_{\lambda,a} = r_{\lambda,a},
\end{equation}
то~\eqref{eq:strict_interpolation} выполняется строго.
Кроме того, несложно показать, что среднее геометрическое вида~\eqref{eq:geometric_mean}
приводит к выполнению дискретного аналога \(H\)-теоремы.
Аналогичные процедуры применяются к \(\delta_{\beta'\gamma}\) и \(\hat{f}_{\beta'}\).

\subsection{Решение задачи Коши}

Если переписать интеграл столкновения~\eqref{eq:discrete_symm_ci} как
\begin{equation}\label{eq:discrete_short_ci}
    \hat{J}_\gamma\left(\hat{f}_\gamma, \hat{f}_\gamma\right) =
    \sum_{j=1}^N \hat{\mathscr{J}}_\gamma^{n+(j-1)/N}
    \left(\hat{f}_\gamma, \hat{f}_\gamma\right), \quad N=|\Nu|,
\end{equation}
то для интервала времени \(\tau = t_{n+1} - t_n\) можно воспользоваться явным методом Эйлера в дробных шагах
\begin{equation}\label{eq:time_scheme}
    \hat{f}_\gamma^{n+j/N} = \hat{f}_\gamma^{n+(j-1)/N} + \tau \hat{\mathscr{J}}_\gamma
    \left(\hat{f}_\gamma^{n+(j-1)/N}, \hat{f}_\gamma^{n+(j-1)/N}\right).
\end{equation}
Если предположить, что все \(\gamma\in\Gamma\) равномерно распределены среди кубатурных точек \(\nu\in\Nu\)
(это достигается случайным перемешиванием точек в \(\Nu\)),
то схема~\eqref{eq:time_scheme} имеет порядок сходимости \(\OO{\tau|\Gamma|/|\Nu|}\) для равномерной сетки.
Если же \(|\Gamma|/|\Nu| = \OO{\tau}\), то схема~\eqref{eq:time_scheme} сходится со вторым порядком.

\subsection{Положительность функции распределения}

Схема~\eqref{eq:time_scheme} допускает отрицательные значения функции распределения,
что противоречит её физической природе. Чтобы обеспечить её положительность,
достаточно потребовать выполнения неравенства
\begin{equation}\label{eq:positive_f}
    \hat{f}_\gamma^{n+(j-1)/N} + \frac{\tau}N \hat{\mathscr{J}}_\gamma^{n+(j-1)/N} > 0
\end{equation}
для всех \(\gamma\in\Gamma\) и \(\nu\in\Nu\).
Если \(\gamma = \alpha\), то получаем оценку
\begin{equation}\label{eq:positive_f_alpha}
    \hat{f}_{\alpha} - \frac{A}{N}\hat{f}_{\alpha}\hat{f}_{\beta} > 0, \quad
    A = \pi\frac{\tau V_\Gamma^2 N B_{\max}}{\sum_{\nu\in\Nu} w_{\alpha}w_{\beta}}
\end{equation}
или
\begin{equation}\label{eq:positive_f_alpha2}
    N > A \hat{f}_{\max},
\end{equation}
где
\begin{equation}\label{eq:hat_f_max}
    \hat{f}_{\max} = \max_{\gamma\in\Gamma} \hat{f}_\gamma, \quad
    B_{\max} = \max_{\boldsymbol\alpha, \bzeta, \bzeta_*} B.
\end{equation}
Такая же оценка справедлива для \(\gamma = \beta\).

Для рассмотрения проекционных скоростей \(\gamma = \lambda+s_a\)
будем использовать интерполяцию вида~\eqref{eq:Kolmogorov_mean}.
Дополнительно положим
\begin{equation}\label{eq:normal_stencil}
    r_{\lambda,a} \leq 1, \quad q_{\lambda,a} \leq 1, \quad p_{\lambda,a} \leq 1.
\end{equation}
Если \(r_{\lambda,a} \geq 0\), то получаем оценку
\begin{equation}\label{eq:positive_f_lambda+}
    \hat{f}_{\lambda+s_a} - \frac{A}{N} w_\alpha w_\beta
    \frac{
        \phi_f^{-1}\left[\sum_{b\in\Lambda} q_{\lambda,b} \phi_f\left(\hat{f}_{\lambda+s_b}\right)\right]
        \phi_f^{-1}\left[\sum_{b\in\Lambda} q_{\mu,b} \phi_f\left(\hat{f}_{\mu+s_b}\right)\right]
    }{
        \phi_w^{-1}\left[\sum_{b\in\Lambda} q_{\lambda,b} \phi_w\left(w_{\lambda+s_b}\right)\right]
        \phi_w^{-1}\left[\sum_{b\in\Lambda} q_{\mu,b} \phi_w\left(w_{\mu+s_b}\right)\right]
    }> 0
\end{equation}
или
\begin{equation}\label{eq:positive_f_lambda2+}
    N > A \hat{f}_{\max} \epsilon_f^2 \epsilon_w^2,
\end{equation}
где наибольшее изменение функции распределения на проекционном шаблоне
\begin{equation}\label{eq:epsilon_f}
    \epsilon_f = \max_{\substack{s_a,s_b\in\mathcal{S}\\\gamma\in\Gamma}} \frac{\hat{f}_{\gamma+s_a}}{\hat{f}_{\gamma+s_b}}
\end{equation}
и максимальное отношение весов скоростной сетки
\begin{equation}\label{eq:epsilon_w}
    \epsilon_w = \max_{\gamma,\varsigma\in\Gamma} \frac{w_\gamma}{w_\varsigma}.
\end{equation}
Для гладкой функции распределения \(\epsilon_f\) меньше для шаблонов с меньшим \emph{диаметром}
\begin{equation}\label{eq:stencil_diameter}
    R_{\mathcal{S}\gamma} = \max_{s_a,s_b\in\mathcal{S}}
        \left| \bzeta_{\gamma+s_a} - \bzeta_{\gamma+s_b} \right|.
\end{equation}
Если же \(r_{\lambda,a} < 0\), то
\begin{equation}\label{eq:positive_f_lambda-}
    \hat{f}_{\lambda+s_a} + \frac{A}{N}r_{\lambda,a} \hat{f}_{\alpha}\hat{f}_{\beta} > 0
\end{equation}
или
\begin{equation}\label{eq:positive_f_lambda2-}
    N > A \hat{f}_{\max} \epsilon_f^2
        \max_{a\in\Lambda,\gamma\in\Gamma}\left(-r_{\gamma,a}\right)
        \max_{\bzeta_\alpha,\bzeta_\beta\in\mathbb{R}^3}
        \frac{\hat{f}_{\alpha}\hat{f}_{\beta}}{\hat{f}_{\alpha'}\hat{f}_{\beta'}}.
\end{equation}
Эта оценка минимальна для распределения Максвелла,
где \(\hat{f}_{\alpha}\hat{f}_{\beta} = \hat{f}_{\alpha'}\hat{f}_{\beta'}\).
Таким образом, чем более неравновесна функция распределения,
тем больше следует брать кубатурных точек.

На практике строгое выполнение условия~\eqref{eq:positive_f} требует больших вычислительных затрат,
поэтому для достижения приемлемой точности достаточно исключать из \(\Nu\) точки,
нарушающие~\eqref{eq:positive_f}. Другими словами, интеграл столкновений можно вычислять как
\begin{equation}\label{eq:discrete_short_ci_discarded}
    \hat{J}_\gamma = \sum_{\nu\in\Nu\setminus\mathcal{M}} \hat{\mathscr{J}}_{\gamma\nu},
\end{equation}
где \(\mathcal{M}\) "--- множество кубатурных точек, исключённых из \(\Nu\).
Чтобы недопустить существенную ошибку при таком методе численного интегрирования,
необходимо контролировать вклад исключённых точек в интеграл столкновения.
Например, достаточно поддерживать малость
\begin{equation}\label{eq:excluded_contribution}
    \frac{\sum_{\nu\in\mathcal{M}} \left| \hat{\mathscr{J}}_{\alpha\nu} \right|}
        {\sum_{\nu\in\Nu\setminus\mathcal{M}} \left| \hat{\mathscr{J}}_{\alpha\nu} \right|}.
\end{equation}

\section{Проекционные шаблоны}

Проекционные шаблоны будем называть \(n\)-точечными,
если проекционный шаблон состоит из \(n\) точек, т.\,е. \(|\mathcal{S}|=n\).
Будем предполагать, что скоростная сетка прямоугольна.
Ввиду симметричности равномерной прямоугольной сетки,
для достижения консервативности достаточно использовать два проекционных узла.
В общем случае необходимо как минимум пять проекционных точек,
чтобы удовлетворить~\eqref{eq:stencil_conservation}.
Диаметр шаблона \(R_{\mathcal{S}\gamma}\) можно уменьшить,
если использовать только ближайшие семь точек.

\subsection{Двухточечная схема для равномерной сетки}

\emph{Двухточечная схема} строится на симметричном проецировании
\begin{equation}\label{eq:uniform_projection}
    \delta_{\alpha'\gamma} = (1-r)\delta_{\lambda\gamma} + r\delta_{\lambda+s,\gamma}, \quad
    \delta_{\beta'\gamma} = (1-r)\delta_{\mu\gamma} + r\delta_{\mu-s,\gamma},
\end{equation}
где \(\bzeta_{\lambda+s} + \bzeta_{\mu-s} = \bzeta_{\lambda} + \bzeta_{\mu}\) и
\begin{equation}\label{eq:stencil_weights2}
    r = \frac{E_0-E_1}{E_2-E_1}, \quad
    E_0 = \bzeta_{\alpha'}^2 + \bzeta_{\beta'}^2, \quad
    E_1 = \bzeta_{\lambda}^2 + \bzeta_{\mu}^2, \quad
    E_2 = \bzeta_{\lambda+s}^2 + \bzeta_{\mu-s}^2.
\end{equation}
При этом справедливы оценки
\begin{equation}\label{eq:weights_ranges2}
    0 \leq r < 1, \quad R_{\mathcal{S}\gamma} = \sqrt3h,
\end{equation}
где \(h^3 = w_\gamma = V_\Gamma/|\Gamma|\).

Рассмотрим интерполяцию на равномерной сетке на основе среднего гармонического:
\begin{equation}\label{eq:symmetric_interpolation}
    f_{\alpha'}f_{\beta'} = \frac{\Delta_1\Delta_2}{r\Delta_1+(1-r)\Delta_2}, \quad
    \Delta_1 = \hat{f}_\lambda \hat{f}_\mu, \quad
    \Delta_2 = \hat{f}_{\lambda+s}\hat{f}_{\mu-s}.
\end{equation}
Она обеспечивает~\eqref{eq:strict_interpolation} с точностью \(\OO{h}\).
Для выполнения~\eqref{eq:positive_f} справедлива оценка
\begin{equation}\label{eq:positive_f_symmetric}
    N > A f_{\max} \epsilon_f.
\end{equation}
Видно, что симметричная интерполяция вида~\eqref{eq:symmetric_interpolation} позволяет обойтись
меньшим числом кубатурных точек по сравнению со интерполяцией~\eqref{eq:geometric_mean},
особенно если функция распределения претерпевает резкие изменения на сетке.

\subsection{Компактная пятиточечная схема}

Пусть \(\boldsymbol{\eta} = \bzeta_{\alpha'} - \bzeta_{\lambda}\),
а \(\bh_+\) и \(\bh_-\) "--- минимальные диагональные смещения на сетке,
такие что \(\bh_+\) направлен в тот же октант, что и \(\boldsymbol{\eta}\),
а \(\bh_-\) лежит в противоположном.
Тогда \emph{компактная 5-точечная схема} строится на узлах
\begin{equation}\label{eq:stencil_nodes5}
    \bzeta_{\lambda+s_0} = \bzeta_{\lambda}, \quad
    \bzeta_{\lambda+s_i} = \bzeta_{\lambda} + (\bh_+\cdot \be_i)\be_i, \quad
    \bzeta_{\lambda+s_4} = \bzeta_{\lambda} + \bh_-,
\end{equation}
где \(\be_i\) "--- базис прямоугольной сетки.
Для выполнения~\eqref{eq:stencil_conservation}, выбираются проекционные веса
\begin{equation}\label{eq:stencil_weights5}
    r_{\lambda,0} = 1 - \sum_{j=1}^4 r_{\lambda,j}, \quad
    r_{\lambda,i} = \frac{\eta_i - r_{\lambda,4}h_{-i}}{h_{+i}}, \quad
    r_{\lambda,4} = \frac{\boldsymbol{\eta}\cdot(\boldsymbol{\eta} - \bh_+)}
        {\bh_-\cdot(\bh_- - \bh_+)}.
\end{equation}
Для равномерной сетки справедливы оценки
\begin{equation}\label{eq:weights_ranges5}
    0 < r_{\lambda,0} \leq 1, \quad
    -\frac1{12} \leq r_{\lambda,i} < \frac{11}{24}, \quad
    -\frac18 \leq r_{\lambda,4} \leq 0, \quad
    R_{\mathcal{S}\gamma} = \sqrt6h.
\end{equation}

\subsection{Симметричная семиточечная схема}

\emph{Симметричная 7-точечная схема} строится на узлах
\begin{equation}\label{eq:stencil_nodes7}
    \bzeta_{\lambda+s_0} = \bzeta_{\lambda}, \quad
    \bzeta_{\lambda+s_{\pm i}} = \bzeta_{\lambda} + (\bh_\pm\cdot \be_i)\be_i.
\end{equation}
Для выполнения~\eqref{eq:stencil_conservation}, выбираются проекционные веса
\begin{equation}\label{eq:stencil_weights7}
    r_{\lambda,0} = 1 - \sum_{j=1}^3 r_{\lambda,j} + r_{\lambda,-j}, \quad
    r_{\lambda,\pm i} = \pm\frac{\eta_i(\eta_i - h_{\mp i})}{h_{\pm i}(h_{+i}-h_{-i})}.
\end{equation}
В~\eqref{eq:stencil_weights7} не используется соглашение Эйнштейна, суммирование по повторяющимся индексам не производится.
Для равномерной сетки справедливы оценки
\begin{equation}\label{eq:weights_ranges7}
    \frac14 \leq r_{\lambda,0} \leq 1, \quad
    0 \leq r_{\lambda,\pm i} \leq \frac38, \quad
    -\frac18 \leq r_{\lambda,\mp i} \leq 0, \quad
    R_{\mathcal{S}\gamma} = 2h.
\end{equation}

\end{document}
