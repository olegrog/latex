%&pdflatex
\documentclass{article}
\usepackage[utf8]{inputenc}
\usepackage[T1]{fontenc}
\usepackage[english]{babel}
\usepackage{csquotes}

%%% functional packages
\usepackage{amssymb, mathtools, amsthm, mathrsfs}
\usepackage{graphicx}
\usepackage{physics}
\usepackage{siunitx}

%%% configuration packages
\usepackage{fullpage}

\usepackage[
    pdfusetitle,
    colorlinks
]{hyperref}

\usepackage[
    backend=biber,
    sorting=none,
    giveninits,
]{biblatex}
\bibliography{phase_field}

\title{A toy phase-field model for crystallization}
\author{Oleg Rogozin}

\begin{document}
\maketitle

% added by Vladimir
\begin{figure}
    \centering
    \includegraphics{SantaRudolph}
    \caption{Santa Claus and his faithful assistant reindeer Rudolph.}
    \label{fig:SantaRudolph}
\end{figure}

\begin{abstract}
Despite the global warming on the Earth, everyone has ever observed
how frost draws intricate patterns on the window glass in winter.
As a child, we were sometimes told that this is the tricks of Santa Claus (Fig.~\ref{fig:SantaRudolph}).
But what is the underlying physics of this phenomenon?
\end{abstract}

\tableofcontents

\section{Model}

%%% Mechanism of solidification
When the weather outside is cold, moisture in a warm room condenses on the inner surface to form a thin water film.
If the temperature outside is sufficiently low, ice nucleates and grows.
The region around the ice crystal becomes depleted in moisture.
The liquid water has to arrive to the ice crystal by diffusion through the depleted zone.
Suppose that a small part of the ice crystal accidentally advances further then the rest of the interface.
The diffusion distance for that perturbation decreases, and hence the perturbation grows faster.
This leads to the formation of a branch, and a \emph{branching instability} is said to have formed.
The mechanism described here is essentially how snowflakes are supposed to form,
by the diffusion of water molecules through air on to the ice crystals.
The essential difference is that snowflakes have the \emph{dendritic} morphology in three dimensions~\cite{dendrites}.

%%% Gibbs--Thomson effect
Branching instability is restrained by increasing surface tension of the solid--liquid interface,
which depresses, in particular, the equilibrium freezing temperature when the interface is convex towards the liquid.
Mathematically, this is expressed by the boundary condition at the interface,
which is called the \emph{Gibbs--Thomson relationship}~\cite{worster2000solidification}:
\begin{equation}\label{eq:Gibbs--Thomson}
    T_I = T_M - \Gamma\div\vu{n}_I,
\end{equation}
where $T_I$ is the equilibrium temperature of the interface, $T_M$ is the melting temperature,
$\vu{n}_I$ is the unit vector normal to the interface and pointing into the liquid,
$\div\vu{n}_I$ is the curvature of the interface,
$\Gamma = \gamma T_M/\rho L$ is the Gibbs--Thomson coefficient (\si{K.m}),
$\gamma$ is the surface energy (\si{J/m^2}), $\rho$ is the density of the solid (\si{kg/m^3}),
and $L$ is the latent heat of fusion per unit mass (\si{J/kg}).

%%% Free energy
To observe the occurrence of branching instability, a phase-field model can be analyzed~\cite{kobayashi1993modeling}.
Let the temperature field $T(\vb{x},t)$ be supplemented by the phase field $\phi(\vb{x},t)$,
which can be considered as an ordering parameter within the Ginzburg--Landau formalism~\cite{landau1950theory}.
The solid--liquid interface is expressed by a steep but smooth transition from the solid phase $\phi=0$
to the liquid phase $\phi=1$.
Let us postulate the following free energy functional of the system:
\begin{equation}\label{eq:Ginzburg--Landau}
    \Phi[\phi] = \int\qty(\frac{\epsilon^2}2|\grad\phi|^2 + F(\phi))\dd{\vb{x}},
\end{equation}
where $\epsilon(\vu{n})$ is a small parameter (\si{m}), which determines the thickness of the phase transition layer.
Dependence of $\epsilon$ on $\vu{n}=\grad\phi/|\grad\phi|$ allows us to introduce kinetic anisotropy into the model.
The function $F(\phi)$ is a double-well potential, which has local minimums at $\phi=0$ and $1$, of the form
\begin{equation}\label{eq:F}
    F(\phi) = \frac14\phi^4 - \qty(\frac12 + \frac{m}3)\phi^3 + \qty(\frac14 + \frac{m}2)\phi^2,
\end{equation}
where parameter $|m|<\frac12$ represents a thermodynamic driving force.
The difference of the two minimum values of $F$ is proportional to $m$,
which gives a difference of chemical potentials of both the phases.

%%% Evolution equation for phase field
In the context of statistical mechanics, the equilibrium function $\phi$ minimizes the free energy $\Phi$
among some suitable class of functions, i.e., equilibrium $\phi$ is found from $\fdv*{\Phi}{\phi} = 0$,
which leads to the Euler--Lagrange equations.
For a non-equilibrium problem, $\phi$ does not minimize $\Phi$ but dissipates in time to equilibrium:
\begin{equation}\label{eq:phi}
    \tau\pdv{\phi}{t} = -\fdv{\Phi}{\phi}
        = \div( \epsilon\pdv{\epsilon}{\grad\phi}|\grad\phi|^2 + \epsilon^2\grad\phi)
        + \phi(1-\phi)\qty(\phi - \frac12 - m),
\end{equation}
where $\tau$ is a relaxation time, whose inverse is an intrinsic interfacial mobility.
For a two-dimensional case, the phase-field equation~\eqref{eq:phi} can be rewritten as
\begin{equation}\label{eq:phi2d}
    \tau\pdv{\phi}{t} =
        - \pdv{x}\qty(\epsilon\epsilon'\pdv{\phi}{y})
        + \pdv{y}\qty(\epsilon\epsilon'\pdv{\phi}{x})
        + \div(\epsilon^2\grad\phi)
        + \phi(1-\phi)\qty(\phi - \frac12 - m),
\end{equation}
where $\epsilon=\epsilon(\Theta)$, $\epsilon' = \dv*{\epsilon}{\Theta}$,
$\Theta$ is an angle between $\grad\phi$ and the $x$ axis.
Eq.~\eqref{eq:phi2d} is derived straightforwardly from Eq.~\eqref{eq:phi} by using
\begin{equation}\label{eq:Dtheta}
    \pdv{\Theta}{\grad\phi} = \frac{\vu{z}\cross\grad\phi}{|\grad\phi|^2},
\end{equation}
where $\vu{z}$ is a unit vector in the $z$ direction.

%%% Driving force
Next, let us assume that the driving force of interfacial motion is proportional to the supercooling
plus thermal fluctuations and therefore take the form
\begin{equation}\label{eq:m}
    m(T) = \frac{a_1}{\pi}\arctan\qty[\pi\beta(T_M-T)] + \frac{a_2}{2}\chi,
\end{equation}
where all coefficients are nonnegative and stochastic term $\chi(\vb{x},t)$
has a uniform distribution on interval $[-1,1]$
with correlation function $\langle\chi(\vb{x},t)\chi(\vb{x}',t')\rangle = \delta(\vb{x}-\vb{x}')\delta(t-t')/3$.
The thermal fluctuations are basically introduced
to force dendritic side-branching of an unstable solid--liquid interface
and are naturally modeled by Gaussian white noise~\cite{karma1993fluctuations},
but a bounded stochastic term is incorporated in Eq.~\eqref{eq:m} to ensure that $|m|<\frac12$.
Due to the same reason, function $\arctan[\pi\beta(T_M-T)]/\pi$ is used instead of $\beta(T_M-T)$.
Thus, inequality $a_1+a_2<1$ should be satisfied.

%%% Asymptotic analysis
Let us introduce $\epsilon_0$ as a mean thickness of the phase-transition layer, i.e.,
\begin{equation}\label{eq:epsilon}
    \epsilon(\Theta) = \epsilon_0\sigma(\Theta), \quad
    \int_0^{2\pi}\sigma\dd\Theta = 0.
\end{equation}
Then, asymptotic analysis of the phase-field equation~\eqref{eq:phi2d} for $m(T) = \beta(T_M-T)$
in the sharp-interface limit~\cite{caginalp1986analysis}:
\begin{equation}\label{eq:sil}
    \epsilon_0\to0, \quad \beta=\order{\epsilon_0}, \quad \tau=\order{\epsilon_0^2},
\end{equation}
yields an anisotropic non-equilibrium Gibbs--Thomson relationship~\cite{mcfadden1993phase}:
\begin{equation}\label{eq:Gibbs--Thomson2}
    T_I = T_M - \frac{1}{\sqrt2\beta} \qty(
        \qty(\epsilon_I + \epsilon_I'')\div\vu{n}_I
        + \frac{\tau}{\epsilon_I} v_I ),
\end{equation}
where $v_I$ is the normal velocity of the interface that is positive for the formation of solid (\si{m/s}),
$\epsilon_I=\epsilon(\Theta_I)$, and $\Theta_I$ is defined as the angle between $\vu{n}_I$ and the $x$ axis.
Eq.~\eqref{eq:Gibbs--Thomson2} in the equilibrium case ($v_I=0$) can be compared with Eq.~\eqref{eq:Gibbs--Thomson}
to relate $\beta$ to the surface energy $\gamma$. The coefficient before $v_I$ is referred to as kinetic.

%%% Temperature
The governing equation for temperature $T$,
\begin{equation}\label{eq:temperature}
    \pdv{T}{t} = D\laplacian{T} + \frac{L}{c_p}\pdv{\phi}{t},
\end{equation}
is derived from the conservation of enthalpy.
Here, $D$ is the thermal diffusivity (\si{m^2/s}) and
$c_p$ is the specific heat capacity at constant pressure (\si{J/kg.K}),
which are both assumed to be constant.

%%% Dimensionless variables
Let us introduce the following nondimensional variables (denoted by hat):
\begin{equation}\label{eq:dimensionless}
    \vb{x} = \hat{\vb{x}}H, \quad t = \hat{t}H^2/D, \quad
    T = T_M + T_C(\hat{T}-1), \quad \tau = \hat{\tau}H^2/D, \quad
    \epsilon_0 = \hat{\epsilon}_0H, \quad \beta = \hat{\beta}/T_C,
\end{equation}
where $H$ is the characteristic length, $T_C$ is the undercooling temperature.

\section{Problem}

The goal of the project is to analyze numerically equations~\eqref{eq:phi2d} and~\eqref{eq:temperature}
in the square 2D domain $\hat{\vb{x}}\in[-\hat{H},\hat{H}]^2$.
The initial conditions are $\hat{T}=0$ and $\phi=1$ plus nucleation of solid phase occurs at $\vb{x}=0$ and $t=0$.
The boundary conditions are $\vu{n}_B\vdot\grad\phi=0$ and $\vu{n}_B\vdot\grad\hat{T}=0$,
where $\vu{n}_B$ is a vector normal to the boundary.
Let the anisotropy function take the form
\begin{equation}\label{eq:sigma}
    \sigma(\Theta) = 1 + \delta\cos\qty[j(\Theta-\Theta_0)],
\end{equation}
where $\delta$ means the strength of anisotropy, $j$ is the number of its modes,
and the dimensionless parameters be specified as follows:
\begin{equation}\label{eq:parameters}
    \begin{gathered}
    \delta=0.04, \quad j=6, \quad \Theta_0=\frac{\pi}2, \quad a_1=0.9, \quad a_2 = 0.01, \\
    \hat{H} = 5, \quad \hat{\epsilon}_0 = 0.01, \quad \hat{\tau} = 0.0003, \quad \hat{\beta} = 10.
    \end{gathered}
\end{equation}
Additionally, let the dimensionless undercooling $S = c_p T_C/L$,
which is often called the Stefan number, vary from $0.5$ to $2$.

\section{Postscriptum}

Finally, it worth noting that the considered model corresponds to the supercooling solidification of water
and does not describe the formation of snowflakes,
since snowflakes are formed by vapor growth which has a different process from melt growth.
In order to simulate such a process, the formation of a facet
and its destabilization should be considered~\cite{yokoyama1990pattern}.
Incorporation of the advanced anisotropy function allows the described phase-field model
to qualitatively reproduce the growth dynamics of complex 3D snowflake morphology~\cite{demange2017phase}.

\appendix

\section{Sharp-interface limit}

%%% Literature review
In this section, we show how the Gibbs--Thomson relationship~\eqref{eq:Gibbs--Thomson2}
can be derived from the phase-field equation~\eqref{eq:phi2d} in the asymptotic limit~\eqref{eq:sil}.
In the absence of anisotropy, systematic analysis were performed in~\cite{caginalp1986analysis},
where the classical method of the matched asymptotic expansions was employed.
The results were generalized for an anisotropic case in~\cite{mcfadden1993phase}.
Note that a wrong formula is derived in~\cite{kobayashi1993modeling},
since the author presumes that $\phi$ is axisymmetric in the vicinity of each point of the interface,
which is not the case for the anisotropic formulation.
Below, a simplified version of the asymptotic analysis is presented.

%%% Polar coordinates
For each point $\vb{x}_I$ on the interface, which is a curve defined by $\phi=1/2$,
let us go to the local polar coordinate system
such that the origin coincides with the center of the interface curvature:
\begin{equation}\label{eq:polar}
    x = r\cos\theta, \quad y = r\sin\theta, \quad r_I = (\div\vu{n}_I)^{-1}, \quad \theta_I = \Theta_I.
\end{equation}
This choice of coordinates provides the following asymptotic relation in the vicinity of $\vb{x}_I$:
\begin{equation}\label{eq:polar_asymptotics}
    \pdv{r} = \order{\epsilon_0^{-1}}, \quad \pdv{\theta} = \order{1}, \quad
    r = r_I + \order{\epsilon_0}, \quad \theta = \theta_I + \order{\epsilon_0}.
\end{equation}
Some gradients in the polar coordinates take the form
\begin{equation}\label{eq:polar_gradients}
    \grad\phi = \phi_r\vu{r} + \frac{\phi_\theta}{r}\vu*\theta, \quad
    |\grad\phi|^2\pdv{\Theta}{\grad\phi} = \phi_r\vu*\theta - \frac{\phi_\theta}{r}\vu{r},
\end{equation}
where $\phi_r$ and $\phi_\theta$ denotes the partial derivatives of $\phi$ with respect to $r$ and $\theta$,
$\vu{r}$ and $\vu*\theta$ are unit vectors in the $r$ and $\theta$ directions, respectively.
The time derivative can be expressed in terms of the gradient,
if we assume that interface is moving with velocity $\vb{v}$, i.e.,
\begin{equation}\label{eq:polar_velocity}
    \pdv{\phi}{t} = -\vb{v}\vdot\grad\phi = -v_r\phi_r - \frac{v_\theta\phi_\theta}{r}.
\end{equation}

%%% Asymptotic expansion
Substituting Eqs.~\eqref{eq:polar_gradients} and~\eqref{eq:polar_velocity}
into the phase-field equation~\eqref{eq:phi2d}, we obtain
\begin{equation}\label{eq:phi2d_polar}
    \tau\qty( v_r\phi_r + \frac{v_\theta\phi_\theta}{r} ) + \div(
        \qty[ \epsilon^2\phi_r - \epsilon\epsilon'\frac{\phi_\theta}{r} ]\vu{r}
        + \qty[ \epsilon\epsilon'\phi_r + \epsilon^2\frac{\phi_\theta}{r} ]\vu*\theta
    ) + \phi(1-\phi)\qty(\phi - \frac12 - m) = 0,
\end{equation}
which can be reduced to the asymptotic form
\begin{equation}\label{eq:phi2d_asymptotic}
    \epsilon^2\phi_{rr} + \qty(
        \tau v_r + \qty[\epsilon^2 +  (\epsilon')^2 + \epsilon\epsilon'']\frac1{r_I}
        + 2\epsilon\epsilon'\frac{\phi_r\phi_{r\theta} - \phi_\theta\phi_{rr}}{r_I\phi_r^2}
    )\phi_r + \phi(1-\phi)\qty(\phi - \frac12 - m) = \order{\epsilon_0^2},
\end{equation}
where $\tau=\order{\epsilon_0^2}$ and $m=\order{\epsilon_0}$ are assumed.
The following asymptotic relations have been used to derive Eq.~\eqref{eq:phi2d_asymptotic}:
\begin{equation}\label{eq:theta_asymptotic}
    \pdv{\Theta}{r} = \frac{\phi_r\phi_{r\theta} - \phi_\theta\phi_{rr}}{r_I\phi_r^2} + \order{\epsilon_0}, \quad
    \pdv{\Theta}{\theta} = 1 + \order{\epsilon_0},
\end{equation}
which are obtained from Eqs.~\eqref{eq:polar_gradients} using the chain rule.
Next, Eq.~\eqref{eq:phi2d_asymptotic} is rewritten as
\begin{equation}\label{eq:phi2d_epsilon_I}
    \tilde\epsilon^2\phi_{rr} + \qty(
        \tau v_r + \qty[\tilde\epsilon^2 + (\tilde\epsilon')^2 + \tilde\epsilon\tilde\epsilon'']\frac1{r_I}
    )\phi_r + 2\tilde\epsilon\tilde\epsilon'\frac{\phi_{r\theta}}{r_I}
    + \phi(1-\phi)\qty(\phi - \frac12 - m) = \order{\epsilon_0^2},
\end{equation}
where $\tilde\epsilon$ denotes $\epsilon(\theta)$ and
\begin{equation}\label{eq:epsilon_asymptotic}
    \epsilon(\Theta) = \epsilon(\theta) + (\Theta - \theta)\epsilon'(\theta) + \order{\epsilon_0^2}, \quad
    \Theta = \theta + \frac{\phi_\theta}{r_I\phi_r} + \order{\epsilon_0^2}.
\end{equation}
The last equation comes from
\begin{equation}\label{eq:delta_theta}
    \order{\epsilon_0^2} + \tan\theta + (\Theta - \theta)(1+\tan^2\theta) = \tan\Theta
        = \tan\theta + \frac{\phi_\theta}{r\phi_r}(1+\tan^2\theta) + \order{\epsilon_0^2}.
\end{equation}
The first equality in Eq.~\eqref{eq:delta_theta} is just a Taylor expansion,
and the second one is obtained from the definition of $\Theta$ in the polar coordinates.

%%% Asymptotic equations
For small $\epsilon_0$, the solution is looked for in a power series $\epsilon_0$, i.e.,
\begin{equation}\label{eq:phi_expansion}
    \phi = \sum_{n=0}^{\infty} \epsilon_0^n \phi^{(n)}.
\end{equation}
The leading-order solution is given by $\phi^{(0)}=1$ and $\phi^{(0)}=0$ in the liquid and solid regions, respectively.
It is naturally to introduce a streched coordinate $\eta = (r-r_I)/\epsilon_0$ such that $\pdv*{\eta} = \order{1}$.
Then, substituting the series~\eqref{eq:phi_expansion} into Eq.~\eqref{eq:phi2d_epsilon_I}
and arranging the terms by the order of $\epsilon_0$, we obtain
\begin{gather}
    \tilde\sigma^2\phi^{(0)}_{\eta\eta} + g(\phi^{(0)}) = 0, \label{eq:phi0}\\
    \tilde\sigma^2\phi^{(1)}_{\eta\eta} + g'(\phi^{(0)})\phi^{(1)} + \qty(
        \frac{\tau v_r}{\epsilon_0^2} + \qty[\tilde\sigma^2 + (\tilde\sigma')^2 + \tilde\sigma\tilde\sigma'']\frac1{r_I}
    )\phi^{(0)}_\eta + 2\tilde\sigma\tilde\sigma\frac{\phi^{(0)}_{\eta\theta}}{r_I}
    - \frac{m}{\epsilon_0}\phi^{(0)}(1-\phi^{(0)}) = 0, \label{eq:phi1}
\end{gather}
where $\tilde\sigma$ denotes $\sigma(\theta)$ and
\begin{equation}\label{eq:g}
    g(\phi) = \phi(1-\phi)\qty(\phi - \frac12).
\end{equation}
The leading-order solution is
\begin{equation}\label{eq:phi0_solution}
    \phi^{(0)}(\eta,\theta) = \frac12\qty( \tanh\qty(\frac{\eta}{2\sqrt2\sigma(\theta)}) + 1 ).
\end{equation}
Differentiation of Eq.~\eqref{eq:phi0_solution} with respect to $\eta$ shows that $\phi^{(0)}_\eta$
is a homogeneous solution of Eq.~\eqref{eq:phi1}.
Since the homogeneous part of Eq.~\eqref{eq:phi1} is self-adjoint in $\eta\in\mathbb{R}$, i.e.,
\begin{equation}\label{eq:phi0_self-adjoint}
    \int_{-\infty}^{+\infty} \phi\mathcal{L}\psi\dd\eta = \int_{-\infty}^{+\infty} \psi\mathcal{L}\phi\dd\eta, \quad
    \mathcal{L} = \pdv[2]{\eta} + g'(\phi^{(0)}),
\end{equation}
the inhomogeneous part of Eq.~\eqref{eq:phi1} must be orthogonal to the homogeneous solution
(the Fredholm solvability condition), i.e.,
\begin{equation}\label{eq:phi1_solvability}
    \int_{-\infty}^{+\infty} \qty[
        \qty(
            \frac{\tau v_r}{\epsilon_0^2} + \qty[\tilde\sigma^2 + (\tilde\sigma')^2 + \tilde\sigma\tilde\sigma'']\frac1{r_I}
        )\phi^{(0)}_\eta + 2\tilde\sigma\tilde\sigma'\frac{\phi^{(0)}_{\eta\theta}}{r_I}
        - \frac{m}{\epsilon_0}\phi^{(0)}(1-\phi^{(0)})
    ]\phi^{(0)}_\eta \dd\eta = 0,
\end{equation}
or, using relations
\begin{equation}\label{eq:phi0_relations}
    \phi^{(0)}(1-\phi^{(0)}) = \sqrt2\tilde\sigma\phi^{(0)}_\eta, \quad
    \dv{\theta}\int_{-\infty}^{+\infty} \qty(\phi^{(0)}_\eta)^2 \dd\eta =
        -\frac{\tilde\sigma'}{\tilde\sigma}\int_{-\infty}^{+\infty} \qty(\phi^{(0)}_\eta)^2 \dd\eta,
\end{equation}
we have
\begin{equation}\label{eq:phi1_solvability2}
    \frac{\tau v_r}{\epsilon_0^2} + \qty[\tilde\sigma^2 + \tilde\sigma\tilde\sigma'']\frac1{r_I} =
        \frac{\sqrt2\tilde\sigma m}{\epsilon_0},
\end{equation}
which is identical to~\eqref{eq:Gibbs--Thomson2} at $\theta=\theta_I$.

\printbibliography

\end{document}
