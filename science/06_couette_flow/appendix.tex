\section{Функции Абрамовица}\label{sec:Abramowitz}

Семейство функций Абрамовица~\cite{Cercignani2000} имеют следующий вид:
\begin{equation}\label{eq:Abramowitz}
    \mathcal{T}_n(s) = \int_0^\infty t^n \exp\left(-t^2-\frac{s}{t}\right) \dd t,
    \quad s\ge0, \quad n \in \mathbb{Z}.
\end{equation}
Из очевидного соотношения \(\dd \mathcal{T}_n/\dd x = -\mathcal{T}_{n-1}\) следует, что
\[
    \lim_{s\to0} \frac{\dd^{n+1} \mathcal{T}_n(s)}{\dd s^{n+1}} = \infty.
\]
Таким образом, функции \(\mathcal{T}_n(s)\) неаналитичны ввиду особенности в точке \(x=0\),
поэтому не могут быть разложены непосредственно в ряд Тейлора.

Найдём, тем не менее, приближение \(\mathcal{T}_0(s)\) при малых \(s\):\footnote{
    В монографии К.~Черчиньяни <<Динамика разреженного газа>>~(2000)~\cite{Cercignani2000}
    допущена ошибка в форм.~(2.4.19).
}
\begin{gather}
    \mathcal{T}_0(s) = \int_0^1 \left(e^{-t^2}-1\right) e^{-\frac{s}{t}} \dd{t}
        + \int_1^\infty e^{-t^2} e^{-\frac{s}{t}} \dd{t}
        + \int_0^1 e^{-\frac{s}{t}} \dd{t} \notag\\
    = \int_0^1 \left(e^{-t^2}-1\right) \sum_{k=0}^2 \frac{(-s)^k}{k!t^k} \dd{t}
        + \int_1^\infty e^{-t^2} \sum_{k=0}^2 \frac{(-s)^k}{k!t^k} \dd{t}
        + s\int_s^\infty \frac{e^{-t}}{t^2} \dd{t} + o(s^2) \notag\\
    = \left(\frac{\sqrt\pi}2-1\right)
        + \frac{\gamma}2 s
        + (1-\sqrt\pi) \frac{s^2}2
        + e^{-s} - \int_s^\infty\frac{e^{-t}}{t} \dd{t} + o(s^2)  \notag\\
    = \frac{\sqrt\pi}2
        + s\ln{s} + \left(\frac{3\gamma}2 - 1\right) s
        - \sqrt\pi\frac{s^2}2 + o(s^2) \label{eq:Abramowitz0}
\end{gather}
Здесь используется интегральная показательная функция
\begin{equation}\label{eq:exp_integral}
    \mathrm{Ei}(s) = -\int_{-s}^\infty \frac{e^{-t}}{t} \dd{t}
        = \gamma + \ln|s| + \sum_{k=1}^{\infty} \frac{s^k}{kk!},
\end{equation}
где \(\gamma\) "--- постоянная Эйлера"--~Маскерони:
\begin{equation}\label{eq:euler-masceroni}
    \gamma = -\int_0^\infty e^{-t}\ln{t} \dd{t}
        = \int_0^1 \frac{1-e^{-t}}{t} \dd{t} - \int_1^\infty \frac{e^{-t}}{t} \dd{t}.
\end{equation}

Дальнейшее разложение в ряд приводит к следующим рекуррентным
соотношениям~\cite{Abramowitz1953,Abramowitz1972}:\footnote{
    В справочнике по специальным функциям М.~Абрамовица, И.~Стиган~(1979)~\cite{Abramowitz1972}
    см. форм.~(27.5.4).
}
\begin{gather}\label{eq:Abramowitz1-full}
    \mathcal{T}_1(s) = \frac12\sum_{k=0}^\infty ( a_k \ln{k} + b_k ) s^k, \\
    \begin{alignedat}{2}
        a_k &= \frac{-2a_{k-2}}{k(k-1)(k-2)}, &\quad b_k &= \frac{-2b_{k-2} - (3k^2-6k+2)a_k}{k(k-1)(k-2)}, \quad k>2, \\
        a_0 &= a_1 = 0, \quad a_2 = -1, &\quad b_0 &= 1, \quad b_1 = -\sqrt\pi, \quad b_2 = \frac32(1-\gamma).
    \end{alignedat} \notag
\end{gather}
В частности, дифференцируя~\eqref{eq:Abramowitz0}, получаем
\begin{equation}\label{eq:Abramowitz-1}
    \mathcal{T}_{-1}(s) = - \ln{s} - \frac{3\gamma}2 + \sqrt\pi s + \mathcal{O}(s^2\ln{s}).
\end{equation}

\section{Численное решение задачи Куэтта для модели БКВ}\label{sec:numerical_bkw}

Решение задачи Куэтта в рамках линеаризованного уравнения БКВ
сводится к решению уравнения~\eqref{eq:bkw_g_equation}.
Основная сложность решения неоднородного интегрального уравнения Фредгольма второго рода для \(g(y)\)
"--- это логарифмические особенности ядра и первой производной свободного члена
[см.~\eqref{eq:Abramowitz0} и~\eqref{eq:Abramowitz-1}].
Как следствие, первая производная от \(g(y)\)
будет также иметь логарифмическую особенность на границе:\footnote{
    В работе Виллиса~(1962)~\cite{Willis1962} в форм. (2.14) допущено две опечатки.
    Правильный вариант: \( (dG/d\eta) \to G(0)J_{-1}(\eta)/2J_0(0) \to -\pi^{-\frac12}G(0)\ln(\eta)\).
}
\begin{equation}
    \frac{\dd{g}}{\dd{y}} = \frac1{k\sqrt\pi}\left[g\left(\frac12\right)-1\right]\ln\left(\frac12-y\right) + \mathcal{O}(1).
\end{equation}

Метод решения подобных интегральных уравнений описан, например, в~\cite{Atkinson1997}
(см. главу 4.2) и основан на исключении из конечно-разностной аппроксимации
соответствующей сингулярности.
Поскольку свободный член в~\eqref{eq:bkw_g_equation} является приближённым решением для \(k\gg1\),
то непосредственное численное конечно-разностное решение уравнения~\eqref{eq:bkw_g_equation}
не вызывает сложностей для \(k\gtrsim1\). Для малых \(k\) решение~\eqref{eq:bkw_g_equation}
требует больших вычислительных затрат, однако уже для \(k \lesssim 0.05\) будет справедливо
асимптотическое решение~\eqref{eq:small_macro}, как минимум с точностью до 4 знаков.

\section{Вычисление транспортных коэффициентов}\label{sec:gamma8_9} %\(\gamma_8\) и \(\gamma_9\)

\newcommand{\Q}{\ensuremath{\mathcal{Q}^{(0)}}}
\newcommand{\B}{\ensuremath{\mathcal{B}^{(4)}}}
\newcommand{\QQ}{\ensuremath{\tilde{\mathcal{Q}}^{(0)}}}

В~\cite{Sone2002} можно найти общие формулы для вычисления \(\gamma_8\) и \(\gamma_9\):
\begin{equation}\label{eq:gamma_8_9}
    \gamma_8 = I_6(\Q_2) - I_6(\QQ_{22}) + \frac17 I_8(\Q_3) - \frac17 I_8(\QQ_3), \quad
    \gamma_9 = -I_6(\B),
\end{equation}
где
\begin{equation}\label{eq:I_n}
    I_n[Z(\zeta)] = \frac{8}{15\sqrt\pi} \int_0^\infty \zeta^n Z(\zeta) \exp(-\zeta^2) \dd\zeta,
\end{equation}
а функции \(\B(\zeta), \Q_2(\zeta), \Q_3(\zeta), \QQ_{22}(\zeta), \QQ_3(\zeta)\) могут быть вычислены
с помощью следующих интегральных уравнений:
\begin{gather}
    \mathcal{L}(\zeta_x\zeta_y\B) = \zeta_x\zeta_y\mathcal{B}, \label{eq:B4_par}\\
    \mathcal{L}[\zeta_x\zeta_y(3\zeta_z^2-\zeta_x^2)\Q_3] =
        -\zeta_x\zeta_y(3\zeta_z^2-\zeta_x^2)\left(2\mathcal{B}-\frac1\zeta\der[\mathcal{B}]{\zeta}\right), \label{eq:Q3_par}\\
    \mathcal{L}[\zeta_x\zeta_y(3\Q_2-\zeta_x^2\Q_3)] =
        -\zeta_x\zeta_y\zeta_z^2\left(2\mathcal{B}-\frac1\zeta\der[\mathcal{B}]{\zeta}\right), \label{eq:Q2_par}\\
    \mathcal{L}\left[ \zeta_x\zeta_y\left( 3\zeta_z^2 - \zeta_x^2 \right)\QQ_3 \right]
        = \mathcal{J}(\zeta_x\zeta_y\mathcal{B}, \zeta_z^2\mathcal{B})
        + 2\mathcal{J}(\zeta_x\zeta_y\mathcal{B}, \zeta_x\zeta_z\mathcal{B})
        - \mathcal{J}(\zeta_x\zeta_y\mathcal{B}, \zeta_x^2\mathcal{B}) \label{eq:QQ3_par}\\
    \mathcal{L}[\zeta_x\zeta_y(\QQ_{22} + \zeta_z^2\QQ_3)] =
        \mathcal{J}(\zeta_x\zeta_z\mathcal{B}, \zeta_y\zeta_z\mathcal{B}), \label{eq:QQ22_par}
\end{gather}
где
\begin{equation}\label{eq:mathcalJ}
    E\mathcal{J}(\phi, \psi) = J(E\phi, E\psi).
\end{equation}

