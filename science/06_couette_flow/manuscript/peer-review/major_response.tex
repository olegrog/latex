%&pdflatex
\documentclass{article}
\usepackage{a4wide}

\usepackage{amssymb, amsmath}
\usepackage{framed}                % for leftbar
\usepackage[%
    font={bfseries},
    leftmargin=0.1\textwidth,
    indentfirst=false
]{quoting}

\title{Response to the Reviewers' Comments}
\author{Oleg Rogozin%
    \thanks{Electronic address: \texttt{oleg.rogozin@phystech.edu}}
}

%%% abbreviations for the manuscript
\newcommand{\Kn}{\mathrm{Kn}}
\newcommand{\NS}{N\!S}
\newcommand{\dd}{\mathrm{d}}
\newcommand{\der}[2][]{\frac{\dd#1}{\dd#2}}
\newcommand{\pder}[2][]{\frac{\partial#1}{\partial#2}}
\newcommand{\pderdual}[2][]{\frac{\partial^2#1}{\partial#2^2}}
\newcommand{\pderder}[3][]{\frac{\partial^2#1}{\partial#2\partial#3}}
\newcommand{\Pder}[2][]{\partial#1/\partial#2}
\newcommand{\dzeta}{\boldsymbol{\dd\zeta}}
\newcommand{\dxi}{\boldsymbol{\dd\xi}}
\newcommand{\bzeta}{\boldsymbol{\zeta}}
\newcommand{\bxi}{\boldsymbol{\xi}}
\newcommand{\bh}{\boldsymbol{h}}
\newcommand{\be}{\boldsymbol{e}}
\newcommand{\Nu}{\mathcal{N}}
\newcommand{\Mu}{\mathcal{M}}
\newcommand{\OO}[1]{O(#1)}
\newcommand{\Set}[2]{\{\,{#1}:{#2}\,\}}

%%% abbreviations for the appendix
\newcommand{\B}{\ensuremath{\mathcal{B}^{(4)}}}
\newcommand{\Q}{\ensuremath{\mathcal{Q}^{(0)}}}
\newcommand{\T}[1]{\ensuremath{\mathcal{T}^{(#1)}}}
\newcommand{\TT}{\ensuremath{\tilde{\mathcal{T}}^{(0)}}}
\newcommand{\QQ}{\ensuremath{\tilde{\mathcal{Q}}^{(0)}}}
\newcommand{\IF}[2][0]{\ensuremath{I{#2}^{(#1)}}}
\newcommand{\IFF}[1]{\ensuremath{I\tilde{#1}^{(0)}}}
\newcommand{\ZD}[2]{\zeta_{#1}\delta_{#2}}
\newcommand{\ZZD}[3]{\zeta_{#1}\zeta_{#2}\delta_{#3}}
\newcommand{\ZZZ}{\zeta_i\zeta_j\zeta_k}
\newcommand{\ZZZZ}{\zeta_i\zeta_j\zeta_k\zeta_l}
\newcommand{\DD}[2]{\delta_{#1}\delta_{#2}}


\usepackage{lipsum}
\usepackage{graphicx}
\usepackage{natbib}

\def\asterism{\par\vspace{1em}{\centering\scalebox{1}{\bfseries *~*~*}\par}\vspace{.5em}\par}

\usepackage[
    pdfauthor={Oleg Rogozin},
    pdftitle={Response to the reviewers' comments},
    pdftex,
    unicode
]{hyperref}

\begin{document}

\maketitle

First of all, I would like to express my gratitude to the reviewers
for the extremely valuable and detailed comments.
Following up on the given suggestions, I have revised the whole
manuscript. I have corrected the mentioned mistakes and ambiguous expressions.
Furthermore, I have improved some of the presented results.
Below I clarify all points raised by the reviewers.
The reviewers' words are shown in bold.
Individual parts of the corrected version of the manuscript
can be recognized by the bold lateral line.

\section{Referee Report \#1}

\begin{quoting}
1. On lines 14-16, it says \emph{"The gain term of the Boltzmann integral is interpolated in such a way
that the discrete collision operator from the Maxwell distribution is strictly equal to zero"},
which is not correct.
Correct is to say that \emph{contributions from the inverse collisions are interpolated in such a way that...}
\end{quoting}

It is true that the gain term is not interpolated entirely, but the interpolation of the individual contributions
is constructed in a special way. Indeed, the projection method without interpolation (\emph{direct collisions method})
\begin{equation}
    f_{\alpha'} f_{\beta'} = - f_\alpha f_\beta
\end{equation}
approximates the gain term through the symmetrized form of the collision integral.
Therefore, I agree with the reviewer's wording.

\begin{quoting}
2. On lines 33-34 discovered a grammatical error.
\end{quoting}

Corrected.

\begin{quoting}
3. In the Lemma, the author made the assumption about the statistical distribution of the "rejected" contributions.
Next, in lines 215--216, the author notes that the number of excluded contributions depends on the form of the distribution function.
So it is not strictly statistically distributed over the space of velocities.
Therefore, the assumption of the Lemma is quite reasonable to assess the impact of rejected contributions, but not strict.
This should be noted in the text.
\end{quoting}

Indeed, the Lemma does not provide a rigorous evaluation of the impact of rejected contributions
on the collision integral. In fact, a number of excluded cubature points is not indicative,
because the most of these points correspond to the tails of the velocity distribution,
which have a very small contribution to the collision integral.
Instead of this, an accuracy of the numerical integration can be estimated
by controlling of the relative contribution of the excluded points.

I have removed the Lemma and Fig.~1,~2. Instead, I have added the following remarks.

\begin{leftbar}
To prevent a significant error in this method of numerical integration,
it is necessary to control the contribution of excluded points in the collision integral.
For example, \(N\) may be chosen such that the following sum is small enough:
\begin{equation}\label{eq:excluded_contribution}
    \epsilon_J = \frac{\pi V_\Gamma^2}{\rho\sum_{\nu\in\Nu} w_{\alpha_\nu}w_{\beta_\nu}}
        \sum_{\nu\in\Mu} \left(
            \hat{f}_{\lambda_\nu}\hat{f}_{\mu_\nu} - \hat{f}_{\alpha_\nu}\hat{f}_{\beta_\nu}
        \right)B_\nu.
\end{equation}

\asterism

The number of cubature points \(N\) is chosen to provide \(\epsilon_J<10^{-4}\)
in each physical cell and never exceed \(5\times10^5\).
The maximum value of \(\epsilon_J\) is observed in the marginal physical cell,
where \(f\) varies stronger near \(\zeta_y = 0\), and practically does not depend on \(\Kn\) for \(Kn<1\).
For large \(\Kn\), it is smaller due to the smaller size of cells in terms of the mean free path.
\end{leftbar}

%%%%%%%%%%%%%%%%%%%
\section{Referee Report \#2}
%%%%%%%%%%%%%%%%%%%

\begin{quoting}
1. Concerning the literature, I should point out that the asymptotic solution of the nonlinear Couette flow for small Knudsen numbers
was obtained on the basis of the BGK model by Sone and Yamamoto long time ago up to the second order in Knudsen number:
Section 4 in Y. Sone and K. Yamamoto, Flow of rarefied gas over plane wall, J. Phys. Soc. Jpn. 29, 495--508 (1970)
[Some data are improved in Y. Sone and Y. Onishi, Flow of rarefied gas over plane wall, J. Phys. Soc. Jpn. 47, 672 (1979)].
\end{quoting}

I have added the proposed citation at the end of introduction,
but I intentionally do not compare the nonlinear asymptotic BGK solution with the hard-sphere one
to avoid an overload of information in the figures.

\begin{quoting}
2. The sentence \emph{"To obtain a rigorous asymptotic solution ... the viscous boundary and Knudsen layers."}
(lines 105--107) is slightly confusing.
It is true in general, but in the plane Couette flow, the viscous boundary layer never appears.
In other words, the viscous boundary layer spreads all over the channel.
In fact, the viscous boundary layer does not appear in the analysis in Sec. 4.3.
\end{quoting}

To clear up this ambiguity, I have added the following remark.

\begin{leftbar}
Owing to the degeneracy of the convection effect in the plane Couette flow,
the viscous boundary layer spreads all over the channel;
therefore, the solution is presented as a sum \(f = f_H + f_K\),
where the subscripts \(H\) and \(K\) denote the Hilbert fluid-dynamic and Knudsen-layer parts, respectively.
\end{leftbar}

\begin{quoting}
3. The coefficient \(\gamma_2\) occurring in Eqs. (41) and (43) is not defined.
\end{quoting}

Corrected.

\begin{leftbar}
The thermal conductivity coefficient
\begin{equation}\label{eq:gamma_2}
    \gamma_2 = \frac{16}{15\sqrt{\pi}}\int_0^\infty \zeta^6 \mathcal{A}(\zeta)E\dd\zeta.
\end{equation}
For a hard-sphere gas, \(\gamma_2 = 1.922284066\).
\end{leftbar}

\begin{quoting}
4. The logic deriving Eq. (51) is not convincing. In fact, in the Hilbert expansion, \(v_{x0}\) and \(T_0\)
should not depend on the expansion parameter \(k\).
Nevertheless, the boundary condition (51) for \(v_{x0}\) and \(T_0\) contain \(k\).
This appears to be contradicting. I would prefer the following logic.
Let us put \(v_x = v_{x0} + v_{x1}k + \OO{k^2}\), \(T = T_0 + T_1k + \OO{k^2}\).
If we add Eq.~(40) and Eq.~(42) multiplied by \(k\) and add Eq.~(41) and Eq.~(43) multiplied by \(k\),
we have, respectively,
\begin{gather}
    \pder{y}\left(\sqrt{T}\pder[v_x]{y}\right) + \OO{k^2} = 0, \\
    \sqrt{T}\left(\pder[v_x]{y}\right)^2 + \frac{5\gamma_2}{4\gamma_1}\pder{y}\left(\sqrt{T}\pder[T]{y}\right) + \OO{k^2} = 0.
\end{gather}
If we neglect \(\OO{k^2}\) terms, we have exactly the same equations as Eqs.~(40) and~(41),
that is, the Navier–Stokes equations. The same procedure for the boundary conditions~(47) and~(48) leads to
\begin{equation}
    v_x = v_{Bx} + k_0\frac{T}{p}\pder[v_x]{y}k + \OO{k^2}, \quad T = T_B - d_1\frac{T}{p}\pder[T]{y}k + \OO{k^2},
\end{equation}
which are the same as Eq.~(51) if we neglect \(\OO{k^2}\) terms.
Therefore, Eqs.~(40) and~(41) with boundary condition~(51) essentially provide the asymptotic solution
that is correct up to \(\OO{k}\).
\end{quoting}

I agree with the referee, so I have rewritten subsection 4.3 according to the proposed logic.

\begin{quoting}
5. Section 4.5 could be improved. Some symbols are not defined clearly.
For instance, the set \(\Gamma\) (line 145) should be defined clearly.
The definition of \(\delta_{\alpha\gamma}\) in Eq.~(70) is not given.
It looks like the Kronecker delta.
But, if so, \(\delta_{\alpha'_\nu\gamma}\) does not make sense because \(\alpha'_\nu\notin\Gamma\).
The meaning of the subscript \(\nu\) in \(\alpha_\nu\) and \(\beta_\nu\) should be explained,
and the space \(\Nu\) should be defined clearly, just after Eq.~(70).
\end{quoting}

Indeed, subsection 4.5 turns out to be very raw, so I have rewritten it as a whole,
taking into account all referee's remarks. Please, see section 5.

\begin{quoting}
6. In regards to the values of the slip coefficients \(k_0 = −1.25395\) (line~78) and \(d_1 = 2.40014\) (line~110),
the recent paper~[20] is quoted. It is true that these accurate values are obtained in [20].
However, they were computed with sufficient accuracy more than 25 years ago in [15] and
in Y. Sone, T. Ohwada, and K. Aoki, Temperature jump and Knudsen layer in a rarefied gas over a plane wall:
Numerical analysis of the linearized Boltzmann equation for hard-sphere molecules, Phys. Fluids A 1, 363--370 (1989).
Quoting only [20] may give the reader a wrong idea that \(k_0\) and \(d_1\) were obtained only recently.
In connection with line~122, \(\Theta_1(\eta)\) and \(\Omega_1(\eta)\) were also obtained first time in the reference given above.
\end{quoting}

To avoid this ambiguity, I have rewritten the specified citations as follows.
\begin{leftbar}
For hard-sphere molecules, \(\gamma_1 = 1.270042427\), \(k_0 = -1.25395\),
and \(Y_0(\eta)\), \(H_A(\eta)\) are tabulated in~\cite{Ohwada1989creep, Sone2002, Sone2007, Takata2015}.
\asterism
..., where \(d_1 = 2.40014\) is another slip coefficient.
\asterism
For hard-sphere molecules, \(\Theta_1(\eta)\) and \(\Omega_1(\eta)\)
are tabulated in~\cite{Ohwada1989jump, Sone2002, Sone2007, Takata2015}.
\end{leftbar}

\begin{quoting}
7. There are some typos:
\begin{itemize}
    \item On the right-hand side of Eq.~(8), \(-\zeta_x\) should be \(\zeta_x\).
    \item What are \(\mathcal{T}_0^+\) and \(\mathcal{T}_0^-\) in Eq.~(28)?
    \item \emph{"convenient"} should be \emph{"convenience"} in line 147.
    \item \(\bzeta_\alpha-\boldsymbol{v}\) and \(\bzeta_\gamma-\boldsymbol{v}\) in Eq.~(71)
        should be \(|\bzeta_\alpha-\boldsymbol{v}|^2\) and \(|\bzeta_\gamma-\boldsymbol{v}|^2\).
    \item In Figs. 4–6, \emph{"Ohwada et al."} should be \emph{"Sone et al."}, since it refers to [2].
\end{itemize}
\end{quoting}

The first remark is wrong. The plane Couette-flow problem is antisymmetric.
I have corrected the paragraph as follows.
\begin{leftbar}
The following antisymmetry condition is imposed at \(y=0\):
\begin{equation}\label{eq:specular_bc}
    f(\zeta_x,\zeta_y,\zeta_z) = f(-\zeta_x,-\zeta_y,\zeta_z), \quad \zeta_y>0.
\end{equation}
\end{leftbar}
The antisymmetry property is also seen from Fig.~1,
where the contour plots of the velocity distribution function are depicted.

The second remark is corrected as follows:
\begin{leftbar}
\begin{equation}\label{eq:bkw_g_equation}
    \sqrt{\pi} g(y) = \mathcal{T}_0 \left(\frac{1-2y}{2k}\right) - \mathcal{T}_0 \left(\frac{1+2y}{2k}\right)
        + \frac1k \int_0^{\frac12} \left[ \mathcal{T}_{-1}\left(\frac{|y-s|}{k}\right)
        - \mathcal{T}_{-1}\left(\frac{y+s}{k}\right) \right] g(s) \dd{s}.
\end{equation}
\end{leftbar}

The other remarks are corrected according to the reviewer's suggestions.

\section{Additional improvements}

\begin{enumerate}
    \item Transport coefficients \(\gamma_8\), \(\gamma_9\), \(\gamma_{10}\) have been recomputed more accurately.
    For this purpose, three-dimensional integral equations for hard-sphere molecules
    have been reduced to the one-dimensional ones. Please, see Appendix.
    \item Some estimates on \(|\Nu|\) for preserving positivity are represented in subsection 5.4.
    \item A detailed description of three projection stencils has been added in subsection 5.5.
\end{enumerate}

\bibliographystyle{plain}
\bibliography{../manuscript}

\end{document}


