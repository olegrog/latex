%&pdflatex
\documentclass[10pt]{article}
\usepackage[utf8]{inputenc}
\usepackage[T2A,T1]{fontenc}
\usepackage[english,russian]{babel}
\usepackage{csquotes}

\usepackage{amssymb, amsmath}
\usepackage{fullpage}
\usepackage{indentfirst}
\usepackage[font=small,labelfont=bf,labelsep=period]{caption}
\usepackage{graphicx} % [demo]
\usepackage{wrapfig}
\usepackage{subcaption}
%\usepackage{paralist}

\usepackage{tikz}
\usepackage{pgfplots}
%\usetikzlibrary{arrows,fit,positioning,shapes.multipart}
%\usetikzlibrary{shapes.geometric}

\usepackage[
    pdfauthor={Oleg Rogozin},
    pdftitle={Temperature ghost effect between two nonuniform heated parallel plates},
    colorlinks,pdftex, unicode]{hyperref}

\newcommand{\Kn}{\mathrm{Kn}}
\newcommand{\Ma}{\mathrm{M}}
\newcommand{\dd}{\:\mathrm{d}}
\newcommand{\pder}[2][]{\frac{\partial#1}{\partial#2}}
\newcommand{\pderder}[2][]{\frac{\partial^2 #1}{\partial #2^2}}
\newcommand{\Pder}[2][]{\partial#1/\partial#2}
\newcommand{\dzeta}{\boldsymbol{\dd\zeta}}
\newcommand{\bzeta}{\boldsymbol{\zeta}}
\newcommand{\Nu}{\mathcal{N}}
\newcommand{\OO}[1]{O(#1)}
\newcommand{\Set}[2]{\{\,{#1}:{#2}\,\}}

\usepackage[
    backend=biber,
    hyperref=true,
    autolang=other,
    maxnames=100,                   % print all authors
    style=gost-numeric,
    movenames=false,                % only for biblatex-gost
    sorting=none,
    url=false]{biblatex}
\bibliography{snit_flows}

\title{О медленных неизотермических течениях газа}
\author{Рогозин Олег}
\date{}

\begin{document}

\maketitle
\begin{abstract}
\begin{flushright}
\vspace{1em}
{\it Памяти Оскара Гаврииловича Фридлендера (1939--2015)}
\vspace{1em}
\end{flushright}
Рассматриваются течения слабо разреженного газа, вызванные значительными градиентами температур,
при условии, что число Маха того же порядка, что и число Кнудсена.
Предлагается метод постановки граничных условий,
позволяющий на основе уравнений гидродинамического типа
получить приближённое решение для малых чисел Кнудсена.
Результаты подкреплены численным анализом уравнения Больцмана
как на равномерной скоростной сетке, так и неравномерной.
\end{abstract}

\tableofcontents

%%%%%%%%%%%%%%%%%%%%%%%%%%%%%%%%%%%%%%%%%%%
\section{Введение}
%%%%%%%%%%%%%%%%%%%%%%%%%%%%%%%%%%%%%%%%%%%

С 1969 по 1974 год Оск\'{а}р Гавриилович Фридлендер (1939--2015)
совместно с Владленом Сергеевичем Галкиным (род.~1932)
под руководством Михаила Наумовича Когана (1925--2011)
развили теорию \emph{медленных неизотермических}
течений газа~\cite{Kogan1970, Kogan1971, Kogan1972, Friedlander1974, Kogan1976}.
Медленность течений следует понимать как малость числа Маха (\(\Ma\ll1\)),
а неизотермичность как присутствие в газе существенного градиента температур.
Основным импульсом упомянутых работ стала попытка учесть влияние температурных напряжений
на конвекцию газа через анализ барнеттовского приближения для слаборазреженного газа.
Несмотря на то что барнеттовские члены второго порядка малости по числу Кнудсена \(\Kn\),
при \(\Ma\sim\Kn\) они становятся сравнимыми с ньютоновскими вязкостными напряжениями.
Таким образом, уравнения Навье"--~Стокса оказываются некорректными для медленных
неизотермических течений.

В~\cite{Kogan1970} была получена система уравнений гидродинамического типа,
адекватно описывающая медленные неизотермические течения слаборазреженного газа.
В дальнейшем, будем использовать устоявшийся термин: \emph{уравнения SNIF} (slow nonisothermal).
Нелинейный характер температурных напряжений в полученных уравнениях приводит к
явлению \emph{термострессовой конвекции} газа~\cite{Kogan1971}.
Кроме того, оказалось, что эти напряжения способны приводить в движение в том числе однородно нагретые тела~\cite{Friedlander1974},
хотя ранее считалось, что для возникновения силы, действующей со стороны газа на тело,
необходимо наличие градиента температур на его поверхности (явление \emph{теплового скольжения} газа).
Однако разработанная теория медленных неизотермических течений долгое время не только не применялась
в прикладных задачах, но даже не была подтверждена экспериментально.
Лишь на рубеже веков были поставлены соответствующие эксперименты~\cite{Friedlander1997, Friedlander2003}.
В это же время, благодаря развитию вычислительной техники, появилась возможность провести численные расчёты
некоторых прикладных задач с помощью уравнений SNIF, а также подтвердить асимптотику на основе модельного уравнения
Больцмана"--~Крука"--~Веландера (БКВ)~\cite{Alexandrov2002, Aoki2006, Alexandrov2008b, Alexandrov2011}
и прямого статистического моделирования (ПСМ, DSMC)~\cite{Alexandrov2008a, Aoki2007}.
Рассматривались также задачи с граничными условиями газ"--~жидкость,
учитывающие процессы испарения и конденсации~\cite{Aoki2007}.

Как известно, разложение Чепмена"--~Энскога позволяет получать \emph{приближённые} решения кинетической теории
для слаборазреженного газа. Для того чтобы получить \emph{асимптотическое} решение для малых \(\Kn\),
необходимо использовать разложение Гильберта, приводящее, однако, к более громоздким выражениям~\cite{Sone2002}.
Тем не менее, оба метода приводят к одинаковым системам уравнений ведущего порядка~\cite{Sone1996}.
В результате строгого асимптотического анализа было показано, что в континуальном пределе (\(\Kn\to0\))
скоростное поле первого порядка конечным образом влияет на температурное распределение нулевого порядка.
Подобные \emph{призрак-эффекты} не прояляются в уравнениях Навье"--~Стокса с граничным условием без скольжения,
но впоследствии были обнаружены при асимптотическом анализе ряда других задач~\cite{Sone2007}.

В приведённых выше работах температурное поле нулевого порядка находилось из уравнений SNIF,
однако попытки получить приближение первого порядка малости от \(\Kn\) в литературе не встречаются.
Кроме того, в области малых \(\Kn\) нет достаточно точных решений уравнения Больцмана.
Распространённый метод ПСМ, из-за присущих ему значительных флуктуаций, плохо пригоден
для медленных течений, особенно при малых \(\Kn\). В неизотермических течениях ситуация
ещё более усложняется, поскольку, в силу призрак-эффекта, ошибка \(\varepsilon\)
при вычислении поля скоростей приводит к ошибке порядка \(\varepsilon/\Kn\) при вычислении поля температур.

%%%%%%%%%%%%%%%%%%%%%%%%%%%%%%%%%%%%%%%%%%%
\section{Основные уравнения}
%%%%%%%%%%%%%%%%%%%%%%%%%%%%%%%%%%%%%%%%%%%

Прежде всего, перейдём к безразмерным переменным.
Пусть \(L\) "--- характерная длина в рассматриваемой задаче,
а \(T^{(0)}\) и \(p^{(0)}\) "--- референсные температура и давление газа.
Тогда макроскопические переменные принимают следующий вид:
температура \(TT^{(0)}\), давление \(pp^{(0)}\),
плотность \(\rho p^{(0)}/RT^{(0)}\), скорость \(v_i(2RT^{(0)})^{1/2}\).
Функция распределения молекулярных скоростей \(f(x_i,\zeta_i)(2p^{(0)})/(2RT^{(0)})^{5/2}\)
определяется в физическом пространстве \(x_iL\) и скоростном \(\zeta_i(2RT^{(0)})^{1/2}\).
Удельная газовая постоянная \(R = k_B/m\), где \(k_B\) "--- постоянная Больцмана,
\(m\) "--- масса отдельной молекулы.
Число Кнудсена \(\Kn = \ell^{(0)}/L\) определяется через референсную длину свободного пробега
\begin{equation}\label{eq:ell}
    \ell^{(0)} = \frac{mRT^{(0)}}{\sqrt2\pi d_m^2 p^{(0)}},
\end{equation}
где радиус действия межмолекулярного потенциала взаимодействия \(d_m\)
совпадает с диаметром молекул для модели твёрдых сфер.

Стационарное уравнение Больцмана в присутствии внешней силы \(F_i (2RT^{(0)})/L\) в безразмерных переменных имеет вид:
\begin{equation}\label{eq:Boltzmann}
    \zeta_i\pder[f]{x_i} + F_i\pder[f]{\zeta_i} = \frac1k J(f,f),
\end{equation}
где интеграл столкновений
\begin{equation}\label{eq:integral}
    J(f,g) = \frac12 \int(f'g'_*+g'f'_*-fg_*-gf_*)B\dd\Omega(\boldsymbol\alpha) \dzeta_*
\end{equation}
и \(k = \sqrt\pi\Kn/2\).
\(\Omega(\boldsymbol{\alpha})\) "--- телесный угол в направлении единичного вектора \(\boldsymbol\alpha\),
\(B\) "--- функционал межмолекулярного потенциала. Для модели твёрдых сфер,
\begin{equation}\label{eq:ci_kernel}
    B = \frac{|\alpha_i(\zeta_{i*}-\zeta_i)|}{4\sqrt{2\pi}}.
\end{equation}
Макроскопические переменные выражаются через моменты функции распределения:
\begin{equation}\label{eq:macro}
    \rho = \int f \dzeta, \quad
    v_i = \frac1{\rho} \int \zeta_i f \dzeta, \quad
    T = \frac{2}{3\rho}\int(\zeta_i-v_i)^2 f \dzeta, \quad
    p = \rho T.
\end{equation}

Граничные условия диффузного отражения задаются следующим образом:
\begin{equation}\label{eq:diffuse_bc}
    f\left(\zeta_i n_i > 0\right) =
        \frac{\sigma_B}{(\pi T_B)^{3/2}} \exp\left(-\frac{\zeta_i^2}{T_B}\right), \quad
    \sigma_B = -2\left(\frac{\pi}{T_B}\right)^{1/2} \int_{\zeta_i n_i < 0} \zeta_j n_j f\dzeta,
\end{equation}
где \(n_i\) "--- единичный вектор нормали к границе, направленный внутрь газа,
а \(T_B\) и \(v_{Bi}\) "--- граничные температура и скорость.
Для стационарных задач предполагается, что \(v_{Bi}n_i = 0\).

%%%%%%%%%%%%%%%%%%%%%%%%%%%%%%%%%%%%%%%%%%%
\section{Асимптотический анализ}
%%%%%%%%%%%%%%%%%%%%%%%%%%%%%%%%%%%%%%%%%%%

В этом разделе рассматривается асимптотическая теория,
подробное изложение которой можно найти
в~\cite{Sone1996, Sone2002, Sone2007}.
Для простоты рассматривается единственный потенциал межмолекулярного взаимодействия "---
модель твёрдых сфер.

\subsection{Система уравнений гидродинамического типа}

Функция распределения \(f(x_i,\zeta_i)\) и макроскопические переменные \(h = \rho, v_i, T, \dots\)
могут быть разложены в ряд по \(k\):
\begin{equation}\label{eq:expansion}
    f = f_0 + f_1k + f_2k^2 + \cdots, \quad h = h_0 + h_1k + h_2k^2 + \cdots.
\end{equation}
Если предположить, что \(\Pder[f]{x_i} = \OO{f}\) (разложение Гильберта),
\(v_{i0} = 0\) (медленные течения), \(F_{i0} = F_{i1} = 0\) (слабое поле внешних сил)
и подставить~\eqref{eq:expansion} в уравнение Больцмана~\eqref{eq:Boltzmann},
то для получаемых интегро-дифференциальных уравнений должны выполняться
условия разрешимости в нулевом порядке:
\begin{equation}
    \pder[p_0]{x_i} = 0, \label{eq:asymptotic0_p}
\end{equation}
в первом:
\begin{gather}
    \pder{x_i}\left(\frac{u_{i1}}{T_0}\right) = 0, \label{eq:asymptotic1_u} \\
    \pder[p_1]{x_i} = 0, \label{eq:asymptotic1_p} \\
    \frac{u_{i1}}{T_0}\pder[T_0]{x_i}
        = \frac{\gamma_2}2\pder{x_i}\left(\sqrt{T_0}\pder[T_0]{x_i}\right), \label{eq:asymptotic1_T} \\
\end{gather}
и во втором:
\begin{gather}
    \pder{x_i}\left(\frac{u_{i2}}{T_0}\right)
        + \pder{x_i}\left[\left(\frac{p_1}{p_0}-\frac{T_1}{T_0}\right)\frac{u_{i1}}{T_0}\right] = 0, \label{eq:asymptotic2_u} \\
    \begin{aligned}
    \pder{x_j}\left(\frac{u_{i1}u_{j1}}{T_0}\right)
        &-\frac{\gamma_1}2\pder{x_j}\left[\sqrt{T_0}\left(
            \pder[u_{i1}]{x_j} + \pder[u_{j1}]{x_i} - \frac23\pder[u_{k1}]{x_k}\delta_{ij}
        \right)\right] \\
        &- \frac{\gamma_7}{T_0}\pder[T_0]{x_i}\pder[T_0]{x_j}\left(\frac{u_{j1}}{\gamma_2\sqrt{T_0}} - \frac{1}4\pder[T_0]{x_j}\right)
        = -\frac12\pder[p_2^\dag]{x_i} + \frac{p_0^2 F_{i2}}{T_0},
    \end{aligned} \label{eq:asymptotic2_p} \\
    \frac{u_{i1}}{T_0}\pder[T_1]{x_i}
        + \left[\frac{u_{i2}}{T_0} + \left(\frac{p_1}{p_0}-\frac{T_1}{T_0}\right)\frac{u_{i1}}{T_0}\right]\pder[T_0]{x_i}
        = \frac{\gamma_2}2\pder{x_i}\left(\sqrt{T_0}\pder[T_1]{x_i} + \frac{T_1}{2\sqrt{T_0}}\pder[T_0]{x_i}\right). \label{eq:asymptotic2_T}
\end{gather}
Здесь обозначены \(u_{i1} = p_0v_{i1}\), \(u_{i2} = p_0v_{i2}\) и
\begin{equation}\label{eq:dag_pressure}
    p_2^\dag = p_0 p_2
        + \frac{2\gamma_3}{3}\pder{x_k}\left(T_0\pder[T_0]{x_k}\right)
        - \frac{\gamma_7}{6}\left(\pder[T_0]{x_k}\right)^2.
\end{equation}
Уравнения~\eqref{eq:asymptotic1_u},~\eqref{eq:asymptotic1_T},~\eqref{eq:asymptotic2_p}
для \(T_0\), \(u_{i1}\) и \(p_2^\dag\), которые будем называть уравнениями SNIF,
содержат единственный член температурных напряжений, отсутствующий в уравнениях Навье"--~Стокса.
Сравнивая его с \(p_0^2F_{i2}/T_0\), можно увидеть, что на единицу массы действует сила
\begin{equation}\label{eq:gamma7_force}
    k^2\frac{\gamma_7}{p_0^2}\pder[T_0]{x_i}\pder[T_0]{x_j}\left(\frac{u_{j1}}{\gamma_2\sqrt{T_0}} - \frac{1}4\pder[T_0]{x_j}\right).
\end{equation}
Для модели твёрдых сфер, эта сила противоположна градиенту температур.
Важно отметить, что \(p_2^\dag\) не входит в уравнение состояния,
поэтому определяется с точностью до константы.
Поскольку член \(\partial{p_2^\dag}/\partial{x_i}\) включён в систему,
как давление в уравнениях Навье"--~Стокса для несжимаемого газа,
то для решения уравнений SNIF применяются соответствующие численные методы~\cite{Aoki2007}.

Безразмерные транспортные коэффициенты для модели твёрдых сфер равны
\begin{alignat*}{2}\label{eq:gamma_coeffs}
    \gamma_1 &= 1.270042427, &\quad \gamma_2 &= 1.922284066, \\
    \gamma_3 &= 1.947906335, &\quad \gamma_7 &= 1.758705.
\end{alignat*}
Первые два коэффициента соответствуют вязкости \(\mu\) и теплопроводности \(\lambda\) газа, т.е.
\begin{equation}\label{eq:mu_lambda}
    \mu = \gamma_1\sqrt{T_0} \frac{p^{(0)}L}{\sqrt{2RT^{(0)}}} k, \quad
    \lambda = \frac{5\gamma_2}{2}\sqrt{T_0} \frac{p^{(0)}RL}{\sqrt{2RT^{(0)}}} k.
\end{equation}
Коэффициент \(\gamma_3\) входит в значения температурных напряжений,
которые создают неоднородное распределение давления в газе,
но не движущую силу. Коэффициент \(\gamma_7\) соответствует
температурной конвекции газа.

\subsection{Граничные условия}

Граничные условия диффузного отражения имеют вид:
\begin{gather}
    T_0 = T_{B0}, \label{eq:bc_T0} \\
    T_1 = T_{B1} + d_1\frac{T_{B0}}{p_0}\pder[T_0]{x_j} n_j, \label{eq:bc_T1} \\
    \left\{
    \begin{aligned}
        & u_{j1} (\delta_{ij}-n_in_j) =
            \left(u_{Bj1} - K_1 \sqrt{T_{B0}} \pder[T_0]{x_j}\right) (\delta_{ij}-n_in_j), \\
        & u_{j1} n_j = 0.
    \end{aligned}
    \right. \label{eq:bc_u1}
\end{gather}
Здесь \(d_1\) "--- коэффициент температурного скачка, а \(K_1\) "--- коэффициент теплового скольжения.
Для модели твёрдых сфер,
\begin{equation}\label{eq:slip_coefficients}
    d_1 = 2.4001, \quad K_1 = -0.6463.
\end{equation}

Решение уравнений SNIF не удовлетворяет в общем случае кинетическому граничному условию~\eqref{eq:diffuse_bc}
поскольку функция распределения изменяется значительно вблизи границы вдоль нормали к ней.
Коррекция макроскопических переменных \(h = h_0 + (h_1 + h_{K1})k + \cdots\) в кнудсеновском слое
принимает следующий вид:
\begin{gather}
    T_{K1} = \frac{T_{B0}}{p_0}\left.\pder[T_0]{x_j}\right|_0 n_j
        \Theta_1\left(\frac{p_0}{T_0}\eta\right), \label{eq:correction_T} \\
    \rho_{K1} = \frac1{T_{B0}}\left.\pder[T_0]{x_j}\right|_0 n_j
        \Omega_1\left(\frac{p_0}{T_0}\eta\right), \label{eq:correction_rho} \\
    \left\{
    \begin{aligned}
        & u_{jK1} (\delta_{ij}-n_in_j) =
            -\frac{\sqrt{T_{B0}}}2 \left.\pder[T_{B0}]{x_j}\right|_0
            Y_1\left(\frac{p_0}{T_0}\eta\right) (\delta_{ij}-n_in_j), \\
        & u_{jK1} n_j = 0,
    \end{aligned}
    \right. \label{eq:correction_u}
\end{gather}
где \(\eta = (x_i-x_{Bi})n_i/k\), \(x_{Bi}\) "--- ближайшая к \(x_i\) точка на границе,
а \(|_0\) означает значение на границе (\(\eta=0\)).
Функции \(\Theta_1(\eta)\), \(\Omega_1(\eta)\), \(Y_1(\eta)\) убывают экспоненциально от \(\eta\)
и табулированы для модели твёрдых сфер в~\cite{Sone2002, Sone2007}.

Разложение в ряд граничной температуры \(T_B = T_{B0} + T_{B1}k + \cdots\) может быть выполнено
произвольным образом. Обычно полагают, что \(T_{B1}=0\)~\cite{Kogan1976, Sone1996},
что позволяет получить температурное поле с точностью \(\OO{k}\).
Если же положить, что \(T_1=0\) на границе,
тогда из~\eqref{eq:bc_T0} и~\eqref{eq:bc_T1} следует,
что \(T_0\) вычисляется из уравнения
\begin{equation}\label{eq:boundary_temp}
    T_B = T_0 \left( 1 - \frac{d_1}{p_0}\pder[T_0]{x_j}n_j k \right).
\end{equation}
При таком граничном условии получается приближённое решение,
для которого температура газа на границе вычисляется с точностью \(\OO{k^2}\).
Однако из уравнения~\eqref{eq:asymptotic2_T} видно, что в общем случае во всей области
решение обладает точностью \(\OO{k}\).

\subsection{Континуальный предел}

Рассмотрим поведение системы уравнений SNIF~\eqref{eq:asymptotic1_u},~\eqref{eq:asymptotic1_T},~\eqref{eq:asymptotic2_p}
в континуальном пределе (\(k\to0\)).
Поле скоростей \(v_i\) стремится к нулю, поскольку \(v_{i0}=0\),
однако скорость \(v_{i1}\), которая вносит инфинитезимальный вклад в \(v_i\),
конечным образом влияет на \(T_0\).
Такое асимптотическое поведение получило название \emph{призрак-эффекта} (ghost effect)~\cite{Sone2002, Sone2007}.

Можно выделить три причины возникновения конечного поля \(v_{i1}\).
Во-первых, если граница движется со скоростью \(v_{B1}\ne0\).
Во-вторых, если на границе присутствует градиент температуры \(\Pder[T_0]{x_i}\ne0\) (эффект \emph{теплового скольжения}).
В-третьих, когда изотермические поверхности не параллельны
\begin{equation}\label{eq:nonparallel}
    e_{ijk}\pder[T_0]{x_j}\pder{x_k}\left(\pder[T_0]{x_l}\right)^2 \ne 0,
\end{equation}
возникает конвекция газа под действием температурных напряжений (\emph{термострессовая конвекция}).

В классической гидродинамике, уравнения Навье"--~Стокса (\(\gamma_7=0\))
с неподвижными (\(v_B=0\)) граничными условиями без скольжения (\(K_1=0\), \(d_1=0\))
приводят к нулевому полю \(v_{i1} = 0\) и к уравнению теплопроводности
\begin{equation}\label{eq:heat_equation}
    \pder{x_i}\left(\sqrt{T_0}\pder[T_0]{x_i}\right) = 0.
\end{equation}
Тем не менее, для корректного описания в континуальном пределе температурного поля \(T\equiv T_0\)
необходимо решать уравнения SNIF~\eqref{eq:asymptotic1_u},~\eqref{eq:asymptotic1_T},~\eqref{eq:asymptotic2_p}
с соответствующими граничными условиями~\eqref{eq:bc_T0},~\eqref{eq:bc_u1}.

%%%%%%%%%%%%%%%%%%%%%%%%%%%%%%%%%%%%%%%%%%%
\section{Метод решения уравнения Больцмана}
%%%%%%%%%%%%%%%%%%%%%%%%%%%%%%%%%%%%%%%%%%%

В настоящей работе уравнение Больцмана решается численно с помощью
симметричного расщепления на уравнение переноса
\begin{equation}\label{eq:split_advection}
    \pder[f]{t} + \zeta_i\pder[f]{x_i} = 0,
\end{equation}
для которого используется метод конечных объёмов с явной TVD схемой второго порядка,
и уравнение релаксации
\begin{equation}\label{eq:split_integral}
    \pder[f]{t} = \frac1k J(f,f),
\end{equation}
для которого используется проекционно-интерполяционный метод дискретных скоростей~\cite{Tcheremissine1997, Tcheremissine2006}.
Несмотря на то что проекционный метод обладает вторым порядком аппроксимации в пространстве скоростей~\cite{Anikin2012},
в граничных задачах медленных течений возникают разрывы функции распределения скоростей.
Таким образом, для получения адекватной аппроксимации в слое Кнудсена необходимо прибегать
к умельчению сетки вблизи линии разрыва~\cite{Rogozin2015}.
Далее коротко описывается методика консервативного вычисления интеграла столкновений Больцмана
для произвольной прямоугольной сетки в пространстве скоростей~\cite{Dodulad2015}.

%%% Numerical approximation
Пусть регулярная скоростная сетка построена таким образом,
что кубатура в пространстве \(\bzeta\) выражается в виде взвешенной суммы
\begin{equation}\label{eq:zeta_cubature}
    \int F(\bzeta) \dzeta \approx \sum_{\gamma\in\Gamma} F_\gamma w_\gamma,
        \quad \sum_{\gamma\in\Gamma} w_\gamma = V_\nu,
        \quad F_\gamma = F(\bzeta_\gamma),
\end{equation}
где \(\mathcal{V} = \Set{\zeta_\gamma}{\gamma\in\Gamma}\) "--- множество точек интегрирования,
\(F\) "--- произвольная функция от \(\bzeta\),
а \(V_\nu\) "--- общий объём скоростной сетки.
Тогда интеграл столкновений, записанный в симметризованной форме
\begin{equation}\label{eq:symm_ci}
    \begin{aligned}
    J(f_\gamma, f_\gamma) = \frac14\int &\left[
        \delta(\bzeta'-\bzeta_\gamma) + \delta(\bzeta'_*-\bzeta_\gamma)
        - \delta(\bzeta-\bzeta_\gamma) - \delta(\bzeta_*-\bzeta_\gamma)\right] \\
        &\times(f'f'_* - ff_*)B \dd\Omega(\boldsymbol{\alpha}) \dzeta\dzeta_*,
    \end{aligned}
\end{equation}
где \(\delta(\bzeta)\) "--- функция Дирака в \(\mathbb{R}^3\),
имеет следующий дискретный аналог:
\begin{equation}\label{eq:discrete_symm_ci}
    w_\gamma J_\gamma = \frac{\pi V_\nu^2}{\sum_{\nu\in\Nu} w_{\alpha_\nu}w_{\beta_\nu}}
        \sum_{\nu\in\Nu} \left(
            \delta_{\alpha'_\nu\gamma} + \delta_{\beta'_\nu\gamma}
            - \delta_{\alpha_\nu\gamma} - \delta_{\beta_\nu\gamma}
        \right)\left(
            \frac{w_{\alpha_\nu}w_{\beta_\nu}}{w_{\alpha'_\nu}w_{\beta'_\nu}}
            \hat{f}_{\alpha'_\nu}\hat{f}_{\beta'_\nu} - \hat{f}_{\alpha_\nu}\hat{f}_{\beta_\nu}
        \right)B_\nu,
\end{equation}
Здесь \(\alpha_\nu\in\Gamma\) и \(\beta_\nu\in\Gamma\), наряду с \(\boldsymbol{\alpha}_\nu\),
берутся из решётчатого правила Коробова~\cite{Korobov1959, Sloan1994}
для восьмимерной кубатуры.
Кроме того, кубатурная решетка случайным образом сдвигается на каждом шаге по времени.
Модифицированная функция распределения \(\hat{f}_\gamma = f_\gamma w_\gamma\) используется для удобства.
Дискретный аналог распределения Максвелла может быть записан в следующем виде:
\begin{equation}\label{eq:discrete_Maxwell}
    \hat{f}_{M\gamma} = \rho\left[\sum_{\alpha\in\Gamma}w_\alpha\exp
            \left(-\frac{\bzeta_\alpha - \boldsymbol{v}}{T}\right)
        \right]^{-1}
        w_\gamma\exp\left(-\frac{\bzeta_\gamma - \boldsymbol{v}}{T}\right).
\end{equation}

%%% Projection and interpolation
В общем случае, скорости после столкновения,
\(\bzeta_{\alpha'_\nu}\) и \(\bzeta_{\beta'_\nu}\), не попадают в \(\mathcal{V}\).
Если они просто заменяются ближайшими сеточными скоростями,
\(\bzeta_{\lambda_\nu}\in\mathcal{V}\) и \(\bzeta_{\mu_\nu}\in\mathcal{V}\),
дискретный интеграл столкновений~\eqref{eq:discrete_symm_ci} теряет свойство консервативности.
Более того, дискретное распределение Максвелла~\eqref{eq:discrete_Maxwell} перестаёт быть равновесным состоянием.
Для решения этих проблем в проекционно-интерполяционном методе применяются следующие две процедуры.
Во-первых, \(\bzeta_{\alpha'_\nu}\) проецируется на \(\bzeta_{\lambda_\nu}\) и
на некоторое множество соседних узлов, \(\Set{\bzeta_{\lambda_\nu+s_a}}{a\in\Lambda}\subset\mathcal{V}\),
степенным образом:
\begin{equation}\label{eq:ci_projection}
    \delta_{\alpha'_\nu\gamma} = \sum_{a\in\Lambda} r_{\lambda,a}\delta_{\lambda_\nu+s_a,\gamma}.
\end{equation}
Веса \(r_{\lambda,a}\) выбираются так, чтобы обеспечить сохранение массы, импульса и энергии, т.е.
\begin{equation}\label{eq:impact_conservation}
    \sum_{a\in\Lambda} r_{\lambda,a} = 1, \quad
    \sum_{a\in\Lambda} r_{\lambda,a} \bzeta_{\lambda_\nu+s_a} = \bzeta_{\alpha'_\nu}, \quad
    \sum_{a\in\Lambda} r_{\lambda,a} \bzeta_{\lambda_\nu+s_a}^2 = \bzeta_{\alpha'_\nu}^2.
\end{equation}
Множество проекционных точек \(\mathcal{S}_{\alpha'_\nu} = \Set{\bzeta_{\lambda_\nu+s_a}}{a\in\Lambda, r_{\lambda,a}\neq0}\)
называется проекционным шаблоном.
Для равномерной сетки, достаточно двух проекционных точек для достижения консервативности,
в то время как для нерномерной сетки минимальный шаблон состоит из пяти точек, \(|\mathcal{S}_{\alpha'_\nu}|=5\).
Во-вторых, чтобы удовлетворить условию \(J_\gamma(\hat{f}_{M\gamma}, \hat{f}_{M\gamma}) = 0\),
\(\hat{f}_{\alpha'_\nu}\) интерполируется следующим образом:
\begin{equation}\label{eq:ci_interpolation}
    \frac{\hat{f}_{\alpha'_\nu}}{w_{\alpha'_\nu}} = \prod_{a\in\Lambda}
        \left(\frac{\hat{f}_{\lambda_\nu+s_a}}{w_{\lambda_\nu+s_a}} \right)^{r_{\lambda,a}}.
\end{equation}
Этот тип интерполяции требует высоких затрат с вычислительной точки зрения,
но для медленных течений, когда функция распределения близка к максвелловской,
позволяет добиться высокой точности и низкого уровня флуктуаций.
Для \(\bzeta_{\beta'_\nu}\) и \(\hat{f}_{\beta'_\nu}\) все формулы аналогичны.

Существует ещё одно важное и необходимое свойство численного алгоритма:
функция распределения должна всегда оставаться положительной.
Если в сумме~\eqref{eq:discrete_symm_ci} присутствует член, нарушающий это свойство,
то его необходимо исключить из \(\mathcal{N}\).
С другой стороны, большое количество таких исключенных столкновений
нарушает кубатурную решетку, следовательно, существенно увеличивается ошибка аппроксимации.
Чтобы предотвратить такую ситуацию, необходимо либо уменьшить временной шаг, либо увеличить \(\mathcal{N}\).
Если мы предположим, что кубатурная решетка создаёт ошибку аппроксимации \(\OO{|\mathcal{N}|^{-1}}\),
то исключение \(M\) случайно распределённых узлов приводит к ошибке \(\OO{\sqrt{M}/|\mathcal{N}|}\).

%%%%%%%%%%%%%%%%%%%%%%%%%%%%%%%%%%%%%%%%%%%
\section{Численный пример}
%%%%%%%%%%%%%%%%%%%%%%%%%%%%%%%%%%%%%%%%%%%

\begin{wrapfigure}{r}{7.4cm}
    \vspace{-10pt}
    \centering
    \includegraphics{tikz/geometry}
    \vspace{-5pt}
    \caption{Геометрия задачи}\label{fig:geometry}
\end{wrapfigure}

Рассмотрим плоскую периодическую геометрию, как на рис.~\ref{fig:geometry}.
Газ расположен между двумя покоящимися (\(v_{Bi} = 0\)) бесконечными параллельными пластинами,
разделёнными на единичное расстояние. Их температура распределена на них по синусоидальному закону:
\begin{equation}
    T_B = 1-\alpha\cos(2\pi x).
\end{equation}
Будем рассматривать случай \(\alpha=1/2\) и полное диффузное отражение от пластин.
Плотность газа нормирована на единицу, т.е.
\begin{equation}\label{eq:total_mass}
    \int_0^1\int_0^1\rho\dd{x}\dd{y} = 1.
\end{equation}

В силу симметрии задачи, расчётная область представляет собой квадрат со стороной \(1/2\).
На рис.~\ref{fig:geometry} выделена серым цветом.
Подобная задача изучалась в~\cite{Sone1996}, но для конечных \(\Kn\) ради простоты
использовалось численное решение модельного уравнения БКВ.
В настоящей работе используется модель твёрдых сфер в качестве молекулярного потенциала
в уравнении Больцмана.
Кроме того, в рамках задачи рассмотрим приближённое решение гидродинамического типа для малых \(\Kn\),
основанное на граничном условии~\eqref{eq:boundary_temp}.

\subsection{Решение задачи в континуальном пределе}

\begin{figure}
    \centering
    \begin{subfigure}[b]{0.5\linewidth}
        \centering
        \includegraphics{contours/heat-temp}
        \caption{уравнение теплопроводности~\eqref{eq:heat_equation}}
        \label{fig:continuum:temp-heat}
    \end{subfigure}%
    \begin{subfigure}[b]{0.5\linewidth}
        \centering
        \includegraphics{contours/snif-0-temp}
        \caption{уравнения SNIF}
        \label{fig:continuum:temp-snif}
    \end{subfigure}
    \caption{Изотермические линии в континуальном пределе}
    \label{fig:continuum:temp}
\end{figure}

\begin{figure}
    \centering
    \begin{subfigure}{0.5\linewidth}
        \centering
        \includegraphics{contours/nonslip-0-flow}
        \caption{уравнения SNIF без скольжения (\(K_1=0\))}
        \label{fig:continuum:flow-nonslip}
    \end{subfigure}%
    \begin{subfigure}{0.5\linewidth}
        \centering
        \includegraphics{contours/snif-0-flow}
        \caption{уравнения SNIF}
        \label{fig:continuum:flow-snif}
    \end{subfigure}
    \caption{Стационарное поле \(v_{i1}\) в континуальном пределе: цветом показана величина, стрелки изображают направление}
    \label{fig:continuum:flow}
\end{figure}

%%% About solver
В настоящей работе уравнения SNIF~\eqref{eq:asymptotic1_u},~\eqref{eq:asymptotic1_T},~\eqref{eq:asymptotic2_p}
решаются с помощью солвера \verb+snifSimpleFoam+~\cite{Rogozin2014},
использующего метод конечных объёмов на основе вычислительной платформы OpenFOAM\textregistered{}.

%%% Space discretization
В физическом пространстве используется прямоугольная сетка:
область \(0<x<1/2\) разбивается на 30 интервалов одинаковой длины,
а область \(0<y<1/2\) на 40 интервалов, сгущающихся к \(y=0\).

%%% Results and discussions
На рис.~\ref{fig:continuum:temp} показано стационарное температурное поле,
получаемое с помощью различных моделей.
В континуальном пределе уравнение теплопроводности получается из уравнений SNIF,
если положить \(\gamma_7=0\) и \(K_1=0\).
Эффект теплового скольжения газа значительно превышает конвекцию под действием температурных напряжений,
что продемонстрированно на рис.~\ref{fig:continuum:flow}, где уравнения SNIF решены
для граничных условий со скольжением и без.
Отметим также, что направления течения газа противоположны на рис.~\ref{fig:continuum:flow-nonslip}
и~\ref{fig:continuum:flow-snif}.

\subsection{Решение для произвольных чисел Кнудсена}

\begin{figure}
    \centering
    \begin{subfigure}[b]{0.5\linewidth}
        \centering
        \includegraphics{{{contours/asym-0.01-temp}}}
        \caption{уравнения SNIF с граничным условием~\eqref{eq:boundary_temp}}
        \label{fig:kn0.01:temp-snif}
    \end{subfigure}%
    \begin{subfigure}[b]{0.5\linewidth}
        \centering
        \includegraphics{{{contours/data-0.01-temp}}}
        \caption{уравнение Больцмана}
        \label{fig:kn0.01:temp-exact}
    \end{subfigure}
    \caption{Изотермические линии для \(\Kn=0.01\)}
    \label{fig:kn0.01:temp}
\end{figure}

\begin{figure}
    \centering
    \begin{subfigure}{0.5\linewidth}
        \centering
        \includegraphics{{{contours/asym-0.01-flow}}}
        \caption{уравнения SNIF с граничным условием~\eqref{eq:boundary_temp}}
        \label{fig:kn0.01:flow-snif}
    \end{subfigure}%
    \begin{subfigure}{0.5\linewidth}
        \centering
        \includegraphics{{{contours/data-0.01-flow}}}
        \caption{уравнение Больцмана}
        \label{fig:kn0.01:flow-exact}
    \end{subfigure}
    \caption{Стационарное поле скоростей для \(\Kn=0.01\): цветом показана величина, стрелки изображают направление}
    \label{fig:kn0.01:flow}
\end{figure}

\begin{figure}
    \centering
    \begin{subfigure}[b]{0.5\linewidth}
        \centering
        \includegraphics{{{contours/data-0.1-temp}}}
        \caption{изотермические линии}
        \label{fig:kn0.1:temp}
    \end{subfigure}%
    \begin{subfigure}[b]{0.5\linewidth}
        \centering
        \includegraphics{{{contours/data-0.1-flow}}}
        \caption{поле скоростей}
        \label{fig:kn0.1:flow}
    \end{subfigure}
    \caption{Решение уравнения Больцмана для \(\Kn=0.1\)}
    \label{fig:kn0.1}
\end{figure}

%%% About solver
Для рассмотрения задачи в произвольном диапазоне чисел Кнудсена необходимо
обратиться к численному решению уравнения Больцмана.
Для этого используется солвер, разработанный в рамках проблемно-моделирующей среды
для анализа газокинетических процессов~\cite{Kloss2011, Kloss2012}.

%%% Space discretization
В физическом пространстве использовалась такая же разностная сетка,
как и при решении уравнений гидродинамического типа,
однако в слое Кнудсена (вблизи \(y=0\)) она дополнительно сгущалась так,
что ширина приграничной ячейки равнялась \(0.02\) от длины свободного пробега.
Для контроля точности использовались две различные сетки в скоростном пространстве.
Сначала задача решалась на равномерной прямоугольной сетке, ограниченной сферой радиусом \(4.24\),
так что на радиусе помещалось 8 ячеек.
Далее результат уточнялся на неравномерной сетке, ограниченной сферой радиусом \(5.3\),
причём вдоль осей \(x\) и \(z\) границы ячеек располагались как корни полинома Эрмита,
а вдоль оси \(y\) сгущались в геометрической прогрессии так,
что ширина ячеек возле \(y=0\) равнялась \(0.0067\).
Таким образом, на радиусе помещалось 9, 26, 9 ячеек соответствено.
Кубатурные сетки также имели разную мощность: около 5000 для равномерной сетки
и около 200000 для неравномерной.
Кроме того, для очень малых \(\Kn\) применялась временн\'{а}я экстраполяция распределения температуры
и поля скоростей, поскольку достижение стационарного состояния вблизи \(y=1/2\)
требует слишком большого числа итераций.

%%% Results and discussions
На рис.~\ref{fig:kn0.01:temp} и~\ref{fig:kn0.01:flow} изображено поле температуры и скорости
для \(\Kn=0.01\) и проводится сравнение между численным решением уравнения Больцмана
и приближённым решением для малых \(\Kn\).
На рис.~\ref{fig:kn0.1} показаны соответствующие распределения для \(\Kn=0.1\).
С увеличением \(\Kn\) возрастает температурный скачок возле границы \(y=0\),
а также отодвигается от пластины область максимальной скорости газа.

\newcommand*{\graphlinewidth}{1}
\pgfplotscreateplotcyclelist{legend}{
    {cyan,  dashdotted, line width=\graphlinewidth},
    {blue,  solid,      line width=\graphlinewidth},
    {green, dashed,     line width=\graphlinewidth},
    {red,   dashed,     line width=\graphlinewidth, mark=o, mark options={solid}},
    {magenta, dotted,     line width=\graphlinewidth, mark=square, mark options={solid}},
}
\newcommand{\pgfplotsReferenceGenerator}[3]{
    \scalebox{0}{
        \begin{tikzpicture}
            \begin{axis}[hide axis, cycle list name = #1]
                \foreach \i   [evaluate=\i] in {1,...,#2}{
                    \addplot (0,0); \label{#3\i}
                }
            \end{axis}
        \end{tikzpicture}
    }
}

\pgfplotsReferenceGenerator{legend}{5}{line}

\begin{figure}
    \centering
    \begin{subfigure}[b]{0.5\linewidth}
        \centering
        \includegraphics{vs_kn/bottom_temp}
        \caption{}\label{fig:comparison:bottomT}
    \end{subfigure}%
    \begin{subfigure}[b]{0.5\linewidth}
        \centering
        \includegraphics{vs_kn/bottom_flow}
        \caption{}\label{fig:comparison:bottomU}
    \end{subfigure}\\
    \begin{subfigure}[b]{0.5\linewidth}
        \centering
        \includegraphics{vs_kn/top_temp}
        \caption{}\label{fig:comparison:topT}
    \end{subfigure}%
    \begin{subfigure}[b]{0.5\linewidth}
        \centering
        \includegraphics{vs_kn/top_flow}
        \caption{}\label{fig:comparison:topU}
    \end{subfigure}
    \caption{
        Некоторые пограничные интегралы в зависимости от \(\Kn\), полученные разными методами:
        уравнение теплопроводности~\ref{line1},
        уравнения SNIF с условием~\eqref{eq:boundary_temp}~\ref{line2} и без~\ref{line3},
        уравнение Больцмана на равномерной~\ref{line4} и неравномерной~\ref{line5} сетках
    }
    \label{fig:comparison}
\end{figure}

%%% Comparison between solutions
Чтобы наглядно продемонстрировать сходимость численного решения уравнения Больцмана к
решению уравнений SNIF в континуальном пределе, рассмотрим рис.~\ref{fig:comparison}.
На рис.~\ref{fig:comparison:topT} отчётливо видно, что решение уравнения Больцмана сходится
именно к решению уравнений SNIF, а не уравнения теплопроводности.
Для самых малых \(\Kn\) погрешность решения уравнения Больцмана возрастает
ввиду получения стационарного значения посредством экстраполяции.
На рис.~\ref{fig:comparison:bottomT} приближённое решение гидродинамического типа
на основе граничного условия~\eqref{eq:boundary_temp} аппроксимирует
точное решение со вторым порядком точности,
в то время как другие решения дают только первый порядок.
В общем приближённое решение отличается от точного менее чем на 10\% в области \(\Kn<0.05\),
однако стоит учитывать, что в рассматриваемой задаче тепловое скольжение значительно превалирует
над термострессовой конвекцией. Для противоположного случая приближённое решение может давать
б\'{о}льшую погрешность.

%%%%%%%%%%%%%%%%%%%%%%%%%%%%%%%%%%%%%%%%%%%
\section{Заключение}
%%%%%%%%%%%%%%%%%%%%%%%%%%%%%%%%%%%%%%%%%%%

С помощью численного решения уравнения Больцмана для модели твёрдых сфер было показано,
что уравнения SNIF с соответствующими граничными условиями адекватно описывают
медленные неизотермические течения газа для малых чисел Кнудсена.
Таким образом, в этой области можно избежать трудоёмкого решения уравнения Больцмана.

\printbibliography

\end{document}


