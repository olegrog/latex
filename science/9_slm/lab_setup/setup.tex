\documentclass{article}
\usepackage[utf8]{inputenc}
\usepackage{mathtools, amssymb, amsthm}
\usepackage{fullpage}
\usepackage{physics}
\usepackage{siunitx}
\usepackage{comment}
\usepackage{subcaption}

\usepackage{graphicx} % [draft]

\usepackage[
    pdfusetitle,
    colorlinks
]{hyperref}

\title{Laboratory Setup for Powder Bed Fusion Technology}
\date{\today}
\author{A. Dyakov, O.A.~Rogozin, I.V.~Shishkovsky}

\begin{document}
\maketitle
\tableofcontents

\section{Introduction}

%%% Current problems
The powder bed fusion (PBF) process on every industrial installation goes in a closed regime without any operator intervention.
In fact, it is a precisely and excessively high added value of powders for the PBF that the manufacturers' costs for the choice of powder materials
and the selection of optimal technological regimes.
Therefore, multi-parametrical theoretical modeling of the PBF processes allows minimizing these costs both in time and money.
The laboratory setup (LS) allows to conduct multivariate analysis and determine the range of laser influence parameters
(energy power, scanning speed, laser beam diameter, etc.) and requirements for powder compositions
(dispersion, content, alloying elements, methods of preparation and laying, etc.).
Multifactorial optimization of the PBF processes on the LS will allow to identify the main (leading) processes
and accelerate the validation of the CAD--CAM--CAE models.

%%% Aim
The current project aims at \emph{design} of the LS for the comprehensive optimization of the PBF technology
and diagnostics of the physical and mechanical properties of \emph{new powder materials} during their 3D printing.
Data to be obtained on the LS will form the basis for \emph{validation}
of the hierarchical multi-physics model for selective laser melting (SLM).

\section{Requirements}

\begin{figure}
    \newcommand{\picwidth}{0.7\textwidth}
    \centering
    \begin{subfigure}{\picwidth}
        \includegraphics[width=\textwidth]{full}
        \caption{full}
        \label{fig:full}
    \end{subfigure}\\
    \begin{subfigure}{\picwidth}
        \includegraphics[width=\textwidth]{front}
        \caption{front}
        \label{fig:front}
    \end{subfigure}\\
    \begin{subfigure}{\picwidth}
        \includegraphics[width=\textwidth]{side}
        \caption{side}
        \label{fig:side}
    \end{subfigure}
    \caption{The general view of the LS}
    \label{fig:ls}
\end{figure}

The principal advantage of the proposed LS from well-known industrial PBF equipment will be ability in the laboratory configuration:
\begin{enumerate}
    \item To stop the synthesis process at any stage in order to change the technical process parameters on the fly, for example,
    add (or remove) powders, mix disparate materials (the so-called ``multi-material approach'', which is only planned in foreign installations),
    change the composition of the protective gas mixture, enable (disable) the processing zone heating.
    \item To carry out the process temperature diagnostics (thermocouple measurement over the powder volume, pyrometer measurements from the surface).
    \item 3D scanning system with ``flat-top'' shape beam.
    \item To determine the deviation of the layer shape from the CAD model.
    \item To evaluate the microelement composition of the surface layer (diagnostics by induction-coupled plasma spectroscopy).
    \item To evaluate stresses and deformations (from the surface by the laser speckle holography, or by volume via ultrasonic diagnostics).
    \item \dots (input into the chamber of an additional energy source, evaluation of the phase composition
    and transformation during 3D part fabrication, etc.).
\end{enumerate}

There capabilities make is possible to reduce drastically time for searching and optimizing regimes of PBF manufacturing
and significantly save on consumable powder materials, since the results of the synthesis and the prospects for creating
a future 3D parts are already visible on the first 1--5 layers.

\section{Design}

The preliminary LS project is designed and shown in Fig.~\ref{fig:ls}.
The laser source, printing chamber, powder cylinders, and inert chamber are arranged one above the other.
The central cylinder is used for printing, while the input and output powders are stored in the left and right ones, respectively.
There is a control unit in the left rack.
A close-up of the printing chamber is shown in Fig.~\ref{fig:chamber}.
The sealed chamber can be filled with an inert gas and preheated.

\begin{figure}
    \newcommand{\picwidth}{0.5\textwidth}
    \centering
    \begin{subfigure}{\picwidth}
        \includegraphics[width=\textwidth]{inside}
        \caption{inside the chamber}
        \label{fig:inside}
    \end{subfigure}\:
    \begin{subfigure}{\picwidth}
        \includegraphics[width=\textwidth]{door}
        \caption{the door}
        \label{fig:door}
    \end{subfigure}
    \caption{The chamber of the LS}
    \label{fig:chamber}
\end{figure}

\end{document}
