\documentclass{article}
\usepackage[utf8]{inputenc}
\usepackage{amsmath, amssymb, amsthm}
\usepackage{fullpage}
\usepackage{comment}
\usepackage{siunitx}

\title{Selective Laser Melting}
\date{\today}
\author{Oleg Rogozin}

\newcommand{\pder}[2][]{\frac{\partial#1}{\partial#2}}
\newcommand{\dder}[2][]{\Delta_{#2}#1}
\newcommand{\pderdual}[2][]{\frac{\partial^2#1}{\partial#2^2}}

\newcommand{\pert}[1]{\delta#1}
\newcommand{\diag}[1]{\left(#1\right)^\mathrm{diag}}

\newcommand{\fusion}[1]{{#1}_\mathrm{fus}}

\begin{document}
\maketitle

\section{Formulation of the problem}

The temperature \(T\) (\si{K}) of metal powder is governed by the heat-conduction equation
\begin{equation}\label{eq:heat_conduction}
	\rho\pder[h(T)]{t} = \pder{x_i}\left(k(\psi,T)\pder[T]{x_i}\right),
\end{equation}
where \(x_i\) are the physical coordinates (\si{m}), \(t\) is the time (\si{s}),
\(\rho\) is the density (\si{kg/m^3}), \(h\) is the specific enthalpy (\si{J/m^3}),
\(k\) is the thermal conductivity (\si{W/m/K}), \(\psi\) is the porosity. Let
\begin{equation}\label{eq:porosity_variables}
	\rho = \rho_0(1-\psi_0), \quad k = k_0(T)(1-\psi),
\end{equation}
where \(\rho_0\) and \(k_0\) is the density and thermal conductivity of the bulk material, respectively,
\(\psi_0\) is the initial porosity.

The initial condition is \(T=T_0\) at \(t=0\).
Consider the following physical domain:
\begin{equation}
	-H/2 < x_1 < L-H/2, \quad -H/2 < x_2 < H/2, \quad 0 < x_3 < H.
\end{equation}
The boundary condition at \(x_3=0\) is expressed as
\begin{equation}\label{eq:bc}
	k\pder[T]{x_3} = \frac{AP}{\pi R^2}\exp\left(-\frac{(x_1-Ut)^2 + x_2^2}{R^2}\right) - \alpha(T-T_0),
\end{equation}
otherwise \(T = T_0\).
\(A\) is the absorptivity, \(P\) is the laser power (\si{W}), \(U\) is the scanning speed (\si{m/s}),
\(R\) is the radius of the laser beam (\si{m}), \(\alpha\) is the convective heat transfer coefficient (\si{W/m^2/K}).

To incorporate the melting process into the model, write the enthalpy and porosity in the following form:
\begin{gather}
	h = c_p(T)(T-T_0) + \fusion{h}\phi(h), \label{eq:enthalpy} \\
	\psi(t) = \psi_0\min_{0\leq\tau\leq t}\left( 1-\phi(h(\tau)) \right), \label{eq:porosity}
\end{gather}
where \(c_p\) is the specific heat capacity (\si{J/kg/K}),
\(\fusion{h}\) is the specific enthalpy of fusion (\si{J/kg}),
\(\phi\) is the liquid fraction (\(0 \leq \phi \leq 1\)), which is monotonic function of \(h\).
Assume that \[ \phi(h\leq h_S) = 0, \quad \phi(h\geq h_L) = 1,\]
where \(h_S = h(T_S)\) and \(h_L = h(T_L)\).
\(T_S\) and \(T_L\) are the solidus and liquidus temperatures, respectively.


\subsection{Dimensionless expressions}

Introduce the following dimensionless variables:
\begin{equation}\label{eq:dimensionless}
    \begin{aligned}
        x_i &= \hat{x_i}R, & L &= \hat{L}R, & H &= \hat{H}R, \\
        t &= \hat{t}t_0, & U &= \hat{U}R/t_0, & T - T_0 &= \hat{T}(\fusion{T}-T_0), \\
        h &= \hat{h}c_p(T_0)(\fusion{T}-T_0), & P &= \hat{P} k(T_0)(\fusion{T}-T_0)R, & \alpha &= \hat{\alpha}k(T_0)/R,
    \end{aligned}
\end{equation}
where \(t_0=\rho_0 c_p(T_0) R^2/k(T_0)\) is the diffusive time scale at \(T=T_0\),
\(\fusion{T}=(T_S+T_L)/2\) is the mean fusion temperature.
Also denote
\begin{equation}
    \hat{c}_p(\hat{T}) = \frac{c_p(T)}{c_p(T_0)}, \quad
    \hat{k}(\hat{T}) = \frac{k(T)}{k(T_0)}, \quad
    \hat{\phi}(\hat{h}) = \phi(h), \quad
    \hat{\psi}(\hat{h}) = \psi(h).
\end{equation}

Let the dependence on temperature be linear:
\begin{gather}
	\hat{c}_p = 1 + \hat{c}_p'\hat{T}, \quad
	\hat{k} = 1 + \hat{k}'\hat{T}, \\
	\hat{\phi} = \begin{cases}
        0 & \quad \hat{h} \leq \hat{h}_S, \\
        \frac{\hat{h}-\hat{h}_S}{\hat{h}_L-\hat{h}_S} & \quad \hat{h}_S < \hat{h} < \hat{h}_L, \\
        1 & \quad \hat{h}_L \leq \hat{h}.
    \end{cases}
\end{gather}
Therefore,
\begin{equation}\label{eq:heat_conduction_hats}
	\pder[\hat{h}]{\hat{t}} = \pder{\hat{x}_i}\left(
	    \frac{1 - \hat{\psi}}{1 - \psi_0}
	    \frac{1 + \hat{k}'\hat{T}}{1 + 2\hat{c}_p'\hat{T}}
	    \pder[\hat{h}_*]{\hat{x}_i}
	\right)
\end{equation}
is the dimensionless form of the heat-conduction equation for \(\hat{h}\),
where \(\hat{h}_* = \hat{h} - \fusion{\hat{h}}\hat{\phi}\) and
\begin{equation}\label{eq:temperature}
	\hat{T} = \frac{\sqrt{1+4\hat{c}_p'\hat{h}_*}-1}{2\hat{c}_p'}
\end{equation}
when \(\hat{c}_p'>0\) and \(\hat{T} = \hat{h}_*\) when \(\hat{c}_p'=0\).
\begin{equation}\label{eq:bc_hats}
	\left( \hat{\alpha} + \frac{1 - \hat{\psi}}{1 - \psi_0}\left(1 + \hat{k}'\hat{T}\right)\pder{\hat{x}_3}\right)\hat{T}
	    = \frac{A\hat{P}}{\pi}\exp\left(-(\hat{x}_1-\hat{U}\hat{t})^2 - \hat{x}_2^2\right)
\end{equation}
is the boundary conditions at \(\hat{x}_3=0\).
\(\hat{h} = 0\) is the initial condition and boundary condition for other faces.

\section{Numerical methods}

For~\eqref{eq:heat_conduction_hats}, we can write
\begin{gather}
    F = \dder{\hat{x}_i}\left(
        \frac{1 - \hat{\psi}}{1 - \psi_0}
	    \frac{1 + \hat{k}'\hat{T}}{1 + 2\hat{c}_p'\hat{T}}
	    \dder[\hat{h}_*]{\hat{x}_i}
	\right), \\
    \pert{F} = \dder{\hat{x}_i}\left[ \frac{1 - \hat{\psi}}{1 - \psi_0} \left(
	    \frac{(\hat{k}' - 2\hat{c}_p')\pert{\hat{h}_*}}{(1 + 2\hat{c}_p'\hat{T})^3}
	     \dder[\hat{h}_*]{\hat{x}_i}
	    +
	    \frac{1 + \hat{k}'\hat{T}}{1 + 2\hat{c}_p'\hat{T}}
	    \dder[\pert{\hat{h}_*}]{\hat{x}_i}
	\right)\right], \\
    \diag{\pert{F}} \approx \frac{1 - \hat{\psi}}{1 - \psi_0} \left(
	    \frac{\hat{k}' - 2\hat{c}_p'}{(1 + 2\hat{c}_p'\hat{T})^3} \dder[\hat{h}_*]{\hat{x}_i} \diag{\dder{\hat{x}_i}}
	    +
	    \frac{1 + \hat{k}'\hat{T}}{1 + 2\hat{c}_p'\hat{T}}
	    \diag{\dder{\hat{x}_i}^2}
	\right) \diag{\pert{\hat{h}_*}}.
\end{gather}
where
\begin{gather}
    \pert{\hat{h}_*} = \pert{\hat{h}} - \fusion{\hat{h}}\pert{\hat\phi}, \quad
    \diag{\hat{h}_*} = \diag{\pert{\hat{h}_*}} = 1 - \fusion{\hat{h}}\diag{\pert{\hat\phi}}, \\
	\pert{\hat{\phi}} = \begin{cases}
        0 & \quad \hat{h} \leq \hat{h}_S, \quad \hat{h}_L \leq \hat{h}, \\
        \pert{\hat{h}}/(\hat{h}_L-\hat{h}_S) & \quad \hat{h}_S < \hat{h} < \hat{h}_L, \\
    \end{cases} \label{eq:pert_phi_h}\\
	\diag{\pert{\hat{\phi}}} = \begin{cases}
        0 & \quad \hat{h} \leq \hat{h}_S, \quad \hat{h}_L \leq \hat{h}, \\
        1/(\hat{h}_L-\hat{h}_S) & \quad \hat{h}_S < \hat{h} < \hat{h}_L. \\
    \end{cases} \label{eq:pert_phi_h_diag}
\end{gather}

Nonlinear dependence~\eqref{eq:temperature} can be linearized; therefore, we have the following linear boundary condition at \(z=0\):
\begin{equation}\label{eq:bc_linearized}
	\left( \hat{\alpha} + \frac{1 - \hat{\psi}^n}{1 - \psi_0}\left(1 + \hat{k}'\hat{T}^n\right)\pder{\hat{x}_3}\right)
	    \left( \hat{T}^n + \frac{\hat{h}_* - \hat{h}_*^n}{1+2\hat{c}_p'\hat{T}^n} \right)
	    = \frac{A\hat{P}}{\pi}\exp\left(-(\hat{x}_1-\hat{U}\hat{t}^n)^2 - \hat{x}_2^2\right),
\end{equation}
where superscripts \(n\) and \(n+1\) correspond to the previous and current iterations, respectively.

\end{document}

0 = T + c'T^2 + phi - h
T = (-1+\sqrt{1+4c'(h-phi)})/2c'
\psi = \psi_0 -- \phi=0
\psi = 0      -- \phi=1
\psi = \psi_0 - \phi\psi_0

