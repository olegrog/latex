\documentclass{article}
\usepackage[utf8]{inputenc}
\usepackage{amsmath, amssymb, amsthm}
\usepackage{fullpage}
\usepackage{comment}
\usepackage{siunitx}

\title{Selective Laser Melting}
\date{\today}
\author{Oleg Rogozin}

\newcommand{\pder}[2][]{\frac{\partial#1}{\partial#2}}
\newcommand{\pderdual}[2][]{\frac{\partial^2#1}{\partial#2^2}}
\newcommand{\idx}[2][]{#1_{\mathrm{#2}}}

\begin{document}
\maketitle

\section{Heat-conduction equation}

Temperature \(T\) of metal powder is governed by the heat-conduction equation
\begin{equation}\label{eq:heat_conduction}
	\pder[\rho(\phi,T) h(T)]{t} = \pder{x_i}\left(k(\phi,T)\pder[T]{x_i}\right),
\end{equation}
where \(\rho\) is the density (\si{kg/m^3}), \(h\) is the specific enthalpy (\si{J/m^3}),
\(k\) is the thermal conductivity (\si{J/m/K/s}), \(\psi\) is the porosity.
Let
\begin{equation}\label{eq:porosity}
	\rho = \rho_0(1-\psi), \quad k = k_0(1-\psi)
\end{equation}
where \(\rho_0(T)\) and \(k_0(T)\) is the density and thermal conductivity
of the bulk material.
To study melting process as well, write the  enthalpy in the following form:
\begin{equation}\label{eq:enthalpy}
	h = c_p(T)T + \Delta\idx[h]{fus}\phi(T),
\end{equation}
where \(c_p\) is the specific heat capacity (\si{J/kg/K}),
\(\Delta\idx[h]{fus}\) is the specific enthalpy of fusion (\si{J/kg}),
\(\phi\) is the liquid fraction (\(0 \leq \phi \leq 1\)).
\(\phi=0\) and \(\phi=1\) corresponds to the solid and liquid states, respectively.

Consider the following physical domain:
\begin{equation}
	0 < x_1 < L, \quad -H/2 < x_2 < H/2, \quad 0 < x_3 < H.
\end{equation}
Boundary conditions at \(x_3=0\):
\begin{equation}\label{eq:bc}
	k\pder[T]{x_3} = \frac{AP}{\pi R^2}\exp\left(-\frac{(x_1-H/2-Ut)^2 + x_2^2}{R^2}\right),
\end{equation}
otherwise \(T = T_0\).
\(A\) is the absorptivity, \(P\) is the laser power (\si{J/s}), \(U\) is the scanning speed (\si{m/s}),
\(R\) is the radius of the laser beam (\si{m}).
The initial condition is \(T=T_0\).

Introduce the following dimensionless variables:
\begin{equation}\label{eq:dimensionless}
    \begin{aligned}
        x_i &= \hat{x_i}R, & L &= \hat{L}R, & H &= \hat{H}R, \\
        t &= \hat{t}t_0, & U &= \hat{U}R/t_0, & T &= (\hat{T}+1)T_0, \\
        h &= \hat{h}c_p(T_0)T_0, & P &= \hat{P} k(T_0)T_0R,
    \end{aligned}
\end{equation}
where \(t_0=\rho(T_0) c_p(T_0) R^2/k(T_0)\) is the diffusive time scale at \(T=T_0\).
Also denote
\begin{equation}
    \hat{\rho}(\hat{T}) = \frac{\rho(T)}{\rho(T_0)}, \quad
    \hat{c}_p(\hat{T}) = \frac{c_p(T)}{c_p(T_0)}, \quad
    \hat{k}(\hat{T}) = \frac{k(T)}{k(T_0)}, \quad
    \hat{\phi}(\hat{T}) = \phi(T).
\end{equation}

Let the dependence on temperature be linear:
\begin{gather}
	\hat{\rho} = 1 + \hat{\rho}'\hat{T}, \quad
	\hat{c}_p = 1 + \hat{c}_p'\hat{T}, \quad
	\hat{k} = 1 + \hat{k}'\hat{T}, \\
	\hat{\phi} = \begin{cases}
        0 & \quad \hat{T} \leq \hat{T}_S, \\
        \frac{\hat{T}-\hat{T}_S}{\hat{T}_L-\hat{T}_S} & \quad \hat{T}_S < \hat{T} < \hat{T}_L, \\
        1 & \quad \hat{T}_L \leq \hat{T},
    \end{cases}
\end{gather}
where \(\hat{T}_S\) and \(\hat{T}_L\) are the solidus and liquidus temperatures, respectively.
Therefore,
\begin{equation}\label{eq:heat_conduction2}
	\pder{\hat{t}}\left[\left(1 + \hat{\rho}'\hat{T}\right)
	    \left(\hat{T} + \hat{c}_p'\hat{T}^2 + \Delta\idx[\hat{h}]{fus}\hat{\phi}(\hat{T})\right)\right]
	    = \pder{\hat{x}_i}\left(\left(1 + \hat{k}'\hat{T}\right)\pder[\hat{T}]{\hat{x}_i}\right)
\end{equation}
is the dimensionless heat-conduction equation.

\begin{comment}
C = C_1 + (C_2-C_1)\frac{T-T_1}{T_2-T_1}
C_0 = C(T=T_0)
C = C_0 + (C_2-C_0)\frac{\hat{T}-1}{\hat{T_2}-1}

[pV] = [Fs] = J
[k] = J/m/K/s
[c] = J/kg/K
[\rho c] = J/K/m^3
[H] = J/kg
[h] = J/m^3 = [H\rho]


\pder[h]{t} = \pder{x_i}\left(k\pder[T]{x_i}\right)
J/m^3/s = K/m^2 [k]
[k] = J/m/K/s
[kT/L] = [P/L^2]
[P] = [kTL]
\end{comment}

\end{document}

