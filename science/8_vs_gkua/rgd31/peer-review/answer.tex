\documentclass{article}
\usepackage{a4wide}
\usepackage[english]{babel}

%%% functional packages
\usepackage{amssymb, mathtools, amsthm, mathrsfs}
\usepackage{framed}                % for leftbar
\usepackage[%
    font={bfseries},
    leftmargin=0.1\textwidth,
    indentfirst=false
]{quoting}

\usepackage{lipsum}
\usepackage{graphicx}
\def\asterism{\par\vspace{1em}{\centering\scalebox{1}{\bfseries *~*~*}\par}\vspace{.5em}\par}

\usepackage[
    backend=biber,
    style=numeric,
    sorting=none,
    maxbibnames=99, minbibnames=99,
    natbib=true,
    url=false,
    eprint=true,
    pagetracker,
    giveninits]{biblatex}
\bibliography{../manuscript}

\graphicspath{{../}}

% user commands
\newcommand{\Kn}{\mathrm{Kn}}
\newcommand{\Ma}{\mathrm{Ma}}
\newcommand{\dd}{\:\mathrm{d}}
\newcommand{\der}[2][]{\frac{\dd#1}{\dd#2}}
\newcommand{\derdual}[2][]{\frac{\dd^2#1}{\dd#2^2}}
\newcommand{\pder}[2][]{\frac{\partial#1}{\partial#2}}
\newcommand{\pderdual}[2][]{\frac{\partial^2#1}{\partial#2^2}}
\newcommand{\pderder}[2][]{\frac{\partial^2 #1}{\partial #2^2}}
\newcommand{\Pder}[2][]{\partial#1/\partial#2}
\newcommand{\dxi}{\dd\boldsymbol{\xi}}
\newcommand{\bxi}{\boldsymbol{\xi}}
\newcommand{\bx}{\boldsymbol{x}}
\newcommand{\Nu}{\mathcal{N}}
\newcommand{\Mu}{\mathcal{M}}
\newcommand{\OO}[1]{O(#1)}
\newcommand{\Set}[2]{\{\,{#1}:{#2}\,\}}

\newcommand{\FigWidth}{0.7}

\title{Response to the Reviewers' Comments}

\begin{document}

\maketitle

We thank the reviewers for the comments and suggestions.
Thanks to them, the manuscript was revised and extended by 2 pages.
Our answer and changes are presented below.
The reviewers' words are shown in bold.
Individual parts of the corrected version of the manuscript
can be recognized by the bold lateral line.

\section*{Referee report \#1}

\begin{quoting}
    1. What is the time discretization used in Tcheremissine's method? Is it explicit or implicit?
\end{quoting}

The explicit time integration scheme was employed for the current study.
The following paragraphs has been added to clarify this point.

\begin{leftbar}
    %%% Evaluation of the collision integral
    Based on cubature~\eqref{eq:xi_cubature}, the collision integral~\eqref{eq:symm_ci} is evaluated in a quasi-Monte Carlo way:
    \begin{equation}\label{eq:J_cubature}
        J(f_\gamma) \approx \frac{\pi V_\Gamma^2}{\displaystyle\sum_{\nu\in\Nu} w_{\nu}w_{*\nu}}
            \sum_{\nu\in\Nu} \left(
                \delta_{\gamma\nu} + \delta_{*\gamma\nu} - \delta'_{\gamma\nu} - \delta'_{*\gamma\nu}
            \right)\left(
                \frac{w_{\nu}w_{*\nu}}{w'_{\nu}w'_{*\nu}} f'_{\nu} f'_{*\nu} -  f_{\nu} f_{*\nu}
            \right)B_\nu.
    \end{equation}
    Set of cubature points \(\{\bxi_\nu, \bxi_{*\nu}, \boldsymbol{\alpha}_\nu\}_{\nu\in\Nu}\)
    is defined by the specific eight-dimensional cubature rule.

    \asterism

    %%% Solution of the Cauchy problem
    Now, let \(f_\gamma^n\) denote the approximate solution of the space-homogeneous Boltzmann equation
    for velocity \(\bxi_\gamma\) at time \(t_n\), \(n\in\mathbb{N}\).
    Approximation~\eqref{eq:J_cubature}, rewritten as
    \begin{equation}\label{eq:J_cubature_short}
        J_\gamma^n = \sum_{j=1}^N \Delta_\gamma^{n+(j-1)/N},
    \end{equation}
    is transformed to a numerical scheme in fractional steps:
    \begin{equation}\label{eq:time_integration_scheme}
         f_\gamma^{n+j/N} =  f_\gamma^{n+(j-1)/N} + \frac{\Delta{t}}{N}\Delta_{\gamma}^{n+(j-1)/N}
        \quad (j = 1,\dotsc,N), \quad \Delta{t} = t_{n+1}-t_n.
    \end{equation}
    Each \(j\) step preserves mass, momentum, kinetic energy and does not decrease entropy
    if positivity of the VDF is maintained.

    \asterism

    It is worth noting that the construction of the implicit schemes is straightforward for the BGK-type models,
    but is a not a trivial problem for the Boltzmann equation~\cite{Aristov1980}.
    This property is usually very attractive from the computational point of view.
\end{leftbar}

\begin{quoting}
    2. I don't understand the bottom half of Figure 4. The GKUA does not give a reasonable solution in this case?
\end{quoting}

We have removed the meaningless bottom half of Figure 4. Additionally, we have extended the caption for this figure.
\begin{leftbar}
    FIGURE 5. \dots
    The exact solution is \(v_z=0\), but an approximate evaluation of the collision integral yields a nonzero field.
\end{leftbar}

\begin{quoting}
    3. Some further discussion regarding the results in Figure 1-4 are necessary.
    The current conclusion remarks are somehow detached from the numerical section.
    That is, I don’t see how the conclusion is directly related to what was observed in numerical results.
\end{quoting}

We agree that connection between some points of the concluding remarks and the presented results was unclear,
however, indeed, all of mentioned points just follow from the excellent agreement of the presented results.
This is a list of conclusions.
\begin{itemize}
    \item The Shakhov model with the appropriate temperature exponent for the viscosity and thermal conductivity
    is a reliable engineering-level model for the re-entering aerodynamics,
    including the hypersonic flows~\cite{Titarev2018, Frolova2018}.
    \item The Tcheremissine regularization method is a promising framework for robust DV algorithms and high-accuracy simulations.
    \item Stochastic computational techniques can improve pure deterministic methods.
    Here, the randomly shifted cubature rule with time averaging considerably improves accuracy for steady-state problems.
    \item It is known that the conservation properties are important for crude meshes and/or internal flows.
    However, lack of them is not the issue for detailed meshes and external flows. The present comparison of
    the non-conservative GKUA and fully conservative finite-volume discrete-velocity method provides evidence.
    \item The ray effect can noticeably distort a solution in the wake region near the boundary.
    Refinement of the velocity grid near the origin helps to reduce the relevant numerical error.
\end{itemize}

We hope that the first two points does not require additional discussion.
They are just consequence of the successful results.
Description of the third and forth points have been extended.
To clarify the fifth point, we have added the missed figure and paragraph (see the answer for the Referee report \#2).

\section*{Referee report \#2}

\begin{quoting}
    1. Please comment on the computational cost of these models.
    The Tcheremissine approach needs eight-dimensional numerical integration, whose efficiency should be commented.
\end{quoting}

We hope that the following paragraph is sufficient to cover the question.
\begin{leftbar}
    Finally, let us compare the computational cost of the two methods considered.
    The model collision term~\eqref{eq:Shakhov} requires macroscopic variables
    and, therefore, \(\OO{|\mathcal{V}|}\) operations.
    In contrast, complexity of evaluating~\eqref{eq:J_cubature} is equal to \(\OO{|\Nu|}\).
    Values \(|\mathcal{V}|\) and \(|\Nu|\) are independent in general.
    However, the number of cubature points providing the desired accuracy are strongly correlated with the Knudsen number.
    For instance, a low cardinality of \(\Nu\) can be sufficient for a highly rarefied gas,
    and, therefore, evaluating the Boltzmann collision integral can be much cheaper than its relaxation model
    (similar to the DSMC method).
\end{leftbar}

\begin{quoting}
    2. The ray effect is mentioned in the concluding remarks, but is not clear in the numerical results.
\end{quoting}

We have included the missed figure and paragraph to clarify this issue.
\begin{leftbar}
    The discrete-velocity methods have a well-known drawback in comparison with the particle-based methods.
    Due to a fixed discretization of the velocity space, the isotropic scattering from the boundary
    is distorted by distinctive directions for a gas flow.
    The so-called ray effect occurs in the highly rarefied regions
    and can be observed in Fig.~\ref{fig:90:temp_behind}.
\end{leftbar}

\begin{figure}
    \centering
    \includegraphics[width=\FigWidth\linewidth, clip, trim={120 30 107 50 mm}]%
        {cylinder-90km/temp-gkua-behind}
    \caption{Contours of temperature for \(\Ma=2.4\), \(\Kn=0.13\).
        A close-up behind the cylinder is shown.
        Arrows shows the selected directions of the gas flow due to the DV model of the diffuse reflection.
        Upper and lower halves correspond to the different solvers.}
    \label{fig:90:temp_behind}
\end{figure}

\printbibliography

\end{document}
