\RequirePackage{amsmath}
\documentclass{aip-cp}

\usepackage[numbers]{natbib}
\usepackage{rotating}
\usepackage{graphicx}

% Authors' packages
\usepackage{bm}
\usepackage{comment}

% user commands
\newcommand{\Kn}{\mathrm{Kn}}
\newcommand{\Ma}{\mathrm{Ma}}
\newcommand{\dd}{\:\mathrm{d}}
\newcommand{\der}[2][]{\frac{\dd#1}{\dd#2}}
\newcommand{\derdual}[2][]{\frac{\dd^2#1}{\dd#2^2}}
\newcommand{\pder}[2][]{\frac{\partial#1}{\partial#2}}
\newcommand{\pderdual}[2][]{\frac{\partial^2#1}{\partial#2^2}}
\newcommand{\pderder}[2][]{\frac{\partial^2 #1}{\partial #2^2}}
\newcommand{\Pder}[2][]{\partial#1/\partial#2}
\newcommand{\dxi}{\dd\boldsymbol{\xi}}
\newcommand{\bxi}{\boldsymbol{\xi}}
\newcommand{\bx}{\boldsymbol{x}}
\newcommand{\Nu}{\mathcal{N}}
\newcommand{\OO}[1]{O(#1)}
\newcommand{\Set}[2]{\{\,{#1}:{#2}\,\}}

\newcommand{\FigWidth}{0.7}

% Document starts
\begin{document}

% Title portion
\title{Regularization and modeling of the Boltzmann collisional operator:
    Tcheremissine and Shakhov approaches}

\author[aff1,aff2]{Oleg A. Rogozin\corref{cor1}}
\author[aff2]{Vladimir V. Aristov}
\author[aff3,aff4]{Aoping Peng}
\author[aff3,aff4]{Zhihui Li}

\affil[aff1]{Center for Design, Manufacturing and Materials,
    Skolkovo Institute of Science and Technology,
    3 Nobel, Moscow 143026, Russia}
\affil[aff2]{Dorodnicyn Computing Center,
    Federal Research Center ``Computing Science and Control'' of Russian~Academy~of~Sciences,
    40 Vavilova, Moscow 119333, Russia}
\affil[aff3]{Hypervelocity Aerodynamics Institute,
    China Aerodynamics Research and Development Center,
    6 South Section, 2 Ring, Mianyang, Sichuan 621000, China}
\affil[aff4]{National Laboratory for Computational Fluid Dynamics,
    Beihang University,
    37 Xueyuan, Beijing 100191, China}
\corresp[cor1]{Corresponding author: oleg.rogozin@phystech.edu}

\maketitle

\begin{abstract}
Solution of the Boltzmann equation involves many difficulties
mainly associated with the nonlinear and nonlocal multidimensional nature of the collision term.
To overcome them, two different strategies are commonly used:
1) highly efficient numerical algorithms
and 2) simplified collision operators capable of reproducing the desired properties of the Boltzmann dynamics.
In this work, two corresponding approaches proposed many years ago are considered and compared for some numerical examples.
The Tcheremissine regularization method provides a flexible framework for designing conservative discrete-velocity methods.
The Shakhov relaxation model is able to mimic the viscosity and thermal conductivity preserving the computational simplicity.
The obtained results are in very good agreement, but, due to uncontrollable inaccuracy of relaxation models,
some subtle discrepancies arise in regions where the velocity distribution function is far from equilibrium.
\end{abstract}

% Head 1
\section{INTRODUCTION}

%%% Relevance of the problem
The Boltzmann equation describes the gas transport phenomena for the full spectrum of flow regimes
and acts as the main foundation for the study of complex gas dynamics including spacecraft re-entering Earth’s atmosphere.
Efficient unified numerical algorithms are especially in demand in the re-entering aerodynamics,
due to the wide range of the gas rarefaction for the considered flow regimes.

%%% Classification of numerical methods for the Boltzmann equation
A huge amount of research is devoted to the numerical solution of the Boltzmann equation.
The diversity of the proposed computational methods is so wide that it is hardly amenable to a clear classification.
However, it is possible to identify three main directions
depending on the method of approximation of the velocity distribution function (VDF)
for the further evaluation of the Boltzmann collision integral:
\begin{itemize}
    \item \emph{statistical simulation} methods are designed on the basis of some Markov-type random process,
    \item \emph{discrete velocity (DV)} methods imply a fixed set of allowed molecular velocities,
    \item \emph{projection (weighted residual)} methods work within a certain function space.
\end{itemize}
Due to the overwhelming popularity of the direct simulation Monte-Carlo (DSMC) method from the first family,
the separation of the numerical methods into the stochastic and deterministic ones is widespread~\cite{Mieussens2014}.
This traditional opposition seems to have a weak rationale from the practical point of view,
since introduction of the stochastic techniques is the classical approach for improving the purely deterministic method,
particularly, for the high-dimensional numerical integration~\cite{Dick2013}.

%%% More detailed classification for projection methods
While historically the DSMC and DV methods were the first that used for practical applications~\cite{Bird1963, Nordsieck1966},
currently, development of the projection methods are the most active area of research.
The Hermite basis is most appropriate, when the VDF is close to Maxwellian~\cite{Gobbert2007}.
The Sonine (associated Laguerre) basis demonstrates the best convergence for the isotropic VDF~\cite{Fonn2014}.
In the absence of a priori information about the VDF,
it is possible to design a fast algorithm for an arbitrary molecular potential based on the Fourier expansion~\cite{Wu2015}.
The discrete Galerkin formulation is widely used for the compactly supported basis functions~\cite{Majorana2011, Alekseenko2014, Kitzler2015}.
Finally, the wavelet-based methods are promising for the efficient approximation of the multi-scale VDF~\cite{Tran2013}.

%%% Requirements for methods
Among the many properties of the robust and efficient computational method for solving the Boltzmann equation,
there are three important ones that are usually emphasized in the literature~\cite{Dimarco2014}:
\begin{itemize}
    \item conservation of mass, momentum, and kinetic energy (\emph{conservation properties});
    \item \(\mathcal{H}\) theorem holds (\emph{entropy properties});
    \item \emph{positivity} of the VDF.
\end{itemize}
Violation of all the listed properties is possible, but usually requires special care.

%%% Regularization of the collisional operator
The DV methods are often associated directly with the \emph{discrete-collision models}
that allow only discrete set of post-collisional velocities~\cite{Goldstein1989}.
The conservation and entropy properties that are inherent at the microscopic level are considered as their fundamental advantage.
However, the disappointing result on the convergence rate of these models~\cite{Palczewski1997}
stimulated the search for a method of \emph{collisional operator regularization}
in order to relax the severe restriction on the collisional set
and achieve the second-order convergence with the mentioned properties.
Three different approaches were put forward immediately as a dialectical answer for the problem posed.
\begin{itemize}
    \item \citet{Buet1998} considered several mollified operators preserving the conservation properties
    in the weak (\emph{macroscopic}) form (for the collision operator as a whole).
    \item \citet{Babovsky1998} designed a simple conservative scheme on the \emph{mesoscopic} level
    (for the whole collisional sphere). His approach was later developed by \citet{Goersch2002}.
    \item \citet{Tcheremissine1998} proposed a new class of \emph{microscopically} conservative DV methods
    based on the splitting of scattered particles.
    The entropic properties were ensured by the special interpolation procedure~\cite{Tcheremissine2006}.
\end{itemize}
The last approach is turned to be the most flexible and efficient from the computational point of view.
The multipoint projection~\cite{Beylich2000, Varghese2007} gives extra freedom to extend the Tcheremissine method
for mixtures~\cite{Dodulad2012} and nonuniform grids~\cite{Rogozin2016}.

%%% Modeling of the collisional operator
Despite the significant progress in the numerical methods for the Boltzmann equation,
the \emph{collisional operator modeling} still remains effective for reducing computational cost
and designing implicit and asymptotic-preserving schemes~\cite{Jin1999}.
Currently, the \emph{relaxation} models are widespread since they are able to reproduce many properties of the Boltzmann equation.
\citet{Holway1966} and~\citet{Shakhov1968} introduced extra freedom to make the Prandlt number
an arbitrary parameter in contrast to the original Krook--Welander model~\cite{Krook1954, Welander1954}.
While the Holway model is more suitable for the temperature-driven flows, since it corrects the stress tensor,
the Shakhov model is known to describe the velocity-driven flows better, because it adjusts only the heat-flux term.
However, until now, it is not possible to make any a priori estimates of the difference
between the relaxation models and Boltzmann solutions.
Therefore, a comparative analysis between them for the crucial flow regimes
is important for the validation of the collisional operator modeling.

%%% Aim of the study
In the present paper, an analysis of this kind is carrying out
for the Tcheremissine method and Shakhov model.
The latter is presented in the gas-kinetic unified algorithm (GKUA)~\cite{Li2004},
established and used to simulate the re-entering aerodynamics
from the highly rarefied free-molecular flow regimes to slightly rarefied continuum ones.

\section{COLLISIONAL OPERATOR MODELING}

%%% Boltzmann equation
Evolution of the VDF \(f(\bx,\bxi, t)\) is governed by the Boltzmann equation
\begin{equation}\label{eq:Boltzmann}
    \pder[f]{t} + \xi_i\pder[f]{x_i} = \frac1k J(f),
\end{equation}
where \(J(f)\) is the collisional operator, \(\Kn=2k/\sqrt\pi\) is the Knudsen number,
\(\bx\), \(\bxi\), and \(t\) are the physical, velocity, and time coordinates, respectively.

%%% Collisional term modeling
The original Boltzmann collision integral
\begin{equation}\label{eq:ci}
    J_B(f) = \int_{\mathbb{R}^3\times S^2} (f'f'_* - ff_*) B \dd \Omega(\boldsymbol{\alpha}) \dxi_*,
\end{equation}
where \(B(|\alpha_i V_i|/V,V)\) is the collision kernel, \(\dd \Omega(\boldsymbol{\alpha})\) is an element of solid angle
in the direction of the unit vector \(\boldsymbol{\alpha}\), determining the velocities after collision:
\begin{equation}\label{eq:after_collision}
    \xi_i' = \xi_i + \alpha_i\alpha_j V_j, \quad \xi_{i*}' = \xi_{i*} - \alpha_i\alpha_j V_j, \quad
    \boldsymbol{V} = \bxi_*-\bxi, \quad V = |\boldsymbol{V}|,
\end{equation}
and the following notation is used:
\begin{equation}\label{eq:Boltzmann_notation}
    f = f(\bx,\bxi,t), \quad f_* = f(\bx,\bxi_*,t), \quad
    f' = f(\bx,\bxi',t), \quad f'_* = f(\bx,\bxi'_*,t),
\end{equation}
with the variable hard sphere (VHS) kernel~\cite{Bird1981vhs}
\begin{equation}\label{eq:ci_kernel_vhs}
    B_\mathrm{VHS} = \frac{|\alpha_i V_i|}{4\sqrt{2\pi}\Gamma(2.5-\omega)}\left(\frac{V}2\right)^{1-2\omega},
\end{equation}
where \(\omega\) is the temperature exponent of the viscosity coefficient,
\(\Gamma(z) = \int_0^\infty t^{z-1}e^{-t}\dd{t}\) is the gamma function,
can be modeled by the Shakhov collisional operator
\begin{equation}\label{eq:shakhov}
    J_S(f) = \frac{\rho T^{1-\omega}}{\gamma_1}\left(f^{\mathrm{(eq)}} - f\right), \quad
    f^{\mathrm{(eq)}} = f_M\left[ 1 + \frac{1-\Pr}5\frac{4c_i q_i}{pT}\left(\frac{c^2}{T}-\frac52\right) \right], \quad
    f_M = \frac{\rho}{(\pi T)^{3/2}}\exp\left(-\frac{c^2}{T}\right),
\end{equation}
where \(c_i = \xi_i - v_i\), \(c = |\boldsymbol{c}|\), \(\Pr=\gamma_1/\gamma_2\) is the Prandtl number,
\(\gamma_1\) and \(\gamma_2\) are the viscosity and thermal conductivity coefficients for a hard-sphere gas~\cite{Sone2007}:
\begin{equation}\label{eq:gammas}
    \gamma_1 = 1.270042427, \quad \gamma_2 = 1.922284066.
\end{equation}
Macroscopic variables are calculated as follows:
\begin{equation}\label{eq:macro}
    \rho = \int f \dxi, \quad
    \rho v_i = \int \xi_i f \dxi, \quad
    p = \rho T = \frac23\int c^2 f \dxi, \quad
    q_i = \int c_i c^2 f \dxi,
\end{equation}
where \(\rho\) is the density, \(v_i\) is the velocity, \(T\) is the temperature, \(p\) is the pressure,
and \(q_i\) is the heat-flux vector.

\section{COLLISIONAL OPERATOR REGULARIZATION}

%%% Symmetrized collision operator
The Tcheremissine method is started from the symmetrized Boltzmann collision integral
\begin{equation}\label{eq:symm_ci}
    J(f_\gamma) = \frac14\int_{\mathbb{R}^6\times S^2} \left(
        \delta_\gamma + \delta_{*\gamma} - \delta'_\gamma - \delta'_{*\gamma}
    \right) (f'f'_* - ff_*)B \dd\Omega(\boldsymbol{\alpha}) \dxi\dxi_*,
\end{equation}
where \(f_\gamma = f(\bxi_\gamma)\),
\(\delta_\gamma = \delta(\bxi-\bxi_\gamma)\) is the Dirac delta function in \(\mathbb{R}^3\)
and the same notation~\eqref{eq:Boltzmann_notation} is assumed for it.
If the continuum velocity space \(\mathbb{R}^3\) is discretized
into net \(\mathcal{V} = \Set{\bxi_\gamma}{\gamma\in\Gamma\subseteq\mathbb{N}}\),
all the primed variables in~\eqref{eq:symm_ci} require regularization,
since the post-collisional velocities, \(\bxi'_\nu\) and \(\bxi'_{*\nu}\),
are included in \(\mathcal{V}\) only for a very small set of the collisional parameters~\cite{Palczewski1997}.

%%% Projection--interpolation regularization
Regularization of~\eqref{eq:symm_ci} can be, generally, performed by the \emph{projection} procedure:
\begin{equation}\label{eq:projection}
    \delta'_\gamma = \sum_{\sigma\in\Gamma} a_\sigma \delta(\bxi_\sigma-\bxi_\gamma) = a_\sigma \delta_{\sigma\gamma}, \quad
    \delta'_{*\gamma} = \sum_{\sigma\in\Gamma} a_{*\sigma} \delta(\bxi_\sigma-\bxi_\gamma) = a_{*\sigma} \delta_{\sigma\gamma},
\end{equation}
and \emph{interpolation} one, written as a weighted Kolmogorov mean:
\begin{equation}\label{eq:interpolation}
    f' = \phi^{-1}\left( \sum_{\sigma\in\Gamma} b_\sigma \phi(f_\sigma) \right), \quad
    f'_* = \phi^{-1}\left( \sum_{\sigma\in\Gamma} b_{*\sigma} \phi(f_\sigma) \right).
\end{equation}
The conservation properties can be guaranteed on the microscopic level (for every \(\gamma\in\Gamma\))
if \(a_\sigma\) and \(a_{*\sigma}\) are satisfy
\begin{equation}\label{eq:conservation}
    \int_{\mathbb{R}^3} \psi_s(\bxi_\gamma) \left(
        \delta'_\gamma + \delta'_{*\gamma} - a_\sigma \delta_{\sigma\gamma} - a_{*\sigma} \delta_{\sigma\gamma}
    \right) \dxi_\gamma = 0, \quad s = 0,\dots,4,
\end{equation}
where set \(\psi_s\) contains all collisional invariants:
\begin{equation}\label{eq:invariants}
    \psi_0 = 1, \quad \psi_i = \xi_i, \quad \psi_4 = |\bxi|^2.
\end{equation}
Set \(\mathcal{S} = \Set{a_\sigma\neq0}{\sigma\in\Gamma}\) is called the \emph{\(|\mathcal{S}|\)-point projection stencil}.
1-point projection to the nearest node, used in the classical Nordsieck--Yen--Hicks (HYN) method~\cite{Nordsieck1966, Yen1984},
yields conservation of mass only.
\citet{Tcheremissine1998} proved that the 2-point projection is sufficient for momentum and energy as well,
when \(\mathcal{V}\) is a uniform rectangular grid.
Five points are needed in the general case if requirement~\eqref{eq:conservation} is replaced to the more strict one~\cite{Varghese2007}:
\begin{equation}\label{eq:conservation2}
    \int_{\mathbb{R}^3} \psi_s(\bxi_\gamma) \left( \delta'_\gamma - a_\sigma \delta_{\sigma\gamma} \right) \dxi_\gamma = 0, \quad
    \int_{\mathbb{R}^3} \psi_s(\bxi_\gamma) \left( \delta'_{*\gamma} - a_{*\sigma} \delta_{\sigma\gamma} \right) \dxi_\gamma = 0.
\end{equation}
The entropic properties can be achieved also on the microscopic level
if the geometric mean with corresponding projection weights is taken
in~\eqref{eq:interpolation}~\cite{Tcheremissine2006, Dodulad2013}:
\begin{equation}\label{eq:geometric_mean}
   \phi(x) = \log(x), \quad b_\sigma = a_\sigma, \quad b_{*\sigma} = a_{*\sigma}.
\end{equation}

\section{COMPUTATIONAL METHODS}

\begin{figure}
    \centering
    \includegraphics[width=\FigWidth\linewidth, clip=true, trim = 90 30 75 50 mm]%
        {cylinder-90km/mach-dsmc}
    \caption{Contours of Mach number for \(\Ma=2.4\), \(\Kn=0.13\).
        Upper and lower halves correspond to the different solvers.}
    \label{fig:90:dsmc}
\end{figure}
\begin{figure}
    \centering
    \includegraphics[width=\FigWidth\linewidth, clip=true, trim = 90 30 75 50 mm]%
        {cylinder-90km/mach-gkua}
    \caption{Contours of Mach number for \(\Ma=2.4\), \(\Kn=0.13\).
        Upper and lower halves correspond to the different solvers.}
    \label{fig:90:gkua}
\end{figure}

%%% Splitting scheme and discretization in the physical space
The both solvers employed for the present comparative study solve the kinetic equation
by the second-order operator-splitting scheme into the collisionless transport equation and space-homogeneous Boltzmann equation.
The Tcheremissine method is supplemented by the finite-volume method to preserve the conservation properties
over the whole computational domain, while the GKUA is based on the finite-difference one.

%%% Discretization in the velocity space
Within the DV approximation, the three-dimensional integration with respect to the molecular velocity \(\bxi\)
is approximated as
\begin{equation}\label{eq:xi_cubature}
    \int_{\mathbb{R}^3} F(\bxi) \dxi \approx \sum_{\gamma\in\Gamma} F_\gamma w_\gamma, \quad
    F_\gamma = F(\bxi_\gamma),
\end{equation}
where \(F\) is an arbitrary function, \(\Gamma\) is some index set of the velocity grid,
\(\Set{(\bxi_\gamma,w_\gamma)}{\gamma\in\Gamma}\) is the set of discrete velocities and corresponding weights,
determined by the specific three-dimensional cubature.
For the supersonic flows considered below, the Gauss--Hermite quadrature is used for each axis.

%%% 8D cubature
The eight-dimensional cubature rule is needed for the evaluation of~\eqref{eq:symm_ci}.
For the present study, the optimal Korobov lattice rules~\cite{Dick2013},
randomly shifted for each time step, is employed.
In order to achieve a high accuracy for the steady state, time averaging is used.

\section{NUMERICAL EXAMPLES}

%%% Formulation of the problem
A steady-state flow around a cylinder is considered under two different conditions corresponding to the free fall at 90 and 70 km altitude.
The complete-diffusion boundary condition is assumed on the boundary surface,
which temperature is assumed to be constant and equal to the stagnation temperature
\begin{equation}\label{eq:stagnation}
    T_B = 1 + \frac{\gamma-1}2 \Ma^2.
\end{equation}
VHS model with \(\omega=0.81\) is used to simulate an argon flow.

%%% Velocity grids
The cylinder axis is directed along \(z\).
The velocity space is discretized to the rectangular grid, which nodes are located as zeros
of the Hermite polynomials rescaled to interval \(|\xi_i| < \xi_\mathrm{max}\).
(40, 40, 16) nodes are used for the corresponding axis.
Cutting the grid by sphere of radius \(\xi_\mathrm{max}\) gives \(|\mathcal{V}| = 25936\).
\(\xi_\mathrm{max}=7.5\) and \(\xi_\mathrm{max}=8.5\) used for 90 and 70 km, respectively.
Finally, note that the relaxation models allow to reduce the considered 5-dimensional problem to the 4-dimensional one.

\subsection{Cylinder at 90 km altitude}

When the Knudsen number is close to unity, the considered problem can be easily solved by the DSMC method.
Despite the visible statistical noise, there is an almost perfect coincidence between solutions obtained by the Bird's solver
and Tcheremissine method in Fig.~\ref{fig:90:dsmc}.
The GKUA solution is also very close to the reference one (Fig.~\ref{fig:90:gkua}),
however, has a longer shock tail, which is a well-known defect of the relaxation models, including the Shakhov one.

\subsection{Cylinder at 70 km altitude}

\begin{figure}
    \centering
    \includegraphics[width=\FigWidth\linewidth, clip=true, trim = 90 30 75 50 mm]%
        {cylinder-70km/mach-gkua}
    \caption{Contours of Mach number for \(\Ma=3.0\), \(\Kn=0.0055\).
        Upper and lower halves correspond to the different solvers.}
    \label{fig:70:gkua}
\end{figure}
\begin{figure}
    \centering
    \includegraphics[width=\FigWidth\linewidth, clip=true, trim = 90 30 75 50 mm]%
        {cylinder-70km/uz-gkua}
    \caption{Contours of \(v_z\) for \(\Ma=3.0\), \(\Kn=0.0055\).
        Upper and lower halves correspond to the different solvers.}
    \label{fig:70:U_z}
\end{figure}

A DSMC solution becomes too expensive for small Knudsen numbers;
therefore, the presented comparison is limited to Fig.~\ref{fig:70:gkua}.
As for the previous case, there is a good agreement between the results,
along with the same remark about the shock approximation.

Fig.~\ref{fig:70:U_z} illustrates the accuracy obtained by the Tcheremissine method.
Since integral~\eqref{eq:symm_ci} is computed approximately,
the deviation from the exact solution \(v_z=0\) can be served as an indirect indicator of the numerical error.
The largest residual is observed in the center of the shock;
however, there are also several local maxima in the wake region.
All of them can be reduced by the appropriate mesh refinement~\cite{Kolobov2013}.
Note that the time-averaged value is shown in Fig.~\ref{fig:70:U_z},
which is smaller than the corresponding value after a single iteration.

\section{CONCLUDING REMARKS}

In summary, the following points can be emphasized:
\begin{itemize}
    \item The Shakhov model with the appropriate temperature exponent for the viscosity and thermal conductivity
    is a reliable engineering-level model for the re-entering aerodynamics,
    including the hypersonic flows~\cite{Titarev2018, Frolova2018}.
    \item The Tcheremissine regularization method is a promising framework for robust DV algorithms and high-accuracy simulations.
    \item Stochastic computational techniques can improve pure deterministic methods.
    \item The conservation properties are important for crude meshes and internal flows.
    Lack of them is not the issue for external flows.
    \item The ray effect can noticeably distort a solution in the wake region near the boundary.
    Refinement of the velocity grid near the origin helps to reduce the relevant numerical error.
\end{itemize}

\section{ACKNOWLEDGMENTS}
The work was supported by Russian Foundation for Basic Research (Grants 18-01-00899, 18-07-01500),
Chinese 973 Program (2014CB744100), NSFC (No. 91530319).

% References

\bibliographystyle{aipnum-cp}%
\bibliography{manuscript}%


\end{document}
