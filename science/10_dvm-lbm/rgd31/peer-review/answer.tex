\documentclass{article}
\usepackage{a4wide}
\usepackage[english]{babel}

%%% functional packages
\usepackage{amssymb, mathtools, amsthm, mathrsfs}
\usepackage{framed}                % for leftbar
\usepackage[%
    font={bfseries},
    leftmargin=0.1\textwidth,
    indentfirst=false
]{quoting}

\usepackage{lipsum}
\usepackage{graphicx}
\def\asterism{\par\vspace{1em}{\centering\scalebox{1}{\bfseries *~*~*}\par}\vspace{.5em}\par}

\usepackage[
    backend=biber,
    style=numeric,
    sorting=none,
    maxbibnames=99, minbibnames=99,
    natbib=true,
    url=false,
    eprint=true,
    pagetracker,
    giveninits]{biblatex}
\bibliography{../../dvm-lbm}

\title{Response to the Reviewers' Comments}

\begin{document}

\maketitle

We thank the Reviewers for the comments and suggestions.
Our answer and changes are presented below.
The reviewers' words are shown in bold.
Individual parts of the corrected version of the manuscript
can be recognized by the bold lateral line.

\section*{Referee report \#1}

\begin{quoting}
    1. There are a lot of literatures on the coupling of the kinetic method and the LB method.
    The authors should make a more comprehensive review on the existing literatures on this topic.
    The comparison and the comments should also be made.
\end{quoting}

As far as the authors are aware, these is no examples of coupling methods (based on the domain-decomposition approach)
of the deterministic models of the Boltzmann equation and LB models in the literature.
We would be very grateful if the reviewer provides us with the correspoding citations.
Our awareness are represented now by the following paragraph added in the introduction.

\begin{leftbar}
    ... This fundamental affinity suggests us to propose new variants of the hybrid schemes
    based on the domain decomposition approach and use the best properties of the both methods for rarefied and continuum regions.
    The kinetic mapping scheme for coupling the different order LB models is presented in~\cite{Meng2011},
    while the possibility of merging the DV and LB methods was noticed in~\cite{Succi2016}.
\end{leftbar}

\begin{quoting}
    2. In the second equation below the line ``for the DV difference scheme on the Grad...'',
    there is a typo: \(H_j\) should be \(H_{\alpha}\).
\end{quoting}

Corrected.

\begin{quoting}
    3. As the author pointed out, in Fig.~1a, there is some discrepancy between the DV and the benchmark profiles.
    The local refinement should be made to eliminate this discrepancy.
\end{quoting}

We decided to restrict ourselves to the case of the uniform velocity grid within the current conference paper.
In the same time, we have already extended our results for the nonuniform grid with local mesh refinement in the full paper.
Therefore, the following remark is added in the considered paragraph.
\begin{leftbar}
    The corresponding results for nonuniform velocity grid are presented in~\cite{Aristov2019}.
\end{leftbar}

\section*{Referee report \#2}

\begin{quoting}
    It is known that BGK fails to give a correct Prandtl number, which is about 2/3 for monatomic gas, but BGK gives 1.
    Meanwhile, the LBM can give a good Prandtl number.
    When Kn is not large, whether the Prandtl number is correct really matters.
    So it may be not a good choice to combine BGK and LBM.
    An option is Shakhov or ES-BGK equation.
    Based on the authors' current work, it would be not hard to extend the hybrid method of Shakhov and LBM.
\end{quoting}

We thank the reviewer for the suggestion and believe that more complicated LB schemes, for instance,
the thermal models with correct half-fluxes~\cite{Ambrus2016mixed}, can be also implemented,
however such kind of investigation is beyond the current topic of the manuscript.

The idea of the extension of the presented scheme is based on the change of the Hermite basis by the Laguerre basis.
Assume that the matching point for the two methods (BGK or S-BGK and LB) the gas distribution function can be expressed
in the form of the finite Laguerre expansions up to the moments of some order which is sufficient
for the reproduction of thermal properties of the gas.

Having the BGK VDF, the projection step (from BGK to LB) can be performed in two steps:
\begin{enumerate}
    \item The Laguerre expansion coefficients are calculated from the BGK DV changing integrals by sums, as a result
    the Laguerre expansion df is recovered.
    \item The Gauss--Laguerre quadratures are applied and appropriate boundary LB distribution is derived.
\end{enumerate}
Now assume that LB distribution is known and now the reconstruction of the BGK DV distribution function is required.
The coefficients of the Laguerre expansion are calculated using LB scheme
(simply as a sum of LB VDF weighted by Laguerre polynomials over the lattice velocity nodes),
and as a result the Laguerre expansion function is recovered.
Next, this function is discretized at the velocity nodes of the BGK scheme.

It is worth to mention that the discussed tailoring scheme is slightly more complicated than in the paper
since it requires more moments to be viable.

As for C

\begin{quoting}
    In Figure 2, there is a discontinuity in the profile of \(q_x\), which is caused by the boundary condition of the interface.
    The authors claim that its amplitude is proportional to the non-equilibrium part of the VDF. It may be true.
    But the fact is that, the two model used in this paper are not consistent.
    For example, they give different Prandtl number.
    If the VDF is not very close to the equilibrium, there is an error between these two models.
    But if the VDF is close to the equilibrium, it is a trivial case.
    The authors might be interesting in choosing a better model to suppress the oscillation.
\end{quoting}

The reproduction of \(q_x\) can be challenging for any LB method
since this heat flux is significantly non-equilibrium and has non-hydrodynamic nature.
Thus, our explanation is that the jump in \(q_x\) is not attributed to the Prandtl mis-reproduction
but reflects the fact that LB can not reproduce the beyond-Navier--Stokes regime.
We have extended discussion on this problem:

\begin{leftbar}
    Due to the diffuse-reflection boundary condition, there is a discontinuity of the VDF on the boundary,
    which decays monotonically and faster than any inverse power of distance from the boundary.
    Therefore, all the breakdown parameters reach their maximum on the boundary;
    however, \(E_\infty\) relaxes in a nonmonotonic way.
    This is probably due to crude approximation of the sharp variations of the VDF in the Knudsen layer.
    For the D3Q96 model, \(E_\infty\) noticeably exceeds \(E_{1,2}\) (Fig.~3b),
    which can be explained by its peculiar properties minimizing the wall moment errors.
    The hermite-based coupling induces oscillations (Fig.~3c,~3d),
    since it is unable to reconstruct nonequilibrium part of the VDF.
    The sharp drop in Fig.~3c indicates that the coupling interface is too close to the boundary,
    while the smoother transition in Fig.~3d can be considered as more acceptable.
\end{leftbar}

Finally, all LB models considered in the manuscript are one-parameter single-relaxation models;
therefore, they can have only the same Prandlt number as the BGK equation
and be consistent with it in this sense.

% As for Couette problem, the ratio between the terms related to viscosity and thermal conductivity
% (as determined by the Prandtl number) is not so important.
% Indeed, in this problem the main role is played by viscous terms, and the heat transfer is of little interest.
% Here in the BGK model equation the frequency is expressed through the correct viscosity coefficient.
% Therefore, it can be expected that the processes of transfer of viscous stresses are described correctly.

\printbibliography

\end{document}
