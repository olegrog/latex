\documentclass{article}
\usepackage{a4wide}
\usepackage[english]{babel}

%%% functional packages
\usepackage{amssymb, mathtools, amsthm, mathrsfs}
\usepackage{framed}                % for leftbar
\usepackage[%
    font={bfseries},
    leftmargin=0.1\textwidth,
    indentfirst=false
]{quoting}

\usepackage{lipsum}
\usepackage{graphicx}
\def\asterism{\par\vspace{1em}{\centering\scalebox{1}{\bfseries *~*~*}\par}\vspace{.5em}\par}

\usepackage[
    backend=biber,
    style=numeric,
    sorting=none,
    maxbibnames=99, minbibnames=99,
    natbib=true,
    url=false,
    eprint=true,
    pagetracker,
    giveninits]{biblatex}
\bibliography{../dvm-lbm}

\title{Response to Reviewers' Comments}

\begin{document}

\maketitle

We are grateful to the Reviewers for their very fruitful comments and suggestions.
Our answer and changes are presented below.
The reviewers' words are shown in bold.
Individual parts of the corrected version of the manuscript
can be recognized by the bold lateral line.

\section*{Referee report \#1}

\begin{quoting}
    1. It is not clear to me what this paper brings in addition to the work of Di Staso et al.
    Please, be specific on this point.
\end{quoting}

...

insert answers here

\begin{leftbar}
    ...

    insert quotes from the manuscript here
\end{leftbar}

\begin{quoting}
    2. It would be very useful if the authors could comment a bit more specifically
    on the potential competitivity with DSMC for the problems they mention in the conclusions
    (supersonic, highly thermal, \dots because this is the real issue at stake).
\end{quoting}

...

\begin{quoting}
    3. Please take a look at recent work of Aiguo Xu (soft matter, PRE) on DVM for
    thermal and multiphase non-equil flows. I think they belong and should be cited.
\end{quoting}

...

\section*{Referee report \#2}

\begin{quoting}
    1. My greatest concern is related to the idea of coupling
    DVM with LB. Especially in the context of Couette flow,
    there are already published pure LB implementations which
    are significantly more efficient and even more accurate
    than the DVM employed in this paper. Thus, further
    motivation should be provided for introducing this scheme,
    perhaps by identifying a fundamental flaw of the LB
    approach which mandates the use of DVM.
\end{quoting}

...

\begin{quoting}
    2. The coupling between the LB and DVM regions is not
    smooth, suggesting that the coupling procedure is not
    fully compatible with the idea of this multi-solver.
    In particular, is there any reason why the series in
    Eq. (13) is truncated at third order with respect to
    the Hermite polynomials? Can this coupling be improved?
    If it can, I recommend that a definitive implementation
    of the coupling is provided in this manuscript; otherwise,
    a convincing proof / argument should be provided why this
    coupling cannot be improved.
\end{quoting}

...

\begin{quoting}
    3. The numerical results indicate that the ``special''
    7th order D3Q96 model performs better than the
    ``non-augmented'' models, even than the 9th order
    D3Q121 model. This improvement can be attributed to
    the half-range capabilities of the augmented models.
    Page 10 contains the remark that ``increasing order of
    the LB model [...] failed to describe its high-order
    relaxation correctly.'' - an important addition should
    be made here, namely that the higher order model is
    ``non-augmented.'' Does the quoted sentence hold when
    a higher order model which is augmented is employed?
\end{quoting}

...

\begin{quoting}
    4. Fig. 5 indicates that the deviation norms are
    non-monotonous in the vicinity of the coupling
    interface - more strikingly on the DVM side of this
    interface. The interface essentially provides boundary
    conditions for the DVM, which are based on the
    full-range Hermite expansion of the Boltzmann
    distribution, given in Eq. (13). It can be seen that
    the augmented model provides a smoother transition than
    the non-augmented ones - this may be due to the
    half-range capabilities of the former. If this is
    the case, can a mapping at the level of a half-range
    expansion of the distribution be more appropriate?
\end{quoting}

...

\begin{quoting}
    5. In this paper, the advection is computed using
    the MCD flux limiter scheme. Is there any reason why
    a linear (first order) extrapolation given in
    Eq. (31) is preferred for the computation of the
    flux at the wall interface? Are there other viable
    extrapolation schemes available?
\end{quoting}

...

\begin{quoting}
    6. The numerical scheme presented in Sec. 5.1 is
    second order by design. Is this order preserved
    when boundary conditions are imposed? In particular,
    can second order convergence be proven with respect
    to the benchmark solution?
\end{quoting}

...

\begin{quoting}
    7. At the coupling, it can be seen that the
    macroscopic profiles exhibit jumps. Are the
    amplitudes of these jumps affected by the spatial
    discretisation? If they remain finite as the
    mesh is refined, the stationary state displays
    a standing discontinuity, which clearly is not
    present in the solution of the kinetic equation -
    this point should be addressed. Also, the stability
    of the scheme should be discussed in this context,
    e.g. by checking if there exists a maximum
    resolution after which the scheme becomes unstable.
\end{quoting}

...

\begin{quoting}
    8. Results should be presented for larger values of
    $k$. The numerical results indicate that the
    discontinuity between the LB and DVM regions increases
    with $k$. Is there a maximum value of $k$ for which
    the scheme is stable?
\end{quoting}

...

\begin{quoting}
    9. Can the scheme be employed for higher Mach number flows?
\end{quoting}

...

\begin{quoting}
    10. There are insufficient details regarding the scheme
    the results to be reproduced - in particular, it is not
    clear how the flux rescaling in Eq. (34) works, i.e. on
    which interfaces it is applied, and whether this affects
    the order of the numerical scheme. In my opinion, the
    readers will benefit from a short summary of the main
    steps, while ensuring that the scheme can be implemented
    based on the details provided in the manuscript.
\end{quoting}

...

\begin{quoting}
    11. Results obtained in the linearised regime are
    reported for the longitudinal heat flux. Since this
    quantity is not covered in Ref.~\cite{Luo2016}, an important
    appendix should be added, detailing the main steps
    taken in deriving these results.
\end{quoting}

...

\printbibliography

\end{document}
