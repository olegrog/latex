\documentclass[]{elsarticle} %review
\usepackage[english]{babel}

%%% functional packages
\usepackage{amssymb, mathtools, amsthm, mathrsfs}
\usepackage{subcaption}
\usepackage{graphicx}
\graphicspath{{pics/}}

%%% configuration packages
\usepackage{fullpage}
\usepackage{indentfirst}
%\usepackage{comment}

\usepackage{xcolor}
\newcommand{\todo}[1]{\textcolor{olive}{#1}}

\usepackage{lineno}
\modulolinenumbers[5]

\usepackage[
    colorlinks,
]{hyperref}

\journal{Journal of \LaTeX\ Templates}

%%%%%%%%%%%%%%%%%%%%%%%
%% Elsevier bibliography styles
%%%%%%%%%%%%%%%%%%%%%%%
%% To change the style, put a % in front of the second line of the current style and
%% remove the % from the second line of the style you would like to use.
%%%%%%%%%%%%%%%%%%%%%%%

%% Numbered
%\bibliographystyle{model1-num-names}

%% Numbered without titles
%\bibliographystyle{model1a-num-names}

%% Harvard
%\bibliographystyle{model2-names.bst}\biboptions{authoryear}

%% Vancouver numbered
%\usepackage{numcompress}\bibliographystyle{model3-num-names}

%% Vancouver name/year
%\usepackage{numcompress}\bibliographystyle{model4-names}\biboptions{authoryear}

%% APA style
%\bibliographystyle{model5-names}\biboptions{authoryear}

%% AMA style
%\usepackage{numcompress}\bibliographystyle{model6-num-names}

%% `Elsevier LaTeX' style
\bibliographystyle{elsarticle-num}
%%%%%%%%%%%%%%%%%%%%%%%

% general
\newcommand{\Kn}{\mathrm{Kn}}
\newcommand{\dd}{\mathrm{d}}
\newcommand{\pder}[2][]{\frac{\partial#1}{\partial#2}}
\newcommand{\pderdual}[2][]{\frac{\partial^2#1}{\partial#2^2}}
\newcommand{\pderder}[3][]{\frac{\partial^2#1}{\partial#2\partial#3}}
\newcommand{\Pder}[2][]{\partial#1/\partial#2}
\newcommand{\Pderdual}[2][]{\partial^2#1/\partial#2^2}
\newcommand{\Pderder}[3][]{\partial^2#1/\partial#2\partial#3}
\newcommand{\Set}[2]{\{\,{#1}:{#2}\,\}}
\newcommand{\OO}[1]{O(#1)}
\newcommand{\transpose}[1]{#1^\mathsf{T}}
\DeclarePairedDelimiter\autobracket()       % mathtools are needed
\newcommand{\br}[1]{\autobracket*{#1}}

% topic-specific
\newcommand{\dxi}{\boldsymbol{\dd\xi}}
\newcommand{\bxi}{{\boldsymbol{\xi}}}
\newcommand{\bxia}{\bxi_\alpha}
\newcommand{\dx}{\boldsymbol{\dd{x}}}
\newcommand{\bx}{\boldsymbol{x}}
\newcommand{\xiai}{\xi_{\alpha i}}
\newcommand{\xiaj}{\xi_{\alpha j}}
\newcommand{\xiak}{\xi_{\alpha k}}
\newcommand{\cai}{c_{\alpha i}}
\newcommand{\caj}{c_{\alpha j}}
\newcommand{\equil}[1]{#1^\mathrm{(eq)}}
\newcommand{\refer}[1]{#1_0}

\begin{document}

\begin{frontmatter}

\title{Multiscale hybrid approach based on the discrete-velocity and lattice-Boltzmann methods under the finite-volume formulation}

\author[ccas]{V.V.~Aristov}

\author[ccas]{O.V.~Ilyin}
\cortext[mycorrespondingauthor]{Corresponding author}
\ead{oilyin@gmail.com}

\author[skoltech,ccas]{O.A.~Rogozin}

\address[ccas]{Dorodnicyn Computing Center,
    Federal Research Center "Computer Science and Control" of Russian Academy of Science, Moscow, Russia}
\address[skoltech]{Center for Design, Manufacturing, and Materials,
    Skolkovo Institute of Science and Technology, Skolkovo, Russia}

\begin{abstract}

A novel hybrid computational approach based on the discrete velocity (DV) approximation
including the lattice-Boltzmann (LB) technique is proposed.
Numerical schemes for kinetic equations are used in regions of rarefied flows and LB schemes are used in continuum flow zones.
Expansion to the Hermite polynomials is used for coupling of DV and LB solutions.
A special attention is paid to the recent high-order LB models.
Additional correction procedures are applied to ensure conservative properties.
The plane Couette flow is investigated as a test problem.
Influence of the breakdown criterion to accuracy and efficiency of the proposed hybrid method is studied.

\end{abstract}

\begin{keyword}
kinetic--kinetic coupling, discrete-velocity method, lattice-Boltzmann method, rarefied and continuum flows, hybrid methods
\end{keyword}

\end{frontmatter}

\linenumbers

\section{Introduction}\label{sec:intro}

So far, a problem of effective numerical simulation of complex flows is challenging despite the efforts of many researchers. This is due, in particular, to a multiscale complicated structure of a flow where there are zones of equilibrium and nonequilibrium states. The use of the kinetic equation in all regions is very computationally consuming. On the other hand, the modern approaches including the LB method is a profitable tool for description near-equilibrium flows, but it is not adequate for regions where the distribution function is not close to Maxwellian and  the contribution of highest moments can not be ignored. So the computational hydrodynamics is being confronted with an issue of constructing the unified numerical method suited for calculations in all regions of flows.

One of the most effective methods for modeling multiscale flows is the construction of hybrid schemes. Some methods include the solution of both kinetic and hydrodynamic equations in the entire spatial region, and the transport coefficients for the hydrodynamic equations are calculated on the basis of the distribution function. The next type includes methods with dividing the physical space region into kinetic and continual subregions using some criterion of domain decomposition \cite{Dimarco2014}. The fluid-kinetic coupling is an effective approach for the description of multiscale flows. The coupling of the Boltzmann and Euler (or NS) equations is one of canonical example of such hybrid schemes (see, e.g., \cite{Bourgat1996} \cite{Tallec1997}) and has been also elaborated and applied within the framework of UFS (Unified Flow Solver) \cite{Kolobov2007}. In the last method the kinetic oriented schemes for approximation of continuum equations are used.

We consider very briefly development and realization of these important ideas. The kinetic-based description of continuum media has been suggested independently and has been widely used since the beginning of 80s, see~\cite{Potkin1975, Pullin1980, Reitz1981, Aristov1983}. Later these kinetic-consistent schemes have been developed in~\cite{Elizarova1985, Deshpande1986, Prendergast1993, Chou1997, Ohwada2004Xu, Ohwada2004Kobayashi, Ohwada2006}. Such methods reproduce the Euler and Navier–Stokes (NS) dynamics. The cellular-automata approximation for the Navier–Stokes equations has been developed in the middle of 80s (see e.g.~\cite{Frisch1986}). Finally, the lattice-gas model based on the BGK equation was proposed in the beginning of 90s (see e.g.~\cite{Qian1992}). It gave rise to a wide class of mentioned numerical methods called LB methods (see e.g.~\cite{Succi2001}).

It is  worth to emphasize that LB method is based, in fact, on some models of DV method. There are obvious relations between these two methods. For instance, one can cite a phrase from \cite{Rivet2001}: ''This type of discrete kinetic theory can be seen as the ''ancestor'' of the lattice gas approach''. The LB method is genetically related to the Broadwell-type models~{\cite{Broadwell1964shock, Gatignol1975}, which use small numbers of discrete velocities to caricature the Boltzmann dynamics. Thus the DV approximation is a natural basis for construction a hybrid fluid--kinetic model.

The  mapping scheme for the mapping of the solutions for low order
and  high order LB models is  presented in \cite{Meng2011}, while the possibility of DVM and LB merging  was noticed in \cite{Succi2016}.
The methods of DSMC type in the kinetic zones and the Euler or Navier-Stokes equations in the continuum are well developed  now, but the usual statistical modeling in the buffer zone yields some statistical noise especially for subsonic flows, which complicates calculations. This hybrid approach has been suggested in~\cite{Staso2016short, Staso2016long}  and in the recent paper \cite{Staso2018} where the solution is obtained by means of hybridization of the DSMC and LB methods.

Unlike DSMC, the methods of direct numerical solution of the Boltzmann equation do not give a statistical noise of macroscopic parameters. Therefore, hybrid methods based on a direct numerical solution of the Boltzmann equation appear in principal to be more promising. The DV schemes with the large number of discrete velocities are used in direct methods for solving BE, BGK, S-model or other kinetic equations. The DV method is applied with the combination of Monte Carlo or quasi-Monte Carlo procedures for evaluating collision integrals and for computing the appropriate moments which are used in the collision integrals of the model kinetic equations. For the near-equilibrium zones the small number of discrete velocities can be considered. Therefore, one can expect that LB approaches are fit for describing flows in these regions.

A novel hybrid kinetic approach for multiscale problems is proposed in the present paper. We attempt to couple DV method for the Boltzmann equation (or its BGK model) and LB method for the Navier--Stokes equations. The first one adequately describes nonequilibrium regions, while the second one provides an adequate description in continuum-flow regions. At the boundaries of the BGK and LBGK domains the coupling method is applied. For mapping we assume that in the overlapping zone of the two methods the distribution function takes the form of truncated Grad expansion.  We employ several LB models including  high order lattices \cite{Shan2006,Feuchter2016}. The hybrid LB+DVM method is tested for the Plane Couette flow and the excellent correspondence with previous results was achieved. No any  additional  regularization procedures \cite{Latt2006, Mont2015} for LB method are needed since LB covers the spatial domain where the distribution function is close to equilibrium.
In the mapping domain the Lattice Boltzmann distribution function is equivalent to the Grad expansion via Gauss-Hermite quadrature. Similar approach of mapping has been suggested in~\cite{Staso2016short, Staso2016long, Staso2018}.

The plan of the present paper is as follows. The employed kinetic equations and nondimensional variables are introduced in Section~\ref{sec:equations}. The numerical methods are described in Section~\ref{sec:numerics}. This section is subdivided in several parts in which in particular the coupling procedure of the hybrid algorithm is presented. In Section~\ref{sec:results} examines the computations obtained by the BGK model in the whole region of the Couette flow, the analogous results by the different LB models and by the hybrid scheme. The efficiency of the hybrid approach is studied. In section~\ref{sec:summary} Conclusions and a perspective development of DV hybrid methods is discussed.

\section{Main equations}

In the present  paper we will consider  two  kinetic models based on the  nonlinear  Boltzmann equation which will be used in our further computations.
Our main object is the Bhatnagar-Gross-Krook (BGK) kinetic model \cite{Kogan1969} which reads  in the  dimensionless variables as
\begin{equation}\label{bgk}
\frac{\partial f}{\partial t}+\boldsymbol{\xi}\frac{\partial f}{\partial \mathbf{x}}=\frac{{1}}{\tau}(f^{eq}-f), \quad f^{eq}=\equiv\frac{\rho}{(2\pi T)^{3/2}}e^{-(\boldsymbol{\xi}-\mathbf{u})^2/2T},
\end{equation}
where  $f(t,\mathbf{x},\boldsymbol{\xi})$  is   distribution function of  a dilute  gas, $\tau$  is dimensionless  relaxation time (which equals  to average number of collisions per particle until relaxation) assumed  to be  constant  in the present study, $\rho,\mathbf{v},T$ are  density, bulk velocity and  temperature. We also denote  $p_{\alpha \beta}, \mathbf{q}$ as stress tensor and  heat flow.
All this quantities are defined as

\begin{equation}\label{eq:macro}
    \begin{gathered}
    \rho = \int f \dxi, \quad
    \mathbf{v} = \frac1{\rho} \int \boldsymbol{\xi} f \dxi, \quad
    T = \frac{2}{3\rho}\int(\boldsymbol{\xi}-\mathbf{v})^2f \dxi = \frac{p_{ii}}{3\rho}, \\
    p_{\alpha \beta} = 2 \int(\xi_{\alpha}-v_{\alpha})(\xi_{\beta}-v_{\beta}) f \dxi, \quad
    q_{\alpha} = \int(\xi_{\alpha}-v_{\alpha})(\boldsymbol{\xi}-\mathbf{v})^2 f \dxi
    \end{gathered}
\end{equation}


The Knudsen number \(\Kn\) can be expressed in terms of the reference gas viscosity \(\refer\mu\):
\begin{equation}\label{eq:Knudsen_number}
    k = \frac{\sqrt\pi}2\Kn = \frac{\refer\mu V}{\refer{p}L}.
\end{equation}



This   model is  derived from the  full Boltzmann equation  substituting the  collision integral term by the  local Maxwell distribution $f^{eq}$.  Formally the nonlinearity in $f^{eq}$ is  more severe than in the  full Boltzmann equation since  $f^{eq}$  depends  on $f$  via the moments $\rho,\mathbf{v},T$  but the BGK model  \eqref{bgk} is  simpler for the  numerical study. The BGK equation can adequately
describe strongly non-equilibrium effects in a dilute gas we will
solve this equation  using the second-order splitting scheme.


The  second  approach is  the  Lattice Boltzmann method which is the further simplification of the BGK model \cite{Succi2001}. We assume that the considered flow is isothermal,  slow,  i.e the  Mach number is  close  to zero, then we  can expand  the local Maxwell term.
in the Taylor series on the bulk velocity $\mathbf{v}$ (or to perform the expansion on the Hermite polynomials)  up to the terms of some finite order (at least second). Moreover, we assume that  the particle can travel with the  velocities $\mathbf{c}_{j}, j=1 \ldots N$ from a finite discrete set of possible velocities and  the values of absolute Maxwellian are  changed  by the lattice weights $w_j$.
Finally the general LB model is  obtained  by the finite difference integration of the BGK equation on the  characteristics,
the third order expansion on $\mathbf{v}$ yields the  following model
$$
f(t+\delta t, \mathbf{x}+\mathbf{c}_j\delta t, \mathbf{c}_j)-f(t, \mathbf{x},\mathbf{c}_j)=
$$
\begin{equation}\label{lbgk}
=\frac{1}{(\frac{1}{2}+\frac{\tau}{\delta t})}\left\{ \rho w_j\left(1+ \frac{\mathbf{c_j}\mathbf{v}}{c_s^2}+\frac{(\mathbf{c_j}\mathbf{v})^2-c_s^2v^2}{2c_s^4}
+\frac{(\mathbf{c}_j\mathbf{v})^3-3c_s^2 v^2(\mathbf{c}_j\mathbf{v})}{6c_s^6}\right)-f(t,\mathbf{x}, \mathbf{c}_j) \right \},
\end{equation}
for $j=1\ldots N$, where $\mathbf{c}_j$ are lattice velocities, $c_s$  is  sound velocity defined by $\sum_jw_j\mathbf{c}^2_j=c_s^2$, $\tau$  is  relaxation time, $\delta t$  is lattice  time step and  $N$  is number of lattice velocities.  The integration on  the  characteristics using the trapezium rule adds the small error terms of the  order  $\delta t^2$.


Several approaches can be applied for the  construction of the LB models like Gauss-Hermite \cite{he1997},\cite{shan1998}, \cite{Shan2006},\cite{shan2010} and the entropic  method \cite{karlin1999},\cite{chikatamarla2006},\cite{chikatamarla2009}.

We adopt the spatial domain splitting procedure  for solving  the  boundary problems  for  rarefied gas dynamics.
Commonly the  non-equilibrium effects in flows appear near boundaries, therefore the  boundary domains are modeled with the BGK model while the internal  zone which is modeled with the Lattice BGK. Previously, the  approach based on solving  the gas  dynamics with DSMC method and LBGK models using  domain splitting was presented in papers \cite{Staso2016short,Staso2016long,Staso2018} where  the Couette and  Poiseuille  flow  was  considered. Nevertheless, LBGK and DSMC coupling leads to appearance  of some  noise \cite{Succi2016} therefore the  introduction of the LBGK and  BGK coupling  method  is  very desirable.


\section{The  mapping  method}

We will introduce the coupling  procedure  in the spatial overlapping  zone  of the BGK and LB models.
First of all, we  assume that in this  domain  the real distribution function of the gas is close  to the Maxwell state with zero bulk velocity,  unit temperature and  can  be represented in the  form of the truncated  Grad expansion up to the third order terms on the velocity
\begin{equation}\label{grad}
f_{Grad}(\mathbf{x},\boldsymbol{\xi})=w(\boldsymbol{\xi})\left(a(\mathbf{y})
+\sum_{\alpha}a_{\alpha}(\mathbf{x})H_{\alpha}+\frac{1}{2}\sum_{\alpha\beta}a_{\alpha \beta}(\mathbf{x})H_{\alpha\beta}+\frac{1}{6}\sum_{\alpha\beta \gamma}a_{\alpha\beta \gamma}(\mathbf{x})H_{\alpha\beta \gamma}\right),
\end{equation}
 where $H_{\alpha}, H_{\alpha\beta}, H_{\alpha\beta \gamma}$ are the  Hermite polynomials of the first, second and third order defined by
  $$
H_\alpha(\boldsymbol{\xi})=\frac{(-1)}{w(\boldsymbol{\xi})}\frac{\partial}{\partial \xi_\alpha}w(\boldsymbol{\xi}),  \quad H_{\alpha\beta}(\boldsymbol{\xi})=\frac{1}{w(\boldsymbol{\xi})}\frac{\partial^2}{\partial \xi_\alpha\partial \xi_\beta}w(\boldsymbol{\xi}),
\quad H_{\alpha\beta \gamma}(\boldsymbol{\xi})=\frac{(-1)}{w(\boldsymbol{\xi})}\frac{\partial^3}{\partial \xi_\alpha\partial \xi_\beta \partial \xi_\gamma}w(\boldsymbol{\xi}),
$$
and
 $$
 w(\boldsymbol{\xi})\equiv \frac{1}{\sqrt{(2\pi})^3}\exp\left(-\frac{\boldsymbol{\xi}^2}{2}\right).
 $$
 The  terms $a, a_{\alpha},a_{\alpha\beta}, a_{\alpha\beta \gamma}$ are coefficients depending on $\mathbf{x}$ (the point in the overlapping domain).
 We will use the function \eqref{grad} for the transfer of the  data between the lattice Boltzmann  and the BGK models. We will discuss this  procedure  in turn.

For the sake of clarity we  assume  that the flow  depends  not on the whole vector $\mathbf{y}$ but on one of the coordinates only, we denote  it  by $y$.

At the  first  step we  update the DV distribution function $f_{DV}(y,\boldsymbol{\xi})$ for the discrete velocities  $\boldsymbol{\xi}_n$ such that $(\boldsymbol{\xi}_n,\mathbf{e})<0$ , where $\mathbf{e}$ is the outer normal to overlapping domain,  starting from the wall spatial nodes and  moving  towards the overlapping zone  (for instance, see Fig. 1 -- Fig. 8 in the case  of the Couette flow, the computations of the BGK scheme starts  at $y=0.5$).
In the spatial domain (physical domain) where  the DV method overlaps the LB method we map the DV distribution function for the DV difference scheme on the Grad distribution function by calculating the following coefficients $a, a_\alpha,a_{\alpha\beta}, a_{\alpha\beta \gamma}$
$$
a(y)=\sum_{m=1}^M  f_{DV}(y,\boldsymbol{\xi}_m),   \quad a_\alpha(y)=\sum_{m=1}^Mf_{DV}(y,\boldsymbol{\xi}_m)H_j(\boldsymbol{\xi}_m),
$$
$$
a_{\alpha\beta }(y)=\sum_{m=1}^Mf_{DV}(y,\boldsymbol{\xi}_m)H_{\alpha\beta}(\boldsymbol{\xi}_m), \quad
a_{\alpha\beta \gamma}(y)=\sum_{m=1}^Mf_{DV}(y,\boldsymbol{\xi}_m)H_{\alpha\beta \gamma}(\boldsymbol{\xi}_m),
$$
where $\boldsymbol{\xi}_m, m=1 \ldots M$ are velocities for DV  difference scheme.
The Grad  distribution function \eqref{grad} is recovered in the overlapping spatial domain.

Next we will map the velocity distribution \eqref{grad} on the lattice Boltzmann distribution using  the  Gauss-Hermite quadrature method.
The idea of the method is based on the fact  that the representation of the distribution function in the  Grad form  is equivalent to the  lattice  Boltzmann method \cite{he1997},\cite{shan1998},\cite{Shan2006}.
We  consider the first moments $a,a_{\alpha},a_{\alpha\beta}, a_{\alpha\beta \gamma}$  in the integral form and then calculate them using Gauss-Hermite quadratures
$$
\{ a,a_{\alpha},a_{\alpha\beta}, a_{\alpha\beta \gamma} \}=\int f_{Grad}(\boldsymbol{\xi})\{1,H_{\alpha}, H_{\alpha\beta} ,H_{\alpha\beta \gamma}\}(\boldsymbol{\xi})d\boldsymbol{\xi}=\sum_{j=1}^N w_j\frac{f_{Grad}(\boldsymbol{c}_j)}{w(\mathbf{c}_j)}
\{1,H_{\alpha}(\mathbf{c_j}),H_{\alpha\beta}(\mathbf{c_j}), H_{\alpha\beta\gamma}(\mathbf{c_j}) \},
$$
where $w_j, \boldsymbol{c}_j$  are the weights  and the  nodes of the Gauss-Hermite quadrature respectively. The  nodes $\mathbf{c}_j$  can be  considered as  the  lattice Boltzmann velocities while $ w_j\frac{f(\mathbf{c}_j)}{w(\mathbf{v}_j)}$ are the lattice Boltzmann distribution function values and
$w_j$ are the  lattice analog of the Maxwell distribution.  Then the  formula
\begin{equation}\label{grad_to_latt}
f_{LB,j}\equiv w_j\frac{f_{Grad}(\mathbf{c}_j)}{w(\mathbf{c}_j)}
\end{equation}
gives  the mapping from  the Grad truncated distribution function $f_{Grad}$ to the lattice Boltzmann distribution function $f_{LB,j}$ for the corresponding velocities $\mathbf{c}_j$. Now having  the values in the overlapping domain we  update  $f_{LB,j}$  for the  velocities $\mathbf{c}_j$  directed from the  overlapping domain into the interior of the LB  domain (the lattice velocities with non-positive $y$ component in Fig. 1-- Fig. 8).


The  second  step consists of the evaluation of the LB   distribution  for the lattice velocities  $\boldsymbol{c}_j$ such that $(\boldsymbol{c}_j,\mathbf{e})<0$ , where $\mathbf{e}$ is the outer normal to overlapping domain.  We evaluate the moments $a,a_{\alpha},a_{\alpha\beta},
a_{\alpha\beta\gamma}$ and finally  update again  the  Grad  distribution function (in  Fig.1--Fig. 8 computations starts from $y=0$) in the overlapping domain. The coefficients $a,a_{\alpha},a_{\alpha\beta},
a_{\alpha\beta\gamma}$ are calculated using the formulas
$$
a=\sum_{j=1}^N f_{LB,j}, \quad a_{\alpha}=\sum_{j=1}^N f_{LB,j}H_{\alpha}(\mathbf{c}_j), \quad  a_{\alpha\beta}=\sum_{j=1}^N f_{LB,j}H_{\alpha\beta}(\mathbf{c}_j),
\quad a_{\alpha\beta\gamma}=\sum_{j=1}^N f_{LB,j}H_{\alpha\beta\gamma}(\mathbf{c}_j).
$$
Now  we  are ready derive the DV distribution function in the  DV and  LB  overlapping domain. This  can be made by a simple
discretization of the Grad distribution functions at  the  nodes  of the  DV  scheme. Finally, we evaluate the DV distribution function for  the  all velocities  $(\boldsymbol{\xi}_j,\mathbf{e}) \leq 0$
in the interior of the  DV  spatial domain.

The described mapping  method can be generalized for the LB models which are not derived on the basis of the Gauss-Hermite quadratures. We assume that after the regularization procedure~\cite{Latt2006, Chen2006} and~\cite{Zhang2006, Mont2015, Mattila2017} the non-equilibrium part of LB distribution function will be  projected in a  finite dimensional velocity space with a basis spanned by  Hermite polynomials. Then the equivalence between the calculated LB distribution and
the expansion of the Grad  type can be achieved  therefore  the proposed mapping method can be applied.



\section{Basic equations}\label{sec:equations}

%%% nondimensional variables
Consider a monatomic ideal gas in \(D\)-dimensions.
Let \(L\), \(\refer\rho\), \(\refer{T}\), \(V = \sqrt{2R\refer{T}}\) and \(\refer{p} = \refer{\rho}R\refer{T}\) be
the reference length, density, temperature, velocity, and pressure, respectively.
The specific gas constant \(R = k_B/m\), where \(k_B\) is the Boltzmann constant and \(m\) is the molar mass.
Then, \(f\refer{\rho}/V^D\) is the one-particle velocity distribution function (VDF)
defined in \((2D+1)\)-dimensional space \((tL/V, x_iL, \xi_iV)\) and
the macroscopic variables take the following form:
\(\rho\refer{\rho}\) is the density, \(v_iV\) is the velocity, \(T\refer{T}\) is the temperature,
\(p_{ij}\refer{p}\) is the stress tensor, \(q_i\refer{p}V\) is the heat-flow vector.
In the dimensionless form, they are calculated as polynomial moments of the VDF:
\begin{equation}\label{eq:macro}
    \begin{gathered}
    \rho = \int f \dxi, \quad
    v_i = \frac1{\rho} \int \xi_i f \dxi, \quad
    T = \frac{2}{D\rho}\int(\xi_i-v_i)^2 f \dxi = \frac{p_{ii}}{D\rho}, \\
    p_{ij} = 2 \int(\xi_i-v_i)(\xi_j-v_j) f \dxi, \quad
    q_i = \int(\xi_i-v_i)(\xi_j-v_j)^2 f \dxi.
    \end{gathered}
\end{equation}
Integration with respect to \(\bxi\) is, hereafter, carried out over \(\mathbb{R}^D\) unless otherwise stated.

%%% Boltzmann equation
The VDF is governed by the Boltzmann equation
\begin{equation}\label{eq:Boltzmann}
    \pder[f]{t} + \xi_i\pder[f]{x_i} = \frac1kJ(f),
\end{equation}
where \(J(f)\) is the collisional operator with a local Maxwellian as the equilibrium function
\begin{equation}\label{eq:equilibrium}
    \equil{f}\br{\xi_i,\rho,v_i,T} = \frac{\rho}{(\pi T)^{D/2}}\exp\br{-\frac{(\xi_i - v_i)^2}T}.
\end{equation}
The Knudsen number \(\Kn\) can be expressed in terms of the reference gas viscosity \(\refer\mu\):
\begin{equation}\label{eq:Knudsen_number}
    k = \frac{\sqrt\pi}2\Kn = \frac{\refer\mu V}{\refer{p}L}.
\end{equation}

%%% BGK equation
In the present paper, we restrict ourselves to the simplest relaxation model~\cite{Krook1954, Welander1954}
\begin{equation}\label{eq:bgk_integral}
    J(f) = \frac1\tau\br{\equil{f}-f}, \quad \tau = \frac{T^{\omega-1}}{\rho},
\end{equation}
often referred as the Bhatnagar--Gross--Krook (BGK) model of the Boltzmann collisional operator.
\(\omega\) is the exponent of the viscosity law of the gas.
The nonlinearity in~\eqref{eq:bgk_integral} is more severe in comparison to the full Boltzmann equation
since \(\equil{f}\) depends on \(f\) via its moments,
but the BGK model is much simpler from the numerical point of view.

%%% Kinetic BC
The gas-surface interaction is described via the diffuse-reflection boundary conditions:
\begin{equation}\label{eq:diffuse}
    f(t,\xi_i) = \br{-2\sqrt\frac\pi{T_B} \int_{\xi'_jn_j<0}\xi'_jn_jf(t,\xi'_i)\dxi'}
        \equil{f}(\xi_i,1,v_{Bi},T_B) \quad (\xi_in_i>0),
\end{equation}
where \(n_i\) is the unit vector normal to the boundary, directed into gas.
\(T_B\) and \(v_{Bi}\) are the boundary temperature and velocity, respectively.
It is also assumed that \(v_{Bi}n_i = 0\).

\section{Numerical method}\label{sec:numerics}

\subsection{Discrete-velocity approximation}\label{sec:dv}

%%% quadratures in velocity space
Formally, the VDF can be represented as a weighted sum
\begin{equation}\label{eq:discrete_velocity}
    f(\bx,\bxi,t) = \sum_\alpha w_\alpha f_\alpha(\bx,t)\delta(\bxi-\bxia),
\end{equation}
where \(\delta(\bxi)\) is the Dirac delta function in the velocity space.
Then, an arbitrary moment \(\phi(\bxi)\) from \(f\) is calculated as
\begin{equation}\label{eq:cubature}
    \int \phi f\dxi = \sum_\alpha w_\alpha \phi(\bxia) f_\alpha.
\end{equation}
It is convenient to deal with weighted values \(\hat{f}_\alpha = w_\alpha f_\alpha\).
The evolution of \(\hat{f}_\alpha\) is governed by the system of partial differential equations
\begin{equation}\label{eq:dvm}
    \pder[\hat{f}_\alpha]{t} + \xiai\pder[\hat{f}_\alpha]{x_i} = \frac1kJ(\hat{f}_\alpha),
\end{equation}
which is called the \emph{discrete-velocity (DV) model} of~\eqref{eq:Boltzmann}~\cite{Cabannes1980}.

%%% conservative and entropic models
It is important for a DV model~\eqref{eq:dvm} to preserve conservation and entropy properties
of the continuous kinetic equation~\eqref{eq:Boltzmann}.
For the BGK model
\begin{equation}\label{eq:dvm-bgk}
    J(\hat{f}_\alpha) = \frac1\tau\br{\equil{\hat{f}}_\alpha-\hat{f}_\alpha},
\end{equation}
it can be accomplished when the discrete equilibrium function \(\equil{\hat{f}}_\alpha\)
is defined from the minimum entropy principle~\cite{Mieussens2000}:
\begin{equation}\label{eq:equilibrium-bgk}
    \equil{f}_\alpha = \exp(\beta_r\psi_{\alpha r}), \quad
    \psi_{\alpha r} = \transpose{\br{1,\bxia, \xi_\alpha^2}},
\end{equation}
where \(\beta_r\in\mathbb{R}^{D+2}\) is unique solution of
\begin{equation}\label{eq:beta}
    \sum_\alpha\psi_{\alpha r}\br{\equil{\hat{f}}_\alpha - \hat{f}_\alpha} = 0.
\end{equation}

%%% DV method
Let the molecular velocities \(\bxia\) be restricted to a Cartesian lattice \(X\)
with uniform spacing \(c\), called the \emph{lattice speed}.
The classical \emph{DV method} of solving~\eqref{eq:Boltzmann} is based on such approximations \(J(\hat{f}_\alpha)\)
that are consistent with \(J(f)\) when \(c\) goes to zero~\cite{Aristov2001}.
Due to exponential decay of the VDF, the given accuracy can be achieved
by cutting off all velocities that satisfy condition \(|\xi_\alpha| > \xi^{(\mathrm{cut})}\) from \(X\).

%%% LB method
A set of velocities with the corresponding weights \(\Set{(\bxia,w_\alpha)}{\alpha=1,\dots,Q}\)
is called the \emph{quadrature} (\emph{cubature} if \(D>1\)) \emph{rule}~\cite{Stroud1971}.
A common notation D\(n\)Q\(m\) means \(D=n\) and \(Q=m\).
Hereinafter, a quadrature rule, together with the discrete operator in form~\eqref{eq:dvm-bgk},
is referred as the \emph{lattice-BGK (LBGK) model}.
When the VDF is close to equilibrium, it can be acceptably approximated using quadratures with small number \(Q\).
LBGK models are capable of reproducing low-order polynomial moments of the VDF accurately and,
therefore, describing a fluid-dynamic behavior of a gas, including that beyond the Navier--Stokes level.
They serve as a foundation for the \emph{lattice Boltzmann (LB) method}.

%%% Difference between DV and LB
Within the introduced terminology, the only difference between the DV and LB methods is an ultimate goal.
The former one strives to capture kinetic properties of highly nonequilibrium flows and, therefore,
forced to have a quite large dimension of the approximation space \(Q\).
The latter one, on the contrary, is based on the assumption of a slightly perturbed equilibrium and, therefore,
tends to describe it in the most efficient way.

%%% Gauss--Hermite
In the present paper, we employ LBGK models based on the projection of the VDF onto Hermite basis~\cite{Shan2006}:
\begin{multline}\label{eq:gauss-hermite}
    \equil{f}_\alpha = \rho\Bigg\{ 1 + 2\xiai v_i + \overbrace{
        2\br{\xiai v_i}^2 - v_i^2 + (T-1)\br{\xiai^2 - \frac{D}2}
    }^\text{second order} + \\ \underbrace{
        \frac{2\xiaj v_j}3\left[ 2\br{\xiai v_i}^2 - 3v_i^2 + 3(T-1)\br{\xiai^2 - 1 - \frac{D}2}\right]
    }_\text{third order} + \cdots \Bigg\}.
\end{multline}
Truncating of~\eqref{eq:gauss-hermite} compromises the positivity condition and,
therefore, these LBGK models are not entropic, but the conservation properties are preserved.

\subsection{Time-integration method}\label{sec:splitting}

%%% Splitting scheme
For the present study, we start from the simplest numerical algorithm providing the second-order accuracy
for both time and physical coordinates.
Equation~\eqref{eq:Boltzmann} is solved by the symmetric Strang's splitting scheme~\cite{Bobylev2001}
\begin{equation}\label{eq:Strang}
    S^{\Delta{t}}_{A+B}(f_0) = S^{\Delta{t}/2}_A \br{S^{\Delta{t}}_B \br{S^{\Delta{t}/2}_A(f_0)} } + \OO{\Delta{t}^3},
\end{equation}
where \(A(f) = -\xi_i\Pder[f]{x_i}\), \(B(f) = J(f)/k\).
\(S^t_P (f_0)\) denotes the solution of the Cauchy problem
\begin{equation}\label{eq:Cauchy}
    \pder[f]{t} = P(f), \quad f|_{t=0} = f_0.
\end{equation}

%%% Space-homogeneous
Important implication of the splitting procedure is that the space-homogeneous BGK equation
\begin{equation}\label{eq:bgk_homogeneous}
    \pder[f]{t} = \frac{\rho}k \br{\equil{f}-f}
\end{equation}
has the exact solution
\begin{equation}\label{eq:bgk_exact}
    f(t) = \equil{f} + \br{f(t_0)-\equil{f}}\exp\br{-\frac{\rho}k (t-t_0)}.
\end{equation}
Moreover, generalization of this algorithm to the original Boltzmann equation is straightforward.

%%% Steady-state solution
To find a steady-state solution of the boundary-value problem,
the time-marching process is started from some initial approximation
and continues until the convergence criterion is met.

\subsection{Finite-volume formulation}\label{sec:fv}

For brevity, we consider a one-dimensional physical space.
The transport equation
\begin{equation}\label{eq:transport}
    \pder[f]{t} + \xi_1\pder[f]{x_1} = 0
\end{equation}
is approximated by the \emph{finite-volume (FV)} method:
\begin{equation}\label{eq:finite_volume}
    f^{n+1}_m = f^n_m - \frac{\Delta{t}}{\Delta{x_m}}\br{F^n_{m+1/2}-F^n_{m-1/2}}, \quad
    m = 1,\dots,M, \quad n\in\mathbb{N},
\end{equation}
where \(\Delta{t}\) is the time step, \(\Delta{x_m}\) is the width of \(m\) cell in the physical space,
\begin{equation}\label{eq:vdf_fv}
    f^n_m(\bxi) = f\br{n\Delta{t}, \frac{\Delta{x_m}}2 + \sum_{k=1}^{m-1}\Delta{x_k}, \bxi}.
\end{equation}
For \(\xi_1>0\), the internal fluxes can be written in the following form:
\begin{equation}\label{eq:internal_fluxes}
    F_{m+1/2}^n = \xi_1\br{ f_m^n + \frac{1-\gamma}2\overline{\Delta{f_m^n}}},
    \quad \gamma = \frac{\xi_1\Delta{t}}{\Delta{x_m}}, \quad m = 1,\dots,M.
\end{equation}
They are calculated by the second-order total variation diminishing (TVD) scheme,
e.g., with the monotonized central (MC) slope limiter
\begin{equation}\label{eq:limiter}
    \overline{\Delta{f_m^n}} = \begin{cases}\min\br{
         2\frac{|D_-|}{h_-}, \frac12\frac{|D_-+D_+|}{h_-+h_+}, 2\frac{|D_+|}{h_+}
    }\Delta{x_m}, \quad D_+D_- > 0, \\
    0, \quad D_+D_- \leq 0,
    \end{cases}
\end{equation}
where
\begin{equation}\label{eq:differences}
    D_\pm = \pm\br{f_{m\pm1}^n - f_m^n}, \quad h_\pm = \frac{\Delta{x_{m\pm1}} + \Delta{x_m}}2.
\end{equation}
The last flux \(F_{M+1/2}^n\) is calculated based on the linear extrapolation of the solution for the ghost cell:
\begin{equation}\label{eq:last_ghost}
    f_{M+1}^n = 2f_M^n - f_{M-1}^n.
\end{equation}
Note that sharp variations (in physical space) of solution can occur even for nearly incompressible flow,
especially for large \(|\bxi|\).

%%% Boundary conditions
The diffuse-reflection boundary condition~\eqref{eq:diffuse}, e.g. at \(x=0\),
is introduced through the first flux and ghost cell:
\begin{gather}
    F_{1/2}^n(\bxia) = \displaystyle\xi_1\frac{\sum_{\xi'_{\alpha1}<0}F_{m+1/2}^n(\bxia')}
        {\sum_{\xi'_{\alpha1}<0}\xi'_{\alpha1}\equil{f}(\bxi'_\alpha,1,v_{Bi},T_B)}
        \equil{f}(\bxi_\alpha, 1, v_{Bi}, T_B) \quad (\xi_{\alpha1}>0), \label{eq:first_flux}\\
    f_0^n(\bxia) = \displaystyle\frac{\sum_{\xi'_{\alpha1}<0}\xi'_{\alpha1}f_1^n(\bxia')}
        {\sum_{\xi'_{\alpha1}<0}\xi'_{\alpha1}\equil{f}(\bxi'_\alpha,1,v_{Bi},T_B)}
        \equil{f}(\bxi_\alpha, 1, v_{Bi}, T_B) \quad (\xi_{\alpha1}>0). \label{eq:first_ghost}
\end{gather}
This implementation yields second-order accuracy along with conservation of mass.
For \(\xi_1<0\), all expressions are analogous.

%%% Why half-integer lattice?
The boundary conditions also dictate a way of discretization in the velocity space.
With respect to the origin of the velocity coordinates, only two types of lattices are symmetric~\cite{Inamuro1990}:
\emph{integer} \(\br{\xiai/c} \in \mathbb{Z}^D\) and \emph{half-integer} \(\br{\xiai/c + e_i/2} \in \mathbb{Z}^D\),
where \(e_i\) is the corresponding orthonormal basis.
For the considered boundary condition at \(x=0\) and \(D>1\), there is a zero-measure set of velocities
\(\Set{\bxi\in\mathbb{R}^D}{\xi_1=0}\), called tangential.
These velocities are immune to the diffuse reflection.
In contrast, the integer lattice contains a substantial subset of tangential velocities.
Therefore, to avoid an additional discretization error, the half-integer lattice should be employed.

%%% Diffuse reflection for LB
In the same manner, LB cubatures without tangential velocities are preferable to the classical ones.
Moreover, LB models can be augmented by special groups of velocities to approximate
the diffuse-reflection boundary condition more accurately~\cite{Feuchter2016}.
These models ensure vanishing errors of the relevant half-space integrals.

\subsection{Coupling algorithm}\label{sec:coupling}

%%% Homogeneous domain decomposition problem
Staying within the FV framework, divide our computational domain in the physical space
into subdomains, each employing its own DV model.
The coupling conditions at the interface between subdomains can be considered as virtual boundary conditions.
They are symmetric due to unified formulation in the physical space.

%%% Projection and reconstruction procedures
The concept of ghost cells suggests the simplest (from the algorithmic point of view) coupling strategy.
If the interface between subdomains lies in the near-continuum region,
it is admissible to exchange information only within a Hilbert subspace \(\mathcal{H}\)
spanned by the first \(D\)-dimensional Hermite polynomials.
Then, all that we need is to supplement each DV model with a mapping to this subspace.
In other words, a projection and reconstruction procedures should be implemented.
While the former one is nothing else but weighted cubatures~\eqref{eq:cubature},
the latter one depends on the underlying DV model.

%%% Specify the reconstruction procedure
In the present paper, we confine \(\mathcal{H}\) to the first \(1+5D/2+D^2/2\) polynomials
corresponding to the macroscopic variables~\eqref{eq:macro}.
Note that pressure tensor \(p_{ij}\) contains \(D(D+1)/2\) independent values due to its symmetric nature.
For the DV model, the desired reconstruction coincides with the Grad's thirteen-moment method:
\begin{equation}\label{eq:interface_dvm}
    f^{\mathcal{H}}_\alpha = \equil{f}_\alpha\br{
        1 + \frac{\cai\caj P_{ij}}{\rho T^2} + \frac{4\cai q_i}{\rho T^2}\br{\frac{\caj^2}{(D+2)T} - \frac12} },
\end{equation}
where \(P_{ij} = p_{ij} - \rho T\delta_{ij}\) and \(\cai = \xiai - v_i\).
For the Gauss--Hermite LB model, the appropriate discrete VDF is recovered as a part of the third-order Hermite expansion:
\begin{equation}\label{eq:interface_lbm}
    f^{\mathcal{H}}_\alpha = \equil{f}_\alpha + \underbrace{ \vphantom{ \br{\frac{2\xiaj^2}{D+2}-1} }
        \xiai\xiaj P_{ij}
    }_\text{second order} + \underbrace{
        2\xiai q_i\br{\frac{2\xiaj^2}{D+2}-1} - 2\xiai P_{ij}\br{ v_j - \xiaj\xiak v_k }
    }_\text{third order}.
\end{equation}

%%% Conservative scheme
At first glance, the proposed reconstruction procedure does not violate the conservation properties of the FV scheme,
because all moments required for the equilibrium function are calculated exactly.
However, the reality is different, since the FV scheme actually deals separately
with velocities directed in the opposite half-spaces with respect to the interface.
For this reason, mass, momentum, and energy fluxes across the coupling interface
are turned out to differ slightly for each DV model.
In the present paper, we employ the polynomial correction (like in~\cite{Aristov1980}):
\begin{equation}\label{eq:poly_correction}
    \bar{F}^{(1)}_\alpha = F^{(1)}_\alpha(1+\gamma_r\psi_{\alpha r}), \quad
    \sum_{\alpha=1}^{Q^{(1)}} \bar{F}^{(1)}_\alpha\psi_{\alpha r} = \sum_{\alpha=1}^{Q^{(2)}} F^{(2)}_\alpha\psi_{\alpha r},
\end{equation}
where \(F^{(s)}\) and \(Q^{(s)}\) are the initial flux and number of velocities of \(s\) model, respectively,
\(\bar{F}^{(1)}\) is the corrected flux, \(\psi_{\alpha r}\) is defined in~\eqref{eq:beta}.
In practice, each component of \(\gamma_r\in\mathbb{R}^{D+2}\) is much less than unity;
therefore, the positivity is also preserved.

%%% 1D example
Finally, let us return to the one-dimensional example outlined in sec.~\ref{sec:fv}
and suppose that \(x=0\) is our interface.
In order to use~\eqref{eq:internal_fluxes}, the VDF should be reconstructed in the ghost cells.
In case of second-order TVD scheme, \(f_{-1}^n\) is used for all \(\bxia\) and, additionally,
\(f_{-2}^n\) is required when \(\xi_{\alpha1}>0\).
In case of first-order scheme, \(f_{-1}^n\) and only for \(\xi_{\alpha1}>0\) is sufficient.

\section{Results and discussions}\label{sec:results}

\begin{figure}[!h]
\centering
\includegraphics[height=80mm]{dvm.pdf}
\caption{The  results of the  numerical study for the BGK nonlinear model $\Delta v=0.02, Kn=0.1$. The  black boxes correspond  to the  tabulated theoretical solutions \cite{Luo2015, Luo2016}. The  point $y=0.5$ corresponds  to the position of the right plane, the point $y=0$ lies in the middle between the moving planes.}
\end{figure}

%\begin{figure}[!h]
%\centering
%\includegraphics[height=80mm]{lbm.pdf}
%\caption{\textcolor{red}{ABSCISSAE IS X HERE !?!}The  results of the  numerical study for the Lattice Boltzmann D3Q19 model $\Delta v=0.02, Kn=0.1$. The  black boxes correspond  to the  tabulated theoretical solutions \cite{Luo2016}. The departure of the  LBGK bulk velocity form the  theoretical values is  observed in the Knudsen layer. The  point $y=0.5$ corresponds  to the position of the right plane, the point $y=0$ lies in the middle between the moving planes.}
%\end{figure}

\begin{figure}[!h]
\centering
\includegraphics[height=80mm]{d3q19.pdf}
\caption{The  results of the  numerical study for the Lattice Boltzmann D3Q19 model $\Delta v=0.02, Kn=0.1$. The  black boxes correspond  to the  tabulated theoretical solutions \cite{Luo2015, Luo2016}.  The  matching  point is  located at $y=0.38$. The  point $y=0.5$ corresponds  to the position of the right plane, the point $y=0$ lies in the middle between the moving planes.}
\end{figure}

\begin{figure}[!h]
\centering
\includegraphics[height=80mm]{hyb-d3q19-narrow.pdf}
\caption{The  results of the simulation for the matching point at $y=0.44$ for D3Q19+BGK model.}
\end{figure}

\begin{figure}[!h]
\centering
\includegraphics[height=80mm]{hyb-d3q19-wide.pdf}
\caption{The  results of the simulation for the matching point at $y=0.26$ for D3Q19+BGK model.}
\end{figure}


\begin{figure}[!h]
\centering
\includegraphics[height=80mm]{d3v64.pdf}
\caption{The  results of the simulation for the matching point at $y=0.38$ for D3Q64.}
\end{figure}


\begin{figure}[!h]
\centering
\includegraphics[height=80mm]{d3v96.pdf}
\caption{The  results of the simulation for the matching point at $y=0.38$ for D3Q96.}
\end{figure}


\begin{figure}[!h]
\centering
\includegraphics[height=80mm]{d3q121.pdf}
\caption{\textcolor{red}{CAN ONE EXPLAIN THE STRANGE RESULT ?} The  results of the simulation  for D3Q121.}
\end{figure}



\begin{figure}[!h]
\centering
\includegraphics[height=80mm]{hyb-d3v96.pdf}
\caption{\textcolor{red}{WHY HYBRID IS EVEN WORSE THAN THE SOLE  MODEL}The  results of the simulation for the matching point at $y=0.38$ for D3Q96+BGK model.}
\end{figure}




\todo{In the present paper, we deal with the flow between the two parallel planes which have non-zero relative velocity (the plane Couette flow) for small Knudsen numbers. For solving this problem we develop a hybrid BGK and LBGK coupling method. The gas dynamics in Knudsen layers is considered using the full nonlinear BGK equation while the internal zone is described by LB model.}

The classical stationary Couette problem has been used as a test for the consideration of the hybrid BGK+LBM BGK scheme. The results of the computations for BGK in the whole 1D computational domain are compared with the computations with the LBM and BGK scheme and with the computations by the hybrid scheme. The gas is confined between two infinite parallel plates placed at $y = \pm 1/2$ with constant temperature $T = 1$ and velocities ($v = \pm\Delta v/2,0,0$). A complete diffuse reflection are assumed at the plates. The average density is equal to unity $\int_{-1/2}^{1/2}\rho dy=1$. The physical space $0 < y < 1/2$ is divided into $N = 40$ nonuniform cells refined near $y = 1/2$. This numerical example was tested for the Knudsen number Kn=0.1. The results of the TVD difference scheme for thee BGK  are presented in Fig.1. The buffer layer is located for \textcolor{red}{the first variant} of computation at a distance of $1.2$ mean free path from the plate. The point of matching has been defined as the boundary between the Knudsen layer and the continuum zone of the flow, of course the switching procedure should be introduced, e.g. by the magnitudes of some moment gradients, in our calculations changing the coupling point has been investigated.


The results of the simulations for the Lattice BGK  (D3Q19) in the whole spatial domain are showed in Fig. 2. One can see the differences for the velocity profile between the LBGK and the tabulated data. Obviously the lattice Boltzmann  model is  unable to describe  the  boundary layer with a good precision.
The  heat flux for lattice Boltzmann model equals zero in the whole domain since this model can not tackle with the third  moment.   We use  the  kinetic (diffusive) boundary conditions at the wall, for the LB models  this boundary condistions are described in
\cite{ansumali2002}. More  precisely, let us define $U_w, c_w^2$  as the wall velocity and  the wall temperature and $\mathbf{n}$ is  the wall inward normal vector, then we  have
\begin{equation}\label{kinet_bound}
f_{latt, s}=
\frac{\sum_{\mathbf{v}_j\mathbf{n}<0} f_{latt,j}|\mathbf{v}_j\mathbf{n}|}{\sum_{\mathbf{v}_j\mathbf{n}<0}
f^{eq}_j|\mathbf{v}_j\mathbf{n}|}f^{eq}_s, \quad  \mathbf{v}_s \mathbf{n}>0,
\end{equation}
where
$$
f^{eq}_{j}=\rho w_j\left(1+ \frac{\mathbf{v_j}\mathbf{U}_w}{c_w^2}+\frac{(\mathbf{v_j}\mathbf{U}_w)^2-c_w^2U_w^2}{2c_w^4}\right).
$$
 The formula  \eqref{kinet_bound}  allows to evaluate  the reflected distribution $\mathbf{v}_s\mathbf{n}>0$ if the  impinging distribution $\mathbf{v}_s\mathbf{n}<0$ is  known.

 The computations by the hybrid method LBGK(D3Q19)+BGK show good correspondence with the tabulated data except of the heat flux which equals zero in the lattice Boltzmann domain, see Fig. 3.

We have the preliminary results regarding the hybrid scheme’s efficiency. The typical CPU time for the BGK(DVM) $\sim 16$ s, the typical CPU time for LBM (D3Q19) $\sim 4$ s, zone of BGK (DVM) $\sim 1/4$ of the total zone, zone of LBM (D3Q19) $\sim 3/4 $ of the total zone, CPU time for BGK in the hybrid scheme $\sim 4$ s, CPU time for LBM (D3Q19) in the hybrid $\sim 3$ s, CPU time of the hybrid $\sim 7$ s, so the efficiency of the hybrid is more than $2$ times.

\begin{figure}[!h]
\centering
\includegraphics[height=80mm]{acceleration.pdf}
\caption{The hybrid method acceleration. $N_0$ is the number of spatial points in the Knudsen layer.\textcolor{red}{HAVE  NO IDEA  WHY THE INCREASE IN NUMBER  OF CELLS AFFECTS THE PRODUCTION}}
\end{figure}

\textcolor{blue} {Hybrid simulations with different positions of the matching points have been also considered and presented in Figs. 4-5.}
\textcolor{blue} {Results of more accurate computations for obtaining the efficiency are presented in Fig. 4.It is important that this efficiency tends to the limit when increasing the number of nodes in the coordinate computational domain. It is obvious because the number of velocity nodes is negligible in comparison with that in BGK and the limit is determined by the ratio of the number of coordinate points in the entire computational domain to the number of coordinate points in the kinetic region with computations on the basis of BGK}

More complex LBM (see ??)  are also considered and results in Figs.  are compared with the standard  well-known LBM schemes with small velocity points.

The last considered simulations for hybrids with more complex LBM are very important because allow us to describe dissipative values namely the heat flux. In Fig. ?? the mentioned macroscopic parameters are showed for the Lattice Model D3Q96. One can see that the heat flux is reproduced by the LBM in the middle continuum zone.

\textcolor{blue} {The criteria of the distinguishing between continuum and rarefied zones according to the value of the heat flux can be used. The threshold value for the transition from the continuum region to the kinetic region could be, e.g. related to the accepted level of the error in the calculations. Namely, if the module of the heat flux is less than this threshold value than this region is related to the continuum area and vice versa. In Fig. 5 the value related to the mentioned criteria is shown for all points of the coordinate computational domain.}

\section{Conclusion}\label{sec:summary}
%%%%%%%%%%%%%%% very new %%%%%%%%%%%%%%%%%%%%%%%%%%%%%%%%%%%%

In the paper, we presented a new algorithm of merging of the numerical solutions for BGK kinetic equation applied in the boundary regions with the Lattice Boltzmann Models used in continuum regions. The buffer cells use the distribution function in a form of the Grad expansion, that allows us to construct a hybrid method. We applied several lattice Boltzmann models ranging from the most common D3Q19 to high-order LBM models with the large number of velocities (in comparison with ordinary samples). This method was tested for a simple one-dimensional Couette problem for the isothermal regime.  The  LB
models  which are not based on the Gauss-Hermite quadrature  method could be implemented  using the regularization procedure

The following problems for the further study can be addressed. The usage of multiple velocities in the difference schemes for the full Boltzmann equation or BGK equation is somewhat overkill since the highest moments can be unimportant for many flows. Therefore the methods for the velocity number reduction  should be developed. The DV methods with different  number of discrete velocities in different flow regions can be perspective hybrid schemes for compressible flows. Such approaches can be analogous to the adaptive schemes in velocity space. In some given methods the separation of the velocity space in the subdomains, in particular for small and high temperature (see [1]) or for slow and fast molecules [2] is made.

The implementation of the Cartesian grids and the advection part in a fashion similar to LB method potentially allows to diminish the interpolation errors in the finite difference expressions for the Boltzmann equation. In [3] a similar idea has been realized for D3Q27 in the advection part of BE.
The other developments could be concerned to the hybrid method for supersonic flows, compressible and  thermal flows~\cite{Chen2010, Frapolli2015, Frapolli2016}. Also  the potential usage of entropic models is  appealing due to its enhanced stability for low viscosities (and large Reynolds numbers).


This work was supported by the Russian Foundation on Basic Researches, Grant 18-01-00899.

\section*{References}

\bibliography{dvm-lbm}

\end{document}
