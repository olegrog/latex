\section{Introduction}
The properties of an electrolyte--electrode interface play a crucial role in the electrochemistry, \red{nanofluidic technologies and biology}. 
Such interfaces form the electric double layer (EDL)---a multi-layered system of the ions on the surface, which accumulates charge and controls the electrostatic field. Many important insights into the EDL were obtained from the Goy--Chapman (GC) theory \red{Ref}, where the electrolyte is treated as an ideal gas mixture. However, the discovery of the room-temperature ionic liquids (IL) shifted the operation conditions of the supercapacitors towards to higher electrolyte density and applied voltages, which cannot be described in terms of the GC model. After years, Kornyshev and co-authors developed mean-field-type models accounting both ion steric effects and short-range electrostatic correlations in Refs.~\cite{kornyshev2007double,goodwin2017mean}. These models correctly describe the capacitance behavior, but neglect another important aspect of the EDL---its \red{spacial heterogeneity}~\cite{fedorov2008ionic}. 

%Overscreening
One of the most stark spatial property of the EDL is the ``overscreening'' phenomenon observed in experiments~\cite{mezger2008molecular,zhou2012nanoscale} and simulations~\cite{lynden2007simulations,fedorov2008ionic,fedorov2010double} for ionic liquids at low applied voltages. After these first results, the EDL structure becomes the subject of active research~\cite{belotti2021experimental,toda2021observation,schlaich2021electronic} due to its importance for energy storage in supercapacitors. 
The overscreened state exhibits a strong interfacial layering, where the first layer accumulates a charge with higher absolute value than the total one on the electrode. 
Further increase of the applied potential induces the transition to the crowding state---two or more subsequent layers of counterions on the electrodes surface.   
Bazant, Storey, and Kornyshev (BSK) developed the first theory of the overscreening/crowding transition for ionic liquids~\cite{bazant2011double}. They derived the modified Poisson equation from the phenomenological Landau--Ginzburg-like functional of the total free energy. 
Considering ionic liquids parameters, the BSK theory predicts the overscreening state at low applied potentials and also shows a transition to crowding, as the applied voltage increases. The recent extension of the BSK model~\cite{de2020interfacial} describes the spatial oscillations of the ion density with an accuracy comparable to more computationally expensive density functional methods~\cite{gavish2018solvent,ma2020classical}. 

With few exceptions, the established electrolyte interface models are dedicated to flat~\cite{kornyshev2007double,goodwin2017mean,bazant2011double} or cylindrical~\cite{janssen2019curvature} electrodes, where defects and corrugations on the electrode surface are neglected. At the same time, an implementation of the electrodes with geometrically heterogeneous (rough, corrugated) surfaces opens a new direction in towards to the enhanced capacity~\cite{vatamanu2015non}. This statement is justified by many simulations for ionic liquids near heterogeneous surfaces~\cite{vatamanu2011influence,xing2012nanopatterning,hu2013molecular,vatamanu2014influence, bedrov2015ionic, vatamanu2017charge} and experimental comparison of two electrodes differing only by morphology~\cite{wei2020surface}. In the latter work, it is demonstrated that the molecular-scale patterns on the surface can significantly (around 50\% in Ref.~\cite{wei2020surface}) improve the capacitance. Notice that the surface roughness impact has been observed for ionic liquids only, while the solvent based electrolytes are insensitive to the geometrical modifications of the surface \red{Ref}.
For the diluted electrolytes, the first theory accounting the surface roughness was developed by Daikhin, Kornyshev and Urbakh using the perturbation methods for both the linear~\cite{daikhin1996double} and nonlinear~\cite{daikhin1998nonlinear} Poisson--Boltzmann equations. The theory was extended to the case of dense ionic liquids near a rough surface in Ref.~\cite{aslyamov2021electrolyte}, where the authors perturb the Poisson equation for the rough surface modeled as the random correlated process. This approach describes a realistic morphology with low ~\cite{aslyamov2017density,aslyamov2019random} and notable roughness~\cite{aslyamov2019theoretical}. Another distinctive feature of the recent model~\cite{aslyamov2021electrolyte} from Refs.~\cite{daikhin1996double, daikhin1998nonlinear} is that it accounts for the ion finite volumes and tends to Kornyshev's result~\cite{kornyshev2007double} as the roughness scale decreases. As a consequence, it also describes the differential capacity for rough electrodes. 

In this paper we extend approach~\cite{aslyamov2021electrolyte} beyond the mean-field approximation and investigate the ionic liquid EDL structure on rough electrodes. More precisely, we focus on the description of the overscreening/crowding states in the presence of surface roughness. 
Following the detailed discussion in~\cite{goodwin2017mean}(see footnotes 1-2), the overscreening phenomenon is inherent in the EDL on ideal smooth electrodes. Indeed, the overscreening state relates to well-defined layering, but roughness destroys the layered structure~\cite{sheehan2016layering}. 
Therefore, it is expected that rough morphology depresses such ``resonance'' effect inducing the transition to the crowding state. 
%We have not seen theoretical studies of this topic and quantative estimations for the roughness influence overscreening/crowding EDL structure.

\red{To do: preview  of  results}
In the present study, we develop a theory describing the structure of the ionic liquid EDL near electrodes with a rough surface. To reach this goal, we use the BSK model as the basis due to its explicit form and ability to describe the EDL structure near flat electrodes. We assume that the surfaces with the lateral length scale (average peak to peak distance) is much larger than the normal roughness (root mean square). Such surfaces often meet in the experiments, for example, nanopatterned substrates with low density of defects, see Ref.~\cite{sheehan2016layering}. 

\Oc{Cite~\cite{dambrine2016numerical} somehow.}


  



