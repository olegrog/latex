%&pdflatex
\documentclass{article}
\usepackage{a4wide}

\usepackage{amssymb, amsmath}
\usepackage{framed}                % for leftbar
\usepackage[%
    font={bfseries},
    leftmargin=0.1\textwidth,
    indentfirst=false
]{quoting}

\title{Response to the Reviewer's Comments}
\author{Oleg Rogozin%
    \thanks{Electronic address: \texttt{o.a.rogozin@gmail.com}}
}

\newcommand{\Kn}{\mathrm{Kn}}
\newcommand{\dd}{\:\mathrm{d}}
\newcommand{\pder}[2][]{\frac{\partial#1}{\partial#2}}
\newcommand{\pderder}[2][]{\frac{\partial^2 #1}{\partial #2^2}}
\newcommand{\Pder}[2][]{\partial#1/\partial#2}

\usepackage{lipsum}
\usepackage{graphicx}

\def\asterism{\par\vspace{1em}{\centering\scalebox{1}{\bfseries *~*~*}\par}\vspace{.5em}\par}

\usepackage[
    pdfauthor={Oleg Rogozin},
    pdftitle={Response to the Reviewer's Comments},
    pdftex,
    unicode
]{hyperref}

\begin{document}

\maketitle

First of all, I express my gratitude for the extremely useful reviewers'
comments, for they supply me with a professional view from the outside,
help to correct important mistakes, and improve my own knowledge of
subject. Following up on the given suggestions, I have revised the whole
paper by adding clarifications at the crucial and vague spots. Some
supplementary discussions have been added. The list of references has
been also extended. The introduction and the conclusion have been
completely rewritten to express the meaning of the paper for the field
more exactly.

Below I indicate for each individual comment how I have dealt with it.

\section{Referee Report \#1}

\begin{quoting}
1. The language should be improved by a native. It is little bit hard to
read.
\end{quoting}

It is the only point that both reviews bring my attention. So I revise
the whole text and correct many grammar mistakes, replace bad language
forms by native ones, and remove some ambiguity sentence. I hope new
version of paper is much easier to read and understand.

\begin{quoting}
2. The inaccuracy of using no-slip boundary condition for the
compressible N-S equation when large temperature variation is
encountered should be discussed much more to let readers appreciate the
importance of your study.
\end{quoting}

Many words about inaccuracy of the compressible Navier--Stokes equation
are told inside the reference books cited in the paper~\cite{Sone2002, Sone2007}.
To improve this point, I have added some explanations in the introduction.

\begin{leftbar}
The Navier--Stokes equations is widely used equations for modeling of a
gas with the vanishing mean free path. Slightly rarefied gas effects,
which is widely thought to take place only in the thin Knudsen layer,
can be taken into account using the slip conditions on the
boundary~\cite{SharipovCoefficients}. In some situations, however, the
Navier--Stokes set fails to describe the correct density and temperature
fields even in the continuum limit~\cite{Kogan1976, GhostEffect}. When
the Reynolds number is finite and the temperature variations are large:
\[ \mathrm{Re} = O(1), \quad \frac1T\left|\pder[T]{x_i}\right| = O(1), \]
higher-order terms of gas rarefaction begin to play a significant
role in the gas behavior.
\end{leftbar}

\begin{quoting}
3. Please explain why you chose OpenFOAM for your implementation
platform. And also give more literature about the advantage and
application of OpenFOAM. Please cite the following papers to enhance
this point: Development of a generalized numerical frame work for
simulating biomass fast pyrolysis in fluidized-bed reactors, Chemical
Engineering Science 99 (2013), 305-313; Realistic wave generation and
active wave absorption for Navier-Stokes models. Application to OpenFOAM®,
Coastal Engineering 71 (2013), 102-118; Modeling effects of operating
conditions on biomass fast pyrolysis in bubbling fluidized bed reactors,
Energy \& Fuels 27 (2013), 5948-5956; Numerical simulation of cavitating
turbulent flow in a high head Francis turbine at part load operation
with OpenFOAM, Procedia Engineering 31 (2012), 156-165.
\end{quoting}

I have tried to explain some reasons for selecting
OpenFOAM\textregistered{} as an appropriate CFD platform in detail. Some
of the references, suggested by the reviewer, that deal with
OpenFOAM\textregistered{} direct were included.

\begin{leftbar}
One of the modern and promising CFD platforms, OpenFOAM\textregistered{}
has been selected as a basis for numerical algorithms developing.
OpenFOAM\textregistered{} is an object-oriented C++ library of classes and routines for parallel computation,
providing a set of high-level tools for writing advanced CFD code~\cite{OpenFOAM1998}.
It has a wide set of basic features, similar to any commercial one~\cite{OpenFOAM2010},
and is a robust and reliable software widely used in the industry~\cite{BoilingFlows2009,
TurbulentCombustion2011, CoastalEngineering2013, BiomassPyrolysis2013}.
As for rarefied gas, OpenFOAM\textregistered{} has a standard solver for DSMC, that
can be extended for hybrid simulations~\cite{HybridSolver2012}.
Finally, the most important advantage of OpenFOAM\textregistered{} is its open-source code,
so it is easy to add any modification to any part of the implementation.
OpenFOAM\textregistered{} is also well documented and has a large and active community of users.
\end{leftbar}

\begin{quoting}
4. I think the ``Liquids and'' is not necessary at the beginning of
section~2.
\end{quoting}

Corrected.

\begin{quoting}
5. Please discuss more about the technical part of the inaccuracy of
classical N-S equation in section~2.
\end{quoting}

I have added a special paragraph in the end of section~2 to clarify this point.

\begin{leftbar}
The behavior of a gas is governed by the conservation equations of mass, momentum and energy:
\begin{gather}
    \pder[\rho]{t} + \pder{x_i}(\rho v_i) = 0, \label{eq:mass}\\
    \pder{t}(\rho v_i) + \pder{x_j}(\rho v_i v_j + p_{ij}) = \rho F_i, \label{eq:momentum}\\
    \pder{t}\left[\rho\left(e+\frac{v_i^2}2\right)\right]
        \pder{x_j}\left[\rho v_j\left(e+\frac{v_i^2}2\right)+v_i p_{ij}+q_j\right] = \rho v_j F_j. \label{eq:energy}
\end{gather}

\asterism

By means of Newton's law for the stress tensor \(p_{ij}\) and Fourier's law the heat-flow vector \(q_i\),
the conservation equations are closed to the Navier--Stokes set of equations
\begin{gather}
    p_{ij} = p\delta_{ij} - \mu\left(\pder[v_i]{x_j}+\pder[v_j]{x_i}-\frac23\pder[v_k]{x_k}\delta_{ij}\right) -
        \mu_B\pder[v_k]{x_k}\delta_{ij}, \label{eq:stress_tensor}\\
    q_i = -\lambda\pder[T]{x_i}. \label{eq:heat_flow}
\end{gather}

\asterism

The classical heat-conduction equation
can be derived from~\eqref{eq:energy} and~\eqref{eq:heat_flow}
in the absence of gas flows (\(v_i = 0\)):
\begin{equation}\label{eq:heat_equation}
    \pder{x_i}\left(\sqrt{T}\pder[T]{x_i}\right) = 0.
\end{equation}

\asterism

A rigorous examination of the continuum limit shows that the
infinitesimal thermal conduction term \(\Pder[q_j]{x_j}\) can be of the
same order as the thermal convective term \(\Pder[pv_j]{x_j}\) in the
energy equation~\eqref{eq:energy}. In such a case, when the Mach number
is the same order of as the Knudsen number, the heat-conduction
equation~\eqref{eq:heat_equation} fails to describe the correct
temperature field. Moreover, additional thermal stress terms of the
second order of \(\Kn\) appears in the momentum
equation~\eqref{eq:momentum}. These important modifications can be
systematically considered only within the scope of kinetic theory.
\end{leftbar}


\begin{quoting}
6. Some literature citations are missing, e.g., (thermal creep flow)~[?]
\end{quoting}

Corrected.

\begin{quoting}
7. The correspondence of the figures to the running cases seems not
correct. Please be serious on your paper and check them carefully.
\end{quoting}

I have not got this point.
Unfortunately, the referee did not specify a place of the fault.
I have double-checked all figures and have found only one misprint:
the axis of abscissas \(\alpha\) was presented instead of \(\alpha-1\) in Fig.~10.
Another referee did not found any ambiguity in figures, too.
I have tried to make more detailed and clear descriptions under the figures.

\begin{quoting}
8. I think there also need comparison in the cases of 4.2 and 4.3
between the results from classical N-S equations and your derived
equations.
\end{quoting}

The classical Navier-Stokes equations have a trivial solution \(v_i = 0\).
I have added an explicit remark about this fact.

\begin{leftbar}
Now, consider the case, where there is no temperature gradient on the surface of the surrounding bodies at rest.
The Navier--Stokes equations with any slip boundary conditions have a trivial solution \(u_i = 0\)
that is, however, not valid for~\eqref{eq:asymptotic1}--\eqref{eq:asymptotic3}.
\end{leftbar}

\begin{quoting}
9. The conclusion should be re-written. It is unprofessional.
\end{quoting}

I have rewritten the conclusion entirely by focusing upon the following points.
\begin{itemize}
\item The main achievements.
\item Studied phenomena.
\item The meaning for the field.
\item Possible applications and extensions.
\end{itemize}

\begin{leftbar}
%%% The main achievements
In the present paper, OpenFOAM\textregistered{}, a free and open-source CFD platform,
widely used and rapidly extending, is upgraded to deal with
slightly rarefied gas flows driven by significant temperature variations.
A reliable and rapid solver has been introduced on the basis of the appropriate
equations and boundary conditions, derived from the asymptotic analysis of the Boltzmann equation.
The program code has been validated by means of some benchmark simulations,
presented as illustrations.

%%% Studied phenomena
Typical temperature driven flows of the first order of the Knudsen number have been considered:
thermal creep flow and nonlinear thermal-stress flow.
The force of nonlinear thermal-stress nature, arising in the gas, is of the second order of
the Knudsen number. It has been studied for several problems in detail.

%%% The meaning for the field
OpenFOAM\textregistered{}, together with the newly implemented numerical algorithm,
is a robust tool for numerical analysis of slightly rarefied gas problems
on the kinetic basis. The principal advantage of the developed code is that
it is easily extensible as an open-source software.

%%% Possible applications and extensions
A hard-sphere gas with the diffuse-reflection boundary condition is considered in this paper,
but various molecular models and boundary conditions can be used as well.
In addition, appropriate equations for gas mixtures can be also naturally implemented.
\end{leftbar}

\section{Referee Report \#3}

\begin{quoting}
1. The description ``the thermal conductivity of an ideal gas is
proportional to T'' (page 4, line 7; here, line number indicates the
number in the left margin on each page) is not correct. This is true for
hard-sphere molecules.
\end{quoting}

This is indeed correct. I have added the suggested remark.

\begin{quoting}
2. I do not understand the sentence ``For small k weak convective flow \(v
= O(k)\) appears in the Navier--Stokes set of equations'' (page 4, line 10).
What does it mean? In this paragraph, the author tries to explain what
is the ghost effect, but the result is not successful. This paragraph
should be revised.
\end{quoting}

I agree with the referee that this paragraph is turn to be vague.
As for the convective flow, I intended to describe, why the simple replacement of the limit \(v\to0\)
by the equality \(v=0\) leads to the illegal equations governing the temperature field.
I have rewritten the paragraph as follows.
\begin{leftbar}
A rigorous examination of the continuum limit shows that the infinitesimal thermal conduction term \(\Pder[q_j]{x_j}\)
can be of the same order as the thermal convective term \(\Pder[pv_j]{x_j}\) in the energy equation~\eqref{eq:energy}.
In such a case, when the Mach number is the same order of as the Knudsen number,
the heat-conduction equation~\eqref{eq:heat_equation} fails to describe
the correct temperature field.
\end{leftbar}
As for the ghost effect, I have inserted an additional short paragraph in section 3.
\begin{leftbar}
Then, the following set of equations for a time-independent case (\(\Pder{t} = 0\))
is obtained for variables \(T_0\), \(u_{i1} = p_0v_{i1}\), \(p_2^\dag\):
\begin{align}
    \pder{x_i}\left(\frac{u_{i1}}{T_0}\right) &= 0, \label{eq:asymptotic1} \\
    \pder{x_j}\left(\frac{u_{i1}u_{j1}}{T_0}\right)
        &-\frac{\gamma_1}2\pder{x_j}\left[\sqrt{T_0}\left(
            \pder[u_{i1}]{x_j}+\pder[u_{j1}]{x_i}-\frac23\pder[u_{k1}]{x_k}\delta_{ij}\right
        )\right] \notag\\
        &- \frac{\gamma_7}{T_0}\pder[T_0]{x_i}\pder[T_0]{x_j}\left(\frac{u_{j1}}{\gamma_2\sqrt{T_0}} - \frac{1}4\pder[T_0]{x_j}\right) \notag\\
        &= -\frac{p_0}{2}\pder[p_2^\dag]{x_i} + \frac{p_0^2 F_{i2}}{T_0}, \label{eq:asymptotic2} \\
    \pder[u_{i1}]{x_i} &= \frac{\gamma_2}2\pder{x_i}\left(\sqrt{T_0}\pder[T_0]{x_i}\right). \label{eq:asymptotic3}
\end{align}
\asterism
The set of equations~\eqref{eq:asymptotic1}--\eqref{eq:asymptotic3} revealed the following interesting fact.
In the continuum limit, the gas flows vanish, proportional to the Knudsen number.
However, the infinitesimally weak flows have a finite impact on the temperature field.
This asymptotic behavior is called the \emph{ghost effect}~\cite{GhostEffect, Sone2002, Sone2007}.
\end{leftbar}
Thus, I hope I have succeeded in separating the problem statement from the result.

\begin{quoting}
3. The fact that ``p0 and p1 are constant'' (page 5, line 1) is the
consequence of the degenerated Euler equation \(\Pder[p_0]{x_i} = 0\) and
\(\Pder[p_1]{x_i} = 0\). Therefore, it is natural to put ``p0 and p1 are constant'' before
Eq.~(13) (or even before the definition \(u_{i1} = p_0v_{i1}\)).
\end{quoting}

The referee's advice is really useful to make the statement more consistent.
I have moved it a little higher and given a short explanation.

\begin{leftbar}
The analysis presented below is based on the conventional Hilbert expansion
of the distribution function \(f\) and macroscopic variables \(h\)~\cite{Hilbert1912}:
\[ f = f_0 + f_1k + f_2k^2 + \cdots, \quad h = h_0 + h_1k + h_2k^2 + \cdots \]
under the additional assumption
\begin{equation}\label{eq:Mach_constraint}
    u_i = O(k)
\end{equation}
that means that the Mach number is of the same order as the Knudsen number.
Moreover, let the external force be weak:
\begin{equation}\label{eq:Force_constraint}
    F_i = O(k^2).
\end{equation}
Conditions~\eqref{eq:Mach_constraint} and~\eqref{eq:Force_constraint} make the pressures
\(p_0\) and \(p_1\) constant due to the degenerated momentum equation~\eqref{eq:momentum}:
\begin{equation}
    \pder[p_0]{x_i} = 0, \quad \pder[p_1]{x_i} = 0.
\end{equation}
\end{leftbar}
It is worth mentioning that I have added the external force in the asymptotic analysis of
the Boltzmann equation after major revision.
Thus, a new assumption~\eqref{eq:Force_constraint} has arisen.

\begin{quoting}
4. Are the symbols \(\mu\) and \(\nu\) in line 28 on page 5 dimensionless?
If so, this should be stated clearly. Then, \(\sqrt T_0\) should be
included on the right-hand sides of the equations for \(\mu\) and
\(\nu\) (line 28). By the way, the word ``transport coefficients'' is
more commonly used than ``transfer coefficients''.
\end{quoting}

All these referee's remarks are valid. I have rewritten the ill-defined statements as follows.
\begin{leftbar}
For a hard-sphere gas, the transport coefficients are equal
\begin{alignat*}{2}
    \gamma_1 &= 1.270042427, &\quad \gamma_2 &= 1.922284066, \\
    \gamma_3 &= 1.947906335, &\quad \gamma_7 &= 1.758705.
\end{alignat*}
The first two ones are connected to the dimensional transport coefficients in the following way:
\begin{equation}
    \mu = \gamma_1\sqrt{T_0} \frac{p^{(0)}L}{\sqrt{2RT^{(0)}}} k, \quad
    \lambda = \frac{5\gamma_2}{2}\sqrt{T_0} \frac{p^{(0)}RL}{\sqrt{2RT^{(0)}}} k,
\end{equation}
and \(\gamma_7\) is related to the \emph{nonlinear thermal-stress flow}.
\end{leftbar}
Moreover, to be save from a possible ambiguity,
I have inserted a short remark in the beginning of section~3.
\begin{leftbar}
Unless explicitly otherwise stated, all variables are further assumed to be dimensionless.
\end{leftbar}

\begin{quoting}
5. The definition of \(\eta\) in Eq. (19) is wrong. The stretched normal
coordinate \(\eta\) is defined as \(x_i = k\eta n_i + x_{Bi}\), where \(n_i\) is the unit normal
vector to the boundary pointing into the gas, and \(x_{Bi}\) is the projection
of \(x_i\) on the boundary; \(x_{Bi}\) and \(n_i\) are functions of \(s_1\) and \(s_2\), a
coordinate system on the boundary. The relation \(x_i = k\eta n_i + x_{Bi}\) gives a
coordinate transformation from \((x_1, x_2, x_3)\) to \((\eta, s_1, s_2)\).
\end{quoting}

The referee is correct that this definition was wrong.
Fortunately, this error had not an impact on the numerical results presented in the paper.
Despite the detailed referee's explanation, I have limited myself by a compact remark.
\begin{leftbar}
The relation \( x_i = k\eta n_i(s_1,s_2) + x_{Bi}(s_1, s_2) \) gives a coordinate transformation
from \((x_1,x_2,x_3)\) to \((\eta,s_1,s_2)\).
\end{leftbar}
Moreover, in accordance with the referee's notation for boundary condition,
I have changed mine in the paper: from \(h_w\) to \(h_B\).

\begin{quoting}
6. It is not clear what the items 1, 2, and 3 (lines 35--39, page~6)
mean. I would suggest ``This condition is not generally satisfied because
\begin{enumerate}
\item If the boundary is moving with the velocity of the order of \(k\), then \(u_{i1} \neq 0\).
\item If the boundary temperature is not uniform, then \(u_{i1} \neq 0\) (thermal creep flow).
\item If the isothermal surfaces are not parallel, i.e.,
\[e_{ijk}\pder[T_0]{x_j}\pder{x_k}\left(\pder[T_0]{x_l}\right)^2 \neq 0, \]
then \(u_{i1} \neq 0\) (nonlinear thermal stress flow).''
\end{enumerate}
\end{quoting}

Indeed, the equal sign has a place instead of the sign of inequality in the last formula.
Moreover, I think the effects, thermal creep flow and nonlinear thermal stress flow, should
be discussed separately from this list of reasons. For example, they are introduced in section~4.
So I have corrected the paragraph, but deviated a little bit from the referee's suggestion.
\begin{leftbar}
Equation~\eqref{eq:asymptotic3} converges to~\eqref{eq:heat_equation} if \(u_{i1} = 0\).
In the absence of an external force, there are several reasons for this condition can be violated:
\begin{enumerate}
    \item The boundary is moving: \(u_{B1i} \neq 0 \).
    \item The boundary temperature is not uniform: \(\Pder[T_B]{x_i} \neq 0 \).
    \item The isothermal surfaces are not parallel:
        \begin{equation}\label{eq:equilibrium}
            e_{ijk}\pder[T_0]{x_j}\pder{x_k}\left(\pder[T_0]{x_l}\right)^2 \neq 0.
        \end{equation}
\end{enumerate}
\end{leftbar}

\begin{quoting}
7. In Figs.~5 and 6, does the color scale on the right margin mean the
contour lines of the speed \(|u_{i1}|\)?
\end{quoting}

Yes, it is true. To prevent this ambiguity, I have added a few clarifications to the figure captions.

\begin{leftbar}
The velocity field \(u_{i1}\) with the Knudsen layer correction for \(\Kn=0.01\).
Curves with arrows indicate the direction, contour lines show the magnitude.
\end{leftbar}

\begin{quoting}
8. First, the system of equations that the author solves in this paper
is the correct system to describe the thermal creep flows and nonlinear
thermal stress flows when the temperature variation is large and the
Knudsen number is small. Secondly, when the imposed temperature
variation is large and when the Knudsen number vanishes (i.e., in the
continuum flow limit), the correct temperature field in the gas is
obtained only by this system (not the Navier--Stokes system). These two
points should be discussed separately. The ghost effect is relevant to
the second point. In the present paper, these two points are sometimes
mixed up. Therefore, it is hard to understand the author's claim. For
example, the title of the paper is misleading. It says ``stationary gas
in the continuum limit''. However, Fig.~6 shows the flow at \(Kn=0.01\).
This contradicts the title. Also in Figs.~7--9, the reader only sees the
flows. The title could be, for example, ``Computer simulation of
slightly rarefied gas flows driven by significant temperature variation
and their continuum limit''.
\end{quoting}

I have realized the idea and revised the whole paper to separate these two points.
I have used the proposed title, too.

\begin{leftbar}
\end{leftbar}

\begin{quoting}
9. The language weakness should be addressed.
\end{quoting}

See comment~1 in referee report~\#1.


\bibliographystyle{spphys}
\bibliography{../springer}

\end{document}


