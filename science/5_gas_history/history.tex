%&pdflatex
\documentclass[a4paper, 12pt, oneside]{article} %a4paper
\usepackage[LGR,T2A,T1]{fontenc}	% greek, cyrillic, latin
\usepackage[utf8]{inputenc}
\usepackage[english,german,frenchb,greek,russian]{babel}
\usepackage{fullpage}

\usepackage{gensymb}				% for \celsius
\usepackage{textcomp}				% for \textnumero
\usepackage{epigraph}
\usepackage[pdftex]{graphicx}

% Bibliography
\usepackage{csquotes}
\usepackage[
	backend=biber,
	hyperref=true,
	autolang=other,					% multi-language
	mincrossrefs=100,				% to prevent implicit inserts into bibliography
	maxnames=100,					% print all authors
	style=gost-numeric,
	movenames=false,				% only for biblatex-gost
	sorting=none,
	doi=false
]{biblatex}
\addbibresource[label=first]{first.bib}
\addbibresource[label=second]{second.bib}

\usepackage[
	pdfauthor={Oleg Rogozin},
	pdftitle={History of kinetic gas theory},
	linktocpage=true,
	pdftex,
	unicode
]{hyperref}

% Some tuning to epigraphs
\setlength{\epigraphwidth}{.6\textwidth}
\setlength{\beforeepigraphskip}{0em}

% Indent a first paragraph
\usepackage{indentfirst}

\usepackage{titlesec}
% Each section from a new page
% Use \phantomsection for hyperref
\newcommand{\sectionbreak}{\clearpage\phantomsection}
% Increase spacing after the section titles
\titlespacing*{\section}{0pt}{0pt}{20pt}
% Increase font size in the section titles
% Use raggedright to disable hyphenation
\titleformat*{\section}{\Huge\bfseries\raggedright}
\titleformat*{\paragraph}{\Huge\bfseries\raggedright}

% Table of contents: sections with dots
%\usepackage[titles]{tocloft}
%\renewcommand{\cftsecdotsep}{\cftdot}

% Bibliography without numbering for secondary sources
\defbibenvironment{plain}
	{\list{}
		{\setlength{\leftmargin}{\bibhang}%
		\setlength{\itemindent}{-\leftmargin}%
		\setlength{\itemsep}{\bibitemsep}% 
		\setlength{\parsep}{\bibparsep}}}
	{\endlist}
	{\item}

\newcommand{\person}[1]{\textsc{#1}}
% Prints author names as small caps
%\renewcommand\mkbibnamelast[1]{\person{#1}}
% "hd" only for biblatex-gost
%\renewcommand{\mkbibhdnamelast}[1]{\person{#1}}


\begin{document}

\begin{titlepage}
\begin{center}

\newcommand{\HRule}{\rule{\linewidth}{0.5mm}}

% Header
\includegraphics[width=0.4\textwidth]{mipt_logo}~\\[1cm]
\textsc{\LARGE Московский физико-технический институт}\\[1.5cm]
\textsc{\Large Кафедра философии}\\[0.5cm]

% Title
\HRule \\[0.4cm]
{ \huge \bfseries История кинетической теории газа \\[0.4cm] }

\HRule \\[1.5cm]

% Author and supervisor
\begin{minipage}{0.4\textwidth}
\begin{flushleft} \large
\emph{Автор:}\\
асп. \textsc{Рогозин О.А.}
\end{flushleft}
\end{minipage}
\begin{minipage}{0.5\textwidth}
\begin{flushright} \large
\emph{Научный руководитель:} \\
д. ф.-.м. н. \textsc{Черемисин Ф.Г.}
\end{flushright}
\end{minipage}

\vfill

% Bottom of the page
{\large Москва}\\[0.1cm]
{\large 2014}

\end{center}
\end{titlepage}

\tableofcontents
\newrefsection[first]

\section*{От автора} \addcontentsline{toc}{section}{От автора}

В отличие от первой половины XX века, для которой характерен тотальный
переворот физической картины мира, последние десятилетия не блещут
великими открытиями и научными революциями. Согласно диалектическому
закону развития природы качественное изменение всегда чередуется с
количественным накоплением знаний и опыта. Так и сейчас наступила эра
высоких технологий, девизом которой стала погоня за точностью и
эффективностью.

Настала эпоха переосмысления огромного экспериментально-теоретического
пласта, накопленного человечеством. Нельзя сказать, что все
фундаментальные законы природы уже открыты, но пришло время их
полномасштабного использования. Бурное развитие компьютерной техники
даёт учёным колоссальные вычислительные возможности. Если раньше наука
давала лишь грубые оценки природных явлений и процессов, то сейчас мы
способны моделировать сложнейшие системы с высокой степенью точности.
Международная интеграция и глобализация науки позволила сконцентрировать
ресурсы всего человечества на постановку масштабных проектов,  которые
не под силу ни индивиду, ни группе учёных, ни даже отдельным странам со
множеством научно-исследовательских институтов: погоня за открытием
новых элементарных частиц, попытки детектировать гравитационные волны,
поиск технологий управляемого термоядерного синтеза, методов лечения
раковых заболеваний и многие другие.

Однако кроме этих передовых направлений борьба за новые технологии
касается практически всех областей человеческих знаний, в том числе и
тех, которые принято считать <<прошлым веком>>. Например, до сих пор
активно решаются фундаментальные задачи механики и газовой динамики.
Вспомним хотя бы достижения наших учителей. Виктор Филиппович Журавлёв,
возглавлявший 8 лет кафедру теоретической механики МФТИ, получил
всемирную известность за свои работы в области гироскопических систем. В
1998 году ему удалось впервые точно решить задачу о твёрдотельном
гироскопе, сведя её к эллиптическим квадратурам. 26 сентября 2011 года
ушёл из жизни корифей кинетической теории Михаил Наумович Коган,
читавший много лет одноимённый курс на ФАЛТ. Его классическая монография
<<Динамика разрежённого газа>> явилась в своё время настольной книгой
учёных всех стран мира, занимавшихся этой тематикой.

На следующих страницах Ваш покорный слуга намеревается не только кратко
изложить историю атомистических представлений, но и пролить больше света
на развитие современной кинетической теории газов, которая, несмотря на
свой возраст и зрелость, продолжает активно развиваться. Кто бы мог,
например, поверить, что в XXI веке будут ставить эксперименты по
исследованию течения Пуазёйля, известное всем из школьных учебников, и
вручать Филдсовские премии за развитие работ Больцмана? Тем не менее,
это так.

\section*{Введение}
\addcontentsline{toc}{section}{Введение}

\epigraph{\textit{<<И всё-таки они движутся!>>}}{\person{Людвиг
Больцман} (1844---1906)}

Молекулярно-кинетическая природа вещества "--- одна из наиболее глубоких
фундаментальных идей, на которых зиждется современное технологическое
развитие человечества. Обыгрыш знаменитого восклицания Галилея не
случаен. Становление как атомистических, так и кинетических
представлений в исторической перспективе исполнено борьбой и драмой
многих философов, физиков и математиков.

Впервые уже в Древней Греции и Древней Индии великие умы пытались
строить свои философско-материалистические представления на основе идей
о делимости и внутреннем строении наблюдаемых тел. В Средние века
аристотелевская традиция в рамках схоластического подхода долгое время
препятствовала развитию этого предположения, как и всей научной мысли в
целом. Только некоторые арабские философы продолжали разделять
атомистические идеи, однако лишь на уровне гипотезы, и существенного
вклада не внесли. В Европе к материализму в общем и атомизму в частности
стали возвращаться в эпоху Возрождения, а уже в Новое время были
получены первые научные результаты. Постепенно гениальная догадка стала
превращаться в научную гипотезу, чему во многом способствовало
становление кинетической теории газов. К концу XIX века фундамент теории
был выстроен и подтверждён экспериментальными фактами, но окончательное
принятие атомно-молекулярной гипотезы состоялось только в начале XX
века. Для новейшего времени характерен глубокий теоретический подход,
построение более точных узкоспециализированных моделей и открытие тонких
эффектов. Значительный вклад в современные методы исследований вносит также
стремительный прогресс вычислительной техники.

Атомистическая картина мира возникла как ответ на непостижимость понятия
бесконечность. Эта изначально философская гипотеза служит началом как
элементарной химии, так и современного представления о внутреннем
устройстве материи. Кинетическая теория, выросшая на почве атомизма,
связала крупные разделы современной физики, предоставив более глубокое
понимание фундаментальных законов. Статистическая интерпретация второго
начала термодинамики позволила соединить её с механикой, а кинетическое
рассмотрение гидродинамики "--- очертить границы применимости модели
сплошной среды и дать теоретическое обоснование феноменологическим
коэффициентам процессов переноса. Современное математическое развитие
кинетической теории дало новый толчок в областях асимптотического
анализа и дифференциальных уравнений в частных производных.

При попытке создать предельно краткий исторический обзор в настоящей
работе особое внимание уделялось диалектическим причинам развития
научных представлений и целостности изложения отдельных направлений,
зачастую в ущерб строгой хронологической последовательности.

\section{Натурфилософская атомистика древности}

\epigraph{\textit{<<Если бы в результате какой-то мировой катастрофы все
накопленные научные знания оказались бы уничтоженными и к грядущим
поколениям живых существ перешла бы только одна фраза, то какое
утверждение, составленное из наименьшего количества слов, принесло бы
наибольшую информацию? Я считаю, что это "--- атомная гипотеза: все тела
состоят из атомов "--- маленьких телец, которые находятся в беспрерывном
движении, притягиваются на небольшом расстоянии, но отталкиваются, если
одно из них плотнее прижать к другому. В одной этой фразе, как вы
убедитесь, содержится невероятное количество информации о мире, стоит
лишь приложить к ней немного воображения и чуть
соображения.>>}}{\person{Ричард Фейнман} (1918---1988)}

\subsection{Родина атомистики}

Современная наука не располагает точными знаниями относительно того, как
возникли атомические представления. Эта, казалось бы, простая идея о
том, что всё состоит из мельчайших неделимых частиц, при глубоком
рассмотрении поднимает множество фундаментальных проблем. Вполне уместно
говорить даже о соперничестве между атомистической и континуальной
научно-философскими программами древности.

Более того, нет и ответа на вопрос, где впервые возникла мысль о
дискретном строении вещества. Сегодня большинство историков
придерживаются гипотезы о независимом и параллельном возникновении идей
атомизма в Древней Греции и Древней Индии, хотя несомненно некоторое
взаимное влияние могло иметь место. В любом случае на лицо факт, что
понятие \emph{атома} (от др.-греч. \foreignlanguage{greek}{ἄτομος} "---
неделимый) "--- это не случайная гениальная догадка великих греков,
предвосхитившая современные научные открытия, а определённая
закономерность развития теоретического мышления. Индийская школа
\emph{вайшешики} изучена не так детально, как античная, поэтому
ограничимся качественным их сравнением, а подробно исторический ход
развития атомизма проследим по работам древнегреческих философов.

Ключевым различием философских школ запада и востока является отношение
между теоретическим мышлением и чувственным восприятием. Если образы
греков всегда отличались высокой степенью абстракции (формы, фигуры,
идеи), то вайшешики полагают, что микрокосмос, скрытый от обычных
чувств, может быть постижим не только мыслью, но и воспринят в особых
йогических состояниях. Если античная философия накапливала знания в
основном для практических нужд, то индийская, напротив, стремилась к
познанию природы исключительно в целях совершенствования личности. По
этим причинам именно древнегреческая традиция способствовала становлению
механической картины в Новое время и использованию математического
аппарата для изучения природы.

\subsection{Идея о микроскопических процессах}

Обычно эволюцию молекулярно-кинетической теории начинают сразу излагать
с классических представителей античного атомизма, однако авторство идеи
о том, что макроскопические процессы обусловлены микроскопическими
(говоря уже современным языком), по всей видимости, принадлежит
древнегреческому философу \person{Анаксагору} (ок. 500---428 до н. э.).
В середине V века до н. э. мировая столица того времени город Афины, где
проживал Анаксагор, переживал золотой век. Заметный научно-технический
подъём, расширение сведений о мире способствовали обобщению множества
частных задач. В лице Анаксагора греческая мысль сделала значительный
шаг вперёд по сравнению с предыдущими естественнонаучными
представлениями.

Развивая попытку милетской школы объяснить разнообразие видимых
предметов и явлений через единую первичную стихию (вода, воздух, огонь и
т.д.), Анаксагор пришёл к выводу о существовании однородных, неизменных
и \emph{бесконечно малых} элементов действительного мира, из сочетаний
которых образуются различные тела природы. Таким образом была решена
проблема выделения тел в окружающей среде: не с помощью качественных
особенностей, а на основе субстанциального различия. В своих работах он
называл их <<семенами вещей>>, позже Аристотель ввёл специальный термин
"--- \emph{гомеомерия} (от др.-греч. \foreignlanguage{greek}{ὅμοιος} "---
подобный и \foreignlanguage{greek}{μερίς} "--- доля). Свойства гомеомерий
непосредственно ненаблюдаемы, мы можем судить о них только при
некотором количественном преобладании данных гомеомерий в
рассматриваемом теле. Это первая зачаточная формулировка очень важной
идеи атомистической концепции о связи микро- и макроскопического
описания.

Картина мира, нарисованная Анаксагором имела сильное влияние на
метафизику Аристотеля, но, несмотря на своё новаторство и глубину, была
ещё далека от современной: 1) разнообразие гомеомерий значительно шире,
чем у атомов в нашем смысле; 2) воззрения Анаксагора основаны на идее о
бесконечной дробимости вещества; 3) он отрицал существование пустоты,
как отсутствия гомеомерий в мире вещей.

На основе своего представления о строении вещества Анаксагор впервые
подробно разработал космологическую концепцию \emph{<<мирового Ума>>}
"--- движущего принципа мирового порядка. В космологии Анаксагора ранняя
Вселенная являет собой хаос из беспорядочно смешанных гомеомерий,
который механически приводится в порядок <<Умом>>, заключающим в себе
все закономерности, царящие в природе.

\subsection{Рождение атомной гипотезы}

Следующим шагом в развитии взглядов на строение вещества стали
знаменитые апории \person{Зенона} (ок. 490---ок. 430 до н. э.) о
движении, в которых он в очень острой форме поставил коренные проблемы
непрерывности пространства и времени. Зенон выводил тезис об
иллюзорности движения на основании того, что понятию бесконечность
нельзя приписывать свойства обычных чисел.

Постулирование неделимых частиц "--- атомов "--- позволяло сразу же
разрешить все парадоксы Зенона, просто убирая оттуда все бесконечности.
Первенство этой идеи принято отдавать древнегреческому философу
\person{Левкиппу} (V век до н.э.). Кроме того, он считал, что
бесконечное число атомов находятся в \emph{вечном движении}. Аристотель
позднее критиковал подобную <<беспечность>> атомистов, однако причину
движения атомов Левкипп рассматривает как фундаментальное свойство самих
атомов.

Атомы у Левкиппа обладают бесконечным числом форм, поскольку в природе
нет оснований, чтобы оно было ограничено определённым значением, чтобы
оно было таким, а не иным. Бытие, по его мнению, "--- это атомы, их
сущность состоит в абсолютной плотности и заполненности. Они носятся в
\emph{пустоте} "--- небытии, существующем с той же реальностью, что и
бытие. Таким образом, например, легко объяснить такое свойство тел,
как \emph{сжимаемость}.

Поскольку от самого Левкиппа осталось только имя, считается, что его
основные идеи перенял и развил до стройной и последовательной концепции
его ученик \person{Демокрит} (ок. 460---ок. 370 до н. э.). За свою
долгую жизнь Демокрит побывал во многих странах, накопил колоссальный
объём естественнонаучный сведений. Аристотель и другие мыслители
последующего периода поражались широте знаний Демокрита. Одна главная
мысль пронизывала все его многочисленные труды "--- мысль о сведении всех
явлений природы к перемещению бескачественных частиц. Демокрит
выстраивал механическую картину мира. Даже звёзды у него "--- это
раскалённые глыбы, вращающиеся вокруг плоской Земли, а не отверстия в
небесной сфере, через которые видна стихия огня. Демокрит пришёл к идее
об однородности мирового пространства, однако изотропность у него
отрицалась существованием абсолютного верха и абсолютного низа,
обусловленные фиксированным направлением силы тяжести.

Из высказываний самого Демокрита видно, что, согласно его учению,
чувственно воспринимаемые качества тел не существуют в телах в
действительности. Различия между телами по цвету, теплоте, вкусу и т. д.
не отвечают природе самих вещей. В атомах действительны не эти качества,
а только различия между ними по форме, по величине, по порядку и по
положению в пространстве. Эти качества, различимые у различных атомов, и
пустота, в которой движутся обладающие этими качествами атомы, "--- это
все то, что по истине существует в действительности. Такое разделение
качеств на первичные (основные физические свойства материальных
элементов природы) и на вторичные (непосредственно воспринимаемые
органами чувств) имело большое влияние на становление всего
естественнонаучного воззрения раннего Нового времени.

\subsection{Эпоха эллинов и римлян}

У \person{Платона} (428/427---348/347 до н. э.) тоже были атомы, но не в
телесной, а в геометрической форме. Созданный им мир эйдосов (от
др.-греч. \foreignlanguage{greek}{εἶδος} "--- образ) достиг наивысшего
уровня абстракции, но одновременно в большей степени идеализирован и
оторван от реальности. Можно сказать, что современная наука о микромире
развивается по направлению от Демокрита к Платону, от наглядности
представления к математическим моделям.

Несколько позже широкое распространение получила физическая картина мира
\person{Аристотеля} (384---322 до н. э.), который решение апорий Зенона
свёл к критике и вернулся к континуальному строению вещества. Аристотель
хорошо был знаком с учениями Демокрита и Платона, но более всего он
доверял своим собственным наблюдениям. Их невиданная широта в сочетании
с диалектическим гением сделали Аристотеля самым авторитетным философом
Древности, оставившим наибольшее натурфилософское наследие. Фундаментом
аристотелевского учения стала целесообразность всех явлений природы, а
безликие атомы, движущиеся по произвольным траекториям, явно выходили за
рамки такой картины.

В эллинистический период атомистика Демокрита возродилась в учении
\person{Эпикура} (342/341---271/270 до н. э.), причём вышла на
качественно новый виток развития. Чтобы избавиться в своём этическом
учении от фатализма, вытекающего из механической причинности всякого
движения, Эпикур впервые связал воедино понятия случая и необходимости,
свободы и детерминизма в самом понятии атома.

Поскольку в то время Земля всё ещё считалась плоской, то античные
атомисты постулировали однородность пространства наряду с выделенным
направлением вдоль вектора силы тяжести. В отличие от Демокрита Эпикур
наделил атомы весом, что заставляло их двигаться параллельно друг другу
вдоль этого выделенного направления. Далее он вводил возможность
спонтанного (не вызванного механической необходимостью) поперечного
смещения атомов, которое приводило к их столкновению и образованию
видимых тел. Это положение, именуемое \emph{clinamen} (от лат.
<<уклонение>>), было основой строго материалистического учения о
сознании и свободе воли и вызывало наибольшее пренебрежение более
поздних мыслителей при физическом рассмотрении концепции Эпикура. По
словам Марка Цицерона(106---43 до н. э.): <<ничего более позорного не
может случиться с физиком>>. Только в 1841 году \person{Карл Маркс}
(1818---1883) задолго до развития квантовой механики в своей диссертации
<<Различие между натурфилософией Демокрита и натурфилософией
Эпикура>>~\cite{marx1956difference} прервал эту традицию, указав на
фундаментальную и философскую глубину идей Эпикура, значительно
опередивших своё время.

Гениальность Эпикура в качественном развитии атомной гипотезы
заслуживает глубокого пиетета. Кроме случайного движения в текстах
косвенно прослеживается его сомнение в тождественности движущихся
неделимых частиц. Эпикур, признавая непрерывность времени в
макроскопических масштабах, говорит о неделимых отрезках времени на
микроуровне, на протяжении которых есть только результат движения, но не
само движение. Всё же понятиями вероятности и статистического ансамбля,
связывающих макро- и микромир, Эпикур не оперировал.

Через Эпикура атомистическая гипотеза передалась от Древней Греции к
Древнему Риму, где она была наиболее полно воспета в знаменитой поэме
\person{Лукреция Кара} (ок. 99---55 до н. э.) <<О природе
вещей>>~\cite{kar1983nature}. Собственно, это произведение служит
главным источником наших знаний об учении Эпикура. У Лукреция можно
встретить характерный аргумент в защиту дискретности вещества "---
конечное число сочетаний атомов. Если бы материя была бесконечно
дробима, то в каждом теле бесконечное число его бесконечно малых
элементов могло бы образовывать бесконечное число сочетаний. В этом
случае в мире не было бы возврата к старым сочетаниям, что противоречит
реально наблюдаемому сохранению форм материи.

Конечно, философские изыскания античных философов были ещё далеки от
настоящего физического описания, но самая большая ценность "--- это те
вопросы, которые древние умы поставили перед будущим поколением
исследователей.

\section{Эволюция атомно-молекулярной гипотезы}

\epigraph{\textit{<<Одна из принципиальных целей теоретического
исследования в любой области знания состоит в том, чтобы найти ту точку
зрения, с позиции которой изучаемый объект проявляется в своей
величайшей простоте.>>}}{\person{Джозайя Гиббс} (1839---1903)}

\subsection{От метафизики к первым научным результатам}

В Средние Века атомистическая гипотеза была принята лишь некоторыми
арабскими учёными, а в западноевропейской культуре телеологический
принцип Аристотеля перерос в схоластическую теологию. Её власть была
слишком обширна, чтобы давать развиваться другим научно-философским
учениям, более того Средневековье характеризуется утратой значительной
части античного наследия. Дошедшим до наших дней трудам древнегреческих
философов мы во многом обязаны арабской цивилизации, впитавшей в себя
учение Аристотеля через переводы.

В период Ренессанса античная идея атомизма получила новое дыхание,
сначала в философской форме в работах многих мыслителей. Полноценное
возвращение к античной атомистике обычно связывают с именем французского
философа \person{Пьера Гассенди} (1592---1655). Он был одним из немногих
учёных XVII века, интересовавшихся историей науки, в частности, в своих
работах Гассенди глубоко изучил представления Эпикура и был их горячим
пропагандистом~\cite{gassendi1966writings}. Гассенди считал, что
первоначально бог сотворил определённое число атомов, из которых и
состоят все существующие тела. По Гассенди, не только материальные тела,
но и \emph{невесомые флюиды}, в частности свет и теплота, состоят из
атомов. Гассенди также принимал существование особых атомов тепла и
холода. Атомы обладают определённой формой, они абсолютно твёрдые и
непроницаемые, характеризуются тяжестью и размерами, непрерывно движутся
в пустоте и сталкиваются. Пустота имеется во всех твёрдых и жидких телах
в виде пор. Все тела состоят не непосредственно из атомов, а из их
сростков, которые Гассенди назвал \emph{молекулами} (от лат. moles "---
масса).

Современник Гассенди \person{Рене Декарт} (1596---1650) также полагал,
что материя состоит из мельчайших частиц, недоступных чувственному
восприятию. Размерам, форме и движению этих частиц Декарт приписывал
основную роль при объяснении разнообразных физических и химических
явлений. В отличие от Гассенди, Декарт принимал, что материя может
делиться до бесконечности, наполняя всё мировое пространство. Он был
противником существования пустоты и утверждал, что сами по себе частицы,
составляющие тела, не обладают тяжестью. Свои частицы Декарт представлял
как вихревые образования в пространстве, заполненном первичной материей
"--- \emph{физическим эфиром} (от др.-греч.
\foreignlanguage{greek}{αἰθήρ} "--- верхний слой воздуха). В сравнении с
материализмом Гассенди, учение Декарта~\cite{descartes1950selected}
следует называть \emph{кинетизмом} (от др.-греч.
\foreignlanguage{greek}{κίνησις} ---
движение)~\cite{kaydakov1998theological}.

Окончательным разгромом аристотелевской динамики можно считать выход в
1638 году знаменитой книги итальянского \person{Галилео Галилея}
(1564---1642) «Беседы и математические доказательства двух новых
наук»~\cite{galileo1964discussions}, где он заложил основы
\emph{кинематики} и пришёл к первому пониманию \emph{закона сохранения
энергии в поле тяжести}. Его ученик и преемник \person{Эванджелиста
Торричелли} (1608---1647) в 1644 году развил теорию \emph{атмосферного
давления}~\cite{torricelli1644opera}, показав, что оно имеет
механическую природу.

Непосредственно история газовых законов берёт начало с работ британского
химика и физика \person{Роберта Бойля} (1627---1691). В отличие от
Гассенди, Бойль называл молекулы и атомы \emph{корпускулами} (от лат.
corpus "--- тело). Особое значение Бойль придавал порам, будто бы
имеющимся в твёрдых и жидких телах. Эти поры, по его мнению, заполнены
очень мелкими корпускулами, которым и принадлежит определяющая роль в
различных химических и физических явлениях. В 1662 году он
установил~\cite{boyle1662new}, что воздух обладает упругими свойствами:
сопротивляется сжатию и заполняет предоставленный ему объём. Пытаясь
объяснить открытое явление, Бойль предположил, что корпускулы воздуха
отталкиваются друг от друга. Независимо к такому же соотношению в 1676
году пришёл французский физик \person{Эдм Мариотт}
(1620---1684)~\cite{mariotte1676new2}. Полученный \emph{закон
Бойля"--~Мариотта} описывает изотермическое поведение идеального газа.

\subsection{Динамический атомизм}

В 1686---87 годах наступила революция в естествознании вместе с выпуском
трёхтомника <<Математические начала натуральной
философии>>~\cite{newton1989principia} \person{Исаака Ньютона}
(1642---1727), в котором были сформулированы закон всемирного тяготения
и три закона движения. В 1704 году выходит в свет вторая знаменитая
монография Ньютона под названием <<Оптика>>~\cite{newton1927opticks}, в
которой он пишет: <<Мне кажется вероятным, что бог вначале дал материи
форму твёрдых, массивных, непроницаемых, подвижных частиц... Они никогда
не изнашиваются и не разбиваются на куски. Никакая обычная сила не
способна разделить то, что создал сам бог при первом творении.>>
Постулируя атомистическую картину мира в глубоко религиозном свете,
Ньютон в то же время не разделял кинетического подхода, а напротив
утверждал, что частицы взаимодействуют с определёнными силами,
подчиняющимися открытым им законам. С философской точки зрения такой
Ньютоновский \emph{динамизм} (от фр.
\foreignlanguage{french}{dynamisme}) перевернул Декартовский кинетизм с
ног на голову: понятие силы стало первостепенным по сравнению с
движением.

Происхождение сил межчастичного взаимодействия оставалась загадкой, но
было очевидно, что в твёрдых телах это силы притяжения, а в газах "---
отталкивания. Такой подход привёл к созданию \emph{статической теории
газов}, в которой элементарные частицы практически не изменяют своего
местоположения (кроме конвекции), а давление газа определяется
исключительно дальнодействующим взаимодействием покоящихся частиц. Чтобы
удовлетворять закону Бойля, силы отталкивания в газе должны быть обратно
пропорциональны расстоянию между частицами (это Ньютон показал ещё в
<<Началах>>). Заметим, что ни Ньютон, ни Бойль не утверждали, что такое
описание "--- единственно верная теория, объясняющая давление газа, но
авторитет Ньютона был настолько велик, что на протяжении всего XVIII
века и первой половины XIX века царствовала заложенная им научная
программа интерпретации всех физических явлений совокупностью сил между
частицами.

Вопрос природы тяготения Ньютон также намеренно обходил стороной, <<не
измышляя гипотез>>. В 1690 году~\cite{duillier1690letter} швейцарский
математик и, пожалуй, единственный друг Ньютона \person{Никола Фатио де
Дюилье} (1664---1753) предложил кинетическую модель гравитации, где
первопричина притяжения есть результат движения крошечных частиц,
движущихся по всей Вселенной. Позже, начиная с 1748 года,
развил~\cite{lesage1784lucrece} эту идею женевский физик
\person{Жорж-Луи Лесаж} (1724---1803). Это механическое объяснение
гравитации никогда не получало широкого признания, однако весьма
положительно повлияло на развитие кинетических идей.

Континентальная линия динамизма была в то время представлена в лице
влиятельного немецкого мыслителя \person{Готфрида Лейбница}
(1646---1716). Если Ньютон не придавал особого значения законам
сохранения в пользу главенства концепции силы, то Лейбниц развил выводы
Галилея и в 1686 году предоставил математическую формулировку
\emph{закона сохранения механической
энергии}~\cite{leibniz1982descartes}, которую он позже называл
\emph{живой силой} (\emph{vis viva})~\cite{leibniz1982dynamics}. В
отличие от сугубо научного подхода Ньютона, Лейбница интересовали также
философские аспекты новой механической картины мира. Здесь Лейбниц
соединил континуальное учение Декарта с динамизмом Ньютона и
постулировал, что протяжённость "--- результат действия непротяжённой
динамической субстанции, вернее, множества отдельных субстанций, которые
он в 1697 году назвал \emph{монадами}~\cite{leibniz1982monadology}.
Монады (от др.-греч. \foreignlanguage{greek}{μόνος} "--- единичный) —--
это не геометрические точки, поскольку последние предполагают
существование пространства, а монады сами, по словам Лейбница, создают
пространство. Более того, Лейбница не устраивала перспектива серого и
однообразного мира механического материализма, поэтому в поисках
объективной причины движения частиц он сделал шаг в сторону идеализма и
наделил монады духовным началом, приписав им свойства внутренней
активности и неповторимости.

Сербско-хорватский учёный \person{Руджер Бошкович} (1711---1787)
соединил идею Лейбница о непространственных монадах с понятием силы у
Ньютона, что привело к созданию \emph{динамической атомистики}. В 1758
году в своей <<Теории натуральной философии, приведённой к единому
закону сил, существующих в природе>>~\cite{boscovich1966philosophiae}
Бошкович обобщил закон тяготения (сохраняя его справедливость на больших
расстояниях) на взаимодействие атомов в виде точечных масс. На предельно
малых расстояниях массы отталкиваются, не позволяя атомам совпасть, а в
промежуточной области силы могут быть как отталкивающими, так и
притягивающими, меняя своё направление несколько раз по мере изменения
расстояния между атомами. С помощью такой модели Бошкович смог
количественно и качественно объяснить большинство известных на то время
свойств материи, значительно преуспев в рамках научной программы
Ньютона. Несмотря на свой масштаб и глубину, <<единый закон сил>> не
был понят и принят современниками. Во-многом потому что Бошкович первым
попытался с помощью одного закона объединить свойства атомов в различных
состояниях вещества, что противоречило устойчивому убеждению среди
физиков XVIII века о различной природе атомов в твёрдых, жидких и
газообразных телах. В целом стоит отметить, что теория Бошковича,
являясь наиболее систематической и стройной атомистической теорией,
повлияла значительно на взгляды учёных уже XIX века.

В рамках ньютоновской программы тепловые процессы успешно толковались с
помощью концепции \emph{теплорода}. Теплота, согласно этой теории,
является не мерой движения частиц, а представляет собой невидимую
субстанцию, состоящую из взаимно отталкивающихся частиц, которые при
этом притягиваются к частицам обыкновенной материи. Считалось, что
температура пропорциональна плотности теплорода, а трение выдавливает
его, производя нагрев. Теплород, как некий невесомый флюид, описывающий
все тепловые явления, принимался многими учёными в той или иной форме с
начала XVIII века, но формальная его теория была сформулирована
основателем современной химии \person{Антуаном Лавуазье} (1743---1794).
В 1783 году Лавуазье~\cite{lavoisier1783reflexions} поставил
окончательную точку между в пользу теплорода против конкурирующей
химической гипотезы горючего флюида "--- \emph{флогистона} (от греч.
\foreignlanguage{greek}{φλογιστός} "--- горючий), а в 1789 году издал
свой знаменитый учебник <<Элементарный курс
химии>>~\cite{lavoisier1793traite}, где теплород выделен как
основополагающий элемент наряду с азотом, кислородом, водородом и
светом. К своём <<курсе химии>> Лавуазье также обосновал и ввёл в
научный обиход \emph{принцип сохранения вещества}. В лице наиболее
влиятельного теоретика конца XVIII и начала XIX веков
\person{Пьера-Симона Лапласа} (1749---1827) французская школа
математической физики продвинула теорию теплорода до апофеоза её
развития~\cite{laplace1824attraction}. Её успех связан также с тем, что
она хорошо объясняла широкий круг известных в то время явлений:
равновесные процессы и фазовые переходы, адиабатические процессы и
распространение звука.

Значительный эмпирический удар по теории теплорода был совершён только в
1798 году, когда \person{Бенджамин Томпсон} (1753---1814) обнаружил, что
при высверливании канала в пушечном стволе, выделяется большое
количество теплоты~\cite{thompson1798heat}. Он родился в Америке, а
переехав в Германию, стал графом Румфордом. Своими опытами Румфорд
показал явную связь между механической работой и внутренней энергией,
что явилось важным подспорьем в принятии \emph{закона сохранения
энергии} в тепловых процессах. Румфорд, выступая против теории теплорода
предполагал молекулярное строение вещества, но был не в силах объяснить
процесс передачи тепла с кинетической точки зрения. Отсутствие
количественного анализа полученных результатов также не способствовало
тому, чтобы поставить окончательную точку в пользу кинетической теории.

\subsection{Химический атомизм}

Развитие молекулярного представления о веществе продолжилось в большей
мере химическим научным сообществом. Основатель современной молекулярной
химии англичанин \person{Джон Дальтон} (1766---1844) придерживался
статической модели газа. В 1803 году он дополнил её предположением, что
различные атомы могут притягиваться, а одинаковые всегда
отталкиваются~\cite{dalton1805absorption}. Так Дальтон объяснил тот
факт, что в воздухе под действием силы тяжести не разделяются атомы
азота и кислорода. С помощью предложенной модели он также обосновал
названный позже в свою честь \emph{закон о парциальных давлениях}. Тогда
же он высказал гипотезу об атомарности химических соединений.

В 1808 году Дальтон представил свои знаменитые атомные
диаграммы~\cite{dalton2010new}. Дальтон исходил из принципа простоты,
поэтому <<атом>> воды у него состоял из одного атома кислорода и одного
водорода. В том же году французский химик \person{Жозеф Гей-Люссак}
(1778---1850) открыл \emph{закон объёмных
отношений}~\cite{gay1809memoire}, показав, что объём газа пропорционален
числу молекул и не зависит от их сорта, что противоречило гипотезе
Дальтона, согласно которой частицы находятся в тесном контакте и имеют
различный размер.

Спор в пользу Гей-Люссака был разрешён итальянским химиком
\person{Амадео Авогадро} (1776---1856), который сформулировал свой
знаменитый закон в 1811 году~\cite{avogadro1811essai}, приведя
химическую атомную теорию к современному виду. Во времена Авогадро его
гипотезу нельзя было доказать теоретически, поэтому многие химики долгое
время исходили из устаревших таблиц Дальтона. Признание
атомно-молекулярной модели случилось только на международном съезде
химиков в 1860 году, на котором итальянец \person{Станислао Канниццаро}
(1826---1910) ясно изложил \emph{закон Авогадро} в свете последних
успехов кинетической теории~\cite{hartley1966stanislao}.

Окончательная точка в принятии атомно-молекулярной гипотезы была
поставлена только в начале XX века с построением теории броуновского
движения. В 1827 году это явление открыл шотландский ботаник
\person{Роберт Броун} (1773---1858), наблюдая в микроскоп движение
цветочной пыльцы, взвешенной в воде~\cite{brown1828motion}. Полная
теория броуновского движения была дана в 1905 году \person{Альбертом
Эйнштейном} (1879---1955) и независимо в 1906 году польским физиком
\person{Марианом Смолуховским} (1872---1917)~\cite{brownian1936motion}.
Экспериментальное подтверждение полученных в рамках кинетической теории
соотношений было дано в серии опытов 1908---1909 годов французского
физика и лауреата Нобелевской премии \person{Жана Перрена}
(1870---1942)~\cite{perrin1909movement}.

\section{Зарождение термодинамики}

\epigraph{\textit{<<Теория производит тем большее впечатление, чем проще
её предпосылки, чем разнообразнее предметы, которые она связывает, и чем
шире область её применения. Отсюда глубокое впечатление, которое
произвела на меня классическая термодинамика. Это единственная теория
общего содержания, относительно которой я убеждён, что в рамках
применимости её основных понятий она никогда не будет
опровергнута.>>}}{\person{Альберт Эйнштейн} (1879---1955)}

На рубеже XVIII и XIX веков французские химики \person{Жак Шарль}
(1746---1823), \person{Жозеф Гей-Люссак} (1778---1850) и англичанин
\person{Джон Дальтон} (1766---1844) экспериментально подтвердили простое
соотношение между давлением, плотностью и температурой идеального газа
для изохорических (\emph{закон Шарля}) и изобарических (\emph{закон
Гей-Люссака}) процессов. Работы Шарля относятся ещё к 1787 году (об этом
пишет в своей работе Гей-Люссак~\cite{gay1802recherches}), но так и не
были им опубликованы, Гей-Люссак независимо от Дальтона (который сделал
открытие годом ранее) выполнил наиболее точные измерения в 1802
году~\cite{gay1802recherches}, получив близкую к современному значению
\emph{температуру абсолютного нуля} (–266,7\celsius). Он же в 1807 году
экспериментально обнаружил~\cite{gay1807premier}, что адиабатическое
истечение газа в вакуум является изотермическим процессом, что явно
противоречило теории теплорода, однако открытое им аномальное явление не
привлекло должного внимания. Закон Гей-Люссака на самом деле следует
называть именем французcкого физика-изобретателя \person{Гийома
Амонтона} (1663---1705), который установил его ещё в 1702 году во время
конструирования своего воздушного
термометра~\cite{amontons1703thermometre}. Расчёты Амонтона веком ранее
давали значение абсолютного нуля –293,8\celsius, при которой, по его
словам, воздух должен полностью терять свою упругость.

Промышленная революция в европейских странах началась с середины XVIII
века и вела за собой бурное развитие паровых машин. На вопрос, как
построить наиболее эффективную тепловую машину, первым с научной точки
зрения ответил французский физик \person{Сади Карно} (1796---1832). В
своей единственной опубликованной работе от 1824
года~\cite{carnot1923reflections} Карно заложил основы
\emph{термодинамики}. Несмотря на то что эта работа основана на модели
теплорода, Карно удалось получить основополагающие результаты в
исследовании КПД тепловых машин, идеальных циклов и обратимости
термодинамических процессов. Кроме того, он приблизился к формулировке
второго начала термодинамики, постулировав невозможность вечного
движения. В 1878 году были опубликованы дневники Карно, где он уже
критически относится к теории теплорода в пользу кинетической модели и
формулирует всеобщий закон сохранения
энергии~\cite{carnot1923reflections}. За свою короткую жизнь Карно не
стал широко известным учёным, но его идеи распространил товарищ по
школьной скамье \person{Эмиль Клайперон}
(1799---1864)~\cite{clapeyron1834memoire}. На этом пике французское
первенство в развитии теории теплоты окончилось, в дальнейшем на работы
Сади Карно ссылались исключительно английские и немецкие физики.

Открытие \emph{первого начала термодинамики} принадлежит немецкому врачу
и естествоиспытателю \person{Роберту Майеру} (1814---1878). Интересно,
что предпосылкой к изучению вопросов термодинамики было исследование
цвета крови пациентов. Майер заметил, что разница в цвете венозной и
артериальной крови находится в соотношении с разницей температур тела и
окружающей среды. В 1842 году он опубликовал
работу~\cite{mayer1933forces}, в которой указал на эквивалентность
затрачиваемой работы и производимого тепла. В 1845
году~\cite{mayer1933organic}, исходя из экспериментальных данных об
удельной теплоёмкости идеального газа, он вычисляет \emph{механический
эквивалент тепла}, пользуясь соотношением позже названным в его честь.

В это же время английский физик \person{Джеймс Джоуль} (1818---1889),
родившийся неподалёку от Манчестера "--- сердца индустриального развития
Англии, проводит свою знаменитую серию экспериментов (1843---1850) для
прецизионного определения механического эквивалента тепла и независимо
от Майера приходит к закону сохранения. В течение этих лет, непрерывно
совершенствуя технику, Джоулю удалось достичь небывалой для своего
времени точности измерений "--- 3 милликельвина~\cite{joule1989equivalent}.

На самом деле в эти годы множество авторов внесли тот или иной вклад в
установление всеобщего закона сохранения энергии. Причиной такого
лавинообразного открытия был особый научный климат, сформировавшийся в
первой половине XIX века. Можно выделить несколько основных его
особенностей: 1) открытие различных процессов преобразования энергии, 2)
направление ньютоновской программы, 3) научный анализ машин, в
частности, развитие концепции <<работа>>, 4) влияние немецкой
натурфилософии, особенно идеи о существовании единого принципа, лежащего
в основе всех явлений природы.

В 1848 году английский физик \person{Уильям Томсон}, более известный как
\person{лорд Кельвин} (1824---1907), детально изучив работу Карно,
указал на удобство использования \emph{абсолютной шкалы
температуры}~\cite{thomson1849carnot}. Кельвин заново открывает идею об
абсолютном нуле температуры (спустя полтора века после Гийома Амонтона)
и вводит в обиход \emph{термодинамическую шкалу температуры}. В 1851
году он приходит к первому формальному изложению \emph{второго начала
термодинамики}~\cite{thomson1853dynamical}, говоря про невозможность
производства работы путём охлаждения самого холодного тела в данной
системе. Годом позже Кельвин заявляет о существовании об универсальной в
природе тенденции к диссипации механической
энергии~\cite{thomson1852dissipation}.

Тем временем в 1850 году немецкий физик \person{Рудольф Клаузиус}
(1822---1888), анализируя всё ту же работу Карно, также приходит к
существованию второго начала, рассуждая про стремление тел к
уравновешиванию температур~\cite{clausius1850carnot}. В 1854 году он
даёт строгую формулировку второго начала близкую к
современной~\cite{clausius1854second}, а в 1865 году вводит созвучное с
энергией понятие \emph{энтропии} (от др.-греч.
\foreignlanguage{greek}{ἐντροπία} "--- превращение) для количественного
описания необратимости термодинамической
системы~\cite{clausius1865entropy}. Так появляется закон неубывания
энтропии замкнутых систем.

\section{Непризнанные пионеры кинетической теории}

\epigraph{\textit{<<Стоит сказать, что молодому автору, который находит
себя способным для великих дел, было бы неплохо обеспечить
благоприятствующее признание в научном мире работой, рамки которой
ограничены и значение которой легко оценить, прежде чем приступать к
более высоким полётам.>>}}{\person{Лорд Релей} (1842---1919)}

\subsection{Первые попытки XVIII века}

В 1738 году швейцарский физик-математик \person{Даниил Бернулли}
(1700---1782) сформулировал первые количественные законы
\emph{кинетической теории} газов в своём фундаментальном труде
<<Гидродинамика>>~\cite{bernulli1959hydro}. Термин <<кинетический>>
используется, чтобы подчеркнуть центральную роль молекулярного движения
по сравнению со статической моделью. Бернулли удалось объяснить
макроскопические свойства газа (в частности, закон Бойля"--~Мариотта) на
основе динамических законов взаимодействия микроскопических частиц с
помощью статистического подхода. Он представлял газ моделью
\emph{бильярдных шаров}, интерпретировал давление как результат
столкновения молекул, а понятие теплоты связал с кинетической энергией
частиц идеального газа.

Несмотря на то что <<Гидродинамика>> Бернулли была широко известна среди
современников, содержащиеся в ней кинетические идеи не получили должного
распространения. Изложенная швейцарцем теория, разумеется, была ещё
далека от своей зрелости, но что более важно, не объясняла и не
предсказывала новых явлений, выделяющих кинетический подход среди
других. Можно сказать, что она значительно опередила своё время:
экспериментального пласта знаний было ещё не достаточно для отказа от
общепринятых концепций в пользу кинетической парадигмы. Передовые идеи
Бернулли не оказали практически никакого влияния на современников и были
утеряны в анналах физики вплоть до 1859 года, когда <<Гидродинамика>>
была переведена на немецкий язык~\cite{bernoulli1859daniel}. Сегодня же
Бернулли заслуженно признаётся основателем кинетической теории.

Первый российский учёный-естествоиспытатель \person{Михаил Ломоносов}
(1711---1765) разработал собственную корпускулярную теорию строения
вещества, которая легла в основу всего его физико-химического наследия.
Можно смело заявить, что она явилась лучшим выражением
атомно-молекулярного учения во всей научной литературе XVIII века.
Исходным пунктом \emph{корпускулярной философии} Ломоносова была критика
учения Лейбница (труды которого он высоко ценил) о непротяжённых
сущностях. Ломоносов стремился полностью изгнать метафизические
спекуляции, присущие в духовным монадам, из естествознания. В понятие
корпускулы Ломоносов вкладывал собрание элементов, образующее одну малую
массу.

Ещё в студенческие годы он отверг господствующие на то время
алхимические представления, а в 1741 году поставил цель создать курс
теоретической химии на основе атомно-молекулярной
теории~\cite{lomonosov1951chemistry}. Следующим шагом было
распространение корпускулярного учения на теорию теплоты. Ломоносов был
знаком с <<Гидродинамикой>> Бернулли, поскольку именно в Петербургской
академии наук она в основном была подготовлена. В отличие от
поступательного движения Бернулли в основе учения Ломоносова о теплоте
лежала мысль о внутреннем вращательном (\emph{коловратном}) движении
материальных корпускул. Ломоносов не просто предложил иной подход к
пониманию теплоты, но и выступил с резкой критикой господствующей теории
теплорода. В 1744 году Ломоносов представляет свою диссертацию
<<Размышления о причине теплоты и
холода>>~\cite{lomonosov1951speculations}, которая была встречена
холодно: <<...похвально прилежание и желание г. адъюнкта заняться
теорией теплоты и холода, но... кажется, что он слишком рано взялся за
дело, которое, по-видимому, пока ещё превышает его силы.>> В этой работе
с помощью теоретических размышлений Ломоносов пришёл к ключевым
открытиям следующего столетия: принцип сохранения вещества и движения,
второе начало термодинамики, существование и недостижимость абсолютного
нуля температуры (<<величайший холод>>).

Перед публикацией диссертации на латинском языке (1750 год) Ломоносов в
своём письме к \person{Леонарду Эйлеру} (1707---1783) от 1748
года~\cite{lomonosov1951letter} выразил свои опасения: <<Хотя... всю
систему корпускулярной философии мог бы я опубликовать, однако боюсь,
как бы не показалось, что я даю учёному миру незрелый плод скороспелого
ума, если я выскажу много нового, что по большей части противоположно
взглядам, принятым великими мужами>>. В этом же письме можно увидеть
первую формулировку общего закона сохранения энергии задолго до Майера и
Джоуля. Работы Ломоносова имели широкий резонанс в европейской науке, но
в большей степени были отвергнуты.

В XVIII веке кинетическая теория не могла была принята научным
сообществом по двум фундаментальным причинам: во-первых, не был
установлен общий закон сохранения энергии, во-вторых, не ясна была
возможность абсолютно упругих столкновений. К началу XIX века ситуация
изменилась. Кроме открытия первого начала термодинамики, накопились
экспериментальные противоречия как со статической теорией газа, так с
теорией теплорода.

\subsection{Провал английской школы}

В 1812 году один из основателей электрохимии \person{Гемфри Дэви}
(1778---1829) первым из английского королевского общества выступил с
критикой ортодоксальной статической теории газа в пользу динамического
описания теплоты~\cite{davy1812elements}. Дэви предположил, что в
твёрдых телах преобладают колебательные движения, а в газах "---
вращательные вокруг собственной оси. В 1820 году Дэви стал президентом
королевского общества, что казалось приведёт к распространению
молекулярных представлений и бурному развитию кинетики, но история
распорядилась иначе.

Два англичанина \person{Джон Герапат} (1790---1868) и \person{Джон
Ватерстон} (1811---1883) независимо выстроили основы кинетической теории,
развивая идеи \emph{теории гравитации Лесажа}. Герапат в 1820 году и Ватерстон в
1845 году отправляли свои оригинальные работы в издательство журнала
<<Философские труды Королевского общества>>, где получили отказ во
многом по причине своей малой известности в научном сообществе.

Рецензентом статьи Герапата был как раз Гемфри Дэви, который хоть и
разделял динамический подход к теплоте, но не мог принять факт
существования абсолютного нуля температуры. Кроме того, постулирование
движения частиц в пустом пространстве между столкновениями казалось
возвращением к античной метафизике, к которой Дэви испытывал отвращение.
Будет не совсем верным возложить вину полностью на Дэви, поскольку
работа Герапата не отличалась своей стройностью и гармоничностью, дабы
убедить всё научное сообщество в необходимости пересмотреть устоявшиеся
взгляды.

Несмотря на отказ Герапат опубликовал свою работу в скромном журнале в
1821 году~\cite{herapath1821mathematical} и продолжил её развитие уже в
собственном <<Железнодорожном журнале>>. В 1836 году он впервые вычислил
среднюю скорость молекул водорода при нормальных
условиях~\cite{herapath1836principles}. В
1847 году Герапат существенно переработал свою теорию, дав
количественное описание последних экспериментальных знаний: истечение
газа через узкое отверстие и давление плотного газа за рамками закона
Бойля~\cite{herapath1847mathematical}.

Важно отметить, что понятие температуры у Герапата было связано не с
кинетической энергией атомов, а с модулем импульса. Дело в том, что до
сих пор не было понятно, каким образом с энергетической точки зрения
могут происходить столкновения, зато импульс всегда сохраняется. Однако
при таком рассмотрении температура смеси несколько ниже, чем
в классическом случае, а также нарушается закон Авогадро.

В 1847 году \person{Джеймс Джоуль} (1818---1889) самостоятельно приходит
к кинетическому рассмотрению теплоты~\cite{joule2003matter}, а годом
спустя знакомится с работой Герапата. По-видимому, Джоуль был
единственным крупным учёным того времени, прочитавшим эту монографию. В
1851 году выходит статья, где для объяснения тепловых явлений в газах
Джоуль использует модель Герапата~\cite{joule1851remarks}:
микроскопические твёрдые шарики, столкновения которых создают давление.
Джоуль правильным образом связывает меру теплоты с кинетической энергией
шариков и, повторяя вывод Герапата, находит более точные оценки
молекулярных скоростей. Однако эта работа была опубликована в
непопулярном журнале и не получила распространение в научных кругах до
тех пор, пока на неё не сослался Рудольф Клаузиус в 1857
году~\cite{clausius1857heat}.

Существенно более зрелая и обширная работа Ватерстона была отклонена
рецензентами банально по причине несогласия с исходными предпосылками.
По правилам тех времён прочитанная статья не возвращалась автору и
оставалась лежать в архивах издательства <<Философских трудов>>. На неё
только в 1891 году обратил внимание \person{лорд Релей} (1842---1919),
занимавший в то время пост главного редактора <<Философских трудов>>, и
опубликовал годом спустя~\cite{waterston1892physics}.

В 1851 году на ежегодном заседании Британского научного сообщества
Ватерстон всё-таки презентовал небольшой доклад. В опубликованной
аннотации~\cite{waterston1851general} установлено первенство открытия
\emph{закона равнораспределения} кинетической энергии в смеси газов. В
1858 году Ватерстон показал, что расчёт Лапласом скорости звука не
является подтверждением теории теплорода, а также согласуется с
кинетическим представлением~\cite{waterston1858sound}.

Подобное пренебрежительное отношение к рецензированию работ
малоизвестных учёных стоило, по-видимому, больше десятилетия развитию
всей молекулярной физики, а слава первооткрывателей современной
кинетической теории ушла из рук англичан к немецкому научному
сообществу.

В течение более века одиночные вспышки кинетической теории затухали, не
успев разгореться, но диалектический закон развития природы необратимо
вёл её к своему становлению, необходимые условия которого были
обеспечены только во второй половине XIX века.

\section{Становление современной кинетической теории}

\epigraph{\textit{<<При помощи соображений, которые, казалось бы, не
имели никакого отношения ни к молекулам, ни к динамике их движений, ни к
логике, ни даже здравому смыслу, Максвелл нашёл формулу, которая,
согласно всем прецедентам и всем правилам научной философии, должна была
бы быть безнадёжно неправильной. В действительности же, как было
впоследствии доказано, она вполне правильна и до наших дней известна как
закон Максвелла.>>}}{\person{Джеймс Джинс} (1877---1946)}

В течение более века одиночные вспышки кинетической теории затухали, не
успев разгореться, но диалектический закон развития природы необратимо
вёл её к своему становлению. Кризис теории теплорода достиг своего
предела в связи с установлением закона сохранения энергии в
термодинамике. Последней каплей, по-видимому, стали эксперименты
Джоуля"--~Томсона. Они исследовали в 1852 году адиабатическое
\emph{дросселирование} газа и показали, что гелий и водород при таком
расширении охлаждаются~\cite{joule1852thermal}.

Сначала в 1856 году основы кинетической теории в немецком научном
обществе высказал в короткой статье~\cite{kronig1856grundzuge} немецкий
химик \person{Август Крёниг} (1822---1879), который, по-видимому, был
знаком с очерком Ватерстона. У Крёнига атомы обладали только
поступательной кинетической энергией, что не соответствовало
эмпирическим данным отношения теплоёмкостей. В общем, работа Крёнига ни
в чём не продвинулась по сравнению с кинетической теорией Бернулли, но
привлекла значительное внимание научной общественности в Германии,
поскольку Крёниг был известным профессором.

\subsection{Отец кинетики}

\person{Рудольф Клаузиус} (1822---1888) ещё до 1850 года, когда вышла
его статья по термодинамике, стал вынашивать кинетические идеи, однако
не торопился с их изложением, пытаясь установить сначала эмпирические
законы тепловых явлений. Появление в печати статьи Крёнига стало удобным
моментом, чтобы опубликовать ключевую в истории всей кинетики работу. В
обширной статье 1857 года Клаузиус не ограничился исследованием газовых
законов, а сразу замахнулся распространить молекулярную теорию на все
агрегатные состояния вещества~\cite{clausius1857heat}. Были детально
раскрыты процессы перехода между ними, описаны также процесс испарения
жидкости и явление \emph{латентной теплоты}, необходимой для перехода
между различными состояниями. Клаузиус, пожалуй впервые среди физиков,
поднял важность химической гипотезы Авогадро и указал на её
совместимость с кинетическим подходом. Наконец, он в отличие от Крёнига
постулировал, что молекулы обладают ещё вращательными и колебательными
степенями свободы.

Работа Клаузиуса имела широкий резонанс в научном обществе. Первым её
критиком стал известный голландский метеоролог \person{Христофор
Бёйс-Баллот} (1817---1890). Он в 1858 году указал, что средняя скорость
молекул, вычисленная Клаузиусом, значительно превышает известную из
опыта скорость диффузии~\cite{buys1858heat}. Они должны были совпадать,
поскольку Клаузиус постулировал \emph{модель идеального газа} с
бесконечно малым размером молекул. Немедленным ответом послужила статья
этого же года~\cite{clausius1858freepath}, где Клаузиус утверждает, что
размер молекулы хоть и мал по сравнению с объёмом газа, но достаточен,
чтобы мешать длительному прямолинейному распространению без
столкновений. Таким образом появляется концепция \emph{среднего
свободного пробега} молекул, позволяющая количественно описать явления
переноса. Стоит отметить, что в отличие от более современного вывода
Ватерстона, Клаузиус в своих доказательствах использовал допущение, что
все частицы газа имеют одинаковую скорость.

Рудольф Клаузиус в серии своих работ~\cite{clausius1937kinetics} заложил
программу развития и основные понятия современной кинетической теории,
став по праву настоящим её отцом. Однако законченность и зрелость
кинетическая теория обрела благодаря двум великим физикам: это британец
шотландского происхождения \person{Джеймс Максвелл} (1831---1879) и
австриец \person{Людвиг Больцман} (1844---1906). Все крупные открытия
кинетики последующих двух десятилетий связаны только с этими именами.

\subsection{Основные открытия}

Сначала в 1860 году вышла монументальная статья Максвелла <<Пояснения к
динамической теории газов>> в трёх
частях~\cite{maxwell1860illustrations}. В первой части Максвелл
сформулировал свой знаменитый закон распределения молекулярных скоростей
в \emph{однородном газе}, находящемся в равновесном состоянии, а также
независимо от Ватерстона установил принцип равнораспределения
кинетической энергии в газовой смеси. Максвелл существенно развил
вероятностный подход Клаузиуса, заложив первый камень в новый раздел
физики "--- \emph{статистическую механику}. Ключевым шагом было
применение распределения Гаусса вместо понятия длины свободного пробега
к описанию молекулярного движения. В то время статистический подход
только начал развиваться и применялся только в исследованиях в области
общественных наук. При выводе закона распределения Максвелл исходил из
неочевидной гипотезы о статистической независимости отдельных компонент
скорости каждой молекулы, так что ему удалось избежать непосредственно
рассмотрения столкновительных процессов.

Максвелл рассмотрел проблему внутреннего трения газов и впервые оценил
значение средней длины пробега, получив правильный порядок величины.
Оказалось, что коэффициент \emph{вязкости} не зависит от плотности газа,
что казалось парадоксальным выводом, но было впоследствии подтверждено
экспериментально. Этот серьёзный успех предсказательной силы
кинетической теории значительно способствовал её научному признанию. По
этому поводу лорд Рэлей писал, что <<во всей области науки нет более
красивого или многозначительного открытия>>. Во второй части статьи
Максвелл рассмотрел с тех же позиций процессы \emph{диффузии} и
\emph{теплопроводности}. Таким образом, все явления переноса
кинетическая теория газов позволила рассмотреть с единой точки зрения.

В третьей части Максвелл обратился к вопросу о вращательном движении
сталкивающихся частиц и впервые получил \emph{закон равнораспределения}
кинетической энергии по поступательным и вращательным степеням свободы.
Больцман в 1876 году обобщил закон на все независимые компоненты
движения и применил его в том числе для объяснения эмпирического закона
Дюлонга"--~Пти, описывающего теплоёмкость твёрдых тел. В дальнейшем
попытки решить проблемы, связанные с равнораспределением энергии,
послужили одним из истоков \emph{квантовой теории}.

Первый строгий математических подход к проблемам \emph{неоднородного
газа} был представлен Максвеллом в статье <<О динамической теории
газов>> 1867 года~\cite{maxwell2012theory}. В этой работе он вывел
\emph{уравнения переноса}, которые дают полную скорость изменения любого
молекулярного признака. Это изменение он разбивал на три части,
обусловленные соответственно молекулярными столкновениями, движением
молекул от точки к точке и действием внешних сил. Основным результатом
работы стало вычисление коэффициентов вязкости, диффузии и
теплопроводности на основе кинетических представлений. Однако все его
детальные результаты относятся к так называемому \emph{максвелловскому
газу}, молекулы которого являются точечными центрами силовых полей,
изменяющихся обратно пятой степени расстояния. В этом частном случае
значительно упрощается вычисление сложных интегралов, выражающих эффекты
молекулярных столкновений. Более того, интегрирование может быть
выполнено без знания \emph{функции распределения }скоростей. Наконец,
Максвелл в этой статье изложил новый вывод равновесного распределения
молекул по скоростям, исходя из \emph{принципа детального равновесия}.

Вскоре на сцену становления кинетической теории в главной роли
постепенно выходит Людвиг Больцман. В 1866~\cite{boltzmann1866second} и
1871~\cite{boltzmann1871equilibrium} годах он публикует первые попытки
обосновать второе начало термодинамики с позиции законов механики, в
1868 году~\cite{boltzmann1868distribution} обобщает распределение
Максвелла на случай воздействия силового поля. Настоящим переворотом
стала работа Больцмана 1872 года, изложенная почти на ста
страницах~\cite{boltzmann1872further}.

Больцман начал с критики метода, с помощью которого Максвелл выводил
свой закон распределения молекулярных скоростей. Действительно, Максвелл
нашёл, какое состояние газа является равновесным, но не показал, что
молекулярные столкновения приводят именно к нему. Ответом на этот вопрос
послужила так называемая \emph{H-теорема}. Больцман обнаружил, что
некоторый функционал от распределения скоростей всегда убывает со
временем, если не достигнуто распределение Максвелла, а значит является
характеристикой \emph{неравновесной энтропии}. Таким образом
Больцман показал что в газе, предоставленном самому себе, молекулярные
столкновения приводят к максвелловскому распределению скоростей,
независимо от начального распределения.

Далее Больцман вывел непосредственно уравнение временн\'{о}й эволюции
функции распределения, а именно знаменитое интегро-дифференциальное
\emph{кинетическое} \emph{уравнение Больцмана}. Существенным
предположением, лежащим в основе уравнения Больцмана, стала гипотеза
\emph{молекулярного хаоса}
(\foreignlanguage{german}{\emph{Stosszahlansatz}}), которая выражается в
отсутствии статистической корреляции между скоростями молекул до их
столкновения. Эта гипотеза вызывала наибольшее недоверие со стороны
современников, поскольку строго детерминированные динамические законы
взаимодействия молекул не могут, казалось, создавать такой элемент
случайных процессов как хаос. Именно это положение лишает уравнение
Больцмана \emph{симметрии во времени}. Строгое обоснование гипотезы
основывается на бесконечно малой вероятности нарушения хаотического
состояния в результате молекулярных столкновений. До настоящего времени
математическая проблема молекулярного хаоса решена лишь частично, однако
физическое толкование появилось в 1911 году~\cite{ehrenfest1911chaos}
под авторством ученика Больцмана \person{Пауля Эренфеста} (1880---1933)
и его русской жены \person{Татьяны Афанасьевой}(1876---1964).

Н-теорема Больцмана вызвала огромную и весьма плодотворную дискуссию.
Основная критика со стороны физиков была заключена в \emph{парадоксе
Лошмидта}. Сначала на эту проблему указал лорд Кельвин в 1874
году~\cite{thomson1874kinetic}, а затем в 1876
году~\cite{loschmidt1877paradox} сформулировал друг Больцмана
\person{Йозеф Лошмидт} (1821---1895). Суть парадокса в том, что
обратимые законы механики приводят к необратимому возрастанию энтропии.

В 1877 году выходят в печать две ключевые работы Больцмана. В
первой он разъясняет суть парадокса Лошмидта, указывая на статистический
характер необратимости~\cite{boltzmann1877loschmidt}. Законы физики не
запрещают события, при которых энтропия убывает, однако вероятность их
наступления чрезвычайно мала. Раскрывая парадокс Лошмидта, Больцман
приходит к более глубокому пониманию второго начала термодинамики, его
статистической природы.

Вторая статья по объёму сравнима с работой 1872 года и по праву служит
началом \emph{статистической механики} как новой
науки~\cite{boltzmann1877stats}. В ней Больцман пересматривает базисные
принципы, с позиции которых он даёт обоснование второму началу
термодинамики. От стационарных распределений Больцман переходит к
наиболее вероятным, стремление к равновесию заменяет эволюцией от
маловероятных состояний к более вероятным.  Как результат, формулирует
центральную идею о фундаментальной связи между макроскопическими и
микроскопическими состояниями. Этот труд Больцмана имел широкое влияние
на взгляды учёных начала XX века. В 1901 году \person{Макс Планк}
(1858---1947) устанавливает закон излучения абсолютно чёрного тела и
называет \emph{постоянную Больцмана} в его
честь~\cite{planck1901constant}. Тогда же открытую Больцманом
зависимость между энтропией и числом допустимых микросостояний Планк
представляет в виде знаменитой формулы, которая теперь посмертно
высечена на могильном камне Больцмана~\cite{planck1901equation}. Эту
формулу позднее \person{Альберт Эйнштейн} (1879---1955) назовёт
\emph{принципом Больцмана}.

В то время как прирождённый атомист Больцман изучал статистическую
природу кинетической теории, Максвелл углублялся в прикладные её аспекты
в поисках очередных доказательств молекулярной гипотезы. В последний год
своей жизни (1879) он развил теорию неоднородных газов до следующего
приближения и показал, что неоднородность температуры в разрежённом газе
приводит к механическим напряжениям~\cite{maxwell2012stresses}. В
частности, он предсказал явление \emph{теплового скольжения} газа вдоль
поверхностей с приложенным градиентом температуры. Независимо от
Максвелла в этом же году его экспериментально обнаружил соотечественник
\person{Осборн Рейнольдс} (1842---1912), рассматривая протекание газа
через неравномерно нагретые пористые
вещества~\cite{reynolds1879transpiration}. Явление теплового скольжения
газа позволило Максвеллу объяснить также \emph{радиометрический эффект
Крукса}. Британский физик и химик \person{Уильям Крукс} (1832---1919)
обнаружил его в 1874 году~\cite{crookes1874attraction}, но ошибочно
полагал, что вращение лопастей вызывается давлением света.

\subsection{Поздние работы Больцмана}

Последующие научные труды Больцмана по кинетической теории редко
цитируются, во многом по причине своей громоздкости и
непоследовательности. В действительности же глубокий критический анализ
позволил Больцману поставить ключевые вопросы дальнейшего развития как
кинетической теории, так и всей статистической физики.

В 1880---1881 годах Больцман в нескольких обширных статьях развил
громоздкий приближенный метод решения с целью вычисления вязкости для
произвольного молекулярного потенциала, а не только для максвелловского
газа. Исследование, которое занимает в целом 168 страниц его собрания
сочинений, не привело к простому результату, но явилась зачатком будущих
попыток \emph{асимптотических решений} кинетического уравнения.

В 1884 году Больцман пишет очередную фундаментальную, но малоизвестную
работу, где он выходит за рамки модели идеального газа, учитывая
потенциальную энергию взаимодействия молекул, закладывает основы
\emph{равновесной статистической механики} и развивает \emph{метод
ансамблей}~\cite{boltzmann1885ansambles}. Обычно рождение этих понятий
связывают с американским физиком-теоретиком \person{Джозайей Гиббсом}
(1839---1903), который на самом деле развил идеи Больцмана в 1902 году в
своей обширной монографии~\cite{gibbs1902ansambles}. Дело в том, что
Гиббс был одним из немногих, кто читал работы Больцмана в оригинале. В
основном с трудами Больцмана знакомились по сокращённым выдержкам,
поскольку его огромный энтузиазм порождал обычно слишком объёмные
изложения. По этому поводу писал даже Максвелл: <<Изучая Больцмана, я не
смог понять его. Он не мог понять меня из-за моей краткости, но его
многословность была и является равноправным камнем преткновения для
меня.>>

Работы Больцмана с 1884 по 1887 год содержат глубокое исследование
статистической справедливости второго начала, в частности
\emph{эргодической гипотезы} (термин введён был Эренфестом). В
дальнейшем эта проблема привела к возникновению одноимённой
\emph{эргодической теории}.

Наследие Больцмана богато не только научными трудами, но и обширной
полемикой с современниками. В спорах и дискуссиях Больцман всегда
выступал активным пропагандистом атомистической картины мира и
кинетического подхода к тепловым явлениям. Широко известны его диспуты с
позитивистом \person{Эрнстом Махом} (1838---1916), энергетистом
\person{Вильгельмом Оствальдом} (1853---1932), регулярная дискуссия,
связанная с парадоксом обратимости (Лошмидта). В 1896 году разгорелся
очередной спор между Больцманом и молодым немецким математиком, учеником
Планка, \person{Эрнстом Цермело}
(1871---1953)~\cite{zermelo1896paradox}. \emph{Парадокс Цермело}
указывал на математическое опровержение второго начала термодинамики,
исходя из теоремы Пуанкаре о возвращении, но также не противоречил
статистической природе закона.

Работы Больцмана являются апофеозом классической кинетики, ключевым
рубежом её зрелости. Дальнейшее развитие "--- это поиск решений
кинетических уравнений, обобщение и расширение области применимости,
построение математически строгой теории.

\section{Развитие газовой кинетики в XX и XXI веке}

\epigraph{\textit{<<Среди самых интересных проблем математической физики
специальное место следует отвести проблемам, связанным с кинетической
теорией газов. Многое уже сделано для их решения, но многое ещё остаётся
сделать. Эта теория представляет вечный парадокс. Мы имеем обратимость в
предпосылках и необратимость в следствиях, и между ними "---
пропасть.>>}}{\person{Анри Пуанкаре} (1854---1912)}

Первый после Максвелла шаг в развитии строгой теории неоднородного газа
был сделан в 1905 году голландским лауреатом Нобелевской премии
\person{Хендриком Лоренцом} (1853---1928) для специального случая
газовой смеси, в которой молекулы одного сорта обладают пренебрежимо
малой массой по сравнению с молекулами другого
сорта~\cite{lorentz1905motion}. Этот случай был рассмотрен в связи с
\emph{теорией электронов в металле}. Полученные Лоренцом результаты
представляют из себя точные решения, но при этом не дают общего метода к
уравнению Больцмана.

В 1909 году датский физик \person{Мартин Кнудсен} (1871"--- 1949)
исследовал протекание газа через трубки при различных отношениях длины
свободного пробега к их диаметру, а именно \emph{чисел
Кнудсена}~\cite{knudsen1909gesetze}. Оказалось, что течение Пуазёйля
имеет минимум в пропускной способности в переходном режиме (число
Кнудсена порядка единицы). Впоследствии это явление было названо
\emph{парадоксом Кнудсена}, а теоретическое обоснование получило только
в 1960-х годах. В 1910 году Кнудсен использовал эффект теплового
скольжения в каскадных системах для создания вакуумных \emph{насосов
Кнудсена}, позволяющих создавать градиент давления без движущихся
механических частей~\cite{knudsen1909revision}.

В 1910 году Нобелевскую премию по физике получает ещё один голландский
физик \person{Ян Ван-дер-Ваальс} (1837---1923) за развитие модели,
единообразно описывающей газообразную и жидкую фазы вещества. В своей
диссертации 1873 года он вывел своё знаменитое \emph{уравнение
Ван-дер-Ваальса}~\cite{waals1873equation}.

\subsection{Теория Чемпена"--~Энскога}

В 1911 году к развитию кинетической теории подключается молодой шведский
учёный \person{Дэвид Энског} (1884---1947). В своей ранней работе Энског
детально изучает подход Больцмана к решению кинетического уравнения и
заключает, что такой метод не даёт полезных
результатов~\cite{enskog1911boltzmann}. В следующей статье 1912 года
Энског отмечает, что в общем случае температурный градиент в газе должен
вызывать диффузию, причём для степенного потенциала межмолекулярного
взаимодействия направление термодиффузии зависит непосредственно от
показателя степени, так что полностью исчезает для максвелловского
газа~\cite{enskog1912electron}. Это объясняет столь позднее открытие
эффекта и тот факт, что он ускользнул от глаз самого Максвелла. Это
явление \emph{термодиффузии} было независимо получено английским
геофизиком и математиком \person{Сидни Чепменом} (1888---1970) и
экспериментально подтверждено его коллегой химиком \person{Фредериком
Дутсоном} (1863---1929) в 1917 году~\cite{chapman1917diffusion}.

Следующей ключевой фигурой становится великий немецкий математик
\person{Давид Гильберт} (1862---1943). В 1912 году он подошёл к изучению
уравнения Больцмана с позиций чистой математики, впервые сделав упор на
необходимость доказательства существования
решения~\cite{hilbert1912grounds}. Гильберт показал, как найти
приближённое решение уравнения Больцмана с помощью разложения в ряд по
малому параметру, обратно пропорциональному плотности газа.
\emph{Разложение Гильберта} позволило получить уравнения Эйлера для
сжимаемого газа, однако не позволяло получить решение для протяжённых
масштабов времени.

Независимо друг от друга к 1917 году Чепмен~\cite{chapman1916kinetic,
chapman1918kinetic} и Энског~\cite{enskog1917thesis} развили иной подход
к приближённому решению уравнения Больцмана, названный позже в их честь
\emph{разложением Чепмена"--~Энскога} и получили в первом приближении
также уравнения Эйлера, а во втором "--- уравнения Навье"--~Стокса для
сжимаемого газа. Несмотря на то что они пришли к одному результату, в
основе лежали разные методы. Чепмен развил подход Максвелла, установив
вид функции распределения из соображений инвариантности, а Энског
видоизменил разложение Гильберта, используя сразу все временн\'{ы}е
масштабы в каждом приближении. В 1922 году Энског, базируясь на
молекулярной модели Ван-дер-Ваальса, обобщил своё решение для газов
большой плотности до \emph{уравнения Энскога}~\cite{enskog1922dense}.

С точки зрения истории науки весьма интересным представляется тот факт,
что ни Чепмен, ни Энског не поднимали вопроса о приоритете создания
своей теории. Такое поведение при первом рассмотрении идёт в разрез с
историческим опытом. Первооткрыватели из разных стран, добившиеся успеха
одновременно и независимо, часто вступают в острые дебаты за пальму
первенства: Лейбниц против Ньютона в открытии
интегрально-дифференциального исчисления, Джоуль против Майера в
открытии закона сохранения энергии и др. В большей степени споры
разгораются за приоритет фундаментальных открытий, имеющих отношение к
широкой области науки. Теория Чепмена"--~Энскога, напротив, является
примером узкой специализации и высокой сложности, где на первый план
выступает борьба непосредственно за принятие самой теории научной
общественностью. Не стоит выпускать из внимания также фактор, на который
указал сам Чепмен: когда два человека одновременно и независимо
публикуют тот же результат, это, как правило, весьма положительно влияет
на его принятие среди других учёных, которые бы не захотели вдаваться в
детали сложных расчётов.

Теория Чепмена"--~Энскога~\cite{chapman1960theory} проложила первый мост
между кинетической теорией и \emph{гидродинамикой} и позволила вычислять
феноменологические транспортные коэффициенты (вязкости, теплопроводности
и диффузии), придав им кинетический смысл. В последующие годы много
работ было посвящено прикладному развитию теории, применению методов в
смежных направлениях, таких как \emph{перенос излучения} и \emph{теории
ионизированных газов} (плазмы). Однако при строгом рассмотрении
разложение Чепмена"--~Энскога представляет собой некорректно
поставленную задачу Коши. С математической точки зрения всё ещё
оставалось множество открытых вопросов.

\subsection{Строгая математическая теория}

Центральной проблемой кинетической теории является вопрос о
\emph{существовании} и \emph{единственности} решения уравнения
Больцмана. Первый строгий результат получил известный шведский математик
\person{Торстен Карлеман} (1892---1949), который в 1933 году доказал
соответствующую теорему \emph{для газа твёрдых сфер} при условии, что функция
распределения не зависит от пространственных
координат~\cite{carleman1933theorie}. Его доказательство было улучшено и
очищено от дополнительных ограничений в посмертной монографии 1957
года~\cite{carleman1960problems}.

В 1946 году выходят три независимые работы по статистической физике, в
которых была описана так называемая \emph{цепочка
Боголюбова"--~Борна"--~Грина"--~Кирквуда"--~Ивона (ББГКИ)} или просто
\emph{цепочка Боголюбова}. Советский физик-теоретик \person{Николай
Боголюбов} (1909---1992) показал с её помощью строгий вывод уравнения
Больцмана из уравнения Лиувилля~\cite{bogolyubov1946kinetic}. Британский
физик \person{Герберт Грин} (1920---1999) под руководством ещё одного
лауреата Нобелевской премии \person{Макса Борна} (1882---1970)
использовал цепочку уравнений для создания общей \emph{кинетической
теории жидкостей}~\cite{born1946general}. Американский физик и химик
\person{Джон Кирквуд} (1907---1959) в свою очередь сделал упор на
процессах переноса~\cite{kirkwood1946statistical}. Вклад французского
математика \person{Жака Ивона} (1903---1979) заключается в исследовании
N-частичной функции распределения в 1935 году~\cite{yvon1935theorie}.
Цепочка Боголюбова математически эквивалентна исходному \emph{уравнению
Лиувилля} и тем самым не описывает необратимость во времени, однако её
обрыв при определённых дополнительных условиях позволяет получать
различные кинетические уравения.

В 1949 году американский математик \person{Харольд Грэд} (1923---1986)
опубликовал широко известную статью~\cite{grad1949kinetic}, в которой
построил бесконечную систему \emph{моментных уравнений}, связывающих
макроскопические параметры газа, которая эквивалентна уравнению
Больцмана. Кроме того, предложил систематический подход к приближённому
численному решению уравнения Больцмана с помощью искусственного обрыва
полученной бесконечной системы. В этой же работе Грэд сформулировал
гипотезу справедливости уравнения Больцмана и показал, что с помощью
\emph{предела Больцмана"--~Грэда} для \emph{больцмановского газа} можно
из цепочки Боголюбова получить \emph{цепочку Больцмана}, а вместе с ней
и само уравнение Больцмана.

При определённых условиях обрыв бесконечной системы моментных уравнений
Грэда позволяет также построить частичные решения уравнения Больцмана:
точно определить динамику некоторых старших моментов, но не саму функцию
распределения. Первые подобные результаты были получены в 1956 году для задачи
сдвигового течения независимо советским математиком \person{Владленом
Галкиным} (род. 1932)~\cite{galkin1956solution} и американским
математиком \person{Клиффордом Трусделлом}
(1919---2000)~\cite{truesdell1956pressures}. Позднее Галкин обобщил
метод на широкий класс \emph{гомоэнергетических} течений, в том числе
для нестационарных задач и смеси газов~\cite{galkin1956solutions}.

Вслед за точными решениями уравнений кинетических моментов исследования
были направлены в сторону точных решений непосредственно уравнения
Больцмана, высокая математическая сложность которого существенно
ограничивает их количество. После равновесного распределения
Максвелла"--~Больцмана первых результатов на этом пути добился советский
математик \person{Александр Никольский} (1919---1976). Решённая им в
1963 году задача однородного расширения
газа~\cite{nikolskii1963selfsimilar} носит \emph{автомодельный}
(\emph{self-similar}) характер: пространственные координаты и время
входят зависимым образом в функцию распределения. Следующий шаг был
сделан в 1975 году советским математиком \person{Александром Бобылевым} (род. 1947).
Бобылев применил преобразование Фурье к уравнению Больцмана и получил
первые нетривиальные аналитические решения \emph{для максвелловских
молекул}~\cite{bobylev1975solution}. В дальнейшем он выстроил общую
теорию пространственно однородной релаксации~\cite{bobylev1984exact}, на
базе которой развиваются \emph{спектральные методы} анализа уравнения
Больцмана.

Попытки нахождения решений уравнения Больцмана в практических задачах
привели на путь упрощения нелинейного интеграла столкновений. Первый
успех был достигнут на стыке с кинетической теорией плазмы. В 1954 году
одновременно в двух работах~\cite{bhatnagar1954model,
welander1954temperature} появилось первое \emph{модельное уравнение}
"--- \emph{уравнение БГК} (Бхатнагара"--~Гросса"--~Крука)\footnote{В
литературе также встречается \emph{уравнение БКВ}
(Больцмана"--~Крука"--~Веландера)}. Другой подход к решению
слабовозмущённых задач "--- это использование линеаризованного уравнения
Больцмана, для которого в 1963 году Грэд успешно применил разложение
Гильберта. Поэтому в эти десятилетия новый пласт математических
результатов был добыт именно в рамках описанных
направлений~\cite{cercignani1973mathematical}.

В 1972 году итальянский математик \person{Карло Черчиньяни}
(1939---2010) показал, что строгий вывод уравнения Больцмана того же
порядка сложности, что и доказательство теоремы о существовании и
единственности его решения~\cite{cercignani1972boltzmann}. Другими
словами, теорема позволяет доказать выдвинутую Больцманом гипотезу
молекулярного хаоса. Через год американский математик \person{Оскар
Ланфорд} (1940---2013) обобщил решение Карлемана \emph{для газа твёрдых
сфер}, получив результат для короткого промежутка времени "--- так
называемая \emph{теорема Ланфорда}~\cite{lanford1975time}. Последнего на
сегодняшний день успеха на этом пути добился американский математик
немецкого происхождения \person{Райнхард Илнер} (род. 1950) совместно с
\person{Марвином Шинбротом} (1928---1987). Они в 1984 году доказали
соответствующую теорему для случая свободного распространения газа в
вакуум~\cite{illner1984boltzmann}. Подробное изложение результатов можно
найти в~\cite{cercignani1994mathematical}.

В 1989 году французский математик \person{Пьер-Луи Лионс} (род. 1956) и
американец \person{Рональд Ди-Перна} (1947---1989) пошли другим путём:
развили \emph{теорию перенормировочных решений} и сумели доказать
существование решения задачи Коши без
единственности~\cite{diperna1989cauchy}. Во многом за это достижение
Лионс в 1994 году был удостоен Филдсовской премии. Большой заслугой
Лионса в не меньшей мере является создание французской математической
школы по кинетической теории.

Больцман с помощью H-теоремы показал, что энтропия неравновесных
процессов возрастает, а в 1982 году Черчиньяни поставил вопрос о
скорости диссипации энтропии и сформулировал так называемую
\emph{гипотезу Черчиньяни}~\cite{cercignani1982conjecture}. С 1999 по
2005 год в серии фундаментальных работ ученик Лионса \person{Седрик
Виллани} (род. 1973) детально рассмотрел и разрешил эту
гипотезу~\cite{yau2010work}. При этом он развил теорию оценки
долговременных асимптотик общего класса гипоэллиптических операторов. В
2010 году Виллани был также удостоен Филдсовской премии. Обозначенные им
направления в настоящее время активно развиваются~\cite{villani2002review}.

\subsection{Динамика разрежённого газа}

С прикладной точки зрения пионерская работа Грэда 1949 года резонирует с
мощной волной интереса к аэрокосмической технике. Кинетическая теория
газов понадобилась для решения задач сопротивления и теплопередачи в
верхних слоях атмосферы. В конкурирующих державах стали развиваться
мощные школы~\cite{grad1958principles, kogan1967dynamics}. В 1958 году
выделился новый раздел кинетической теории "--- \emph{динамика
разрежённого газа}, "--- изучающий поведение газа в широком диапазоне чисел Кнудсена.
С этого времени каждые два года проводится международный
одноимённый симпозиум.

Интересно проследить статистику трудов симпозиума, проводимого уже более
полувека. На самой заре динамики разрежённого газа преобладали
экспериментальные работы, а теоретические были ещё далеки от решения
элементарных задач. В 60-х годах начала бурно развиваться вычислительная
техника, и появились первые численные
решения~\cite{cercignani1978applications}. Постепенно экспериментальные
работы уступали место компьютерному моделированию, гегемония которого в
XXI веке уже не оспаривается.

Следующий шаг на пути открытия явлений разрежённого газа "---
после теплового скольжения и термодиффузии "--- был сделан в 1969 году
группой советских учёных под руководством \person{Михаила Когана}
(1925---2011). Анализ третьего приближения в разложении
Чепмена"--~Энскога (\emph{уравнения Барнетта}) привёл к обнаружению
нелинейного эффекта "--- \emph{термострессовой
конвекции}~\cite{kogan1976stresses}. Как и в случае с термодиффузией,
долгое время считалось невозможным движение газа, обусловленного
температурными напряжениями. К такому выводу пришёл Максвелл в
упомянутой ранее работе 1879 года, рассмотрев только линеаризованное
выражение напряжений.

Следующей волной интереса к динамике разреженного газа стало развитие
\emph{микроэлектромеханических систем}
(\emph{МЭМС})~\cite{karniadakis2006microflows}. В меньшей степени
применения затрагивают вакуумную индустрию и даже экологические
проблемы. На сегодняшний день множество монографий посвящено прикладным
аспектам кинетической теории. Основные интересные эффекты разреженного
наблюдаются в так называемом \emph{переходном режиме} (число Кнудсена
порядка единицы), лежащем между предельными случаями:
\emph{свободномолекуляные} течения (число Кнудсена стремится к
бесконечности) и \emph{континуальные} (число Кнудсена стремится к нулю).

Сегодня основная тенденция развития динамики разрежённого газа как
прикладной науки "--- это усложнение спектра решаемых задач и развитие
соответствующих методов компьютерного моделирования. Можно выделить
несколько ключевых направлений современных исследований:
физико-химические процессы, граничные условия раздела двух фаз и
квантовомехенические эффекты. Первое направление включает моделирование
реагирующих смесей газов, в том числе процессов горения и детонации. На
границе с жидкой фазой изучаются вопросы испарения и конденсации, до сих
пор проводятся экспериментальные работы по определению коэффициентов
аккомодации при взаимодействии газа с твёрдыми поверхностями. Квантовая
механика кроме прочего позволяет учесть неупругие столкновения. Более
того, ввиду высокой вычислительной сложности уравнения Больцмана
актуальными становятся \emph{гибридные} методы моделирования,
объединяющие численные решения кинетики и сплошной среды.


\printbibheading
\addcontentsline{toc}{section}{Список литературы}

\printbibliography[
	heading=subbibliography,
	title={Первичные источники}]

\newrefsection[second]
\nocite{*}
\printbibliography[
	nottype=mvbook,
	sorting=nty,
	env=plain,				% See definition in the heading
	heading=subbibliography,
	title={Вторичные источники}]

\end{document}
