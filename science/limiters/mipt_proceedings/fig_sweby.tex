	\centering
	\subfloat[Классические схемы второго порядка (\textit{\mbox{L-W}, \mbox{B-W}, Fromm}) и их линейные комбинации,
		ограниченные условием TVD (\textit{MC, Koren}). Область ТVD схем выделена на рисунке серым цветом.]{
	\label{fig:sweby:a}
	\begin{tikzpicture}[>=latex',thick, scale=1.5, domain=0:4]
		\draw[draw=gray!20,fill=gray!20] (0,0) -- (1,2) -- (4.1,2) -- (4.1,0) -- cycle;
		\node[above] at (.8,0) {TVD};
		\draw[<->] (4.2,0) node[right] {\(\theta\)} -- (0,0) -- (0,2.7) node[above] {\(\varphi(\theta)\)};
		\foreach \x in {0,1,2,3,4}
			\draw (\x cm,1pt) -- (\x cm,-1pt) node[anchor=north] {\(\x\)};
		\foreach \y in {0,1,2}
			\draw (1pt,\y cm) -- (-1pt,\y cm) node[anchor=east] {\(\y\)};
		\node[draw,circle,minimum size=.3cm] (sec) at (1,1) {};
		\node[text width=1cm,text centered] (cap) at (0.8,1.7) {second order};
		\draw[->] (cap.south) to (sec);

		\draw[color=purple!50!blue] (0,0) -- (0.25,0.5) -- (2.5,2) -- (2.7,2) node[above] {Koren} -- (4,2);
		\draw[color=green!50!black] (0,0) -- (0.333,0.667) -- (3,2) node[below right] {MC} -- (4,2);
		\draw[color=blue] (-.1,-.1) -- (2.2,2.2) node[above left] {Beam"--~Warming} -- (2.9,2.9);
		\draw[dashed,color=green] (-.1,.45) -- (4,2.5) node[above left] {Fromm};
		\draw[color=red] (-.1,1) -- (2.5,1) node[below right] {Lax"--~Wendroff} -- (4,1);
	\end{tikzpicture}}\quad
	\subfloat[Другие нелинейные лимитеры. \textit{Van Leer} и \textit{van Albada} гладки в точке (1,1)
		и в пределе ведут себя как схемы \textit{Fromm} и \textit{L-W} соответственно.
		\textit{Minmod} и \textit{superbee} лишены второго порядка в точках перегиба.]{
	\begin{tikzpicture}[>=latex',thick, scale=1.5, domain=0:4]
		\draw[draw=gray!20,fill=gray!20] (0,0) -- (1,2) -- (4.1,2) -- (4.1,0) -- cycle;
		\node[above] at (.8,0) {TVD};
		\draw[<->] (4.2,0) node[right] {\(\theta\)} -- (0,0) -- (0,2.7) node[above] {\(\varphi(\theta)\)};
		\foreach \x in {0,1,2,3,4}
			\draw (\x cm,1pt) -- (\x cm,-1pt) node[anchor=north] {\(\x\)};
		\foreach \y in {0,1,2}
			\draw (1pt,\y cm) -- (-1pt,\y cm) node[anchor=east] {\(\y\)};
		\node[draw,circle,minimum size=.3cm] (sec) at (1,1) {};
		\node[text width=1cm,text centered] (cap) at (0.8,1.7) {second order};
		\draw[->] (cap.south) to (sec);

		\draw[color=blue] (0,0) -- (1,1) -- (4,1) node[below left] {minmod};
		\draw[color=green!50!black] (0,0) -- (.5,1) -- (1,1) -- (2,2) -- (4,2) node[above left] {superbee};
		\draw[color=purple!50!blue,smooth] plot[id=leer] function{(2*x)/(1+x)} node[below left] {van Leer};
		\draw[color=olive,smooth] plot[id=albada] function{(2*x)/(1+x*x)} node[below left] {van Albada};
	\end{tikzpicture}} \quad
	\subfloat[Улучшенные лимитеры, зависящие от числа Куранта \(\gamma\) и вписанные в расширенное условие TVD (выделено светло-серым фоном).
		Приведён частный случай \(\gamma=\sfrac{1}{3}\).]{
	\label{fig:sweby:c}
	\begin{tikzpicture}[>=latex',thick, scale=1, domain=0:4]
		\draw[draw=gray!10,fill=gray!10] (0,0) -- (.5,3) -- (6.1,3) -- (6.1,0) -- cycle;
		\draw[draw=gray!20,fill=gray!20] (0,0) -- (1,2) -- (6.1,2) -- (6.1,0) -- cycle;
		\node[below] at (1.5,3) {TVD \(\gamma=\frac1{3}\)};
		\node[above] at (1,0) {TVD \(\forall\gamma\)};
		\draw[<->] (6.3,0) node[right] {\(\theta\)} -- (0,0) -- (0,3.5) node[above] {\(\varphi(\theta)\)};
		\foreach \x in {0,1,2,3,4,5,6}
			\draw (\x cm,1pt) -- (\x cm,-1pt) node[anchor=north] {\(\x\)};
		\foreach \y in {0,1,2,3}
			\draw (1pt,\y cm) -- (-1pt,\y cm) node[anchor=east] {\(\y\)};
		\node[draw,circle,minimum size=.3cm] (sec) at (1,1) {};
		\node[text width=1cm,text centered] (cap) at (3,.8) {second order};
		\draw[->] (cap.west) to (sec);

		\draw[color=red] (0,0) -- (.1666,1) -- (1,1) -- (3,3) -- (4,3) node[above] {superbee (\(\gamma={}^1/_3\))} -- (6,3);
		\draw[color=blue] (0,0) -- (.1,.6) -- (3,1.888) node[below right] {third (\(\gamma={}^1/_3\))} -- (5.5,3) -- (6,3);
	\end{tikzpicture}}
	\caption{Диаграммы Sweby}\label{fig:sweby}
