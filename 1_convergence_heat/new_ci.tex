\documentclass[english,russian,a4paper,12pt]{article}
\usepackage[utf8]{inputenc}
\usepackage[T2A]{fontenc}
\usepackage{babel}
\usepackage{csquotes}

\usepackage{fullpage}
\usepackage{indentfirst}
\usepackage[font=small,labelfont=bf,labelsep=period]{caption}
\usepackage{graphicx}

\usepackage{amssymb, amsmath}

\usepackage[
	pdfauthor={Oleg Rogozin},
	pdftitle={Convergence of new CI module},
	colorlinks,pdftex, unicode]{hyperref}

\begin{document}
\section*{Сходимость нового модуля CI}

Анализируется одномерная задача переноса тепла между двумя плоскими параллельными пластинами.
Исследуется сходимость коэффициента теплопроводности \(\lambda\) для газа, состоящего из твёрдых сфер, к эталонному
\[ \lambda_\mathrm{ref} = 2.129475. \]

Задача моделировалась для \(\xi_{cut} = 4.3 \sqrt{2RT}\) при различных объёмах скоростной сетки.
Результаты представлены на рис.~\ref{fig:deviation}.
Для уточнения значений проведено усреднение по 50 временным точкам в стационарном режиме.

Для старого модуля явно обнаруживается сходимость второго порядка по числу узлов на радиусе скоростной сетки \(N_R\).
Для нового модуля сходимость также имеется, но с систематической ошибкой.

\begin{figure}[ht]
	\centering
	\includegraphics{conver_heat.pdf}
	\caption{Отклонение коэффициента теплопроводности от эталонного}\label{fig:deviation}
\end{figure}

\end{document}


