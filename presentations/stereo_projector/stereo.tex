\documentclass[ucs]{beamer}
\usetheme{Frankfurt}
\usepackage[utf8x]{inputenc}
\usepackage[T2A]{fontenc}
\usepackage[english,russian]{babel}

\usepackage{amssymb, amsmath, amsfonts}

\usepackage{tikz}
\usetikzlibrary{arrows,fit,positioning,shapes.multipart,shapes.geometric,shapes.symbols,mindmap}

\title{Стереовизуализация}
\author{Рогозин Олег}
\institute{
	Научно-исследовательский центр <<Курчатовский институт>>\\
	Московский физико-технический институт (государственный университет)
}
\date{\today}

\newcommand{\dd}{\:\mathrm{d}}
\newcommand{\Kn}{\mathrm{Kn}}
\newcommand{\TV}{\mathrm{TV}}

\begin{document}

\frame{\titlepage}
\begin{frame}
	\frametitle{Содержание}
	\tableofcontents
\end{frame}

\section{Обзор}

\begin{frame}
	\frametitle{Стереоскопические технологии}
	\begin{center}
		\vspace{-20pt}
		\begin{tikzpicture}[node distance=4cm,every path/.style={draw,>=latex',very thick},thick,
							block/.style={rectangle,text width=3cm,text=white,text badly centered,rounded corners,minimum height=3em}]
			\node[block,fill=blue] (Gla) {Стереоочки};
			\node[block,fill=green!50!black,below of=Gla,node distance=2.5cm] (Pol) {Поляризационные};
			\node[block,fill=green!50!black,left of=Pol] (Ana) {Анаглифические};
			\node[block,fill=green!50!black,right of=Pol] (Shut) {Затворные};
			\draw[->] (Gla.west) to [bend right=30] (Ana);
			\draw[->] (Gla.east) to [bend left=30] node[above right] () {активные} (Shut);
			\draw[->] (Gla) to node[left] () {пассивные\(\quad\quad\)} (Pol);
			\node[block,fill=olive,below of=Pol,node distance=2.5cm] (Pol1) {Кинотеатры};
			\node[block,fill=olive,left of=Pol1] (Ana1) {Любители};
			\node[block,fill=olive,right of=Pol1] (Shut1) {Геймеры \\ Дом. кинотеатры \\ Профессионалы};
			\draw[->] (Ana) to  (Ana1);
			\draw[->] (Shut) to (Shut1);
			\draw[->] (Pol) to (Pol1);
		\end{tikzpicture}
	\end{center}
\end{frame}

\begin{frame}
	\frametitle{Стереоскопические технологии}
	\begin{center}
		\vspace{-20pt}
		\begin{tikzpicture}[node distance=4cm,every path/.style={draw,>=latex',very thick},thick,
							block/.style={rectangle,text width=3cm,text=white,text badly centered,rounded corners,minimum height=3em}]
			\node[block,fill=blue] (Vars) {Варианты реализации};
			\node[block,fill=green!50!black,below of=Vars,node distance=2.5cm] (Side) {Side-by-side};
			\node[block,fill=green!50!black,left of=Side] (V3D) {nVidia 3D Vision};
			\node[block,fill=green!50!black,right of=Side] (FP) {Frame packing};
			\draw[->] (Vars.west) to [bend right=30] node[above left] () {визуализация} (V3D);
			\draw[->] (Vars.east) to [bend left=30] node[above right] () {кино} (FP);
			\draw[->] (Vars) to (Side);
			\node[block,fill=olive,below of=Side,node distance=2.5cm] (Side1) {Горизонтальная \\ стереопара};
			\node[block,fill=olive,left of=Side1] (V3D1) {Мерцающее \\ изображение};
			\node[block,fill=olive,right of=Side1] (FP1) {Двойное \\ изображение};
			\draw[->] (V3D) to node[right,text width=2cm] () {DirectX \\ OpenGL} (V3D1);
			\draw[->] (FP) to node[left] () {HDMI 1.4} (FP1);
			\draw[->] (Side) to (Side1);
			\node[below of=V3D1,node distance=1cm] () {видеокарта};
			\node[below of=Side1,node distance=1cm] () {проектор};
			\node[below of=FP1,node distance=1cm] () {проектор};
		\end{tikzpicture}
	\end{center}
\end{frame}

\section{Технология nVidia 3D Vision}

\begin{frame}
	\frametitle{Маркентинг nVidia}
	\begin{center}
		\vspace{-20pt}
		\begin{tikzpicture}[node distance=4cm,every path/.style={draw,>=latex',very thick},thick,
							block/.style={rectangle,text width=3cm,text=white,text badly centered,rounded corners,minimum height=3em}]
			\node[block,fill=blue] (nVid) {nVidia};
			\node[block,fill=green!50!black!30!white,below of=nVid, node distance=2.5cm] (Tesl) {Tesla};
			\node[block,fill=green!50!black,left of=Tesl] (Quad) {Quadro};
			\node[block,fill=green!50!black,right of=Tesl] (GeF) {GeForce};
			\draw[->] (nVid.west) to [bend right=30] node[above left] () {визуализация} (Quad);
			\draw[->] (nVid.east) to [bend left=30] node[above right] () {игры} (GeF);
			\draw[->] (nVid) to (Tesl);
			\node[block,fill=olive!30!white,below of=Tesl, node distance=2.5cm] (Tesl1) {GPU};
			\node[block,fill=olive,left of=Tesl1] (Quad1) {3D Vision Pro \\ OpenGL Stereo \\ Quad Buffering};
			\node[block,fill=olive,right of=Tesl1] (GeF1) {3D Vision \\ Direct3D \\ Full Screen};
			\draw[->] (Quad) to (Quad1);
			\draw[->] (GeF) to (GeF1);
			\draw[->] (Tesl) to (Tesl1);
		\end{tikzpicture}
	\end{center}
\end{frame}

\begin{frame}[fragile]
	\frametitle{Конфигурация 3D Vision}
	\begin{block}{Требования}
		\begin{itemize}
			\item монитор или проектор, поддерживающий 3D Vision \\
			\item кабель с высокой пропускной способностью \\
			\item фирменные затворные очки 3D очки \\
			\item излучатель для синхронизации \\
			\item драйвера nVidia версией >= 260.19.36
		\end{itemize}
	\end{block}
	\begin{block}{Настройка}
		\scriptsize
		\begin{verbatim}
			Section "Screen"
			    ...
			    Option "Stereo" "11"
			EndSection
		\end{verbatim}
		\begin{verbatim}
			sudo nvidia-xconfig --stereo=11
		\end{verbatim}
	\end{block}
\end{frame}

\section{Стереопара Side-by-Side}

\begin{frame}[fragile]
	\frametitle{Режима Side-by-Side}
	Самый простой и распространённый
	\begin{block}{Требования}
		Для реализации необходим только:
		\begin{itemize}
			\item стереопроектор, поддерживающий режим Side-by-Side \\	
		\end{itemize}
		Для JVC DLA-X7 понадобятся:
		\begin{itemize}
			\item синхронизирующий ИК-эмиттер JVC PK-EM \\
			\item затворные очки JVC PK-AG1-B \\
		\end{itemize}
	\end{block}
	\begin{block}{Настройка}
		Необходимо подать на видеовыход
		\begin{itemize}
			\item горизонтальная стереопара (left/right half width)
			\item на весь экран
		\end{itemize}
	\end{block}
\end{frame}

\section{Технология Frame Packing}

\begin{frame}[fragile]
	\frametitle{Режим Frame Packing}
	\begin{block}{Требования}
		\scriptsize
		\begin{itemize}
			\item стереопроектор, поддерживающий режим Frame Packing \\
			\item хороший видеокабель (например, HDMI High Speed Cable) \\
		\end{itemize}
		Поддерживаемые режимы:
		\begin{itemize}
			\item 1280x1470@60 (рекомендован для игр) \\
			\item 1920x2205@24 (рекомендован для кинопросмотра) \\
		\end{itemize}
	\end{block}
	\begin{block}{Конфигурация}
		\tiny
		\begin{verbatim}
			Section "Monitor"
			    Identifier     "Projector"
			    Modeline       "1280x1470_60" 148.5  1280 1390 1430 1650 1470 1475 1480 1500 +hsync +vsync
			    Modeline       "1920x2205_24" 148.32 1920 2558 2602 2750 2205 2209 2214 2250 +hsync +vsync
			EndSection
			...
			Section "Screen"
			    ...
			    Monitor        "Projector"
			    Option         "ExactModeTimingsDVI" "True"
			    Option         "metamodes" "DFP-2: 1920x2205_24"
			EndSection
		\end{verbatim}
	\end{block}
	\scriptsize
	nVidia продаёт за \$40 3DTV Play
\end{frame}

\section*{}
\begin{frame}[fragile]
	\frametitle{Использование}
	\scriptsize
	\begin{block}{OpenGL Stereo}
		\begin{verbatim}
			paraview --stereo --stereo-type="Crystal Eyes"
			mplayer -vo gl:stereo=3
			bino -n -f -o stereo demo.avi
		\end{verbatim}
	\end{block}
	\begin{block}{Side-by-Side}
		\begin{verbatim}
			echo "stereo on" > script && ./jmol.sh -k -s script file.pdb
			bino -n -f -o left-right-half demo.avi
		\end{verbatim}
	\end{block}
	\begin{block}{Frame Packing}
		\begin{verbatim}
			bino -n -f -o hdmi-frame-pack demo.avi
		\end{verbatim}
	\end{block}
\end{frame}

\begin{frame}
	\frametitle{Заключение}
	\begin{itemize}
		\item профессинальное стерео --- OpenGL Stereo --- 3D Vision \\
		\item простое и нетребовательное стерео --- стереопары \\
		\item высокое качество видео --- HDMI Frame Packing \\
	\end{itemize}
\end{frame}

\end{document}
