\documentclass[11pt]{article}
\usepackage[utf8]{inputenc}
\usepackage[T1]{fontenc}
\usepackage[english]{babel}

\usepackage{amssymb, amsmath}
\usepackage{physics}
\usepackage[normalem]{ulem}         % for \sout
\usepackage{fullpage}
\usepackage{hyperref}

\title{Referee report on \\
"An ultra-sparse approximation of kinetic solutions to spatially homogeneous flows of non-continuum gas"
by A.~Alekseenko, A.~Grandilli, and A.~Wood}
\date{}

\newcommand{\Kn}{\mathrm{Kn}}
\newcommand{\Ma}{\mathrm{Ma}}
\newcommand{\bxi}{\vb*{\xi}}

\usepackage[
    backend=biber,
    sorting=none,
    giveninits,
]{biblatex}
\bibliography{report}

\begin{document}

\maketitle

%%% What is this about? What was done?
The present manuscript is devoted to sparse approximations of the space-homogeneous Boltzmann equation (BE).
The velocity distribution function (VDF) is represented in form of 2-3 Max\-wellians.
The coefficients of this approximation are obtained from a regression analysis of the BE.
The authors try to capture rarefied gas effects to some extent, but number of degrees of freedom is too small
and it seems that it can be significantly increased only at high computational costs.
Therefore, the presented ideas cannot serve as a foundation for a numerical method for solving the BE.
However, the aim of this paper is different.
The authors just try to analyze capability of the ultra-sparse Maxwellian-based approximation
to describe the behavior of a rarefied gas. It is not a new, but a promising idea.

%%% Why major revision?
The major disadvantage of the presented manuscript is that it is not based on the previous works on this topic.
The Soviet scientists Enders (husband and wife), together with their colleagues, published a series of papers in the 1980s,
where they developed several numerical methods based on the expansion of the VDF into the Maxwellian series
for the analysis of the spacially homogeneous relaxation problem, especially the isotropic case.
It seems that the authors are not aware of these studies,
many of which, however, go far beyond the scope of the authors' work.
Unfortunately, the pioneer Enders' papers are written in Russian, but some of them are translated
and can be definetely found nowadays, e.g.~\cite{Ender1988}.

Futhermore, there are some points in the paper that can be improved and/or clarified:
\begin{enumerate}
    \item The authors claim that ``\emph{Our model ... is related to the method of moments.}''
    However, their method is closer to the collocation family, since the residual of the VDF approximation
    are equated in the set of collocation points $\bxi_j$.
    Term~\emph{method of moments} is referred when sequences of the polynomial moments are equated to zero
    for the residual of the approximation in order to find the expansion coefficients.
    The Grad thirteen-moment method and Mott-Smith solution for the normar shock wave (SW) problem,
    mentioned in the paper, are indeed the moment methods~\cite{Sone2007}.

    \item The conclusion about applicability and accuracy of the method is made only
    for the narrow class of initial-value problems (IVP), associated with the SW problem,
    which is, moreover, quite natural for the Gaussian-based approximation.
    In contrast, the majority of the boundary-value problems (BVP) deals with a discontinuous VDF.
    It is obvious that the sum of smooth Maxwellians is bad suited for such cases,
    but the problem of stability with respect to the initial conditions is quite intriguing.
    Furthermore, the standard Bobylev--Krook--Wu (BKW) testing is also very desirable.

    %%% The situation is correct in the paper since \rho and T is normalized to unity.
    % \item The authors provide a comparison with the BGK model, but do not specify the exact form of the collision term.
    % It is essential for high-\(\Ma\) flows. For example, the classical BGK dimensionless collision term
    % is \(J = \rho(M[f]-f)\)~\cite{Sone2007}, where \(\rho\) is the density, \(f\) is the VDF, \(M[f]\) is the equilibrium VDF
    % with the same density, velocity, and temperature as for \(f\). In contrast, the mathematicians obtained many results
    % for the linear BGK model \(J = M[f]-f\). The BGK model is usually dimensionalized though the viscosity
    % at the reference temperature. However, high-\(\Ma\) flows are accompanied with a wide range of temperatures.
    % Therefore, it is important to preserve the equality of viscosities for models that are compared.
    % It is known that viscosity \(\mu=\mu_0\sqrt{T}\) for the hard-sphere (HS) model
    % and \(\mu=\mu_0T\) for the \(J = \rho(M[f]-f)\).
    % Therefore, it is more correct to use BGK model \(J = \rho(M[f]-f)/\sqrt{T}\),
    % which has the same viscosity as for HS model for all temperatures~\cite{Mieussens2000}.
    % The ES-BGK model will have the same viscosity and thermal conductivity for all temperatures.

    \item There are many references to the Grad's moment method and its regularized variants in the manuscript.
    Moreover, one of the goals of the paper is the competition with this approach.
    However, no relevant comparison is provided. The R13 and R26 models are very simple in the spacially homogeneous case.
    Therefore, it does not require much effort to incorporate the related profiles into the figures.
    Relaxation of shear stress and heat flow is constant as for the BGK model,
    but the higher moments of the R26 system do not decay exactly exponentially in general.
    It would be interesting to include these results in the comparison list.

    \item The linear least squares problem is solved based on the random set of collocation points.
    The authors present the results for various number of these points $N$,
    but do not provide any convergence information with respect to $N$.
    It is interesting to analyze some figures, where $N$ corresponds to the abscissa axis.
    If genuine random numbers are used, dependences of $N^{-1/2}$ are probably expected.
    Moreover, quasi-random sequences can be employed to improve the convergence order~\cite{Dick2013}.

    \item Three different algorithms are used to sample random points in the paper.
    However, all of them are far from the optimal one. The are some variance reduction techniques,
    which are used to improve the random sampling~\cite{Dick2013}.
    For example, the classical DSMC method employs the strategy of importance sampling,
    which is also used for dicrete-velocity solutions of the BE~\cite{Varghese1994, Kolobov2013}.
    It is also advisable to compare the sampling algorithms in some figure, where $N$ corresponds to the abscissa axis.

    \item In contrast, it is very difficult to distinguish curves in Fig.~1 and, especially, to determine, which algorithm is best.
    From Fig.~1, one can admit that algorithms are almost identical.

    \item In the third sampling approach presented in the paper, $\rho$ is used as a scaling parameter,
    but there is no information on what values of $\rho$ is used in the numerical experiments.

    \item Every time-integration step is followed by a conservative correction procedure.
    There are many approaches on how to handle the latter task.
    Some of them do not preserve positivity and, therefore, can lead to the stability issues.
    Consequently, the method of conservative correction should be specified in the formulas
    and justified at best.

    \item ``\emph{Substituting (1) into (2) and using the chain rule,
    we obtain the following overdetermined functional equation (3).}''
    Rigorously speaking, it is not correct to say about overdetermination for $\bxi\in\mathbb{R}^D$.
    Instead the system of functional equations can become overdetermined
    if (3) is replaced by the discrete-velocity analog (as in case of the collocation method)
    for $\bxi_j$, $j=1,\dots,r$ and $r>5k$.
    It is also worth emphasizing that the least squares problem (4) is well posed only if $r>5k$.

    \item The caption of Fig.~1 comprises some unclear points. First, which quadratures are used to compute the moments?
    Second, form of $f(t=0)$ is specified only in the next section and there is no reference to exact formula.

    \item Dimensioned quantites are redundant within the scope of the paper.

    \item Axes should be signed in Fig.~7 and colorbars should be provided.

    \item ``\emph{Our ultra-sparse approximation model provides a convenient mechanism
    to explore the process of relaxation in non-continuum gases.}''
    It is a confusing statement. First, the presented method cannot serve for accurate numerical analysis
    of spacially homogeneous relaxation, but can only be considered as efficient crude approximation.
    There is no evidence in the paper that Maxwellian-based approximation converge fast to the exact solution
    even for the considered class of problems.
    As for inhomogeneous problem, it is too early to give a positive conclusion.

    \item The abstract can also be improved in term of clearness for the general reader.
    The following comments can be taken into account:
    \begin{itemize}
        \item Term \emph{macro-parameter} is kind of jargon.
        Macroscopic variables/quantities/parameters are used instead.
        \item Term \emph{approximating stream} is also not established.
        \emph{Gaussians} or \emph{Maxwellians} are more common.
        \item ``\emph{... small to medium deviations ...}''. Sounds too vaguely.
        Instead it can be specified quantitavely, perhaps in terms of the Mach number.
        For instance, the supersonic and hypersonic cases are considered in the results.
        \item Terms \emph{equilibrium} or \emph{continuum limit}~\cite{Sone2007} is preferable over \emph{continuum}.
    \end{itemize}

    \item Finally, some typos are detected:
    \begin{itemize}
        \item Page 1, paragraph 2: ``intr\sout{i}oducing``
        \item Abbreviation \emph{MoM} is not transcripted.
        \item Legend for Fig.~3 informs that $P=5$ and $P=40$ are used, but $P=80$ is mentioned in the text.
    \end{itemize}
\end{enumerate}



%%% Conclusions
In summary, the submitted manuscript is recommended for a major revision
in order to examine the mentioned Enders' work~\cite{Ender1988},
compare with it, and, perhaps, incorporate some of findings.
In addition, the manuscript is deserved to be polished, according to the comments listed above.

\printbibliography

\end{document}
