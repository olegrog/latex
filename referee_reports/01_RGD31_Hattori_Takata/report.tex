\documentclass[11pt]{article}
\usepackage[utf8]{inputenc}
\usepackage[T1]{fontenc}
\usepackage[english]{babel}

\usepackage{amssymb, amsmath}
\usepackage[normalem]{ulem}         % for \sout
\usepackage{fullpage}

\title{Referee report on \\ "Slip/Jump Coefficients and Knudsen-Layer Corrections for the Shakhov Model
    Occuring in the Generalized Slip-Flow Theory" by M.~Hattori and S.~Takata}
\date{}

\newcommand{\Kn}{\mathrm{Kn}}
\newcommand{\Ma}{\mathrm{Ma}}

\begin{document}

\maketitle

%%% What is this about? What was done?
In the presented manuscript, the linear second-order Knudsen-layer problem is solved numerically with a high accuracy
for unsteady monatomic gas obeying the Shakhov model of the Boltzmann equation.
The obtained data are used for quantitative description of the gas behavior
within the generalized slip-flow theory, developed by the authors in the recent years.

%%% Why is the topic important? Key result.
The relaxation models of the Boltzmann equation are widely used for simulation of rarefied gas flows,
especially for engineering applications.
However, there is no any apriori estimates on their predictive capabilities.
This work provides important insights into influence of relaxation model on the gas behaviour in the Knudsen layer.

%%% How results was presented?
It is worth emphasizing that the manuscript is well-organized.
A quite complicated material is presented in a clear and self-consistent manner.
The referee hopes that the following optional minor comments could help authors to polish their manuscript.
Some of them are related to the style of presentation and represent the view from the side, rather that an objective facts.
\begin{enumerate}
    \item The diffusive time scale is assumed for the generalized slip-flow theory;
    however, the same slip/jump coefficients and Knudsen-layer corrections seem to be valid in the acoustic regime as well.
    Clarification of this point may increase the relevance of the obtained results.

    \item Some words about difference between ``elemental'' and ``elementary''.
    The former is mainly chemistry-related, while the latter is widely used for mathematical objects.
    To the referee's aware, term ``elemental'' was introduced into the field by one of the authors more that decade ago,
    however, still sounds dissonant.
    Moreover, ``elemental functions'' and ``elemental solutions'' looks confusing
    in comparison to classical ``solutions to the elementary half-space problems''.
    The Green's functions considered in [Takata09] are also formulated for \emph{elementary} sources,
    as would be commonly mentioned in the community.

    \item Typo after Eq.~(3): \(Lx_{xi}\) is the position o\sout{n}f the boundary.

    \item It should be mentioned in the text that the linearized scattering operator \(\mathcal{K}\)
    is associated with the kinetic boundary conditions.
    The expression of diffuse-reflection \(\mathcal{K}\) is also missing,
    however is assumed for the presented numerical results.

    \item Knudsen-layer function \(H_2^{(1)}\) is not used in the paper and can be easily removed from Eq.~(6).

    \item ``... because \(c_3^{(0)}\) is positive, irrespective of the gas model (see Table 1)''.
    To be exact, there is no proof that \(c_3^{(0)}\) is positive for all possible gas models.
    Instead, there is an evidence that it is positive for the considered models for \(\Pr=2/3\).

    \item ``... \(F_T\), which involves a second order jump coefficient, depends strongly on the molecular model''.
    This claim sounds unfounded, since in fact only one molecular model is being considered (hard spheres).
    The BGK-type models, on the other hand, are collision operator models.

    \item One of the important results of the present work is the provided opportunity
    to compare ES and Shakhov models quantitatively. It is seen that all slip/jump coefficients
    computed for the Shakhov model are closer to the hard-sphere ones in comparison to the ES model.
    \(b_8^{(1)}\) is the only exception. Is it a pure accident or has some explanation?

    \item Finally, the abstract lacks explicit words about the computational problem,
    which, nevertheless, has been solved skillfully and accurately.
    The modest word ``obtain'' can form a misconception about the main challanges associated with the problem,
    especially for readers who are not familiar with the details of the previous authors' papers.
    It seems worth to mention that the presented work is essentially about sophisticated computations.
\end{enumerate}

%%% Conclusions
In summary, the work is accomplished and presented at a high scientific level.
The obtained results are of wide interest and certainly deserve to be published.

\end{document}
