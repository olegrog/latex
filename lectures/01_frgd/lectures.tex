\documentclass[ucs,english,russian]{beamer}
\usepackage[utf8x]{inputenc}
\usepackage[T2A]{fontenc}
\usepackage{iwona} % [math]

\usetheme{Antibes} % AnnArbor, Antibes, JuanLesPins
\usecolortheme{crane} % seahorse, crane
\usefonttheme{professionalfonts}
\useoutertheme{mytree}

\title{Динамика жидкости и разреженного газа}
\author{Рогозин Олег}
\institute{Московский физико-технический институт}
\date{}

%%%%%%%%%%%%%%%%%%%%%%%%%%%%
% choose the lecture below %
%%%%%%%%%%%%%%%%%%%%%%%%%%%%
\includeonlylecture{week1}
%%%%%%%%%%%%%%%%%%%%%%%%%%%%

\AtBeginLecture{
	\frame{
		\centering{\Large Лекция \insertshortlecture/8} \\ \bigskip
		\begin{beamercolorbox}[ht=5.5ex,dp=2.5ex]{section in head/foot}
			\centering\Huge\insertlecture
		\end{beamercolorbox}
	}
}
\AtBeginPart{
	\frame{
		\frametitle{Содержание}\tableofcontents
	}
}
\newcommand{\dd}{\:\mathrm{d}}
\newcommand{\D}{\mathrm{d}}
\newcommand{\Kn}{\mathrm{Kn}}
\newcommand{\Hence}{\quad\Longrightarrow\quad}
\DeclareMathOperator*{\Div}{div}
\DeclareMathOperator*{\Grad}{grad}
\DeclareMathOperator*{\Rot}{rot}
\DeclareMathOperator*{\Const}{const}
\newcommand{\blue}[1]{ {\color{blue} #1} }

\begin{document}

\frame{\titlepage}

\lecture[1]{Идеальная жидкость}{week1}
\part{Идеальная жидкость}
\section{Уравнения движения}
\begin{frame}
	\frametitle{Математическое описание}
	Используем \blue{5 независимых функций}:
		\( \mathbf{v}(\mathbf{r},t),\quad p(\mathbf{r},t),\quad \rho(\mathbf{r},t) \)
	В идеальной жидкости пренебрегаем \alert{диссипативными} и \alert{тепловыми} процессами.
	
	Соотвественно \blue{5 уравнений движений}:
	\begin{itemize}
		\item уравнение непрерывности \( \frac{\D\rho}{\D t} = 0  \)
		\item уравнения Эйлера \( \frac{\D\mathbf{v}}{\D t} = - \frac1\rho\Grad{p} \)
		\item адиабатичность движения \( \frac{\D s}{\D t} = 0 \)
	\end{itemize}
	\( \frac\D{\D t} \) --- субстанциональная производная
\end{frame}

\begin{frame}
	\frametitle{Уравнение непрерывности}
	\begin{block}{}
		\vspace{-10pt}
		\[
			\alert{\frac\D{\D t}\int\rho\dd V \equiv}
			\frac{\partial}{\partial t}\int\rho\dd V + \oint\rho\mathbf{v}\dd\mathbf{f} = 0
			\Hence \frac{\partial\rho}{\partial t} + \Div\rho\mathbf{v} = 0
		\]
	\end{block}
\end{frame}

\begin{frame}
	\frametitle{Уравнение Эйлера}
	\begin{block}{}
		\[
			\int\rho\frac{\D\mathbf{v}}{\D t}\dd V = -\oint p\dd\mathbf{f}
			\Hence \rho\frac{\D\mathbf{v}}{\D t} + \Grad p = 0
			% for proof consider \int\mathrm{div}g\mathbf{A} = \oint g\mathbf{A}\dd f
		\]
		\[
			\frac{\D\mathbf{v}}{\D t} = \frac{\partial\mathbf{v}}{\partial t} + (\mathbf{v}\nabla)\mathbf{v}
			\Hence \frac{\partial\mathbf{v}}{\partial t} + (\mathbf{v}\nabla)\mathbf{v} = -\frac1\rho\Grad{p}
		\]
	\end{block}
\end{frame}

\begin{frame}
	\frametitle{Адиабатичность движения}
	\begin{block}{тепловая функция единицы массы}
		\[ \D w = T\dd s + \frac1\rho\dd p = \frac1\rho\dd p \]
	\end{block}
	\begin{block}{из векторного анализа}
		\[ \frac1{2}\Grad{\mathbf{v}^2} = [\mathbf{v}\Rot\mathbf{v}] + (\mathbf{v}\nabla)\mathbf{v} \]
	\end{block}
	\vspace{-20pt}
	\[
		\frac{\partial\mathbf{v}}{\partial t} - [\mathbf{v}\Rot\mathbf{v}] = -\Grad\left(w+\frac{v^2}2\right)
		\Hence \frac{\partial}{\partial t}\Rot\mathbf{v} = \Rot[\mathbf{v}\Rot\mathbf{v}]
	\]
\end{frame}

\section{Гидростатика}


\section{Уравнение Эйлера}
\begin{frame}
% 	\frametitle{Расщепление уравнения Больцмана}
	\begin{definition} % theorem, lemma, proof, corollary, or example
	A \alert{prime number} is a number that has exactly two divisors.
	\end{definition}
	Hello!
	\[f(x)=\int x^2 dx\]
	\begin{itemize}
	\item Answered Questions
	\begin{itemize}
	\item How many primes are there?
	\end{itemize}
	\item Open Questions
	\begin{itemize}
	\item Is every even number the sum of two primes?
	\end{itemize}
	\end{itemize}
\end{frame}

\section{Уравнение Бернулли}
\begin{frame}
 	\frametitle{Расщепление уравнения Больцмана}
	\begin{definition} % theorem, lemma, proof, corollary, or example
	A \alert{prime number} is a number that has exactly two divisors.
	\end{definition}
	Hello!

\end{frame}

\section{Теорема Томпсона}
\section{Несжимаемая жидкость}
\subsection{Функция тока}
\section{Потенциальное движение}
\subsection{Парадокс Даламбера}

\lecture[2]{Вязкая жидкость}{week2}
\part{Вязкая жидкость}

\end{document}

